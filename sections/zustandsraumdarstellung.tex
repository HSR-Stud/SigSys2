\section{Zustandsraumdarstellung \formelbuch{265} \tiny{$Revision: 999 $}}
Darstellung einer Differentialgleichung $n$. Ordnung durch ein
Differentialgleichungssystem von $n$ Gleichungen 1. Ordnung.

\subsection{Definition \formelbuch{267}}
\begin{tabular}{ll}
\parbox{10cm}{
	\includegraphics[width=10cm]{./images/zrd-schema.png}
	}
	& \parbox{8cm}{
		$\dot{\underline{x}}(t) = {\boldsymbol A} \underline{x}(t) + {\boldsymbol B}
		\underline{u}(t)$ \\
		$\underline{y}(t) = {\boldsymbol C} \underline{x}(t) + {\boldsymbol D}
		\underline{u}(t)$\\ 
		
		${\boldsymbol A}$: Systemmatrix ($n$ x $n$): Spalten entsprechen Ausgängen
		der Integratoren, Zeilen Eingänge ; \\ 
		${\boldsymbol B}$: Steuer- oder
		Eingangsmatrix ($n$ x $m$) ``senkrecht''; \\ ${\boldsymbol C}$: Beobachtungs- oder Ausgangsmatrix ($k$ x $n$)
		``waagrecht''; \\
		${\boldsymbol D}$: Übergangs- oder Durchgangsmatrix ($k$ x
		$m$)\\
		
		\textbf{m}: Anzahl Eingänge \\
		\textbf{n}: Anzahl Integratoren (Ordnung) \\
		\textbf{k}: Anzahl Ausgänge
	}
	Bestimmung der Matrizen A,B,C,D siehe auch \formelbuch{298}
 \end{tabular}

\subsection{ZRD im Zeitbereich \formelbuch{270}}

\subsection{ZRD im Frequenzbereich \formelbuch{147} \matlab{ss2tf}}
$$\boldsymbol{H(s)} = \frac{\underline{Y}(s)}{\underline{U}(s)} =
\boldsymbol{C}\left(s\boldsymbol{I_n}-\boldsymbol{A}\right)^{-1}\boldsymbol{B}+\boldsymbol{D}$$
\\
Die Grösse der Matrix $\boldsymbol {H(s)}$ entspricht der Grösse der
Durchgangsmatrix $\boldsymbol D$. $\boldsymbol{I_n}$ sei die Einheitsmatrix mit
Grösse $n$ x $n$.

\subsection{Übertragungsmatrizen \formelbuch{276}}
\[
H(s)=\frac{Y(s)}{U(s)}=\frac{b_{m} s^{m} + b_{m-1} s^{m-1} +\cdots+b_{1} s 
+ b_{0}}{s^{n} + a_{n-1} s^{n-1} + \cdots + a_{1} s + a_{0}}
\]\\
Allgemeine Formel für $m=1$ Eingang, $k=1$ Ausgang, $n=2$ Integratoren:
\begin{align*}
H(s) &= \left[
	\begin{array}{c c}
		C_{11}  & C_{12}\\
	\end{array}
	\right] \cdot \left( \left[
	\begin{array}{cc}
		s & 0\\
		0 & s\\
	\end{array}
	\right] - \left[
	\begin{array}{cc}
		A_{11} & A_{12}\\
		A_{21} & A_{22}\\
	\end{array}
	\right] \right)^{-1} \cdot \left[
	\begin{array}{c}
		B_{11}\\
		B_{21}\\
	\end{array}
	\right ]+ D \\
	&=  \frac{B_{11}C_{11}(s-A_{22}) + B_{11}C_{12}A_{21} + B_{21}C_{11}A_{12} + B_{21}C_{12}(s-A_{11})}
		{(s-A_{22})(s-A_{11}) - A_{12}A_{21}} + D
\end{align*}
		
Ist die Übertragungsfunktion $H_{ba}(s)$ vom \textbf{Eingang b} zum \textbf{Ausgang a} gesucht, so gilt:
\[
  H_{ba}(s) = \frac{Y_a(s)}{U_b(s)} = C(a,:) \cdot (s \cdot I -A)^{-1} \cdot B(:,b) + D(a,b)
\]


\subsection{Stabilität \formelbuch{285}}
Wenn alle Realteile der Eigenwerte $\lambda$ der Systemmatrix ${\boldsymbol A}$
negativ sind, ist ein LTI-System asymptotisch stabil, jedoch nicht umgekehrt:
$\left | \lambda\boldsymbol{I} - \boldsymbol{A} \right |   =0 \rightarrow \forall~\lambda \quad\Re \{\lambda\}<0$ \\
Sind die Realteile aller $\lambda$ kleiner oder gleich Null, so ist das System grenzstabil.

\newpage
\subsection{Beobachtbar- \& Steuerbarkeit \formelbuch{287}}
\subsubsection{Steuerbarkeit \matlab{ctrb}}
Gibt es Zustände von $\underline{x} (t)$ die nicht von den
Eingängen $\underline{u} (t)$ beeinflusst werden? Wenn ja,
dann ist das System nicht steuerbar!

Wenn $|Q_{Steuerbarkeit}|= \left| \left [ \boldsymbol{B~~AB~~ A^2B~\ldots~
A^{n-1}B} \right ] \right|  \neq 0$, dann ist das System vollständig steuerbar.

\subsubsection{Ausgangssteuerbarkeit}
Ein System ist vollständig ausgangssteuerbar, wenn es eine Steuerfunktion $\underline{u} (t)$ gibt,
welche die Ausgänge $\underline{y}(t)$ innerhalb einer endlichen Zeitspanne in einen Endwert bringt.
Wenn $rang(Q_{AusgStrbrkeit}) = rang \left( \left [ \boldsymbol{CB~~ CAB~~ CA^2B~\ldots~
CA^{n-1}B ~~ D}\right ] \right) = k$, dann ist das System vollständig ausgangssteuerbar.

\subsubsection{Beobachtbarkeit \matlab{obsv}}
Gibt es Zustände $\underline{x}(t)$ die keinen Einfluss auf die Ausgänge
$\underline{y}(t)$ haben? Wenn ja, kann man aus dem Verhalten von 
$\underline{y}(t)$ nicht auf die Zustände $\underline{x}(t)$ schliessen!
Das System ist nicht beobachtbar!


Wenn $|Q_{Beobachtbarkeit}| = \left| \left [ \boldsymbol{
\begin{array}{c}
 C\\
 CA\\
CA^2\\
\vdots \\
CA^{n-1}\\
\end{array}}\right ] \right| \neq 0$, dann ist das System vollständig
beobachtbar.

\subsubsection{Regelungsnormalform \formelbuch{277}}
\includegraphics[width=10cm]{./images/zrd-regelungsnormalform.png} \\
\scriptsize
\begin{equation*}
\left [ 
\begin{array}{c}
\dot{x}_1(t)\\
\dot{x}_2(t)\\
\vdots\\
\dot{x}_{n-1}(t)\\
\dot{x}_{n}(t)\\
\end{array}
\right ] =
\left [ 
\begin{array}{c c c c c}
0 & 1 & 0 & \ldots & 0\\
0 & 0 & 1 & \ldots & 0\\
\vdots & \vdots & \vdots & \ddots & \vdots\\
0 & 0 & 0 & \ldots & 1\\
-a_0 & -a_1 & -a_2 & \ldots & -a_{n-1}\\
\end{array}
\right ]\cdot
\left [ 
\begin{array}{c}
x_1(t)\\
x_2(t)\\
\vdots \\
x_{n-1}(t)\\
x_{n}(t)\\
\end{array}
\right ]+
\left [ 
\begin{array}{c}
0 \\
0\\
\vdots\\
0\\
1\\
\end{array}
\right ]\cdot
u(t),
\end{equation*}
\begin{equation*}
y(t) = 
\left [ 
\begin{array}{c c c c}
b_0-a_0b_n & b_1-a_1b_n & \ldots & b_{n-1}-a_{n-1}b_n\\
\end{array}
\right ] \cdot
\left [ 
\begin{array}{c}
x_1(t)\\
x_2(t)\\
\vdots \\
x_{n-1}(t)\\
x_{n}(t)\\
\end{array}
\right ]+
\left [ 
\begin{array}{c}
b_n \\
\end{array}
\right ] \cdot
u(t).
\end{equation*}
\normalsize


\begin{samepage}
\subsubsection{Beobachtungsnormalform \formelbuch{279}}
\includegraphics[width=10cm]{./images/zrd-beobachtungsnormalform.png} \\
\scriptsize
\begin{eqnarray*}
\left [ 
\begin{array}{c}
\dot{x}_1(t)\\
\dot{x}_2(t)\\
\vdots\\
\dot{x}_{n-1}(t)\\
\dot{x}_n(t)\\
\end{array}
\right ] &=&
\left [ 
\begin{array}{c c c c c}
0 & 0 & 0 & \ldots & -a_0\\
1 & 0 & 0 & \ldots & -a_1\\
0 & 1 & 0 & \ldots & -a_2\\
\vdots & \vdots &  \ddots & 0 & \vdots\\
0 & 0 & \ldots & 1 & -a_{n-1}\\

\end{array}
\right ]\cdot
\left [ 
\begin{array}{c}
x_1(t)\\
x_2(t)\\
\vdots \\
x_{n-1}(t)\\
x_{n}(t)\\
\end{array}
\right ]+
\left [ 
\begin{array}{c}
b_0-a_0b_n \\
b_1-a_1b_n\\
b_2-a_2b_n\\
\vdots\\
b_{n-1}-a_{n-1}b_n\\
\end{array}
\right ]\cdot
u(t),\\
y(t) &= &
\left [ 
\begin{array}{c c c c}
0 & 0 & \ldots & 1\\
\end{array}
\right ] \cdot
\left [ 
\begin{array}{c}
x_1(t)\\
x_2(t)\\
\vdots \\
x_{n-1}(t)\\
x_{n}(t)\\
\end{array}
\right ]+
\left [ 
\begin{array}{c}
b_n \\
\end{array}
\right ] \cdot
u(t).
\end{eqnarray*}
\normalsize
\end{samepage}
