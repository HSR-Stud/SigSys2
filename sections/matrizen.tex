\section{Matrizenrechnung}
\subsection{Übersicht}
	\begin{tabular}{l l}
    	Transponierte Matrix: & $A^T=[a_{ik}^T]=[a_{ki}]$ vertauschen der Zeilen
    	mit Spalten\\
    	Einheitsmatrix:& $I_n= 
			    	\begin{bmatrix} 
			        	1&0 & 0\\
			        	0&1&0\\
			        	0&0&1                               
			        \end{bmatrix}
$		    
    \end{tabular}

\subsection{Determinante}

	\textbf{2x2 Matrix}    
	$$ \det \begin{bmatrix} a_{11} & a_{12} \\ a_{21} & a_{22} \end{bmatrix} =
	a_{11} a_{22} - a_{12} a_{21}.  $$
	
	\textbf{3x3 Matrix}
	$$ \det \begin{bmatrix} a_{11} & a_{12} & a_{13} \\ a_{21} & a_{22}& a_{23} \\
	a_{31} & a_{32} & a_{33} \end{bmatrix} \\ = a_{11} a_{22} a_{33} + a_{12}
	a_{23} a_{31} + a_{13} a_{21} a_{32} - a_{13} a_{22} a_{31} - a_{12} a_{21}
	a_{33} - a_{11} a_{23} a_{32}.  $$
	
	\textbf{Dreiecksmatrix} - Alle Elemente entweder ober- oder unterhalt der Hauptdiagonale $= 0$
	$$\det A =a_{11}\cdot a_{22}\dotsb a_{nn} \quad  \quad \text{Die Det. ist das Produkt
	der Hauptdiagonal-Einträge. Gilt somit auch für Diagonalmatritzen.} $$
	
	\textbf{Null $(|A| = 0)$} - Wenn $A$ eine (n,n)-Matrix ist, so wird $|A| = 0$ unter einer der
	folgenden Bedingungen:
	\begin{itemize}
    	\item Zwei Zeilen/Spalten sind linear abhängig (gleich oder ein Vielfaches der anderen).
    	\item Alle Elemente einer Zeile/Spalte sind Null. \\
  	\end{itemize} 
	
	\textbf{Allgemein:}
	$$A\epsilon M_n: \det A =    
	\begin{vmatrix}
    	a_{11} & a_{12}& \ldots & a_{1n}\\
    	a_{21}& &\ldots & \\
    	\ldots \\
    	a_{n1} & & \ldots & a_{nn}    			
    \end{vmatrix}=
	(-1)^{1+1}a_{11}D_{11} + (-1)^{1+2}a_{12}D_{12}+ \ldots +
	(-1)^{1+n}a_{1n}D_{1n}$$
	
	\subsubsection{Unterdeterminante}
	$$D_{11}=
	\begin{vmatrix}
    	a_{22} & \ldots & a_{2n}\\
    	\ldots\\
    	a_{n2}& \ldots & a_{nn}
    \end{vmatrix} 	\\
	D_{12}=
	\begin{vmatrix}
    	a_{21} & a_{23}& \ldots & a_{2n}\\
    	\ldots\\
    	a_{n1}& a_{n3}&\ldots & a_{nn}
    \end{vmatrix}$$\\
	$D_{ij}$ die (n-1)$ \times $(n-1)-Untermatrix von D ist, die durch Streichen der
	i-ten Zeile und j-ten Spalte entsteht.\\
	Diese Methode ist zu empfehlen, wenn die Matrix in einer Zeile oder Spalte
	bis auf eine Stelle nur Nullen aufweisst.
	Dies lässt sich meist mit dem Gausverfahren bewerkstelligen.
	
\subsection{Gaussverfahren}
	Durch Addition und Subtraktion einzelner Zeilen (auch von Vielfachen einer
	Zeile) werden einzelne Stellen auf Null gebracht. zB:\\
	$\begin{bmatrix}
    	a_{11} & a_{12}& \ldots & a_{1n}\\
    	a_{21}& &\ldots & \\
    	\ldots \\
    	a_{n1} & & \ldots & a_{nn}    			
    \end{bmatrix}=
	\begin{bmatrix}
    	a_{11} & a_{12}& \ldots & a_{1n}\\
    	k a_{21}-n a_{11}& ka_{22}-n a_{12}&\ldots & k a_{2n} - n a_{1n}\\
    	\ldots \\
    	a_{n1} & & \ldots & a_{nn}    			
    \end{bmatrix}$ \\
	Die n * erste Zeile wurde von der k * zweiten Zeile abgezogen ($a_{2.}= 
	k a_{2.}- n a_{1.}$)
	
\subsection{Inverse Matrix \small{(Existiert nur wenn Matrix regulär: $\det A \neq 0$)}}
\begin{minipage}{7cm}
	\textbf{2x2 Matrix:}    
	$$ A^{-1} = \begin{bmatrix} a & b \\ c & d \\ \end{bmatrix}^{-1} = \frac{1}{ad
	- bc} \begin{bmatrix} d & -b \\ -c & a \\ \end{bmatrix} $$
\end{minipage}
\begin{minipage}{11cm}
	\textbf{3x3 Matrix:}
  $$  A^{-1} = \begin{bmatrix} a & b & c\\ d & e & f \\ g & h & i \\ \end{bmatrix}^{-1} =
  \frac{1}{\det(A)} \begin{bmatrix} ei - fh & ch - bi & bf - ce \\ fg - di & ai
  - cg & cd - af \\ dh - eg & bg - ah & ae - bd \end{bmatrix} $$
\end{minipage}\\

\textbf{Diagonalmatrix} (Alle Elemente ausserhalb der Hauptdiagonale $= 0$, Elemente auf
Hauptdiagonale sind Eigenwerte $\lambda_i$): \\ 
Alle Elemete elementweise invertieren - Kehrwert. $\quad \Rightarrow \quad $\textit{Gilt nur wenn
alle Elemente auf der Hauptdiagonale $\neq 0$ sind.}\\

\textbf{Allgemein:}\\
	$A^{-1}= \begin{bmatrix}
    	a_{11} & a_{12}& \ldots & a_{1n}\\
    	a_{21}& &\ldots & \\
    	\ldots \\
    	a_{n1} & & \ldots & a_{nn}    			
    \end{bmatrix}^{-1}$
	\begin{enumerate}
		\item $A^T$ bestimmen (Zeilen und Spalten vertauschen) $A^{T}= \begin{bmatrix}
    	a_{11} & a_{21}& \ldots & a_{n1}\\
    	a_{12}& &\ldots & \\
    	\ldots \\
    	a_{1n} & & \ldots & a_{nn}    			
    \end{bmatrix}$	
		\item Bei $A^T$ jedes Element $a_{ij}$ durch Unterdet. $D_{ij}$ mit
		richtigem Vorzeichen ersetzen $A^*=	\begin{bmatrix}
			(-1)^{1+1}D_{11} &  \ldots	& (-1)^{1+n} D_{1n}\\
			\ldots\\
			(-1)^{n+1} D_{n1}& \ldots  & (-1)^{n+n} D_{nn}
		\end{bmatrix}$
		\item $A^{-1} = \frac{A^*}{\det A}$ 
    \end{enumerate}
 
 \subsection{Diagonalisierung}
 	\begin{enumerate}
       \item Eigenwerte $\lambda$ ausrechnen: $\det (A - I_n \lambda)=0$
       \item Eigenvektoren $\vec{v}$ bilden: $(A- \lambda I_n)\vec{v}=0$
       \item Transformationsmatrix: $T= [\vec{v_1} \ldots \vec{v_n}]$
       \item $T^{-1}$ berechnen (Achtung ist A symmetrisch, dh. $A^T=A$ und
       oder alle EV senktrecht zueinander, dann $T^{-1}=T^T$)
       \item $D=\begin{bmatrix}
                	\lambda_1 &0 &0\\
                	0& \lambda_2 &0\\
                	0& 0& \lambda_3
                \end{bmatrix} = A_{diag} = T^{-1}AT$
     \end{enumerate}
     
     
\subsection{Eigenwerte}
Die Eigenwerte $\lambda$ erhält man folgendermassen ($I$ ist die Einheitsmatrix):
\[
	|\lambda I - A| = 0 \qquad \text{nach } \lambda \text{ auflösen}
\]

