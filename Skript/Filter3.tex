% Gauss filter Tabellen LC etc.

\clearpage
\setcounter{section}{0}
\renewcommand{\thesection}{\thechapter.\Alph{section}}
\vspace*{20pt}
{\LARGE\textbf{Anhang zum Kapitel~\thechapter}}
\section{Nomogramme\index{Nomogramm} zur Bestimmung der Filterordnung\index{Filter!ordnung}}
Die ben"otigte Filterordnung kann mit Hilfe der nachfolgenden
Nomogramme bestimmt werden.  Dazu bestimmt man die Lage des Punktes
$P_{5}$ aus den Werten $A_{\max}$, $A_{\min}$ und $\Omega_{S}$ wie in
der folgenden Skizze gezeigt.  Die ben"otigte Filterordnung f"ur das Tiefpassfilter entspricht
dann dem Wert der n"achst h"oher gelegenen Kurve ($\rightarrow n$).
\begin{figure}[!htb]
\begin{center}
  \bild{/filter/FIL79.fig.eps,width=0.7}\caption{Gebrauch der Nomogramme}
\end{center}
\vspace*{-4mm}
\end{figure}\\
\nit Aus den folgenden drei Nomogrammen (Abb.~\ref{nomo-BW}-\ref{nomo-CC}) ist ersichtlich, dass f"ur die Filterordnung $n$ (f"ur fixe $A_{\min}$, $A_{\max}$ und $\Omega=\Omega_{S}/\Omega_{D}$) gilt: $n_{\text{Butterworth}}\geq n_{\text{Tschebyscheff}_{(\text{I,II})}}\geq n_{\text{Cauer}}$.\\
\nit Zur Bestimmung der Filterordnung k"onnen auch die Formel f"ur die Ordnung oder der entsprechende \mb verwendet werden.
\clearpage
\begin{figure}[!htb]% matlab nomogramm.m
\begin{center}\vspace*{-5.5mm}\hspace*{-5mm}
  \bild{/filter/FILn021.eps,width=1.1}\vspace*{-1.8cm}\caption{Nomogramm f"ur Butterworth-Filter der Ordnung $n=1\ldots 20$\label{nomo-BW}}
\end{center}
\end{figure}
\vspace*{-0.7cm}\nit Es kann auch der \mb {\tt buttord}\index{buttord@{\tt buttord}} verwendet werden oder Formel~\ref{eq: butterwort_ordnung}:
{\footnotesize\begin{equation*}
n \geq\frac{\log{\left[\displaystyle\frac{10^{A_{\min}/10}-1}
{10^{A_{\max}/10}-1}\right]}}
{2 \cdot \log{\klam{\Omega_S/\Omega_D}}}.
\end{equation*}}

\clearpage
\begin{figure}[!htb]% matlab nomogramm.m 
\begin{center}\vspace*{-5.5mm}\hspace*{-5mm}
  \bild{/filter/FILn022.eps,width=1.1}\vspace*{-1.7cm}\caption{Nomogramm f"ur -Filter der Ordnung $n=1\ldots 20$\label{nomo-Tsche}}
\end{center}
\end{figure}
\vspace*{-0.7cm}\nit Man kann auch den \mb {\tt cheb1ord}\index{cheb1ord@{\tt cheb1ord}} ({\tt cheb2ord}\index{cheb2ord@{\tt cheb2ord}}) verwenden oder Formel~\ref{ordnung_tschebyscheff}:
{\footnotesize
\begin{equation*}
n\geq\frac{{\rm Arcosh}\sqrt{\displaystyle\frac{10^{A_{\min}/10}-1}
{10^{A_{\max}/10}-1}}}{{\rm Arcosh}\klam{\Omega_{S}/\Omega_{D}}}.
\end{equation*}}
\clearpage
\begin{figure}[!htb]% matlab nomogramm.m
\begin{center}
\vspace*{-5.5mm}\hspace*{-7mm}
  \bild{/filter/FILn023.eps,width=1.1}\vspace*{-1.7cm}\caption{Nomogramm f"ur Cauer-Filter (elliptische Filter) der Ordnung $n=1\ldots 13$\label{nomo-CC}}
\end{center}
\end{figure}
\vspace*{-0.7cm}\nit Man kann auch den \mb {\tt ellipord} verwenden oder Formel~\ref{ordnung_cauer}:
{\footnotesize\begin{equation*}
\hspace*{-3mm}n\geq\frac{K\left(\left( \frac{\Omega_D}{\Omega_S}\right)^2\right)  K\left(1-\frac{10^{A_{\max}/10}-1}{10^{A_{\min}/10}-1}\right) } {K\left(1-\left(\frac{\Omega_D}{\Omega_S}\right)^2\right )K\left(\frac{10^{A_{\max}/10}-1}{10^{A_{\min}/10}-1} \right)},\text{mit}\quad K(k)=\int_0^{\frac{\pi}{2}}\frac{d\theta}{\sqrt{1-k\sin^2\theta}}
\end{equation*}}
\section{Normierte Tiefpass"ubertragungsfunktionen}
\subsection{Butterworth-Filter}
\[ \mbox{Ansatz:} \hspace{1cm} |T(j\Omega)|^{2}=\frac{1}{1+\Omega^{2n}} \]
Zur Tabellierung der auf die {\bf 3dB-Grenzfrequenz\index{Grenzfrequenz} normierten
"Ubertragungsfunktionen} $T(S)$ wurde $\maxs{S}|T(S)|=1$ gew"ahlt, wobei
%\begin{eqnarray*}
%T(S) &=& \frac{1}{D(S)} \\
%D(S) &=& S^{n}+b_{n-1}S^{n-1}+ \ldots + b_{2}S^{2}+b_{1}S+b_{0}. 
%\end{eqnarray*}
\begin{equation*}
T(S)=\frac{1}{D(S)}\quad\text{und}\quad D(S)=S^{n}+b_{n-1}S^{n-1}+ \ldots + b_{2}S^{2}+b_{1}S+b_{0}. 
\end{equation*}
Die Koeffizienten von $D(S)$ k"onnen den Tabellen~\ref{koef-BW}
und~\ref{fak-BW} entnommen oder mit Hilfe des \matlogo-Befehls {\tt buttap}\index{buttap@{{\tt buttap}}}
bestimmt werden.  Die 3~dB-Grenzfrequenz kann mit der Beziehung
\[ \omega_{\text{3dB}}=\sqrt[2n]{\frac{1}{10^{A_{\max}/10}-1}}\cdot \omega_{D}\]
aus den Gr"ossen $\omega_{D}$ und $A_{\max}$ berechnet werden.
\begin{table}[!htb]
\begin{center}
{\scriptsize
\begin{tabular}{|c||c|c|c|c|c|c|c|c|c|c| }\hline
$n$ & $b_0$ & $b_1$ & $b_2$ & $b_3$ & $b_4$ & $b_5$ & $b_6$ & $b_7$ & $b_8$ & $b_9$ \\ \hline\hline
  1 & 1 & & & & & & & & &   \\ \hline
 2 & 1 & $\sqrt{2}$ & & & & & & & & \\ \hline
 3 & 1 & 2       & 2 & & & & &  & & \\ \hline
 4 & 1 & 2.61313 & 3.41421 & 2.61313 & & & & & & \\ \hline
 5 & 1 & 3.23607 & 5.23607 & 5.23607 & 3.23607 & & & & &  \\ \hline
 6 & 1 & 3.86370 & 7.46410 & 9.14162 & 7.46410 & 3.86370 & & & & \\ \hline   
 7 & 1 & 4.49396 & 10.09783 & 14.59179 & 14.59179 & 10.09783 & 4.49396 & & &  \\ \hline   
 8 & 1 & 5.12583 & 13.13707 & 21.84615 & 25.68836 & 21.84615 & 13.13707 & 5.12583 & & \\ \hline
 9 & 1 & 5.75877 & 16.58172 & 31.16344 & 41.98638 & 41.98638 & 31.16344 & 16.58172 & 5.75877  & \\ \hline
 10 & 1 &  6.39245 & 20.43173 & 42.80206 & 64.88240 & 74.23343 & 64.88240 & 42.80206 & 20.43173 & 6.39245  \\ \hline
\end{tabular}\vspace*{-2mm}\caption{Koeffizienten $b$ von $D(S)$. \label{koef-BW}}
}
\end{center}
\vspace*{-8mm}
\end{table}~\\
\nit Das bedeutet z.~B.,  dass die UTF des normierten Butterworth-Tiefpass Filters 4.~Ordnung wie folgt aussieht:\\
\begin{equation*}
T(S)=\frac{1}{S^4+2.61313\cdot S^3+3.41421\cdot S^2+2.61313\cdot S+1}.
\end{equation*}
\begin{table}[!htb]
\begin{center}
{\footnotesize
\begin{tabular}{|c||l|}\hline
$n$ & $D(S)$ \\ \hline\hline
 1 & $(1+S)$     \\ \hline
 2 & $(1+\sqrt{2}S+S^2)$ \\ \hline
 3 & $(1+S)(1+S+S^2)$  \\ \hline
 4 & $(1+0.765S+S^2)(1+1.848S+S^2)$ \\ \hline
 5 & $(1+S)(1+0.618S+S^2)(1+1.618S+S^2)$    \\ \hline
 6 & $(1+0.518S+S^2)(1+1.414+S^2)(1+1.932S+S^2)$  \\ \hline   
 7 & $(1+S)(1+0.445S+S^2)(1+1.247S+S^2)(1+1.802S+S^2)$   \\ \hline   
 8 & $(1+0.390S+S^2)(1+1.111S+S^2)(1+1.663S+S^2)(1+1.962S+S^2)$  \\ \hline
 9 & $(1+S)(1+0.347S+S^2)(1+S+S^2)(1+1.532S+S^2)(1+1.880S+S^2)$  \\ \hline
10 & $(1+0.313S+S^2)(1+0.908S+S^2)(1+1.414S+S^2)(1+1.782S+S^2)(1+1.975S+S^2)$  \\ \hline
\end{tabular}\vspace*{-2mm}\caption{Faktorzerlegung von $D(S)$ \label{fak-BW}}
}
\end{center}
\vspace*{-8mm}
\end{table}


\clearpage

\begin{table}[!htb]
Die konjugiert-komplexen Polpaare weisen folgende Polg"uten auf:
\begin{center}
{\footnotesize
\begin{tabular}{|c||c|c|c|c|c|c|c|c|c|  }\hline
$n$ & 2 & 3 & 4 & 5 & 6 & 7 & 8 & 9 & 10 \\ \hline\hline
$q_{p1}$ & 0.707 & 1 & 1.307 & 1.618 & 1.932 & 2.247 & 2.563 & 2.8794  & 3.1962\\ \hline
$q_{p2}$ &      &        & 0.541 & 0.618 & 0.707 & 0.802 & 0.900 & 1 &  1.1013 \\ \hline
$q_{p3}$ &      &        &      &      & 0.518 & 0.555 & 0.601 & 0.6527 & 0.7071\\ \hline
$q_{p4}$ &      &        &      &      &      &      & 0.510 & 0.5321 & 0.5612\\ \hline
$q_{p5}$ &      &        &      &      &      &      &   &  & 0.5062\\ \hline
\end{tabular}
}
\end{center}\vspace*{-2mm}
\caption{Polg"uten der konj.-kompl. Polpaare der Butterworth-Filter\label{polg-BW}}
\end{table}
\aufg
Bestimmen Sie die Polg"uten der Butterworth-Filter von Tabelle~\ref{polg-BW} mit Hilfe des \mb
{\tt buttap}.\index{buttap@{{\tt buttap}}}


\clearpage
% GAUSS filter
\subsection{Kritische-ged"ampfte Filter (Gauss-Filter)}\label{anhang_kritisch}\index{Gauss-Filter}\index{Filter!kritisch-ged{\"a}mpft}
\nit Zur Tabellierung der normierten "Uber\-tra\-gungs\-funk\-tionen wurde $\maxs{S}|H(S)|=1$, {\bf sowie $H(j)=\frac{1}{\sqrt{2}}$ (auf 3~dB-Grenzfrequenz\index{Grenzfrequenz} normierte
"Ubertragungsfunktionen)} gew"ahlt:
\begin{equation*}
H(S)=\frac{K}{D(S)},\text{ wobei}\qquad D(S)=S^{n}+b_{n-1}S^{n-1}+ \ldots + b_{2}S^{2}+b_{1}S+b_0,
\end{equation*}
wobei immer gilt: $K=b_0$.

\begin{table}[!htb]
\begin{center}
{\scriptsize
\begin{tabular}{|c||c|c|c|c|c|c|c|c|c|c| }\hline
$n$ & $b_0=K$ & $b_1$ & $b_2$ & $b_3$ & $b_4$ & $b_5$ & $b_6$ & $b_7$ & $b_8$ & $b_9$ \\ \hline\hline
  1 & 1 & & & & & & & & &   \\ \hline
 2 & 2.4142 & 3.1075 & & & & & & & & \\ \hline
 3 & 7.5464 & 11.5420 & 5.8844 & & & & &  & & \\ \hline
 4 & 27.9335 & 48.6020 & 31.7113 & 9.1958 & & & & & & \\ \hline
 5 & 117.2829 & 226.1297 & 174.3977 & 67.2502 & 12.9663 & & & & &  \\ \hline
 6 & 544.50 & 1143.3 & 1000.2 & 466.69 & 122.49 & 17.146 & & & & \\ \hline  
 7 & 2748.4 & 6206.9 & 6007.6 & 3230.4 & 1042.2 & 201.75 & 21.697 & & &  \\ \hline   
 8 & 14902.4 & 35866.6 & 37766.0 & 22723.4 & 8545.3 & 2056.6 & 309.37 & 26.5918 & & \\ \hline
 9 & 86027.3 & 219071.5 & 247943.6 & 163695.5 & 69476.1 & 19658.1 & 3708.2 & 449.66 & 31.8079 & \\ \hline
 10 & 525025.6 &  1406573.6 & 1695730.8 & 1211455.9 & 567972.9 & 182595.8 & 40765.4 & 6240.7 & 626.97 & 37.327  \\ \hline
\end{tabular}\vspace*{-2mm}\caption{Koeffizienten $b$ von $D(S)$ und $K$ der normierten UTF $H(S)=\frac{K}{D(S)}$ der kritisch-ged"ampften Filter ($H(j)=\frac{1}{\sqrt{2}}$)}
}
\end{center}
\vspace*{-8mm}
\end{table}~\\
Die Entormierung der Tabellenwerte erfolgt mit $S=\frac{s}{\omega_{3\text{dB}}}$ mit der maximalen D"ampfung  $A_{\text{max}}$ im Durchlassbereich $\omega_D$ und der Ordnung $n$
\begin{equation}
\omega_{3\text{dB}}=\frac{\omega_D \cdot{\sqrt{2^{1/n}-1}} }{\sqrt{10^{\frac{A_{\text{max}}}{10\cdot n}}-1}}. 
\end{equation}

\bsp{Die entnormierte UTF eines kritisch-ged"ampften Filters der Ordnung 3 ist, mit $A_{\text{max}}=1$~dB,$\omega_D=1000$, und  somit $\omega_{3\text{dB}}=\frac{\omega_D\cdot{\sqrt{2^{1/n}-1}} }{\sqrt{10^{\frac{A_{\text{max}}}{10\cdot n}}-1}}=1805.0$ und $S=\frac{s}{\omega_{3\text{dB}}}=\frac{s}{1805.0}$ }
\begin{equation*}
H(s)=\frac{4.43811e10}{s^3+ 1.062154e4s^2+    3.76057e7s+    4.43811e10}
\end{equation*}

\begin{table}[!htb]
\begin{center}
{\footnotesize
\begin{tabular}{|c||c|c|}\hline
$n$ & $D(S)$ & $\omega_c=\frac{1}{\sqrt{\sqrt[n]{2}-1}}$ \\ \hline\hline
  1 & $(\omega_c +S)^1$ & $1$   \\ \hline
  2 & $(\omega_c +S)^2$ & $1.5538$   \\ \hline
  3 & $(\omega_c +S)^3$ & $1.9615$   \\ \hline
  4 & $(\omega_c +S)^4$ & $2.2990$   \\ \hline
  5 & $(\omega_c +S)^5$ & $2.5933$   \\ \hline
  6 & $(\omega_c +S)^6$ & $2.8576$   \\ \hline
  7 & $(\omega_c +S)^7$ & $3.0995$   \\ \hline
  8 & $(\omega_c +S)^8$ & $3.3240$   \\ \hline
  9 & $(\omega_c +S)^9$ & $3.5342$   \\ \hline
 10 & $(\omega_c +S)^{10}$ & $3.7327$   \\ \hline
\end{tabular}\vspace*{-2mm}\caption{Faktorzerlegung von $D(S)$}
}
\end{center}
\vspace*{-8mm}
\end{table}~\\



\vspace*{-6mm}
\subsubsection{Bemerkung:}
W"ahlt man als Normierung f"ur jedes der $n$ Teilfilter eines kritisch-ged"ampften Filters $n.$~Ordnung $\omega_c=1$, so erh"alt man $n$ identische Filter mit jeweils 3~dB D"ampfung bei $\omega=1$, was bedeutet, dass die gesamte UTF bei $\omega=1$ eine D"ampfung von $n\cdot 3$~dB hat. Stellt man diese Filter dar, so erh"alt man f"ur den Z"ahler der UTF jeweils 1 und die Nennerkoeffizienten ergeben sich zu den Binomialkoeffizienten\index{Binomialkoeffizienten}, d.h. zum Pascalschen Dreieck.\index{Pascalsches Dreieck}

\bsp{UTF des kritisch-ged"ampften Filters 5.~Ordnung mit $\omega_c=1$}
\begin{equation*}
H(s)=\frac{1}{s^5+5S^4+10S^3+10S^2+5S+1}
\end{equation*}~\\

\nit Mit dem Wissen des Pascalschen Dreiecks k"onnen wir die UTF jedes kritisch-ged"ampften Filters $n.$~Ordnung, mit $A_{\text{max}}$, $\omega_D$, und $\omega_c=\frac{\omega_D}{\sqrt{10^{\frac{A_{\text{max}}}{10\cdot n}}-1}}$ direkt aufstellen: 
\begin{equation}
H(s)=\frac{\omega_c^n}{\sum\limits_{i=0}^{n} {n \choose i} s^i\omega_c^{n-i} }.
\end{equation}

\bsp{UTF  des kritisch-ged"ampften Filters 5.~Ordnung mit $A_{\text{max}}=0.5$~dB und $\omega_D=100$}
\begin{equation*}
\omega_c=\frac{100}{\sqrt{10^{\frac{0.5}{10\cdot n}}-1}}=655.2203
\end{equation*}
und 
\begin{equation*}
H(s)=\frac{1.2079e14}{s^5+3.2761e3s^4+4.2931e6s^3+2.8130e9s^2+9.2155e11s+1.2079e14}
\end{equation*}
und mit dem {\tt bodemag} erhalten wir folgende Abbildung:\index{bodemag@{\tt bodemag}}
\begin{figure}[!htb]\vspace*{-3mm}
\begin{center}
  \bild{/filter/FIL_krit_BSP1.eps,width=0.5}\vspace*{-4mm}\caption{Amplitudengang des kritisch-ged"ampften Filters 5.~Ordnung mit $A_{\text{max}}=0.5$~dB und $\omega_D=100$}
\end{center}
\vspace*{-6mm}
\end{figure}



% Matlab Befehle: [z,b,k]=cheb1ap(8,1) (n,Amax)
% poly(p) -> D(S)
\subsection{Tschebyscheff-Filter (Tschebyscheff-I)}\label{anhang_Tschebyscheff}
\[\mbox{Ansatz}\hspace{1cm}|T(j\Omega)|^{2}=\frac{1}{1+e^{2}C_{n}^{2}(\Omega)}\]
Die Tschebyscheff-Polynome k"onnen der Tabelle~\ref{c-poly} entnommen werden.
\begin{table}[!htb]
\begin{center}
{\footnotesize
\begin{tabular}{|c||l|}\hline
$n$ &  $C_{n}(\Omega)$\\ \hline\hline
 0 & 1 \\ \hline
 1 & $\Omega$     \\ \hline
 2 & $2\Omega^2-1$ \\ \hline
 3 & $4\Omega^3-3\Omega$  \\ \hline
 4 & $8\Omega^4-8\Omega^2+1$ \\ \hline
 5 & $16\Omega^5-20\Omega^3+5\Omega$    \\ \hline
 6 & $32\Omega^6-48\Omega^4+18\Omega^2-1$  \\ \hline   
 7 & $64\Omega^7-112\Omega^5+56\Omega^3-7\Omega$   \\ \hline   
 8 & $128\Omega^8-256\Omega^6+160\Omega^4-32\Omega^2+1$  \\ \hline
\end{tabular}\caption{Tschebyscheff-Polynome $C_{n}(\Omega)$ \label{c-poly}}
}
\end{center}
\end{table}\\
\nit Zur Tabellierung der auf die {\bf Rippelgrenzfrequenz\index{Rippel!grenzfrequenz} normierten 
"Uber\-tra\-gungs\-funk\-tionen} wurde $\maxs{S}|T(S)|=1$ gew"ahlt:
\begin{equation*}
T(S)=\frac{K}{D(S)},\text{ wobei}\qquad D(S)=S^{n}+b_{n-1}S^{n-1}+ \ldots + b_{2}S^{2}+b_{1}S+b_{0}. 
\end{equation*}
\nit Die Koeffizienten k"onnen mit Hilfe vom \matlogo-Befehl {\tt
  cheb1ap} bestimmt werden, oder den Tabellen~\ref{koef-0.1}
bis~\ref{fak-3} entnommen werden.  Die
3~dB-Grenzfrequenz\index{3-dB!Grenzfrequenz} $(\omega_{\text{3dB}})$
kann mit der Beziehung
\[\omega_{\text{3dB}}=\cosh{\left[\left(\frac{1}{n}\right){\rm Arcosh}\left(\frac{1}{e}\right)\right]} \cdot
\omega_{D} \] aus den Gr"ossen $\omega_{D}$ (Rippelgrenzfrequenz) und
$e=\sqrt{10^{A_{\max}/10}-1}$ bestimmt werden, oder der
Tabelle~\ref{3db-rippel} entnommen werden.
\begin{table}[!htb]
\begin{center}
{\footnotesize
\begin{tabular}{|c||c|c|c|c|c|c|c|c|}\hline
{} & $n=1$ & $n=2$ & $n=3$ &  $n=4$ &  $n=5$ & $n=6$ & $n=7$ & $n=8$ \\ \hline\hline
$A_{\max}$=0.1dB & 6.55220 & 1.94322 & 1.38899 & 1.21310 & 1.13472 & 1.09293 & 1.06800 & 1.05193 \\ \hline
$A_{\max}$=0.5dB & 2.86278 & 1.38974 & 1.16749 & 1.09310 & 1.05926 & 1.04103 & 1.03009 & 1.02301 \\ \hline
$A_{\max}$=1dB & 1.96523 & 1.21763 & 1.09487 & 1.05300 & 1.03381 & 1.02344 & 1.01721 & 1.01316 \\ \hline
$A_{\max}$=2dB & 1.30756 & 1.07414 & 1.03273 & 1.01837 & 1.01174 & 1.00815 & 1.00598 & 1.00458 \\ \hline
$A_{\max}$=3dB & 1.00238 & 1.00059 & 1.00026 & 1.00015 & 1.00010 & 1.00007 & 1.00005 & 1.00004 \\ \hline
\end{tabular}\vspace*{-3mm}\caption{Verh"altnis $\left(\frac{\omega_{\text{3dB}}}{\omega_D}\right)$ zwischen der 3dB-Grenzfrequenz $(\omega_{\text{3dB}})$ und der Rippelgrenzfrequenz $(\omega_D)$} \label{3db-rippel}
}
\end{center}
\vspace*{-6mm}
\end{table}


\aufg
Wie gross ist das Verh"altnis von $\left(\frac{\omega_{\text{3dB}}}{\omega_D}\right)$ bei einem Tschebyscheff-Filter der Ordnung $n=23$ mit Rippel $e=0.05$? Wie gross ist $A_{\max}$?\\
% Loesung
% A_max=0.01084381=10*log10(e^2+1)
% w3db/wd=1.012885058703690=cosh(1/23*acosh(1/0.05))


\clearpage

\begin{table}[!htb]
\begin{center}
{\footnotesize
\begin{tabular}{|c||c|c|c|c|c|c|c|c||c|}\hline
$n$ & $b_0$ & $b_1$ & $b_2$ & $b_3$ & $b_4$ & $b_5$ & $b_6$ & $b_7$ & $K$\\ \hline\hline
 1 & 6.55220 & & & & & & & & 6.55220   \\ \hline
 2 & 3.31329 & 2.37209 & & & & & & & 3.27610\\ \hline
 3 & 1.63809 & 2.62953 & 1.93883 & & & & & & 1.63805  \\ \hline
 4 & 0.82851 & 2.02550 & 2.62680 & 1.80377 & & & & & 0.819025 \\ \hline
 5 & 0.40951 & 1.42556 & 2.39696 & 2.77071 & 1.74396 & & & & 0.4095127  \\ \hline
 6 & 0.20713 & 0.90176 & 2.04784 & 2.77908 & 2.96575 & 1.71217 & & & 0.2047564  \\ \hline   
 7 & 0.10238 & 0.56179 & 1.48293 & 2.70514 & 3.16925 & 3.18350 & 1.69322 & & 0.102378 \\ \hline   
 8 & 0.05179 & 0.32645 & 1.06667 & 2.15932 & 3.41855 & 3.56485 & 3.41297 & 1.68104 & 0.0511891 \\ \hline
\end{tabular}\vspace*{-1mm}\caption{Koeffizienten von $D(S)$ f"ur eine UTF mit 0.1~dB Rippel $(e=0.152620=\sqrt{10^{0.1/10}-1})$}\label{koef-0.1}
}
\end{center}
\vspace*{-4mm}
\end{table}
\aufg Bestimmen Sie f\"ur ein normiertes Tschebyscheff-TP-Filter der
Ordnung 6 alle Koeffizienten $b_0$ bis $b_6$ sowie $K$. Was ist die
Beziehung von $K$ und $b_0$?
\begin{table}[!htb]
\begin{center}
{\footnotesize
\begin{tabular}{|c||c|c|c|c|c|c|c|c||c|}\hline
$n$ & $b_0$ & $b_1$ & $b_2$ & $b_3$ & $b_4$ & $b_5$ & $b_6$ & $b_7$ & $K$\\ \hline\hline
 1 & 2.86278 & & & & & & & & 2.86278   \\ \hline
 2 & 1.51620 & 1.42562 & & & & & & & 1.43129\\ \hline
 3 & 0.71569 & 1.53490 & 1.25291 & & & & & & 0.71569  \\ \hline
 4 & 0.37905 & 1.02546 & 1.71687 & 1.19739 & & & & & 0.35785 \\ \hline
 5 & 0.17892 & 0.75252 & 1.30957 & 1.93737 & 1.17249 & & & & 0.17892  \\ \hline
 6 & 0.09476 & 0.43237 & 1.17186 & 1.58976 & 2.17184 & 1.15918 & & & 0.08946  \\ \hline   
 7 & 0.04473 & 0.28207 & 0.75565 & 1.64790 & 1.86941 & 2.41265 & 1.15122 & & 0.04473 \\ \hline   
 8 & 0.02369 & 0.15254 & 0.57356 & 1.14859 & 2.18402 & 2.14922 & 2.65675 & 1.14608 & 0.02237 \\ \hline
\end{tabular}\vspace*{-1mm}\caption{Koeffizienten von $D(S)$ f"ur eine UTF mit 0.5~dB Rippel $(e=0.349)$} \label{koef-0.5}
}
\end{center}
\vspace*{-4mm}
\end{table}
\nit Das bedeutet z.~B.,  dass die UTF des normierten Tschebyscheff-I Tiefpass-Filters 4.~Ordnung mit 0.5~dB Rippel wie folgt aussieht:\\
\begin{equation*}
T(S)=\frac{0.35785}{S^4+1.19739\cdot S^3+1.71687\cdot S^2+1.02546\cdot S+0.37905}.
\end{equation*}


\begin{table}[!htb]
\begin{center}
{\footnotesize
\begin{tabular}{|c||c|c|c|c|c|c|c|c||c|}\hline
$n$ & $b_0$ & $b_1$ & $b_2$ & $b_3$ & $b_4$ & $b_5$ & $b_6$ & $b_7$ & $K$\\ \hline\hline
 1 & 1.96523 & & & & & & & & 1.96523   \\ \hline
 2 & 1.10251 & 1.09773 & & & & & & & 0.98261 \\ \hline
 3 & 0.49131 & 1.23841 & 0.98834 & & & & & & 0.49131  \\ \hline
 4 & 0.27563 & 0.74262 & 1.45392 & 0.95281 & & & & & 0.24565 \\ \hline
 5 & 0.12283 & 0.58053 & 0.97440 & 1.68882 & 0.93682 & & & & 0.12283  \\ \hline
 6 & 0.06891 & 0.30708 & 0.93935 & 1.20214 & 1.93083 & 0.92825 & & & 0.06143  \\ \hline   
 7 & 0.03071 & 0.21367 & 0.54862 & 1.35754 & 1.42879 & 2.17608 & 0.92312 & & 0.03071 \\ \hline   
 8 & 0.01723 & 0.10734 & 0.44783 & 0.84682 & 1.83690 & 1.65516 & 2.42303 & 0.91981 & 0.01535 \\ \hline
\end{tabular}\vspace*{-1mm}\caption{Koeffizienten von $D(S)$ f"ur eine UTF mit 1~dB Rippel $(e=0.509)$} \label{koef-1}
}
\end{center}
\vspace*{-4mm}
\end{table}

\begin{table}[!htb]
\begin{center}
{\footnotesize
\begin{tabular}{|c||c|c|c|c|c|c|c|c||c|}\hline
$n$ & $b_0$ & $b_1$ & $b_2$ & $b_3$ & $b_4$ & $b_5$ & $b_6$ & $b_7$ & $K$\\ \hline\hline
 1 & 1.30756 & & & & & & & & 1.30756   \\ \hline
 2 & 0.82302 & 0.80382 & & & & & & & 0.65378 \\ \hline
 3 & 0.32689 & 1.02291 & 0.73782 & & & & & & 0.32689  \\ \hline
 4 & 0.20577 & 0.51680 & 1.25648 & 0.71622 & & & & & 0.16345 \\ \hline
 5 & 0.08172 & 0.45935 & 0.69348 & 1.49954 & 0.70646 & & & & 0.08172  \\ \hline
 6 & 0.05144 & 0.21027 & 0.77146 & 0.86701 & 1.74586 & 0.70123 & & & 0.04086  \\ \hline   
 7 & 0.02042 & 0.16609 & 0.38251 & 1.14444 & 1.03922 & 1.99353 & 0.69789 & & 0.02043 \\ \hline   
 8 & 0.01286 & 0.07294 & 0.35870 & 0.59822 & 1.57958 & 1.21171 & 2.24225 & 0.69606 & 0.01022 \\ \hline
\end{tabular}\vspace*{-1mm}\caption{Koeffizienten von $D(S)$ f"ur eine UTF mit 2~dB Rippel $(e=0.765)$}\label{koef-2}
}
\end{center}
\vspace*{-6mm}
\end{table}


\begin{table}[!htb]
\begin{center}
{\footnotesize
\begin{tabular}{|c||c|c|c|c|c|c|c|c||c|}\hline
$n$ & $b_0$ & $b_1$ & $b_2$ & $b_3$ & $b_4$ & $b_5$ & $b_6$ & $b_7$ & $K$\\ \hline\hline
 1 & 1.00238 & & & & & & & & 1.00238   \\ \hline
 2 & 0.70795 & 0.64490 & & & & & & & 0.50119 \\ \hline
 3 & 0.25059 & 0.92835 & 0.59724 & & & & & & 0.25059  \\ \hline
 4 & 0.17699 & 0.40477 & 1.16912 & 0.58158 & & & & & 0.12530 \\ \hline
 5 & 0.06264 & 0.40794 & 0.54886 & 1.41498 & 0.57443 & & & & 0.06265  \\ \hline
 6 & 0.04425 & 0.16343 & 0.69910 & 0.69061 & 1.66285 & 0.57070 & & & 0.03132  \\ \hline   
 7 & 0.01566 & 0.14615 & 0.30002 & 1.05184 & 0.83144 & 1.91155 & 0.56842 & & 0.01566 \\ \hline   
 8 & 0.01106 & 0.05648 & 0.32076 & 0.47190 & 1.46670 & 0.97195 & 2.16071 & 0.56695 & 0.00783 \\ \hline
\end{tabular}\vspace*{-1mm}\caption{Koeffizienten von $D(S)$ f"ur eine UTF mit 3~dB Rippel $(e=0.99762)$} \label{koef-3}
}
\end{center}
\vspace*{-4mm}
\end{table}

\begin{table}[!htb]
\begin{center}
{\footnotesize
\begin{tabular}{|c||l||c|}\hline
$n$ & $D(S)$ & $K$ \\ \hline\hline
 1 & $(6.55220+S)$  & 6.55220      \\ \hline
 2 & $(3.31329+2.37209S+S^2)$& 3.27610 \\ \hline
 3 & $(0.969+S)(1.690+0.969S+S^2)$ & 1.63805  \\ \hline
 4 & $(1.330+0.528S+S^2)(0.623+1.275S+S^2)$ & 0.819025 \\ \hline
 5 & $(0.539+S)(1.195+0.333S+S^2)(0.636+0.872S+S^2)$ & 0.4095127   \\ \hline
 6 & $(1.129+0.229S+S^2)(0.696+0.627S+S^2)(0.263+0.856S+S^2)$ & 0.2047564 \\ \hline   
 7 & $(0.377+S)(1.092+0.168S+S^2)(0.753+0.470S+S^2)(0.330+0.679S+S^2)$  & 0.102378  \\ \hline   
 8 & $(1.069+0.128S+S^2)(0.799+0.364S+S^2)(0.416+0.545S+S^2)(0.146+0.643S+S^2)$  & 0.0511891 \\ \hline
\end{tabular}\vspace*{-2mm}\caption{Faktorzerlegung von $D(S)$ f"ur 0.1~dB Rippel $(e=0.153)$}\label{fak-0.1}
}
\end{center}
\vspace*{-6mm}
\end{table}

\begin{table}[!htb]
\begin{center}
{\footnotesize
\begin{tabular}{|c||l||c|}\hline
$n$ & $D(S)$ & $K$ \\ \hline\hline
 1 & $(2.86278+S)$  & 2.86278     \\ \hline
 2 & $(1.51620+1.42562S+S^2)$  & 1.43129\\ \hline
 3 & $(0.626+S)(1.142+0.626S+S^2)$  & 0.71569\\ \hline
 4 & $(1.064+0.351S+S^2)(0.356+0.847S+S^2)$  & 0.35785\\ \hline
 5 & $(0.362+S)(1.036+0.224S+S^2)(0.477+0.586S+S^2)$  & 0.17892  \\ \hline
 6 & $(1.023+0.155S+S^2)(0.590+0.424S+S^2)(0.157+0.580S+S^2)$ & 0.08946  \\ \hline   
 7 & $(0.256+S)(1.016+0.114S+S^2)(0.677+0.319S+S^2)(0.254+0.462S+S^2)$   & 0.04473 \ \\ \hline   
 8 & $(1.012+0.087S+S^2)(0.741+0.248S+S^2)(0.359+0.372S+S^2)(0.088+0.439S+S^2)$ & 0.02237 \\ \hline
\end{tabular}\vspace*{-2mm}\caption{Faktorzerlegung von $D(S)$ f"ur 0.5~dB Rippel $(e=0.349)$}\label{fak-0.5}
}
\end{center}
\vspace*{-6mm}
\end{table}

\begin{table}[!htb]
\begin{center}
{\footnotesize
\begin{tabular}{|c||l||c|}\hline
$n$ & $D(S)$ & $K$\\ \hline\hline
 1 & $(1.96523+S)$  & 1.96523      \\ \hline
 2 & $(1.10251+1.09773S+S^2)$  & 0.98261\\ \hline
 3 & $(0.494+S)(0.994+0.494S+S^2)$ & 0.49131   \\ \hline
 4 & $(0.98650+0.2791S+S^2)(0.2794+0.67374S+S^2)$ & 0.24565\\ \hline
 5 & $(0.289+S)(0.988+0.179S+S^2)(0.429+0.468S+S^2)$  & 0.12283  \\ \hline
 6 & $(0.991+0.124S+S^2)(0.558+0.340S+S^2)(0.125+0.464S+S^2)$ & 0.06143 \\ \hline   
 7 & $(0.205+S)(0.993+0.091S+S^2)(0.653+0.256S+S^2)(0.230+0.370S+S^2)$  & 0.03071 \\ \hline   
 8 & $(0.994+0.070S+S^2)(0.724+0.199S+S^2)(0.341+0.298S+S^2)(0.070+0.352S+S^2)$ & 0.01535  \\ \hline
\end{tabular}\vspace*{-2mm}\caption{Faktorzerlegung von $D(S)$ f"ur 1~dB Rippel $(e=0.509)$} \label{fak-1}
}
\end{center}
\vspace*{-6mm}
\end{table}

\begin{table}[!htb]
\begin{center}
{\footnotesize
\begin{tabular}{|c||l||c|}\hline
$n$ & $D(S)$ & $K$\\ \hline\hline
 1 & $(1.30756+S)$  & 1.30756   \\ \hline
 2 & $(0.82302+0.80382S+S^2)$ & 0.65378 \\ \hline
 3 & $(0.369+S)(0.886+0.369S+S^2)$ & 0.32689  \\ \hline
 4 & $(0.929+0.210S+S^2)(0.222+0.506S+S^2)$ & 0.16345\\ \hline
 5 & $(0.218+S)(0.952+0.135S+S^2)(0.393+0.353S+S^2)$   & 0.08172  \\ \hline
 6 & $(0.966+0.094S+S^2)(0.533+0.257S+S^2)(0.100+0.351S+S^2)$  & 0.04086 \\ \hline   
 7 & $(0.155+S)(0.975+0.069S+S^2)(0.635+0.194S+S^2)(0.212+0.280S+S^2)$  & 0.02043  \\ \hline   
 8 & $(0.980+0.053S+S^2)(0.710+0.151S+S^2)(0.327+0.226S+S^2)(0.057+0.266S+S^2)$ & 0.01022 \\ \hline
\end{tabular}\vspace*{-2mm}\caption{Faktorzerlegung von $D(S)$ f"ur 2~dB Rippel $(e=0.765)$} \label{fak-2}
}
\end{center}
\vspace*{-6mm}
\end{table}
\begin{table}[!htb]
\begin{center}
{\footnotesize
\begin{tabular}{|c||l||c|}\hline
$n$ & $D(S)$ & $K$ \\ \hline\hline
 1 & $(1.00238+S)$   & 1.00238    \\ \hline
 2 & $(0.70795+0.64490S+S^2)$ & 0.50119 \\ \hline
 3 & $(0.299+S)(0.839+0.299S+S^2)$  & 0.25059\\ \hline
 4 & $(0.903+0.170S+S^2)(0.196+0.411S+S^2)$ & 0.12530\\ \hline
 5 & $(0.178+S)(0.936+0.110S+S^2)(0.377+0.287S+S^2)$ & 0.06265     \\ \hline
 6 & $(0.955+0.076S+S^2)(0.522+0.209S+S^2)(0.089+0.285S+S^2)$ & 0.03132 \\ \hline   
 7 & $(0.126+S)(0.966+0.056S+S^2)(0.627+0.158S+S^2)(0.204+0.228S+S^2)$  & 0.01566 \\ \hline   
 8 & $(0.974+0.043S+S^2)(0.704+0.123S+S^2)(0.321+0.184S+S^2)(0.050+0.217S+S^2)$  & 0.00783 \\ \hline
\end{tabular}\vspace*{-2mm}\caption{Faktorzerlegung von $D(S)$ f"ur 3~dB Rippel $(e=0.99762)$} \label{fak-3}
}
\end{center}
\vspace*{-6mm}
\end{table}
\begin{table}[!htb]
Die konj.-kompl. Polpaare weisen folgende Polg"uten auf:
\begin{center}
{\footnotesize
\begin{tabular}{|c||c|c|c|c|c|c|c|}\hline
$n$ & 2 & 3 & 4 & 5 & 6 & 7 & 8 \\ \hline\hline
$q_{p1}$ & 0.767 & 1.341 & 2.183 & 3.282 & 4.633 & 6.233 & 8.082 \\ \hline
$q_{p2}$ &       &       & 0.619 & 0.915 & 1.332 & 1.847 & 2.453 \\ \hline
$q_{p3}$ &       &       &      &        & 0.600 & 0.847 & 1.183 \\ \hline
$q_{p4}$ &       &       &      &        &       &       & 0.593 \\ \hline
\end{tabular}
}
\end{center}
\vspace*{-4mm}
\caption{Polg"uten der konj.-kompl. Polpaare f"ur 0.1~dB Rippel der normierten Tschebyscheff-TP-Filter
\label{pg-0.1}}
\end{table}

\begin{table}[!htb]
\begin{center}
{\footnotesize
\begin{tabular}{|c||c|c|c|c|c|c|c|}\hline
$n$ & 2 & 3 & 4 & 5 & 6 & 7 & 8 \\ \hline\hline
$q_{p1}$ & 0.864 & 1.706 & 2.941 & 4.545 & 6.513 & 8.842 & 11.531 \\ \hline
$q_{p2}$ &       &       & 0.705 & 1.178 & 1.810 & 2.576 & 3.466 \\ \hline
$q_{p3}$ &       &       &       &       & 0.684 & 1.092 & 1.611 \\ \hline
$q_{p4}$ &       &       &       &       &       &       & 0.677 \\ \hline
\end{tabular}
}
\end{center}\vspace*{-4mm}\caption{Polg"uten der konj.-kompl. Polpaare f"ur 0.5~dB Rippel der normierten Tschebyscheff-TP-Filter}\label{pg-0.5}
\end{table}

\begin{table}[!htb]
\begin{center}
{\footnotesize
\begin{tabular}{|c||c|c|c|c|c|c|c|}\hline
$n$ & 2 & 3 & 4 & 5 & 6 & 7 & 8 \\ \hline\hline
$q_{p1}$ & 0.957 & 2.018 & 3.559 & 5.556 & 8.004 & 10.899 & 14.241 \\ \hline
$q_{p2}$ &       &       & 0.785 & 1.399 & 2.198 & 3.156 & 4.266 \\ \hline
$q_{p3}$ &       &       &       &       & 0.761 & 1.297 & 1.956 \\ \hline
$q_{p4}$ &       &       &       &       &       &       & 0.753 \\ \hline
\end{tabular}
}
\end{center}\vspace*{-4mm}\caption{Polg"uten der konj.-kompl. Polpaare f"ur 1~dB Rippel der normierten Tschebyscheff-TP-Filter} \label{pg-1}
\end{table}


\clearpage


\begin{table}[!htb]
\begin{center}
{\footnotesize
\begin{tabular}{|c||c|c|c|c|c|c|c|}\hline
$n$ & 2 & 3 & 4 & 5 & 6 & 7 & 8 \\ \hline\hline
$q_{p1}$ & 1.129 & 2.552 & 4.594 & 7.232 & 10.462 & 14.280 & 18.687 \\ \hline
$q_{p2}$ &       &       & 0.929 & 1.775 & 2.844 & 4.115 & 5.584 \\ \hline
$q_{p3}$ &       &       &       &       & 0.902 & 1.645 & 2.533 \\ \hline
$q_{p4}$ &       &       &       &       &       &       & 0.892 \\ \hline
\end{tabular}
}
\end{center}\vspace*{-4mm}\caption{Polg"uten der konj.-kompl. Polpaare f"ur 2~dB Rippel der normierten Tschebyscheff-TP-Filter} \label{gp-2}
\end{table}

\begin{table}[!htb]
\begin{center}
{\footnotesize
\begin{tabular}{|c||c|c|c|c|c|c|c|}\hline
$n$ & 2 & 3 & 4 & 5 & 6 & 7 & 8 \\ \hline\hline
$q_{p1}$ & 1.305 & 3.068 & 5.578 & 8.818 & 12.780 & 17.464 & 22.870 \\ \hline
$q_{p2}$ &       &       & 1.076 & 2.138 & 3.458 & 5.021 & 6.825 \\ \hline
$q_{p3}$ &       &       &       &       & 1.044 & 1.983 & 3.080 \\ \hline
$q_{p4}$ &       &       &       &       &       &       & 1.034 \\ \hline
\end{tabular}
}
\end{center}\vspace*{-4mm}\caption{Polg"uten der konj.-kompl. Polpaare f"ur 3~dB Rippel der normierten Tschebyscheff-TP-Filter} \label{gp-3}
\end{table}



\subsection{Bessel-Filter}
\[\mbox{Ansatz:} \hspace{1cm} T(S)=K \cdot e^{-ST_{0}} \approx \frac{K}{B_{n}(S)}=\frac{K}{S^{n}+\beta_{n-1}S^{n-1}+ \ldots + \beta_{2}S^{2}+\beta_{1}S+\beta_{0}} \]
Die Bessel-Polynome\index{Bessel!Polynom} k"onnen der Tabelle~\ref{b-poly} entnommen werden.
% Matlab bessel_polynom
\begin{table}[!htb]
\begin{center}{\scriptsize
\begin{tabular}{|c||r|r|r|r|r|r|r|r|r|r|r|}\hline
  $n$ & $\beta_0=K$ & $\beta_1$ & $\beta_2$ & $\beta_3$ & $\beta_4$ & $\beta_5$ & $\beta_6$ & $\beta_7$ & $\beta_8$ & $\beta_9$   \\ \hline\hline
   1  & 1     &       &       &       &       &      &   & &   &\\ \hline
   2  & 3     & 3     &       &       &       &      &   & &   &\\ \hline
   3  & 15    & 15    & 6     &       &       &      &   & &   &\\ \hline
   4  & 105   & 105   & 45    & 10    &       &      &   & &   &\\ \hline
   5  & 945   & 945   & 420   & 105   & 15    &      &   & &   &\\ \hline
   6  & 10395 & 10395 & 4725  & 1260  & 210   & 21   &   & &   &\\ \hline
   7  & 135135 & 135135 & 62370 & 17325 & 3150 & 378 & 28 & &   &\\ \hline
   8  & 2027025 & 2027025 & 945945 & 270270 & 51975 & 6930 & 630 & 36 & & \\ \hline
   9  & 34459425 & 34459425 & 16216200 & 4729725 & 945945 & 135135 & 13860 & 990 & 45 & \\ \hline
  10  & 654729075 & 654729075 & 310134825 & 91891800 & 18918900 & 2837835 & 315315 & 25740 & 1485 & 55 \\ \hline
\end{tabular}}
\end{center}\vspace*{-5mm}
\caption{Koeffizienten $\beta$ der Bessel-Polynome $B_n(S)$ ($K=\beta_0$) mit $T_0=1$}\label{b-poly}
\end{table}\vspace*{-5mm}
\aufg
Bestimmen Sie die Koeffizienten des Bessel-Polynomes $B_{13}(S)$ (Tipp: Verwenden Sie die Rekursionsformel~\ref{eq: bessel_rekursion}).\\
% Loesung mit bessel_polynom.m
% 7905853580625 790585380625 3794809718700 1159525191825 252070693875 41247931725 
% 5237832600 523783260 41351310 2552550 120120 4095 91 1
\nit Zur Tabellierung der auf die {\bf 3~dB-Grenzfrequenz\index{3-dB!Grenzfrequenz} normierten
"Ubertragungsfunktionen}
\begin{eqnarray*}
T(S)=\frac{K}{D(S)} \qquad \text{mit}\qquad &D(S)&=S^{n}+b_{n-1}S^{n-1}+ \ldots + b_{2}S^{2}+b_{1}S+b_{0} 
\end{eqnarray*}
wurde $\maxs{S}|T(S)|=1$ gew"ahlt. Die Koeffizienten $b$ k"onnen den
Tabellen~\ref{koef-b} und~\ref{fak-b} entnommen werden.  Der
Zusammenhang zwischen dem Verh"altnis der 3~dB-Grenz\-frequenz $\omega_{\text{3dB}}$ und
$\omega_{D}$ sowie $A_{\max}$ kann
mit folgender Abb.~\ref{gb} bestimmt werden:\\


\clearpage

\begin{figure}[!htb]% matlab bessel_plot.m fig 3.
\vspace*{-1mm}
\begin{center}
  \bild{/filter/FILn011.eps,width=0.7}\vspace*{-3mm}\caption{Graphen $A_{\max}$ der normierten Bessel-Filter der Ordnung $1\ldots 10$ in Funktion von $\Omega=\omega_D/\omega_{\text{3dB}}$ \label{gb}}
\end{center}
\vspace*{-6mm}
\end{figure}

\begin{table}[!htb] % made mit bessel_polynome.m
\begin{center}{\footnotesize
\begin{tabular}{|c||c|c|c|c|c|c|c|c|c|c|}\hline
$n$ & 1 & 2 & 3 & 4 & 5 & 6 & 7 & 8 & 9 & 10 \\ \hline
$T_0$ & 1 & 1.3617 & 1.7557 & 2.1139 & 2.4274 & 2.7034 & 2.9517 & 3.1796 & 3.3917 & 3.5910 \\ \hline \hline
$n$ & 11 & 12 & 13 & 14 & 15 & 16 & 17 & 18 & 19 & 20\\ \hline
$T_0$ & 3.7796 & 3.9592 & 4.1308 & 4.2956 & 4.4542 & 4.6074 & 4.7556 & 4.8993 & 5.0389 & 5.1747\\ \hline
\end{tabular}\caption{Werte der Gruppenlaufzeit $T_0$ zur Umnormierung der Bessel-Polynome $B_n(S)$ auf $A(1)=\text{3~dB}$ 
\label{werte-glaufz}}}
\end{center}
\end{table}\vspace*{-5mm}
\aufg
Bestimmen Sie $T_0$ f"ur das Bessel-Polynom $B_{20}(S)$ und $B_{25}(S)$ (siehe Tabelle~\ref{werte-glaufz}).\\
% Loesung mit bessel_polynome.m
% T_0=5.8062
F"ur die Koeffizienten $b$ von $D(S)$ des auf die 3~dB-Grenzfrequenz\index{3-dB!Grenzfrequenz} normierten Bessel-Tiefpassfilter der Ordnung $n$ gilt: $b_i=\frac{\beta_i}{T_0^{(n-i)}}$, wobei $i=0,\ldots, n.$
\bsp{Berechnung von $b_8$ des normierten Bessel-Filters der Ordnung $n=10$}
Mit $i=8$ und $n=10$ folgt mit Tabelle~\ref{werte-glaufz}: $b_8=\frac{\beta_i}{T_0^{(n-i)}}=\frac{1485}{3.5910^{(10-8)}}=115.16$.
\begin{table}[!htb] % made with bessel_polynome.m a_scaled 
\begin{center}
{\footnotesize
\begin{tabular}{|c||c|c|c|c|c|c|c|c|c|c|}\hline
$n$ & $b_0=K$ & $b_1$ & $b_2$ & $b_3$ & $b_4$ & $b_5$ & $b_6$ & $b_7$ & $b_8$ & $b_9$ \\ \hline\hline
1 & 1 & & & & & & & & &\\ \hline
2 & 1.6180 & 2.2032 & & & & & & & &\\ \hline
3 & 2.7718 & 4.8664 & 3.4175 & & & & & & &\\ \hline
4 & 5.2582 & 11.115 & 10.070 & 4.7306 & & & & & &\\ \hline
5 & 11.213 & 27.218 & 29.364 & 17.820 & 6.1794  & & & & & \\ \hline
6 & 26.630 & 71.991 & 88.463 & 63.774 & 28.734 & 7.768 & & & &   \\ \hline   
7 & 69.22 & 204.32 & 278.35 & 228.23 & 122.49 & 43.39 & 9.4860 & & &  \\ \hline   
8 & 194.03 & 616.93 & 915.41 & 831.62 & 508.5 & 215.58 & 62.315 & 11.3221  & & \\ \hline
9 & 580.18 & 1967.8 & 3140.7 & 3107 & 2107.6 & 1021.2 & 355.23 & 86.06 & 13.268 &  \\ \hline
10 & 1836.2 & 6593.9 & 11216 & 11934 & 8823 & 4752.5 & 1896.2 & 555.8 & 115.16 & 15.316  \\ \hline
\end{tabular}\vspace*{-1mm}\caption{Koeffizienten $b$ von $D(S)$, sowie $K$ der UTF $T(S)$} \label{koef-b}}
\end{center}
\vspace*{-6mm}
\end{table}
\begin{table}[!htb] % made with besel_polynome.m
\vspace*{-3mm}\begin{center}
{\footnotesize
\begin{tabular}{|c||l|}\hline
$n$ & $D(S)$ {}\\ \hline\hline
 1 & $(1+S)$     \\ \hline
 2 & $(1.6180+2.2032S+S^2)$ \\ \hline
 3 & $(1.3227+S)(2.0956+2.0948S+S^2)$  \\ \hline
 4 & $(2.0454+2.7401S+S^2)(2.5708+1.9904S+S^2)$ \\ \hline
 5 & $(1.5023+S)(2.4222+2.7618S+S^2)(3.0814+1.9154S+S^2)$    \\ \hline
 6 & $(2.5726+3.1430S+S^2)(2.8533+2.7637S+S^2)(3.6279+1.8613S+S^2)$  \\ \hline   
 7 & $(1.6844+S)(2.9459+3.2241S+S^2)(3.3212+2.7578S+S^2)(4.2004+1.8197+S^2)$   \\ \hline   
 8 & $(3.1629+3.5148S+S^2)(3.3566+3.2739S+S^2)(3.815+2.7477S+S^2)(4.7905+1.7857S+S^2)$  \\ \hline
 9 & {\scriptsize $(1.8566+S)(3.5284+3.6143S+S^2)(3.7942+3.3048S+S^2)(4.3281+2.7352S+S^2)(5.3932+1.7568S+S^2)$} \\ \hline
10 & {\tiny $(3.7741+3.8552S+S^2)(3.9226+3.6844S+S^2)(4.2527+3.3236S+S^2)(4.8565+2.7214S+S^2)(6.0056+1.7315S+S^2)$} \\ \hline
\end{tabular}\vspace*{-2mm}\caption{Faktorzerlegung von $D(S)$ } \label{fak-b}
}
\end{center}
\vspace*{-6mm}
\end{table}


Die konjugiert-komplexen Polpaare von $D(S)$ weisen folgende Polg"uten auf:
\begin{table}[!htb]
\begin{center}
\begin{tabular}{|c||c|c|c|c|c|c|c|c|c|}\hline
$n$ & 2 & 3 & 4 & 5 & 6 & 7 & 8 & 9 & 10\\ \hline\hline
$q_{p1}$      & 0.577 & 0.691 & 0.806 & 0.916 & 1.023 & 1.126 & 1.226 & 1.322 & 1.415 \\ \hline
$q_{p2}$      &       &       & 0.522 & 0.563 & 0.611 & 0.661 & 0.711 & 0.761 & 0.810 \\ \hline
$q_{p3}$      &       &       &       &       & 0.510 & 0.532 & 0.560 & 0.589 & 0.620 \\ \hline
$q_{p4}$      &       &       &       &       &       &       & 0.506 & 0.520 & 0.538 \\ \hline
$q_{p5}$      &       &       &       &       &       &       &       &       & 0.504 \\ \hline
\end{tabular}
\end{center}
\vspace*{-5mm}\caption{Polg"uten $q_{p_i}$ der konj.-kompl. Polpaare der Bessel-Filter der Ordnung $n$ \label{pg-b}}
\end{table}\vspace*{-5mm}

s
\clearpage

\subsection{Cauer-Filter (elliptische Filter)}
Beim Cauer-Filter (Tschebyscheff-Cauer, elliptische Filter, CC -
Filter) lassen sich die "Ubertragungsfunktionen nicht mehr einfach
tabellieren. Die Ordnung l\"asst sich jedoch mittels dem \mb {\tt
  ellipord} einfach bestimmen und die UTF kann mit {\tt ellipap}
eruiert werden.  Ebenfalls k"onnen die UTFs anhand der Pol-
Nullstellenangaben\index{Nullstellen} in den $LC$-Filtertabellen
(Anhang~\ref{Anh-Tab}) bestimmt werden.  \\
\aufg
Bestimmen Sie die Filterordnung f"ur ein Cauer-TP-Filter mit $A_{\min}=50$~dB, $A_{\max}=1$~dB und $\Omega_S/\Omega_D=3$.


\section{Tabellen zum Entwurf von {\boldmath $LC$}-Filtern \label{Anh-Tab}}
\begin{figure}[!htb]
\vspace*{-8mm}
\begin{center}
  \hspace*{-0cm}\bild{/filter/FIL_nw.ps,width=1.05}\vspace*{-1mm}\caption{M"ogliche Anordnungen der $LC$-Filter. (Werte gem"ass den Tabellen~\ref{el-BW1} bis \ref{filter-tsch05db}).}
\end{center}
\vspace*{-6mm}
\end{figure}

\clearpage~~\vspace*{-13mm}
\begin{table}[!htb]
\begin{center}
{\tiny
\begin{tabular}{|c||r|r|r|r|r|r|r|r|r|r|}\hline
$n$ & $R_1$   & $C_1$ & $L_2$ & $C_3$ & $L_4$ & $C_5$ & $L_6$ & $C_7$ & $L_8$ & $C_9$\\ \hline\hline
\textbf{ 2}&$\infty$&1.4142&0.7071&     &     &     &     &     &     &     \\
&10.0000&0.0743&14.8138&     &     &     &     &     &     &     \\
&5.0000&0.1557&7.7067&     &     &     &     &     &     &     \\
&3.3333&0.2447&5.3126&     &     &     &     &     &     &     \\
&2.5000&0.3419&4.0951&     &     &     &     &     &     &     \\
&2.0000&0.4483&3.3461&     &     &     &     &     &     &     \\
&1.6667&0.5657&2.8284&     &     &     &     &     &     &     \\
&1.4286&0.6971&2.4387&     &     &     &     &     &     &     \\
&1.2500&0.8485&2.1213&     &     &     &     &     &     &     \\
&1.1111&1.0353&1.8352&     &     &     &     &     &     &     \\
&1.0000&1.4142&1.4142&     &     &     &     &     &     &     \\ \hline
\textbf{ 3}&$\infty$&1.5000&1.3333&0.5000&     &     &     &     &     &     \\
&0.1000&5.1672&0.1377&15.4554&     &     &     &     &     &     \\
&0.2000&2.6687&0.2842&7.9102&     &     &     &     &     &     \\
&0.3000&1.8380&0.4396&5.3634&     &     &     &     &     &     \\
&0.4000&1.4254&0.6042&4.0642&     &     &     &     &     &     \\
&0.5000&1.1811&0.7789&3.2612&     &     &     &     &     &     \\
&0.6000&1.0225&0.9650&2.7024&     &     &     &     &     &     \\
&0.7000&0.9152&1.1652&2.2774&     &     &     &     &     &     \\
&0.8000&0.8442&1.3840&1.9259&     &     &     &     &     &     \\
&0.9000&0.8082&1.6332&1.5994&     &     &     &     &     &     \\
&1.0000&1.0000&2.0000&1.0000&     &     &     &     &     &     \\ \hline
\textbf{ 4}&$\infty$&1.5307&1.5772&1.0824&0.3827&     &     &     &     &     \\
&10.0000&0.0392&11.0942&0.1616&15.6421&     &     &     &     &     \\
&5.0000&0.0804&5.6835&0.3307&7.9397&     &     &     &     &     \\
&3.3333&0.1237&3.8826&0.5072&5.3381&     &     &     &     &     \\
&2.5000&0.1692&2.9858&0.6911&4.0094&     &     &     &     &     \\
&2.0000&0.2175&2.4524&0.8826&3.1868&     &     &     &     &     \\
&1.6667&0.2690&2.1029&1.0824&2.6131&     &     &     &     &     \\
&1.4286&0.3251&1.8618&1.2913&2.1752&     &     &     &     &     \\
&1.2500&0.3882&1.6946&1.5110&1.8109&     &     &     &     &     \\
&1.1111&0.4657&1.5924&1.7439&1.4690&     &     &     &     &     \\
&1.0000&0.7654&1.8478&1.8478&0.7654&     &     &     &     &     \\ \hline
\textbf{ 5}&$\infty$&1.5451&1.6944&1.3820&0.8944&0.3090&     &     &     &     \\
&0.1000&3.1522&0.0912&14.0945&0.1727&15.7103&     &     &     &     \\
&0.2000&1.6077&0.1861&7.1849&0.3518&7.9345&     &     &     &     \\
&0.3000&1.0937&0.2848&4.8835&0.5367&5.3073&     &     &     &     \\
&0.4000&0.8378&0.3877&3.7357&0.7274&3.9648&     &     &     &     \\
&0.5000&0.6857&0.4955&3.0510&0.9237&3.1331&     &     &     &     \\
&0.6000&0.5860&0.6094&2.5998&1.1255&2.5524&     &     &     &     \\
&0.7000&0.5173&0.7313&2.2849&1.3326&2.1083&     &     &     &     \\
&0.8000&0.4698&0.8660&2.0605&1.5443&1.7380&     &     &     &     \\
&0.9000&0.4416&1.0265&1.9095&1.7562&1.3887&     &     &     &     \\
&1.0000&0.6180&1.6180&2.0000&1.6180&0.6180&     &     &     &     \\ \hline
\textbf{ 6}&$\infty$&1.5529&1.7593&1.5529&1.2016&0.7579&0.2588&     &     &     \\
&10.0000&0.0263&7.7053&0.1222&15.7855&0.1788&15.7375&     &     &     \\
&5.0000&0.0535&3.9170&0.2484&8.0201&0.3628&7.9216&     &     &     \\
&3.3333&0.0816&2.6559&0.3788&5.4325&0.5517&5.2804&     &     &     \\
&2.5000&0.1108&2.0275&0.5139&4.1408&0.7450&3.9305&     &     &     \\
&2.0000&0.1412&1.6531&0.6542&3.3687&0.9423&3.0938&     &     &     \\
&1.6667&0.1732&1.4071&0.8011&2.8580&1.1431&2.5092&     &     &     \\
&1.4286&0.2072&1.2363&0.9567&2.4991&1.3464&2.0618&     &     &     \\
&1.2500&0.2445&1.1163&1.1257&2.2389&1.5498&1.6881&     &     &     \\
&1.1111&0.2890&1.0403&1.3217&2.0539&1.7443&1.3347&     &     &     \\
&1.0000&0.5176&1.4142&1.9319&1.9319&1.4142&0.5176&     &     &     \\ \hline
\textbf{ 7}&$\infty$&1.5576&1.7988&1.6588&1.3972&1.0550&0.6560&0.2225&     &     \\
&0.1000&2.2571&0.0665&10.7004&0.1417&16.8222&0.1823&15.7480&     &     \\
&0.2000&1.1448&0.1350&5.4267&0.2874&8.5263&0.3692&7.9079&     &     \\
&0.3000&0.7745&0.2055&3.6706&0.4373&5.7612&0.5600&5.2583&     &     \\
&0.4000&0.5899&0.2782&2.7950&0.5917&4.3799&0.7542&3.9037&     &     \\
&0.5000&0.4799&0.3536&2.2726&0.7512&3.5532&0.9513&3.0640&     &     \\
&0.6000&0.4075&0.4322&1.9284&0.9170&3.0050&1.1503&2.4771&     &     \\
&0.7000&0.3571&0.5154&1.6883&1.0910&2.6177&1.3498&2.0277&     &     \\
&0.8000&0.3215&0.6057&1.5174&1.2777&2.3338&1.5461&1.6520&     &     \\
&0.9000&0.2985&0.7111&1.4043&1.4891&2.1249&1.7268&1.2961&     &     \\
&1.0000&0.4450&1.2470&1.8019&2.0000&1.8019&1.2470&0.4450&     &     \\ \hline
\textbf{ 8}&$\infty$&1.5607&1.8246&1.7287&1.5283&1.2588&0.9371&0.5776&0.1951&     \\
&10.0000&0.0198&5.8479&0.0949&12.7455&0.1547&17.4999&0.1846&15.7510&     \\
&5.0000&0.0400&2.9608&0.1921&6.4523&0.3133&8.8538&0.3732&7.8952&     \\
&3.3333&0.0608&1.9995&0.2919&4.3563&0.4757&5.9714&0.5650&5.2400&     \\
&2.5000&0.0822&1.5201&0.3945&3.3106&0.6424&4.5308&0.7594&3.8825&     \\
&2.0000&0.1042&1.2341&0.5003&2.6863&0.8139&3.6678&0.9558&3.0408&     \\
&1.6667&0.1272&1.0455&0.6102&2.2740&0.9912&3.0945&1.1530&2.4524&     \\
&1.4286&0.1513&0.9138&0.7257&1.9852&1.1760&2.6879&1.3490&2.0017&     \\
&1.2500&0.1774&0.8199&0.8499&1.7779&1.3721&2.3874&1.5393&1.6246&     \\
&1.1111&0.2075&0.7575&0.9925&1.6362&1.5900&2.1612&1.7092&1.2671&     \\
&1.0000&0.3902&1.1111&1.6629&1.9616&1.9616&1.6629&1.1111&0.3902&     \\ \hline
\textbf{ 9}&$\infty$&1.5628&1.8424&1.7772&1.6202&1.4037&1.1408&0.8414&0.5155&0.1736\\
&0.1000&1.7558&0.0521&8.5074&0.1153&14.1930&0.1638&17.9654&0.1862&15.7504\\
&0.2000&0.8878&0.1054&4.3014&0.2333&7.1750&0.3312&9.0766&0.3757&7.8838\\
&0.3000&0.5987&0.1600&2.9006&0.3539&4.8373&0.5022&6.1128&0.5680&5.2249\\
&0.4000&0.4545&0.2159&2.2019&0.4775&3.6706&0.6771&4.6310&0.7624&3.8654\\
&0.5000&0.3685&0.2735&1.7846&0.6046&2.9734&0.8565&3.7426&0.9579&3.0223\\
&0.6000&0.3117&0.3330&1.5092&0.7361&2.5124&1.0410&3.1516&1.1533&2.4328\\
&0.7000&0.2719&0.3954&1.3162&0.8734&2.1885&1.2323&2.7314&1.3464&1.9812\\
&0.8000&0.2434&0.4623&1.1777&1.0200&1.9542&1.4336&2.4189&1.5318&1.6033\\
&0.9000&0.2242&0.5388&1.0835&1.1859&1.7905&1.6538&2.1796&1.6930&1.2446\\
&1.0000&0.3473&1.0000&1.5321&1.8794&2.0000&1.8794&1.5321&1.0000&0.3473\\ \hline
\hline
$n$ & $1/(R_1)$ & $L_1$ & $C_2$ & $L_3$ & $C_4$ & $L_5$ & $C_6$ & $L_7$ & $C_8$ & $L_9$  \\ \hline
\end{tabular}\vspace*{-1mm}\caption{Normierte Elementwerte f"ur Butterworth-TP-Filter (frequenznormiert auf 
die 3~dB-Grenzfrequenz bei $\Omega=1$) \label{el-BW1}}}
\end{center}
\vspace*{-4mm} 
\end{table}% made with matlab make_LC_butt_tabelle -> butt_tabelle.txt
\vspace*{-9mm} 
\aufg
Bestimmen Sie f"ur $n=2$ und $R_1=2\cdot R_2$ die Elementwerte $C_1$ und $L_2$. Gehen Sie dabei analog wie Beispiel~\ref{Kapitel_Filter}.\ref{FIl_bsp_kettenbruch} vor.
\vspace*{-14mm} 
\clearpage


\begin{table}[!htb]
\begin{center}
{\tiny
\begin{tabular}{|c||r|r|r|r|r|r|r|r|r|}\hline
$n$ & $R_1$   & $C_1$ & $L_2$ & $C_3$ & $L_4$ & $C_5$ & $L_6$ & $C_7$ & $L_8$\\ \hline\hline
\textbf{ 2}&$\infty$&1.3617&0.4539&     &     &     &     &     &     \\
&10.0000&0.0469&14.5097&     &     &     &     &     &     \\
&5.0000&0.0965&7.6876&     &     &     &     &     &     \\
&3.3333&0.1486&5.4050&     &     &     &     &     &     \\
&2.5000&0.2032&4.2577&     &     &     &     &     &     \\
&2.0000&0.2601&3.5649&     &     &     &     &     &     \\
&1.6667&0.3191&3.0993&     &     &     &     &     &     \\
&1.4286&0.3801&2.7638&     &     &     &     &     &     \\
&1.2500&0.4433&2.5096&     &     &     &     &     &     \\
&1.1111&0.5084&2.3097&     &     &     &     &     &     \\
&1.0000&0.5755&2.1478&     &     &     &     &     &     \\ \hline
\textbf{ 3}&$\infty$&1.4631&0.8427&0.2926&     &     &     &     &     \\
&0.1000&2.9825&0.0860&15.4697&     &     &     &     &     \\
&0.2000&1.5176&0.1752&8.1403&     &     &     &     &     \\
&0.3000&1.0283&0.2673&5.6888&     &     &     &     &     \\
&0.4000&0.7829&0.3618&4.4573&     &     &     &     &     \\
&0.5000&0.6353&0.4587&3.7144&     &     &     &     &     \\
&0.6000&0.5365&0.5576&3.2159&     &     &     &     &     \\
&0.7000&0.4657&0.6584&2.8575&     &     &     &     &     \\
&0.8000&0.4124&0.7609&2.5867&     &     &     &     &     \\
&0.9000&0.3708&0.8650&2.3745&     &     &     &     &     \\
&1.0000&0.3374&0.9705&2.2034&     &     &     &     &     \\ \hline
\textbf{ 4}&$\infty$&1.5012&0.9781&0.6127&0.2114&     &     &     &     \\
&10.0000&0.0214&6.2086&0.0993&15.8372&     &     &     &     \\
&5.0000&0.0434&3.1416&0.2013&8.3185&     &     &     &     \\
&3.3333&0.0658&2.1174&0.3056&5.8048&     &     &     &     \\
&2.5000&0.0887&1.6040&0.4120&4.5430&     &     &     &     \\
&2.0000&0.1120&1.2952&0.5202&3.7824&     &     &     &     \\
&1.6667&0.1356&1.0886&0.6299&3.2727&     &     &     &     \\
&1.4286&0.1596&0.9406&0.7410&2.9066&     &     &     &     \\
&1.2500&0.1839&0.8292&0.8534&2.6304&     &     &     &     \\
&1.1111&0.2085&0.7423&0.9670&2.4143&     &     &     &     \\
&1.0000&0.2334&0.6725&1.0815&2.2404&     &     &     &     \\ \hline
\textbf{ 5}&$\infty$&1.5125&1.0232&0.7531&0.4729&0.1618&     &     &     \\
&0.1000&1.6349&0.0478&7.6043&0.1036&15.9487&     &     &     \\
&0.2000&0.8251&0.0964&3.8352&0.2095&8.3747&     &     &     \\
&0.3000&0.5548&0.1457&2.5768&0.3174&5.8433&     &     &     \\
&0.4000&0.4194&0.1958&1.9464&0.4270&4.5731&     &     &     \\
&0.5000&0.3380&0.2465&1.5672&0.5382&3.8077&     &     &     \\
&0.6000&0.2836&0.2977&1.3138&0.6506&3.2952&     &     &     \\
&0.7000&0.2447&0.3494&1.1323&0.7642&2.9272&     &     &     \\
&0.8000&0.2154&0.4016&0.9959&0.8789&2.6497&     &     &     \\
&0.9000&0.1926&0.4542&0.8894&0.9945&2.4328&     &     &     \\
&1.0000&0.1743&0.5072&0.8040&1.1110&2.2582&     &     &     \\ \hline
\textbf{ 6}&$\infty$&1.5124&1.0329&0.8125&0.6072&0.3785&0.1287&     &     \\
&10.0000&0.0130&3.8146&0.0612&8.1860&0.1045&15.9506&     &     \\
&5.0000&0.0261&1.9209&0.1232&4.1204&0.2110&8.3775&     &     \\
&3.3333&0.0395&1.2890&0.1859&2.7633&0.3193&5.8467&     &     \\
&2.5000&0.0530&0.9725&0.2492&2.0837&0.4292&4.5770&     &     \\
&2.0000&0.0666&0.7824&0.3131&1.6752&0.5405&3.8122&     &     \\
&1.6667&0.0804&0.6553&0.3775&1.4022&0.6530&3.3001&     &     \\
&1.4286&0.0943&0.5644&0.4424&1.2069&0.7665&2.9325&     &     \\
&1.2500&0.1082&0.4961&0.5076&1.0600&0.8810&2.6554&     &     \\
&1.1111&0.1223&0.4429&0.5732&0.9456&0.9964&2.4388&     &     \\
&1.0000&0.1365&0.4002&0.6392&0.8538&1.1126&2.2645&     &     \\ \hline
\textbf{ 7}&$\infty$&1.5087&1.0293&0.8345&0.6752&0.5031&0.3113&0.1054&     \\
&0.1000&1.0612&0.0313&5.0616&0.0679&8.3967&0.1040&15.9166&     \\
&0.2000&0.5338&0.0630&2.5448&0.1365&4.2214&0.2100&8.3623&     \\
&0.3000&0.3579&0.0951&1.7051&0.2058&2.8280&0.3177&5.8380&     \\
&0.4000&0.2698&0.1274&1.2847&0.2755&2.1304&0.4269&4.5718&     \\
&0.5000&0.2168&0.1599&1.0321&0.3457&1.7111&0.5374&3.8090&     \\
&0.6000&0.1815&0.1927&0.8634&0.4163&1.4312&0.6491&3.2984&     \\
&0.7000&0.1562&0.2257&0.7428&0.4873&1.2308&0.7618&2.9319&     \\
&0.8000&0.1372&0.2589&0.6521&0.5586&1.0803&0.8754&2.6556&     \\
&0.9000&0.1224&0.2923&0.5815&0.6302&0.9630&0.9899&2.4396&     \\
&1.0000&0.1106&0.3259&0.5249&0.7020&0.8690&1.1052&2.2659&     \\ \hline
\textbf{ 8}&$\infty$&1.5044&1.0214&0.8392&0.7081&0.5743&0.4253&0.2616&0.0883\\
&10.0000&0.0089&2.6307&0.0427&5.7710&0.0711&8.4376&0.1032&15.8768\\
&5.0000&0.0179&1.3218&0.0859&2.8981&0.1429&4.2389&0.2083&8.3441\\
&3.3333&0.0269&0.8852&0.1294&1.9396&0.2151&2.8380&0.3151&5.8271\\
&2.5000&0.0360&0.6667&0.1732&1.4599&0.2878&2.1367&0.4233&4.5645\\
&2.0000&0.0452&0.5354&0.2173&1.1718&0.3608&1.7154&0.5329&3.8041\\
&1.6667&0.0545&0.4477&0.2616&0.9794&0.4342&1.4340&0.6435&3.2949\\
&1.4286&0.0637&0.3850&0.3061&0.8418&0.5078&1.2328&0.7552&2.9295\\
&1.2500&0.0731&0.3380&0.3509&0.7385&0.5817&1.0816&0.8678&2.6541\\
&1.1111&0.0825&0.3013&0.3958&0.6580&0.6559&0.9639&0.9813&2.4388\\
&1.0000&0.0919&0.2719&0.4409&0.5936&0.7303&0.8695&1.0956&2.2656\\ \hline\hline
$n$ & $1/(R_1)$ & $L_1$ & $C_2$ & $L_3$ & $C_4$ & $L_5$ & $C_6$ & $L_7$ & $C_8$\\ \hline
\end{tabular}\vspace*{-1mm}\caption{Normierte Elementwerte f"ur Bessel-TP-Filter (frequenznormiert
auf die 3~dB-Grenzfrequenz bei $\Omega=1$)}}
\end{center}\vspace*{-1cm}
\end{table}% made with matlab make_LC_bessel_tabelle -> bessel_tabelle.txt
\aufg
Bestimmen Sie f"ur $n=2$ und $R_1=10\cdot R_2$ die UTF von $U_{R_2}(S)/U_0(S)$. Wie gross ist $U_{R_2}(S)/U_0(S)$ bei $S=j0$ ausgedr"uckt durch $R_1$ und $R_2$? Bestimmen Sie f"ur $n=2$ und $R_1=10\cdot R_2$ die UTF von $I_{R_2}(S)/I_0(S)$. Wie gross ist $I_{R_2}(S)/I_0(S)$ bei $S=j0$ ausgedr"uckt durch $R_1$ und $R_2$?
% alle vier anordnungen sind moeglich

\clearpage




\begin{table}[!htb]
\begin{center}
{\tiny
\begin{tabular}{|c||r|r|r| }\hline
$n$ & $R_1$   & $C_1$ & $L_2$ \\ \hline\hline



\textbf{ 2}&$\infty$&1.2872&0.3218\\
&100.0000&1.2968&0.3226\\
&10.0000&1.3830&0.3295\\
&5.0000&1.4773&0.3365\\
&3.3333&1.5705&0.3429\\
&2.5000&1.6625&0.3488\\
&2.0000&1.7536&0.3543\\
&1.6667&1.8438&0.3594\\
&1.4286&1.9333&0.3642\\
&1.2500&2.0219&0.3687\\
&1.1111&2.1100&0.3730\\
&1.0000&2.1974&0.3770\\
&0.5000&3.0455&0.4080\\
&0.3333&3.8616&0.4291\\
&0.2500&4.6571&0.4447\\
&0.2000&5.4380&0.4570\\
&0.1667&6.2080&0.4671\\
&0.1429&6.9691&0.4755\\
&0.1250&7.7231&0.4827\\
&0.1111&8.4712&0.4890\\
&0.1000&9.2141&0.4945\\
&0.0100&71.4711&0.5853\\ \hline



\hline\hline
$n$ & $1/(R_1)$ & $L_1$ & $C_2$ \\ \hline

\end{tabular}\vspace*{-1mm}\caption{Normierte Elementwerte
    f"ur kritisch-ged"ampfte Filter (Gauss-Filter) (frequenznormiert
    auf die 3~dB-Grenzfrequenz bei $\Omega=1$) }}
\end{center}
\vspace*{-6mm}
\end{table}% made with matlab LC_kritisch.m ->krit_table.txt'
\index{Filter!kritisch-ged{\"a}mpft}\index{Gauss-Filter}\index{Filter!Gauss-}





\clearpage 
\begin{table}[!htb]
\begin{center}
{\tiny
\begin{tabular}{|c||r|r|r|r|r|r|r|r|r|}\hline
$n$ & $R_1$   & $C_1$ & $L_2$ & $C_3$ & $L_4$ & $C_5$ & $L_6$ & $C_7$ & $L_8$\\ \hline\hline
\textbf{ 2}&$\infty$&1.3911&0.8191&     &     &     &     &     &     \\
&10.0000&0.0868&14.4332&     &     &     &     &     &     \\
&5.0000&0.1841&7.4256&     &     &     &     &     &     \\
&3.3333&0.2933&5.0502&     &     &     &     &     &     \\
&2.5000&0.4169&3.8265&     &     &     &     &     &     \\
&2.0000&0.5597&3.0538&     &     &     &     &     &     \\
&1.6667&0.7326&2.4885&     &     &     &     &     &     \\
&1.4286&0.9771&1.9824&     &     &     &     &     &     \\
&1.3554&1.2087&1.6382&     &     &     &     &     &     \\ \hline
\textbf{ 3}&$\infty$&1.5133&1.5090&0.7164&     &     &     &     &     \\
&0.1000&7.5121&0.1549&15.4656&     &     &     &     &     \\
&0.2000&3.9418&0.3172&7.8503&     &     &     &     &     \\
&0.3000&2.7630&0.4860&5.2788&     &     &     &     &     \\
&0.4000&2.1857&0.6603&3.9675&     &     &     &     &     \\
&0.5000&1.8530&0.8383&3.1595&     &     &     &     &     \\
&0.6000&1.6475&1.0174&2.6026&     &     &     &     &     \\
&0.7000&1.5210&1.1927&2.1901&     &     &     &     &     \\
&0.8000&1.4511&1.3557&1.8711&     &     &     &     &     \\
&0.9000&1.4258&1.4935&1.6219&     &     &     &     &     \\
&1.0000&1.4328&1.5937&1.4328&     &     &     &     &     \\ \hline
\textbf{ 4}&$\infty$&1.5107&1.7682&1.4550&0.6725&     &     &     &     \\
&10.0000&0.0704&14.8873&0.1802&15.2297&     &     &     &     \\
&5.0000&0.1475&7.6072&0.3670&7.6142&     &     &     &     \\
&3.3333&0.2329&5.1777&0.5602&5.0301&     &     &     &     \\
&2.5000&0.3288&3.9606&0.7599&3.6977&     &     &     &     \\
&2.0000&0.4398&3.2268&0.9672&2.8563&     &     &     &     \\
&1.6667&0.5764&2.7304&1.1851&2.2425&     &     &     &     \\
&1.4286&0.7789&2.3480&1.4292&1.7001&     &     &     &     \\
&1.3554&0.9924&2.1476&1.5845&1.3451&     &     &     &     \\ \hline
\textbf{ 5}&$\infty$&1.5613&1.8049&1.7659&1.4173&0.6507&     &     &     \\
&0.1000&6.7870&0.1447&17.9569&0.1820&15.7447&     &     &     \\
&0.2000&3.5457&0.2950&9.1272&0.3659&7.8890&     &     &     \\
&0.3000&2.4765&0.4509&6.1861&0.5503&5.2373&     &     &     \\
&0.4000&1.9538&0.6119&4.7193&0.7333&3.8861&     &     &     \\
&0.5000&1.6535&0.7777&3.8446&0.9126&3.0548&     &     &     \\
&0.6000&1.4694&0.9469&3.2688&1.0846&2.4835&     &     &     \\
&0.7000&1.3580&1.1170&2.8679&1.2437&2.0621&     &     &     \\
&0.8000&1.2998&1.2824&2.5819&1.3815&1.7384&     &     &     \\
&0.9000&1.2845&1.4329&2.3794&1.4878&1.4883&     &     &     \\
&1.0000&1.3013&1.5559&2.2411&1.5559&1.3013&     &     &     \\ \hline
\textbf{ 6}&$\infty$&1.5339&1.8838&1.8306&1.7485&1.3937&0.6383&     &     \\
&10.0000&0.0666&14.2200&0.1777&18.4267&0.1901&15.3495&     &     \\
&5.0000&0.1393&7.2500&0.3613&9.2605&0.3835&7.6184&     &     \\
&3.3333&0.2195&4.9266&0.5514&6.1947&0.5795&4.9962&     &     \\
&2.5000&0.3095&3.7652&0.7492&4.6513&0.7781&3.6453&     &     \\
&2.0000&0.4137&3.0679&0.9575&3.7118&0.9794&2.7936&     &     \\
&1.6667&0.5422&2.6003&1.1830&3.0641&1.1850&2.1739&     &     \\
&1.4286&0.7347&2.2492&1.4537&2.5437&1.4051&1.6293&     &     \\
&1.3554&0.9419&2.0797&1.6581&2.2473&1.5344&1.2767&     &     \\ \hline
\textbf{ 7}&$\infty$&1.5748&1.8577&1.9210&1.8270&1.7340&1.3786&0.6307&     \\
&0.1000&6.5695&0.1405&17.6031&0.1838&19.3760&0.1862&15.8127&     \\
&0.2000&3.4278&0.2862&8.9371&0.3692&9.7697&0.3723&7.8901&     \\
&0.3000&2.3917&0.4369&6.0535&0.5557&6.5685&0.5569&5.2167&     \\
&0.4000&1.8853&0.5926&4.6179&0.7423&4.9702&0.7384&3.8552&     \\
&0.5000&1.5948&0.7529&3.7642&0.9276&4.0150&0.9142&3.0182&     \\
&0.6000&1.4170&0.9169&3.2052&1.1092&3.3841&1.0807&2.4437&     \\
&0.7000&1.3100&1.0826&2.8192&1.2833&2.9422&1.2326&2.0207&     \\
&0.8000&1.2550&1.2449&2.5481&1.4430&2.6242&1.3619&1.6967&     \\
&0.9000&1.2422&1.3946&2.3613&1.5784&2.3966&1.4593&1.4472&     \\
&1.0000&1.2615&1.5196&2.2392&1.6804&2.2392&1.5196&1.2615&     \\ \hline
\textbf{ 8}&$\infty$&1.5422&1.9106&1.9008&1.9252&1.8200&1.7231&1.3683&0.6258\\
&10.0000&0.0652&13.9469&0.1749&18.3007&0.1942&19.0437&0.1922&15.3880\\
&5.0000&0.1364&7.1050&0.3554&9.1917&0.3917&9.5260&0.3863&7.6164\\
&3.3333&0.2147&4.8250&0.5421&6.1483&0.5930&6.3423&0.5820&4.9811\\
&2.5000&0.3025&3.6860&0.7364&4.6191&0.7990&4.7388&0.7787&3.6241\\
&2.0000&0.4042&3.0029&0.9415&3.6917&1.0118&3.7619&0.9767&2.7690\\
&1.6667&0.5298&2.5460&1.1643&3.0568&1.2367&3.0869&1.1769&2.1477\\
&1.4286&0.7186&2.2054&1.4350&2.5554&1.4974&2.5422&1.3882&1.6029\\
&1.3554&0.9234&2.0454&1.6453&2.2826&1.6841&2.2300&1.5092&1.2515\\  \hline\hline
$n$ & $1/(R_1)$ & $L_1$ & $C_2$ & $L_3$ & $C_4$ & $L_5$ & $C_6$ & $L_7$ & $C_8$\\ \hline
\end{tabular}\vspace*{-1mm}\caption{Normierte Elementwerte
    f"ur Tschebyscheff-I-TP-Filter mit $A_{\max}=0.1$~dB (frequenznormiert
    auf die 3~dB-Grenzfrequenz bei $\Omega=1$) \label{el-C1}}}
\end{center}
\vspace*{-6mm}
\end{table}% made with matlab make_LC_tscheb01_tabelle -> tscheb01_tabelle.txt



\clearpage
\begin{table}[!htb]
\begin{center}
{\tiny
\begin{tabular}{|c||r|r|r|r|r|r|r|r|r|r|r|}\hline
$n$ & $R_1$   & $C_1$ & $L_2$ & $C_3$ & $L_4$ & $C_5$ & $L_6$ & $C_7$ & $L_8$ & $C_9$ & $L_{10}$\\ \hline\hline
\textbf{ 2}&$\infty$&1.3067&0.9748&     &     &     &     &     &     &     &     \\
&10.0000&0.1052&13.3222&     &     &     &     &     &     &     &     \\
&5.0000&0.2282&6.6994&     &     &     &     &     &     &     &     \\
&3.3333&0.3754&4.4110&     &     &     &     &     &     &     &     \\
&2.5000&0.5635&3.1648&     &     &     &     &     &     &     &     \\
&2.0000&0.9086&2.1029&     &     &     &     &     &     &     &     \\
&1.9841&0.9827&1.9497&     &     &     &     &     &     &     &     \\ \hline
\textbf{ 3}&$\infty$&1.5720&1.5179&0.9318&     &     &     &     &     &     &     \\
&0.1000&9.8899&0.1534&16.1177&     &     &     &     &     &     &     \\
&0.2000&5.2543&0.3087&8.2251&     &     &     &     &     &     &     \\
&0.3000&3.7292&0.4633&5.5762&     &     &     &     &     &     &     \\
&0.4000&2.9854&0.6146&4.2416&     &     &     &     &     &     &     \\
&0.5000&2.5571&0.7592&3.4360&     &     &     &     &     &     &     \\
&0.6000&2.2889&0.8937&2.8984&     &     &     &     &     &     &     \\
&0.7000&2.1135&1.0149&2.5172&     &     &     &     &     &     &     \\
&0.8000&1.9965&1.1203&2.2368&     &     &     &     &     &     &     \\
&0.9000&1.9175&1.2086&2.0255&     &     &     &     &     &     &     \\
&1.0000&1.8636&1.2804&1.8636&     &     &     &     &     &     &     \\ \hline
\textbf{ 4}&$\infty$&1.4361&1.8888&1.5211&0.9129&     &     &     &     &     &     \\
&10.0000&0.0975&15.3521&0.1940&14.2616&     &     &     &     &     &     \\
&5.0000&0.2100&7.7076&0.3996&6.9874&     &     &     &     &     &     \\
&3.3333&0.3440&5.1196&0.6208&4.4790&     &     &     &     &     &     \\
&2.5000&0.5162&3.7660&0.8693&3.1206&     &     &     &     &     &     \\
&2.0000&0.8452&2.7197&1.2383&1.9848&     &     &     &     &     &     \\
&1.9841&0.9202&2.5864&1.3036&1.8258&     &     &     &     &     &     \\ \hline
\textbf{ 5}&$\infty$&1.6299&1.7400&1.9217&1.5138&0.9034&     &     &     &     &     \\
&0.1000&9.5560&0.1525&19.6465&0.1731&16.5474&     &     &     &     &     \\
&0.2000&5.0639&0.3060&10.0537&0.3430&8.3674&     &     &     &     &     \\
&0.3000&3.5877&0.4590&6.8714&0.5075&5.6245&     &     &     &     &     \\
&0.4000&2.8692&0.6091&5.2960&0.6640&4.2447&     &     &     &     &     \\
&0.5000&2.4571&0.7537&4.3672&0.8098&3.4137&     &     &     &     &     \\
&0.6000&2.2006&0.8901&3.7651&0.9420&2.8609&     &     &     &     &     \\
&0.7000&2.0347&1.0150&3.3525&1.0582&2.4704&     &     &     &     &     \\
&0.8000&1.9257&1.1261&3.0599&1.1569&2.1845&     &     &     &     &     \\
&0.9000&1.8540&1.2220&2.8478&1.2379&1.9701&     &     &     &     &     \\
&1.0000&1.8069&1.3025&2.6914&1.3025&1.8069&     &     &     &     &     \\ \hline
\textbf{ 6}&$\infty$&1.4618&1.9799&1.7803&1.9253&1.5077&0.8981&     &     &     &     \\
&10.0000&0.0958&15.1862&0.1974&17.6807&0.2017&14.4328&     &     &     &     \\
&5.0000&0.2059&7.6144&0.4064&8.7318&0.4121&7.0310&     &     &     &     \\
&3.3333&0.3370&5.0553&0.6323&5.6993&0.6348&4.4809&     &     &     &     \\
&2.5000&0.5056&3.7219&0.8900&4.1092&0.8808&3.1025&     &     &     &     \\
&2.0000&0.8304&2.7041&1.2913&2.8720&1.2372&1.9556&     &     &     &     \\
&1.9841&0.9053&2.5774&1.3676&2.7133&1.2991&1.7962&     &     &     &     \\ \hline
\textbf{ 7}&$\infty$&1.6464&1.7772&2.0306&1.7892&1.9239&1.5034&0.8948&     &     &     \\
&0.1000&9.4555&0.1513&19.6486&0.1778&20.6314&0.1761&16.6655&     &     &     \\
&0.2000&5.0070&0.3034&10.0491&0.3524&10.4959&0.3478&8.4041&     &     &     \\
&0.3000&3.5456&0.4548&6.8674&0.5221&7.1341&0.5129&5.6350&     &     &     \\
&0.4000&2.8348&0.6035&5.2947&0.6846&5.4698&0.6690&4.2428&     &     &     \\
&0.5000&2.4275&0.7470&4.3695&0.8377&4.4886&0.8137&3.4050&     &     &     \\
&0.6000&2.1744&0.8824&3.7717&0.9786&3.8524&0.9441&2.8481&     &     &     \\
&0.7000&2.0112&1.0070&3.3638&1.1050&3.4163&1.0582&2.4554&     &     &     \\
&0.8000&1.9045&1.1182&3.0761&1.2149&3.1071&1.1546&2.1681&     &     &     \\
&0.9000&1.8348&1.2146&2.8691&1.3080&2.8829&1.2335&1.9531&     &     &     \\
&1.0000&1.7896&1.2961&2.7177&1.3848&2.7177&1.2961&1.7896&     &     &     \\ \hline
\textbf{ 8}&$\infty$&1.4710&2.0022&1.8248&2.0440&1.7911&1.9218&1.5003&0.8926&     &     \\
&10.0000&0.0951&15.1014&0.1969&17.7748&0.2081&18.0544&0.2035&14.4924&     &     \\
&5.0000&0.2044&7.5682&0.4052&8.7770&0.4257&8.8832&0.4146&7.0453&     &     \\
&3.3333&0.3344&5.0234&0.6304&5.7322&0.6577&5.7760&0.6370&4.4806&     &     \\
&2.5000&0.5017&3.6988&0.8878&4.1404&0.9184&4.1470&0.8814&3.0954&     &     \\
&2.0000&0.8249&2.6915&1.2919&2.9133&1.3160&2.8799&1.2331&1.9448&     &     \\
&1.9841&0.8998&2.5670&1.3697&2.7585&1.3903&2.7175&1.2938&1.7852&     &     \\ \hline
\textbf{ 9}&$\infty$&1.6533&1.7890&2.0570&1.8383&2.0481&1.7910&1.9199&1.4981&0.8911&     \\
&0.1000&9.4131&0.1507&19.5995&0.1779&20.8006&0.1822&20.8588&0.1770&16.7140&     \\
&0.2000&4.9830&0.3021&10.0212&0.3526&10.5818&0.3600&10.5925&0.3491&8.4189&     \\
&0.3000&3.5279&0.4528&6.8474&0.5223&7.1951&0.5318&7.1876&0.5142&5.6390&     \\
&0.4000&2.8203&0.6008&5.2792&0.6850&5.5207&0.6957&5.5023&0.6700&4.2416&     \\
&0.5000&2.4150&0.7436&4.3573&0.8385&4.5355&0.8493&4.5087&0.8140&3.4010&     \\
&0.6000&2.1634&0.8786&3.7621&0.9801&3.8985&0.9900&3.8647&0.9436&2.8426&     \\
&0.7000&2.0013&1.0028&3.3565&1.1075&3.4635&1.1157&3.4232&1.0568&2.4489&     \\
&0.8000&1.8955&1.1139&3.0709&1.2189&3.1565&1.2246&3.1102&1.1523&2.1611&     \\
&0.9000&1.8267&1.2103&2.8658&1.3135&2.9353&1.3165&2.8834&1.2302&1.9458&     \\
&1.0000&1.7822&1.2921&2.7162&1.3922&2.7734&1.3922&2.7162&1.2921&1.7822&     \\ \hline
\textbf{ 10}&$\infty$&1.4753&2.0107&1.8386&2.0733&1.8432&2.0494&1.7904&1.9183&1.4965&0.8900\\
&10.0000&0.0948&15.0578&0.1965&17.7624&0.2086&18.2313&0.2107&18.1645&0.2041&14.5199\\
&5.0000&0.2037&7.5446&0.4042&8.7694&0.4266&8.9726&0.4300&8.9248&0.4154&7.0518\\
&3.3333&0.3332&5.0070&0.6289&5.7273&0.6594&5.8398&0.6631&5.7947&0.6376&4.4803\\
&2.5000&0.4999&3.6869&0.8857&4.1383&0.9216&4.2020&0.9238&4.1540&0.8812&3.0919\\
&2.0000&0.8224&2.6845&1.2901&2.9166&1.3246&2.9390&1.3191&2.8782&1.2306&1.9397\\
&1.9841&0.8972&2.5610&1.3683&2.7632&1.4009&2.7795&1.3927&2.7148&1.2908&1.7801\\  \hline \hline
$n$ & $1/(R_1)$ & $L_1$ & $C_2$ & $L_3$ & $C_4$ & $L_5$ & $C_6$ & $L_7$ & $C_8$ & $L_9$ & $C_{10}$\\ \hline
\end{tabular}\vspace*{-1mm}\caption{Normierte Elementwerte
    f"ur Tschebyscheff-I-TP-Filter mit $A_{\max}=0.5$~dB (frequenznormiert
    auf die 3~dB-Grenzfrequenz bei $\Omega=1$) \label{filter-tsch05db} }}
\end{center}
\vspace*{-6mm}
\end{table}% made with matlab make_LC_tscheb05_tabelle -> tscheb05_tabelle.txt
\clearpage

\begin{table}[!htb]
\begin{center}
  \bild{/filter/FIL_Zver2.ps,width=0.85}\caption{Normierte Elementwerte f"ur Cauer-Filter 3.~Ordnung f"ur $\rho=15\%\equiv 0.1~$dB (frequenznormiert auf die Rippelgrenzfrequenz) (Auszug aus \cite{ZVE:67} Seite 176) \label{el-CC1}}
\end{center}
\vspace*{-6mm}
\end{table}
\clearpage

\begin{table}[!htb]
\begin{center}\vspace*{-2mm}
  \bild{/filter/FIL_Zver1.ps,width=0.75}\vspace*{-2mm}\caption{Normierte Elementwerte f"ur Cauer-Filter 3.~Ordnung f"ur $\rho=20\%\equiv 0.178~$dB (frequenznormiert auf die Rippelgrenzfrequenz) (Auszug aus \cite{ZVE:67} Seite 177) \label{el-CC2}}
\end{center}
\vspace*{-6mm}
\end{table}


