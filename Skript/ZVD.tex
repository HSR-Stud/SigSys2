\renewcommand{\thesection}{\thechapter.\arabic{section}}
\setcounter{Aufgabe}{0}\setcounter{Beispiel}{0}
\chapter{Zustandsraumdarstellung\index{Zustandsraumdarstellung}}\label{kapitel_ZVD} 
\section{Einf\"uhrung}
In diesem Kapitel werden wir die {\bf Grundlagen} der {\bf Zustandsraumdarstellung} von {\bf analogen
LTI-Systemen}\index{LTI-System} betrachten.  Die {\bf
  Zustandsraumdarstellung} wird auch h\"aufig {\bf
  Zustandsvariablenarstellung}
genannt\index{Zustandsvariablendarstellung}. In der englischsprachigen
Literatur wird  f\"ur die Zustandsraumdarstellung der Ausdruck {\bf State-Space Analysis}\index{State Space Analysis@{State-Space Analysis}} verwendet.\\

\nit Die Zustandsraumdarstellung wird haupts\"achlich in der
Regelungstechnik\index{Regelungstechnik} verwendet \cite{UNB:81}.  Der
Grundgedanke der {\bf Zustandsraumdarstellung} besteht darin, die {\bf
  Differentialgleichung}\index{Differentialgleichung} {\boldmath
  $n.$}~{\bf Ordnung}\index{Ordnung}, welche ein LTI-System beschreibt, durch ein {\bf
  Differentialgleichungssystem}\index{Differentialgleichungssystem}
von {\bf {\boldmath $n$} Gleichungen 1.~Ordnung} darzustellen \cite{WOS:88}.\\

\nit Mit der Zustandsraumdarstellung kann man einen Einblick in den {\bf
  inneren Aufbau} eines Systems und eine {\bf einheitliche} Darstellung
von Systemverhalten erhalten \cite{GIR:RAB:STE:05}. Das ``{\bf Innenleben}'' des Systems kann
durch die {\bf Energiespeicher}\index{Energiespeicher} des Systems
vollst\"andig beschrieben werden. Im Fall von elektrischen Systemen sind dies {\boldmath
  $L$} und {\boldmath $C$}. Dieser Systemzustand\footnote{d.h., die Str\"ome durch die Induktivit\"aten $L$ und die Spannungen \"uber den Kapazit\"aten $C$}
enth\"alt alle {\bf Information}\index{Information}, um mit den
gegebenen Eingangssignalen das Systemverhalten f\"ur alle Zeiten {\bf
  eindeutig} zu bestimmen \cite{GIR:RAB:STE:05}.\\ 

\nit Aus praktischen
Gr\"unden werden nur {\bf Integratoren} verwendet um die Systeme zu realisieren und durch die {\bf Zustandsraumdarstellung}
darzustellen \cite{WEI:89}.



\newpage
\subsection{Verwendung von Integratoren}\index{Integrator}
Prinzipiell k"onnen {\bf Integratoren}\index{Integrator} oder {\bf Differentiatoren}\index{Differentiator} f"ur die {\bf Zustandsraumdarstellung} verwendet werden. In der Praxis hat sich die Darstellung mit {\bf Integratoren} durchgesetzt, da die Signale meistens mit einem {\bf Rauschen}\index{Rauschen} behaftet sind, das h"oherfrequente Frequenzanteile, als das Signal, besitzt. 

\begin{figure}[!htb]
  \vspace*{-1mm}\begin{center}
    \bild{/ZVD/ZVD_rauschen_ableitung.eps,width=1}\vspace*{-3mm}\caption{a) Sinussignal mit und ohne Rauschen; b) Ableitung von a); c) Integration von a). Die Ableitung und die Integration ergibt jeweils eine Phasenverschiebung um $\pm \frac{\pi}{2}$.}
\end{center}
\vspace*{-6mm}
\end{figure}
\index{Ableitung}\index{Phasenverschiebung}

%\newpage
\section{Zustandsraumdarstellung}
Mit den {\bf Vektoren} \vector{u}$(t)$, \vector{x}$(t)$ und \vector{y}$(t)$ und den {\bf Matrizen}\index{Matrizen}  {\boldmath $A$, $B$, $C$} und  {\boldmath $D$} ergibt sich die standartisierte Form einer {\bf Matrixdifferentialgleichung}\index{Matrixdifferentialgleichung} erster Ordnung (Formel~\ref{ZRD_FORM_01}) und eine {\bf algebraische Gleichung}\index{Gleichung!algebraische} (Formel~\ref{ZRD_FORM_02}) \cite{WEI:89}:\\~\\
\myboxx{\begin{eqnarray}
\dot{\underline{x}}(t) &= &{\boldsymbol A} \underline{x}(t) + {\boldsymbol B} \underline{u}(t),\label{ZRD_FORM_01}\\
\underline{y}(t) &=& {\boldsymbol C} \underline{x}(t) + {\boldsymbol D} \underline{u}(t).\label{ZRD_FORM_02}
\end{eqnarray}}\\~\\

\nit Im Allgemeinen Fall von $m$ {\bf Eingangssignalen} \vector{u}$(t)$ und
$k$ {\bf Ausgangssignalen} \vector{y}$(t)$ ist \vector{u}$(t)$ ein
{\bf Spaltenvektor}\index{Spaltenvektor} mit $m$ Eintr\"agen,
\vector{x}$(t)$ ein Spaltenvektor mit $n$ {\bf
  Zustandsgr\"ossen}\index{Zustandsgrossen@{Zustandsgr\"ossen}} und
\vector{y}$(t)$ ein Spaltenvektor mit $k$ Ausgangssignalen
\cite{GIR:RAB:STE:05}.
Die Gleichung~\ref{ZRD_FORM_01} heisst {\bf Zustandsgleichung}\index{Zustandsgleichung} und die Gleichung~\ref{ZRD_FORM_02} heisst {\bf Ausgangsgleichung}\index{Ausgangsgleichung}. Beide Gleichungen zusammen bilden die {\bf Zustandsraumdarstellung}\index{Zustandsraumdarstellung} im Zeitbereich. Die Bezeichnungen von {\boldmath $A$, $B$, $C$} und  {\boldmath $D$} sind \cite{UNB:81, WOS:88}:\\
\begin{enumerate}
\item[] {\boldmath $A$}: {\bf Systemmatrix} ($n\times n$-Matrix)\index{Systemmatrix}. Sie bestimmt das Verhalten des {\bf ungest\"orten  Systems}\index{System!ungestort@{ungest\"ort}} (\vector{u}$(t)=$\vector{0}).
\item[] {\boldmath $B$}: {\bf Steuermatrix} oder
 {\bf Eingangsmatrix}\index{Eingangsmatrix}
  ($n\times m$-Matrix)\index{Steuermatrix}. Sie bestimmt dir Wirkung der {\bf Steuergr\"osse}\index{Steuergrosse@{Steuergr\"osse}} \vector{u}$(t)$ auf die {\bf Zustandsgr\"ossen}\index{Zustandsgrossen@{Zustandsgr\"ossen}} \vector{x}$(t)$.
\item[] {\boldmath $C$}: {\bf Beobachtungs- oder Ausgangsmatrix}
  ($k\times n$-Matrix)\index{Ausgangsmatrix}\index{Beobachtungsmatrix}. Sie kennzeichnet die Abh\"angigkeit des {\bf Zustandes}\index{Zustand} \vector{x}$(t)$ durch die beobachtbare Ausgangsgr\"osse \vector{y}$(t)$.
\item[] {\boldmath $D$}: {\bf \"Ubergangsmatrix} oder {\bf Durchgangsmatrix}\index{Durchgangsmatrix}
  ($k\times m$-Matrix)\index{Ubergangsmatrix@{\"Ubergangsmatrix}}. Sie bestimmt die unmittelbare Wirkung der Eingangsgr\"osse \vector{u}$(t)$ auf den Ausgang \vector{y}$(t)$.\\
\end{enumerate}
\begin{figure}[!htb]
\vspace*{-4mm}\begin{center}
  \bild{/zvd/ZVD001.ps,width=0.95}\vspace*{-3mm}\caption{Blockdiagramm der Zustandsraumdarstellung mit den {\bf Matrizen}\index{Matrizen}  {\boldmath $A$, $B$, $C$} und  {\boldmath $D$} und den {\bf Integratoren}\index{Integrator} $\int$.}\label{ZVD_BILD_001}
\end{center}
\vspace*{-6mm}
\end{figure}

\bsp{Aufstellen der Zustandsraumdarstellung}
\begin{figure}[!htb]
  \vspace*{-6mm}\begin{center}
    \bild{/fourier/FOU012.ps,width=0.4}\vspace*{-3mm}\caption{Serieschwingkreis}\label{ZVD_ABB_003}\index{Serieschwingkreis}
\end{center}
\vspace*{-6mm}
\end{figure}\\
\nit Mit der Zustandsgr\"osse $x_1(t)\equiv i(t)$ und $x_2(t)\equiv u_c(t)$  wird die  Zustandsraumdarstellung der Schaltung von Abb.~\ref{ZVD_ABB_003}, wobei wir folgende Zusammenh\"ange verwenden
\begin{eqnarray}
  C\dot{u}_c(t)&=&i(t)\quad\text{oder}\quad C\dot{x}_2(t)=x_1(t),\nonumber \\
 y(t)&\equiv& u_c(t),\nonumber\\
 u_0(t)&=&u_c(t)+i(t)\cdot R+L\frac{\partial i(t)}{\partial t}=x_2(t)+x_1(t)\cdot R + \dot{x}_1(t)\cdot L,\nonumber\\
 u_0(t)&=&\dot{y}(t)\cdot CR + y(t) +\ddot{y}(t)\cdot LC,\label{ZVD_RES_001}
\end{eqnarray}
  zu
\begin{eqnarray*}
\left [ 
\begin{array}{c}
\dot{x}_1(t)\\
\dot{x}_2(t)\\
\end{array}
\right ] &=&
\left [ 
\begin{array}{c c}
\frac{-R}{L} & \frac{-1}{L}\\
\frac{1}{C} & 0\\
\end{array}
\right ]\cdot
\left [ 
\begin{array}{c}
x_1(t)\\
x_2(t)\\
\end{array}
\right ]+
\left [ 
\begin{array}{c}
\frac{1}{L}\\
0\\
\end{array}
\right ]\cdot
u_0(t),\\
y(t) &= &
\left [ 
\begin{array}{c c}
0 & 1\\
\end{array}
\right ] \cdot
\left [ 
\begin{array}{c}
x_1(t)\\
x_2(t)\\
\end{array}
\right ]+
\left [ 
\begin{array}{c}
0 \\
\end{array}
\right ] \cdot
u_0(t).
\end{eqnarray*}
%cite HSU: S.374 und unb:81
\aufg Bestimmen Sie f\"ur
die folgende Schaltung die Zustandsraumdarstellung.\\ 
\begin{figure}[!htb]
\vspace*{-4mm}\begin{center}
  \bild{/zvd/ZVD004.ps,width=0.45}\vspace*{-3mm}\caption{$RC$-Tiefpass}\index{RC-Tiefpass@{$RC$-Tiefpass}}\label{ZVD_ABB_004}
\end{center}
\vspace*{-8mm}
\end{figure}\\
\nit W\"ahlen Sie f\"ur
$R$ und $C$ von Abb.~\ref{ZVD_ABB_004} realistische Werte und verwenden Sie $u_c(t)$ als Zustandsgr\"osse, $y(t)$ als Ausgang und $u(t)$ als Eingang. Wie lautet die UTF dieser Schaltung? Kontrollieren Sie Ihre Resultate mit den \mb\hspace*{-1.6mm}en {\tt ss2tf} und {\tt tf2ss}.

\subsection{Ordnung eines Systems}\index{Ordnung}
Die Ordnung eines Systems ist definiert als die {\bf kleinste Anzahl von Zustandsgr"ossen} $\vector{x}(t)$ \cite{KIE:JAE:05}. \"Aquivalent dazu kann die Ordnung des Systems auch als die {\bf Anzahl der unabh"angigen Energiespeicher} definiert werden \cite{GIR:RAB:STE:05}.  
\bsp{Der Schwingkreis von Abb.~\ref{ZVD_ABB_003} hat die Ordnung 2, da $L$ und $C$ unabh"angig sind.}\index{unabhangig@unabh\"angig}

\subsection{\"Aquivalente Zustandsraumdarstellungen, Willk\"urlichkeit der Zustandsgr\"ossen}\index{Zustandsgrossen@{Zustandsgr\"ossen}!Willkurlichkeit@{Willk\"urlichkeit}}
Mit einer {\bf Transformationsmatrix}\index{Transformationsmatrix}
$\boldsymbol{T}$ ($n\times n-$Matrix, nicht {\bf
  singul\"ar}\index{singular@{singul\"ar}}\footnote{D.h, die {\bf
  Determinante}\index{Determinante} von \mat{T} ist ungleich Null.}, \mat{TT^{-1}=I}) kann man
verschiedenste {\bf Zustandsgr\"ossen}\index{Zustandsgrossen@{Zustandsgr\"ossen}} und {\bf
  Zustandsraumdarstellungen}\index{Zustandsraumdarstellung} erhalten, die aber alle ein {\bf identisches} Systemverhalten aufweisen \cite{GIR:RAB:STE:05, HSU:95}.\\
\begin{figure}[!htb]
\vspace*{-4mm}\begin{center}
  \bild{/zvd/ZVD002.ps,width=0.95}\vspace*{-3mm}\caption{Blockdiagramm einer anderen Zustandsraumdarstellung mit identischem Systemverhalten wie in Abbildung~\ref{ZVD_BILD_001}.}\label{ZVD_BILD_002}
\end{center}
\vspace*{-6mm}
\end{figure}\\
\nit In Abb.~\ref{ZVD_BILD_002} ist $\boldsymbol{\hat{A}}=\boldsymbol{TAT^{-1}}$, $\boldsymbol{\hat{B}}=\boldsymbol{TB}$, $\boldsymbol{\hat{C}}=\boldsymbol{CT^{-1}}$, $\boldsymbol{\hat{D}}=\boldsymbol{D}$ und somit ergibt ist die 
Zustandsraumdarstellung mit den Zustandsgr\"ossen \vector{\xi}$(t)$\\~\\
\myboxx{\begin{eqnarray}
\dot{\underline{\xi}}(t) &=& \boldsymbol{TAT^{-1}} \underline{\xi}(t) + \boldsymbol{TB} \underline{u}(t),\label{ZRD_FORM_10}\\
\underline{y}(t) &=& \boldsymbol{CT^{-1}} \underline{\xi}(t) + \boldsymbol{D} \underline{u}(t).\label{ZRD_FORM_11}
\end{eqnarray}}\\~\\
Multiplizieren wir Formel~\ref{ZRD_FORM_10} von links mit
$\boldsymbol{T^{-1}}$, so erhalten wir 
\begin{equation*}
\boldsymbol{
    T^{-1}}\dot{\underline{\xi}}(t) = \boldsymbol{AT^{-1}}
  \underline{\xi}(t) + \boldsymbol{B} \underline{u}(t).
\end{equation*}
Vergleichen wir Formeln~\ref{ZRD_FORM_10} und \ref{ZRD_FORM_11} mit
Formeln~\ref{ZRD_FORM_01} und \ref{ZRD_FORM_02}, so erkennen wir, dass
\vector{u}$(t)$ und \vector{y}$(t)$ identisch sind und die
Zustandsgr\"ossen $\boldsymbol{
  T^{-1}}\underline{\xi}(t)=\underline{x}(t)$, d.h., $\boldsymbol{
  T}\underline{x}(t)=\underline{\xi}(t)$ ist.  Beide
Zustandsraumdarstellungen Formeln~\ref{ZRD_FORM_01} und \ref{ZRD_FORM_02},
sowie \ref{ZRD_FORM_10} und \ref{ZRD_FORM_11} sind {\bf \"aquivalent}
bez\"uglich \vector{y}$(t)$ und \vector{u}$(t)$, wobei wir gezeigt
haben, dass die Zustandsgr\"ossen $\underline{\xi}(t)$ und
$\underline{x}(t)$ willk\"urlich gew\"ahlt werden k\"onnen, solange
$\boldsymbol{T}$ nicht singul\"ar ist.\\ \nit Eine sehr gute Wahl der
{\bf Zustandsgr\"ossen}\index{Zustandsgrossen@{Zustandsgr\"ossen}} bei
elektrischen {\bf LTI-Systemen} sind die {\bf
  Spannungen}\index{Spannungen} \"uber den {\bf
  Kapazit\"aten}\index{Kapazitat@{Kapazit\"at}} und die {\bf
  Str\"ome}\index{Strome@{Str\"ome}} durch die {\bf
  Induktivit\"aten}\index{Induktivitat@{Induktivit\"at}}. Man sollte
aber beachten, dass keine zus\"atzlichen, unn\"otigen
Zustandsgr\"ossen eingef\"uhrt werden, wie z.B., bei der {\bf
  Serieschaltung}\index{Serieschaltung} oder {\bf Parallelschaltung}\index{Parallelschaltung} von Kapazit\"aten und Induktivit\"aten.


\subsection{L\"osung der Zustandsgleichung im Zeitbereich}
\subsection{Fundamentalmatrix}\index{Fundamentalmatrix}
Mit der Entwicklung von $e^{at}$ ($a\in\mathbb{C}$) nach {\bf Taylor}\index{Taylor} oder {\bf MacLaurin}\index{MacLaurin} \cite{BRO:SEM:91} 
\begin{equation*}
e^{at}=1+at+\frac{a^2}{2!}t^2+\ldots ++\frac{a^k}{k!}t^k+\ldots
\end{equation*}
definieren wir die Entwicklung von $e^{\boldsymbol{A}t}$ mit \cite{HSU:95, UNB:89}
\begin{equation}
e^{\boldsymbol{A}t}=\boldsymbol{I}+\boldsymbol{A}t+\frac{\boldsymbol{A}^2}{2!}t^2+\ldots +\frac{\boldsymbol{A}^k}{k!}t^k+\ldots~.
\end{equation}
Bei $t=0$ ist $e^{\boldsymbol{A}\cdot 0}=\boldsymbol{I}$. Weiter gilt, $e^{\boldsymbol{A}t}e^{-\boldsymbol{A}t}=e^{\boldsymbol{A}\cdot 0}=\boldsymbol{I}$, d.h. \cite{HSU:95}
\begin{equation*}
e^{-\boldsymbol{A}t}=\klam{e^{\boldsymbol{A}t}}^{-1}.
\end{equation*}
Leiten wir $e^{\boldsymbol{A}t}$ nach $t$ ab erhalten wir nach dem Umformen \cite{UNB:89}
\begin{equation*}
\frac{\partial e^{\boldsymbol{A}t}}{\partial t}=\boldsymbol{A}e^{\boldsymbol{A}t}=e^{\boldsymbol{A}t}\boldsymbol{A}.
\end{equation*}
Multiplizieren wir nun die {\bf Zustandsgleichung}\index{Zustandsgleichung} (Formel~\ref{ZRD_FORM_01}) mit $e^{-\boldsymbol{A}t}$ ergibt sich
\begin{equation*}
e^{-\boldsymbol{A}t}\dot{\underline{x}}(t) = e^{-\boldsymbol{A}t}\boldsymbol{A} \underline{x}(t) + e^{-\boldsymbol{A}t}\boldsymbol{B} \underline{u}(t).
\end{equation*}
Mit der {\bf Kettenregel}\index{Kettenregel} der {\bf Differentialgleichung}\index{Differentialgleichung} \cite{BRO:SEM:91}, ist
\begin{equation*}
\frac{\partial e^{-\boldsymbol{A}t}\underline{x}(t)}{\partial t}= e^{-\boldsymbol{A}t}\dot{\underline{x}}(t)-\boldsymbol{A}e^{-\boldsymbol{A}t}\underline{x}(t),
\end{equation*}
und somit erhalten wir
\begin{equation*}
\frac{\partial e^{-\boldsymbol{A}t}\underline{x}(t)}{\partial t}=e^{-\boldsymbol{A}t}\boldsymbol{B}\underline{u}(t).
\end{equation*}
Mit der Integration von 0 bis $t$ auf beiden Seiten und anschliessender Multiplikation mit  $e^{\boldsymbol{A}t}$ finden wir \cite{DIS:STU:WIL:90}\\~\\
\myboxx{\begin{equation}
\underline{x}(t)=e^{\boldsymbol{A}t}\underline{x}(0)+\int\limits_{0}^{t}e^{\boldsymbol{A}(t-\tau)}\boldsymbol{B}\underline{u}(\tau)d\tau.\label{ZVD_FORM_09}
\end{equation}}\\~\\
$e^{\boldsymbol{A}t}$ ist die {\bf Fundamentalmatrix}\index{Fundamentalmatrix} oder {\bf \"Ubergangsmatrix} ({\bf Transitionsmatrix})\index{Ubergangsmatrix@{\"Ubergangsmatrix}}\index{Transitionsmatrix} und wird wie folgt bezeichnet \cite{UNB:89}:\\~\\
\myboxx{\begin{equation}
\boldsymbol{\Phi}(t)=e^{\boldsymbol{A}t}
\end{equation}}\\~\\
Ist der Anfangszeitpunkt $t$ nicht bei Null, sondern $t_0$, so ergibt sich Formel~\ref{ZVD_FORM_09} zu
\begin{equation*}
\underline{x}(t)=\underbrace{\boldsymbol{\Phi}(t-t_0)}_{e^{\boldsymbol{A}(t-t_0)}}\underline{x}(t_0)+\int\limits_{t_0}^{t}\boldsymbol{\Phi}(t-\tau)\boldsymbol{B}\underline{u}(\tau)d\tau.
\end{equation*}
Zum Schluss erhalten wir f\"ur die Ausgangssignale \cite{LUT:WEN:05}\\~\\
\myboxx{\begin{equation}
    \underline{y}(t)=\boldsymbol{C}\boldsymbol{\Phi}(t)\underline{x}(0)+\int\limits_{0}^{t}\boldsymbol{C\Phi}(t-\tau)\boldsymbol{B}\underline{u}(\tau)d\tau+\boldsymbol{D}\underline{u}(t),\label{ZRD_FORMEL_ausgang}
\end{equation}}\\~\\
was die L\"osung der Zustandsraumdarstellung im Zeitbereich ist. Die Bestimmung von \mat{\Phi}$(t)$ ist im Allgemeinen sehr aufw\"andig.
\bsp{F\"ur \mat{A}$=\left [ 
\begin{array}{c c}
0 & 6 \\
-1 & -5\\
\end{array}
\right ] $ ist \mat{\Phi}$(t)=\left [ 
\begin{array}{c c}
\klam{3e^{-2t}-2e^{-3t}} & \klam{6e^{-2t}-6e^{-3t}} \\
\klam{-e^{-2t}+e^{-3t}} & \klam{-2e^{-2t}+3e^{-3t}}\\
\end{array}
\right ]$.}

\subsection{Berechnung Fundamentalmatrix $\boldsymbol{\Phi}(t)$}
Es gibt mehrere Methoden die quadratische ($n\times n$) {\bf Fundamentalmatrix}\index{Fundamentalmatrix} $\boldsymbol{\Phi}(t)$  zu bestimmen \cite{HSU:95, UNB:89}:\\
\begin{enumerate}
 \item[]Methode 1:  Mit der {\bf Laplace-Transformation}\index{Laplace} \cite{HSU:95, UNB:89} 
\begin{equation*}
\boldsymbol{\Phi}(t)=\mathcal{L}^{-1}\left\{ (s\boldsymbol{I}-\boldsymbol{A})^{-1}   \right\}
\end{equation*}
 \item[]Methode 2: {\bf Diagonalisierung}\index{Diagonalisierung} von $\boldsymbol{\Phi}(t)$ \cite{HSU:95, UNB:89}. Wenn $\boldsymbol{A_{\text{diagonal}}}=\boldsymbol{T^{-1}AT}$ ist, dann ist
\begin{equation*}
\boldsymbol{\Phi}(t)=e^{\boldsymbol{A}}=\boldsymbol{T}\underbrace{\left [ 
\begin{array}{c c c c}
e^{\lambda_1 t} & & \cdots & 0\\
 & e^{\lambda_2 t} & &\vdots \\
 \vdots & & \ddots & \\
 0 & \cdots& & e^{\lambda_n t}\\
\end{array}
\right ]
}_{\boldsymbol{\Phi}_\text{diagonal}(t)}\boldsymbol{T}^{-1}.
\end{equation*}
 \item[]Methode 3: {\bf Spektrale Zerlegung}\index{Spektrale Zerlegung} \cite{HSU:95, UNB:89}
 \item[]Methode 4: Mit dem Satz von {\bf Cayley-Hamilton}\index{Cayley-Hamilton} \cite{HSU:95, UNB:89}
\end{enumerate}

\bsp{Methode 1: {\bf Laplace-Transformation}\index{Laplace}\\ Mit der
  {\bf Systemmatrix}\index{Systemmatrix} $\boldsymbol{A}=\left
    [\begin{array}{c c} -1 & 0 \\ 1 &-2 \\ \end{array}\right ]$ ergibt
  sich $(s\boldsymbol{I}-\boldsymbol{A})=\left [ \begin{array}{c c}
      s+1 & 0 \\ -1 & s+2\\ \end{array}\right ]$. Somit ist
   $(s\boldsymbol{I}-\boldsymbol{A})^{-1}= \frac{\left [
   \begin{array}{c c} s+2 & 0 \\ 1 & s+1 \\ \end{array}\right ]}{ (s+1)(s+2)} \IFT \left [ \begin{array}{c c}
    e^{-t} & 0 \\    e^{-t} -e^{-2t}& e^{-2t}\\ \end{array} \right ] = \boldsymbol{\Phi}(t)$.}

\bsp{Methode 2: Diagonalisierung von \mat{A} und Bestimmung von \mat{\Phi}$(t)$ \cite{UNB:89}}%s.60 bsp 1.6.3
Mit der {\bf Systemmatrix}\index{Systemmatrix} $\boldsymbol{A}=\left
[ \begin{array}{c c} -1 & 0 \\ 1 &-2 \end{array}\right ]$, welche das
{\bf charakteristische Polynom}\index{charakteristisches Polynom}\index{Polynom!charakteristisches}
\begin{equation*}
 \left | \lambda\boldsymbol{I}-\boldsymbol{A} \right | =\left
  |\begin{array}{c c} \lambda+1 & 0 \\ -1 & \lambda+2
  \end{array}\right |=(\lambda+1)(\lambda+2)=0
\end{equation*}
 ergibt, erhalten wir
die {\bf Eigenwerte}\index{Eigenwert} $\lambda_1=-1$ und $\lambda_2=-2$. Die {\bf
  Transformationsmatrix}\index{Transformationsmatrix} \mat{T} und
\mat{T^{-1}} welche \mat{A} diagonalisieren sind somit z.B.
\begin{equation*}
\boldsymbol{T}=\left [ 
\begin{array}{c c}
1 & 0 \\
1 & 1\\
\end{array}
\right ],\quad \boldsymbol{T^{-1}}= 
\left [ 
\begin{array}{c c}
1 & 0 \\
-1 & 1\\
\end{array}
\right ],\text{~und damit ist~}\boldsymbol{A_{\text{diagonal}}}=\boldsymbol{T^{-1}AT}=
\left [ 
\begin{array}{c c}
-1 & 0 \\
0 & -2\\
\end{array}
\right ].
\end{equation*}
Da $\boldsymbol{\Phi}_{\text{diagonal}}(t)=e^{\boldsymbol{A}_{\text{diagonal}}t}$ ist, ist $\boldsymbol{\Phi}(t)$
\begin{equation*}
\boldsymbol{\Phi}(t)=e^{\boldsymbol{A}t}=\boldsymbol{T\Phi}_{\text{diagonal}}(t)\boldsymbol{T}^{-1}=\left [ 
\begin{array}{c c}
1 & 0 \\
1 & 1\\
\end{array}
\right ] 
\left [ 
\begin{array}{c c}
e^{-t} & 0 \\
0 & e^{-2t}\\
\end{array}
\right ]
\left [ 
\begin{array}{c c}
1 & 0 \\
-1 & 1\\
\end{array}
\right ]=
\left [ 
\begin{array}{c c}
e^{-t} & 0 \\
e^{-t} -e^{-2t}& e^{-2t}\\
\end{array}
\right ]
\end{equation*}
F\"ur eine genaue Erl\"auterung der Methoden~1 bis 4 ist auf die Referenzen \cite{HSU:95, UNB:89} verwiesen.
\aufg Bestimmen Sie \mat{\Phi}$(t)$ von \mat{A}$=\left [ 
\begin{array}{c c}
0 & 6 \\
-1 & -5\\
\end{array}
\right ] $ mit Methode 1. 
%\newpage
%\vspace*{-9mm}
\bsp{Mit \matlogo~~({\tt expm} und nicht {\tt exp})  erhalten wir z.B.:}
\begin{verbatim}
>> syms t
>> A=[0 6;-1 -5];
>> expm(A*t)
 ans =
[  3*exp(-2*t)-2*exp(-3*t), -6*exp(-3*t)+6*exp(-2*t)]
[      exp(-3*t)-exp(-2*t), -2*exp(-2*t)+3*exp(-3*t)]
\end{verbatim}\index{syms@{\tt syms}}\index{expm@{\tt expm}}\index{exp@{\tt exp}}

\subsection{Eigenschaften der Fundamentalmatrix}
Unter anderem hat die {\bf Fundamentalmatrix}\index{Fundamentalmatrix} $\boldsymbol{\Phi}(t)$ folgende Eigenschaften:\\
\begin{enumerate}
  \item[a)] $\boldsymbol{\Phi}(0)=\boldsymbol{I}$
  \item[b)] $\boldsymbol{\Phi}^{-1}(t)=\boldsymbol{\Phi}(-t)$   ($\boldsymbol{\Phi}(t)$ ist stets invertierbar)
  \item[c)] $\boldsymbol{\Phi}^k(t)=\boldsymbol{\Phi}(kt)$
  \item[d)]  $\boldsymbol{\Phi}(t_1)\boldsymbol{\Phi}(t_2)=\boldsymbol{\Phi}(t_1+t_2)$
  \item[e)]  $\boldsymbol{\Phi}(t_2-t_1)\boldsymbol{\Phi}(t_1-t_0)=\boldsymbol{\Phi}(t_2+t_0)$
\end{enumerate}
\aufg Versuchen Sie die Eigenschaften der Fundamentalmatrix zu beweisen, indem Sie  $\boldsymbol{\Phi}(t)=e^{-\boldsymbol{A}t}$ verweden.

\subsection{Fundamentalmatrix und Impulsantwort}\index{Impulsantwort}
Mit Formel \ref{ZRD_FORMEL_ausgang} erhalten wir:\\~\\
\begin{eqnarray*}
    \underline{y}(t)&=&\boldsymbol{C}\boldsymbol{\Phi}(t)\underline{x}(0)+\int\limits_{0}^{t}\boldsymbol{C\Phi}(t-\tau)\boldsymbol{B}\underline{u}(\tau)d\tau+\boldsymbol{D}\underline{u}(t)\\
   \underline{y}(t)&=&\int\limits_{-\infty}^{t}\boldsymbol{C\Phi}(t-\tau)\boldsymbol{B}\underline{u}(\tau)d\tau+\boldsymbol{D}\underline{u}(t)\\
\underline{y}(t)&=&\boldsymbol{C\Phi}(t)\boldsymbol{B}\ast\underline{u}(t)+\boldsymbol{D}\underline{u}(t)
\end{eqnarray*}
Somit erhalten wir f"ur SISO\index{SISO}-Systeme:
\begin{eqnarray}
y(t)&=&\boldsymbol{C\Phi}(t)\boldsymbol{B}\ast u (t)+\boldsymbol{D}u(t)\nonumber\\
y(t)&=& h(t)\ast u(t)  \nonumber\\
\rightarrow h(t)&=&\boldsymbol{C\Phi}(t)\boldsymbol{B}+\boldsymbol{D}\delta(t)
\end{eqnarray}
F"ur SISO\index{SISO}-Systeme ist $\boldsymbol{D}$  eine $1\times 1$-Matrix, $\boldsymbol{\Phi}(t)$  eine $n\times n$-Matrix, $\boldsymbol{C}$  eine $1\times n$-Matrix und $\boldsymbol{B}$  eine $n\times 1$-Matrix.

%\newpage
\section{Zustandsraumdarstellung im Frequenzbereich}\index{Bildbereich}\index{Frequenzbereich}
\subsection{Laplace-Transformation der Zustandsraumdarstellung}
Werden die Gleichungen~\ref{ZRD_FORM_01} und \ref{ZRD_FORM_02} der
{\bf Zustandsraumdarstellung}\index{Zustandsraumdarstellung} mit der
Laplace-Transformation\index{Laplace-Transformation} in den
Frequenzbereich (Bildbereich) transformiert, so erhalten wir mit der
Formel~\ref{FOU_FORM_ABLEITUNG} (d.h., mit
$\mathcal{L\{}$\vector{\dot{x}}$(t)\}=s$\vector{X}$(s)-$\vector{x}$(0)$)
die {\bf Zustandsraumdarstellung} im Frequenzbereich \cite{GIR:RAB:STE:05}:\\~\\
\myboxx{\begin{eqnarray}
    s\underline{X}(s)-\underline{x}(0) &=& \boldsymbol{A} \underline{X}(s) + \boldsymbol{B} \underline{U}(s),\label{ZRD_FORM_03}\\
    \underline{Y}(s) &=& \boldsymbol{C} \underline{X}(s) +
    \boldsymbol{D} \underline{U}(s).\label{ZRD_FORM_04}
\end{eqnarray}}\\~\\
Mit der Einheitsmatrix $\boldsymbol{I}$\index{Einheitsmatrix} (nur die Diagonalwerte haben den Wert 1, alle anderen Eintr\"age sind Null \cite{BRO:SEM:91}) erhalten wir\\
\begin{equation}
\underline{X}(s)=\klam{s\boldsymbol{I}-\boldsymbol{A}}^{-1}\underline{x}(0)+\klam{s\boldsymbol{I}-\boldsymbol{A}}^{-1}\boldsymbol{B}\underline{U}(s).\label{ZVD_FORM_05}
\end{equation}
Mit der inversen Laplace-Transformation kann aus Formel~\ref{ZVD_FORM_05} der zeitliche Verlauf der Zustandsgr\"ossen berechnet werden \cite{GIR:RAB:STE:05}. Mit Formel~\ref{ZVD_FORM_05} in Formel~\ref{ZRD_FORM_04} eingesetzt erhalten wir:\\~\\
\myboxx{\begin{eqnarray}
\underline{Y}(s)&\!\!=\!\!&\boldsymbol{C}\klam{s\boldsymbol{I}-\boldsymbol{A}}^{-1}\underline{x}(0)+\boldsymbol{C}\klam{s\boldsymbol{I}-\boldsymbol{A}}^{-1}\boldsymbol{B}\underline{U}(s)+\boldsymbol{D}\underline{U}(s),\label{ZRD_FORM_06}\\
\underline{Y}(s)&\!\!=\!\!&\boldsymbol{C}\klam{s\boldsymbol{I}-\boldsymbol{A}}^{-1}\underline{x}(0)+\underbrace{\klam{\boldsymbol{C}\klam{s\boldsymbol{I}-\boldsymbol{A}}^{-1}\boldsymbol{B}+\boldsymbol{D}}}_{\boldsymbol{H(s)}}\underline{U}(s).\label{ZRD_FORM_07}
\end{eqnarray}}\\~\\
Sind die {\bf Anfangsbedingungen}\index{Anfangsbedingungen} Null (\vector{x}$(0)=$\vector{0}), ergibt sich\\~\\
\myboxx{\begin{equation}
\underline{Y}(s)=\boldsymbol{H(s)}\cdot\underline{U}(s)\quad\rightarrow\quad \boldsymbol{H(s)}=\frac{\underline{Y}(s)}{\underline{U}(s)}=\boldsymbol{C}\klam{s\boldsymbol{I}-\boldsymbol{A}}^{-1}\boldsymbol{B}+\boldsymbol{D},
\end{equation}}\\~\\
was der {\bf UTF}\index{UTF} entspricht, aber im allgemeinen Fall eine Matrix ist, n\"amlich die {\bf \"Uber\-tragungs\-matrix}\index{Ubertragungsmatrix@{\"Ubertragungsmatrix}}.

\clearpage
\bsp{UTF der Schaltung (Serieschwingkreis) von Abb.~\ref{ZVD_ABB_003}}\label{ZVD_BSP_04}
\begin{enumerate}
\item[a)] Die {\bf UTF}\index{UTF} $H(s)$ ist mit der {\bf Impedanzrechnung}\index{Impedanzrechnung}\\
 \begin{equation*}
 H(s)=\frac{\frac{1}{sC}}{sL + R +\frac{1}{sC}}=\frac{1}{s^2LC +sCR +1}.
\end{equation*}
\item[b)] Mit den Anfangsbedingungen alle Null erhalten wir mit der transformierten Formel~\ref{ZVD_RES_001}$\FT U(s)=s^2Y(s)LC + RC\cdot Y(s)+ Y(s)$ und somit ist\\ 
\begin{equation*}
 H(s)=\frac{Y(s)}{U(s)}=\frac{1}{s^2LC +sCR +1}.
\end{equation*}
\item[c)] Mit den Matrizen {\boldmath $A$, $B$, $C$} und  {\boldmath $D$} folgt\footnote{Die Inverse einer Matrix \mat{K} ist $\boldsymbol{K}^{-1}=\frac{\text{adj($\boldsymbol{K}$)}}{\vert\boldsymbol{K}\vert}$, wobei $\text{adj}(\boldsymbol{K})$ die {\bf Adjungierte}\index{Adjungierte} von \mat{K} ist und zum Teil auch {\bf Adjunkte}\index{Adjunkte} gennant wird \cite{BRO:SEM:91}.}:\\
\begin{eqnarray*}
H(s) &=& \boldsymbol{C}\klam{s\boldsymbol{I}-\boldsymbol{A}}^{-1}\boldsymbol{B}+\boldsymbol{D}\\
& =& 
\left [ 
\begin{array}{c c}
0  & 1\\
\end{array}
\right ]\cdot
\left (
\left [ 
\begin{array}{cc}
 s & 0\\
0 & s\\
\end{array}
\right ] -
\left [ 
\begin{array}{cc}
 \frac{-R}{L} & \frac{-1}{L}\\
\frac{1}{C}& 0\\
\end{array}
\right ]
\right )^{-1}\cdot
\left [ 
\begin{array}{c}
 \frac{1}{L}\\
0\\
\end{array}
\right ]+0\\
 &=& 
\left [ 
\begin{array}{c c}
0  & 1\\
\end{array}
\right ]\cdot
\left [
\begin{array}{cc}
 s+\frac{R}{L} & \frac{1}{L}\\
\frac{-1}{C}& s\\
\end{array}
\right ]^{-1}\cdot
\left [ 
\begin{array}{c}
 \frac{1}{L}\\
0\\
\end{array}
\right ]=
\left [ 
\begin{array}{c c}
0  & 1\\
\end{array}
\right ]\cdot
\frac{\left [
\begin{array}{cc}
 s& \frac{-1}{L}\\
\frac{1}{C}& s+\frac{R}{L}\\
\end{array}
\right ]}{s^2 +\frac{sR}{L}+\frac{1}{LC}}
\cdot
\left [ 
\begin{array}{c}
 \frac{1}{L}\\
0\\
\end{array}
\right ]\\
&=&
\frac{
\left [ 
\begin{array}{c c}
\frac{1}{C} & s+\frac{R}{L}\\
\end{array}
\right ]\cdot\left [ 
\begin{array}{c}
 \frac{1}{L}\\
0\\
\end{array}
\right ]
}{s^2 +\frac{sR}{L}+\frac{1}{LC}}
=\frac{\frac{1}{LC}}{s^2 +\frac{sR}{L}+\frac{1}{LC}}=
\frac{1}{s^2LC +sRC+1}.
\end{eqnarray*}
\end{enumerate}
Somit sehen wir, dass die {\bf
  Differentialgleichung}\index{Differentialgleichung}, die {\bf
  Impedanzrechnung}\index{Impedanzrechnung} sowie die {\bf
  Zustandsraumdarstellung}\index{Zustandsraumdarstellung} in Zeit- und
Frequenzbereich \"aquivalent sind.  
\bsp{}\label{ZRD_BSP_016}\\
\nit Die Umrechnung von der
{\bf Zustandsraumdarstellung} zur {\bf \"Ubertragungsfunktion} von
{\bf LTI-Systemen}\index{SISO}\index{LTI} (SISO und SIMO)\index{SIMO}
kann mit dem \mb {\tt tf2ss}\index{tf2ss@{\tt tf2ss}} und {\tt ss2tf}\index{ss2tf@{\tt ss2tf}} bewerkstelligt werden.\\ F\"ur Beispiel~\ref{kapitel_ZVD}.\ref{ZVD_BSP_04} mit $R=3$, $L=2$ und $C=1$ in \matlogo~ umgesetzt erh\"alt man
folgende Befehlseingabe und \matlogo-Ausgaben:\\
\begin{small}
\begin{verbatim}
>> [zaehler,nenner]=ss2tf([-3/2 -1/2;1 0],[1/2;0],[0 1],[0],1)
zaehler =         0         0    0.5000
nenner  =    1.0000    1.5000    0.5000
\end{verbatim}
\end{small}
was  $H(s)=\frac{\frac{1}{2}}{s^2+\frac{3}{2}s+\frac{1}{2}}=\frac{1}{2s^2+3s+1}$ ist und dem Resultat von Beispiel~\ref{kapitel_ZVD}.\ref{ZVD_BSP_04} entspricht.
%\bsp{Zustandsraumdarstellung mit Anfangsbedingungen alle Null \cite{HSU:95}}

%%\newpage
\subsection{\"Ubertragungsmatrix und \"Ubertragungsfunktion}\index{Ubertragungsfunktion@{\"Ubertragungsfunktion}}
\index{Ubertragungsmatrix@{\"Ubertragungsmatrix}}
Im allgemeinen Fall besitzt ein LTI-System mehrere Eing\"ange \vector{u}$(t)$ und Ausg\"ange \vector{y}$(t)$ (MIMO-System)\index{MIMO}. Das {\bf Mehrgr\"ossensystem}\index{Mehrgrossensystem@{Mehrgr\"ossensystem}} in Matrizenform ist \cite{UNB:81}% s.91
\begin{equation}
\underline{Y}(s)=\boldsymbol{H(s)}\underline{U}(s).\label{ZVD_FORM_MGSys}
\end{equation}
In Formel~\ref{ZVD_FORM_MGSys} ist \mat{H(s)} die {\bf \"Ubertragungsmatrix}.

%\newpage
%\vspace*{-9mm}
\bsp{}
\begin{figure}[!htb]
\vspace*{-4mm}\begin{center}
  \bild{/zvd/ZVD003.ps,width=0.78}\vspace*{-3mm}\caption{Beispiel mit zwei Eingangsgr\"ossen und zwei Ausgangsgr\"ossen.}\label{ZVD_ABB_MEHRG}
\end{center}
\vspace*{-6mm}
\end{figure}\\
Das Mehrgr\"ossensystem von Abb.~\ref{ZVD_ABB_MEHRG} wird beschrieben durch\\
\begin{eqnarray*}
 Y_1(s) &=& H_{11}(s) U_1(s) + H_{12}(s) U_2(s),\\
 Y_2(s) &=& H_{21}(s) U_1(s) + H_{22}(s) U_2(s),\\
\end{eqnarray*}\\
und die Matrizenschreibweise ist
\begin{equation*}
\underbrace{\left [
\begin{array}{c}
 Y_1(s)\\
Y_2(s)\\
\end{array}
\right ]}_{\underline{Y}(s)}=
\underbrace{
\left [
\begin{array}{cc}
 H_{11}(s) & H_{12}(s)\\
 H_{21}(s) & H_{22}(s)\\
\end{array}
\right ]}_{\boldsymbol{H(s)}}
\underbrace{\left [
\begin{array}{c}
 U_1(s)\\
U_2(s)\\
\end{array}
\right ]}_{\underline{U}(s)}.
\end{equation*}
Die \"Ubertragungsmatrix\index{Ubertragungsmatrix@{\"Ubertragungsmatrix}} ${\boldsymbol{H(s)}}$ hat in diesem Beispiel die Gr"osse $2\times 2$ (die Gr"osse der \"Ubertragungsmatrix ist identisch zu der Gr"osse der Durchgangsmatrix ${\boldsymbol{D}}$).\index{Durchgangsmatrix} Im Speziellen ist bei einem einzigen Eingangssignal $u(t)$ und einem einzigen Ausgangssignal $y(t)$, \mat{H(s)}$=H(s)$, was die {\bf UTF}\index{UTF} eines {\bf SISO-LTI-Systemes}\index{SISO}\index{LTI} ist.

\subsection{Bestimmung der Zustandsraumdarstellung aus der allgemeinen UTF}
Die  allgemeine Differentialgleichung\index{Differentialgleichung} der Form
\begin{equation*}
\frac{\partial^{n} y}{\partial t^{n}} + a_{n-1}\frac{\partial^{n-1} y}{\partial t^{n-1}} + \cdots +
a_{1}\frac{\partial y}{dt} + a_{0} y=
b_{m}\frac{\partial^{m} u}{\partial t^{m}} + b_{m-1}\frac{\partial^{m-1} u}{\partial t^{m-1}} + \cdots +
b_{1}\frac{\partial u}{\partial t} + b_{0} u
\end{equation*} \\~\\
ergibt mit der Laplace-Transformation\index{Laplace} (mit $m \leq n$) \\~\\
\myboxx{
\begin{equation*}
H(s)=\frac{Y(s)}{U(s)}=\frac{b_{m} s^{m} + b_{m-1} s^{m-1} +\cdots+b_{1} s 
+ b_{0}}{s^{n} + a_{n-1} s^{n-1} + \cdots + a_{1} s + a_{0}}.
\end{equation*}} \\~\\
Es kann gezeigt werden, dass diese UTF mit verschiedenen
Zustandsraumdarstellungen abgebildet werden kann \cite{GIR:RAB:STE:05,
  HSU:95, UNB:89}. Wir beschr\"anken uns meistens auf die {\bf
  Regelungsnormalform}\index{Regelungsnormalform} und auf die {\bf
  Beobachtungsnormalform}\index{Beobachtungsnormalform} \cite{UNB:89}.

\subsubsection{Regelungsnormalform}\index{Regelungsnormalform}
Mit Hilfe der Signalflussdiagramme\index{Signalflussdiagramm} (Kapitel~\ref{KAP_SFD}) k"onnen wir das Blockdiagramm\index{Blockdiagramm} der Regelungsnormalform direkt aus der UTF aufstellen. Dies gilt auch f"ur die Beobachtungsnormalform\index{Beobachtungsnormalform} und die Jordan'sche Normalform (Diagonalform)\index{Diagonalform}. 
\begin{figure}[!htb]
\vspace*{-4mm}\begin{center}
  \bild{/zvd/ZVD009.ps,width=1}\vspace*{-3mm}\caption{Blockdiagramm der Regelungsnormalform der UTF $H(s)=\frac{b_{m} s^{m} + b_{m-1} s^{m-1} +\cdots+b_{1} s 
+ b_{0}}{s^{n} + a_{n-1} s^{n-1} + \cdots + a_{1} s + a_{0}}$ mit $m<n$.}
\end{center}
\vspace*{-6mm}
\end{figure}\\
\nit Die Regelungsnormalform ist ($m=n$): 
\begin{equation*}
\left [ 
\begin{array}{c}
\dot{x}_1(t)\\
\dot{x}_2(t)\\
\vdots\\
\dot{x}_{n-1}(t)\\
\dot{x}_{n}(t)\\
\end{array}
\right ] =
\left [ 
\begin{array}{c c c c c}
0 & 1 & 0 & \ldots & 0\\
0 & 0 & 1 & \ldots & 0\\
\vdots & \vdots & \vdots & \ddots & \vdots\\
0 & 0 & 0 & \ldots & 1\\
-a_0 & -a_1 & -a_2 & \ldots & -a_{n-1}\\
\end{array}
\right ]\cdot
\left [ 
\begin{array}{c}
x_1(t)\\
x_2(t)\\
\vdots \\
x_{n-1}(t)\\
x_{n}(t)\\
\end{array}
\right ]+
\left [ 
\begin{array}{c}
0 \\
0\\
\vdots\\
0\\
1\\
\end{array}
\right ]\cdot
u(t),
\end{equation*}
\begin{equation*}
y(t) = 
\left [ 
\begin{array}{c c c c}
b_0-a_0b_n & b_1-a_1b_n & \ldots & b_{n-1}-a_{n-1}b_n\\
\end{array}
\right ] \cdot
\left [ 
\begin{array}{c}
x_1(t)\\
x_2(t)\\
\vdots \\
x_{n-1}(t)\\
x_{n}(t)\\
\end{array}
\right ]+
\left [ 
\begin{array}{c}
b_n \\
\end{array}
\right ] \cdot
u(t).
\end{equation*}
\nit Die Regelungsnormalform wird auch Steuerungsnormalform\index{Steuerungsnormalform}, 1. Standardform\index{1. Standardform} oder Frobenius-Form\index{Frobenius-Form} genannt \cite{LUT:WEN:05}.

\newpage
\subsubsection{Beobachtungsnormalform}\index{Beobachtungsnormalform}
\begin{figure}[!htb]
  \vspace*{-4mm}\begin{center}
    \bild{/zvd/ZVD010.ps,width=1}\vspace*{-3mm}\caption{Blockdiagramm der Beobachtungsnormalform der UTF $H(s)=\frac{b_{m} s^{m} + b_{m-1} s^{m-1} +\cdots+b_{1} s 
+ b_{0}}{s^{n} + a_{n-1} s^{n-1} + \cdots + a_{1} s + a_{0}}$ mit $m<n$.}
\end{center}
\vspace*{-6mm}
\end{figure}
\nit Die Beobachtungsnormalform ist ($m=n$): 
\begin{eqnarray*}
\left [ 
\begin{array}{c}
\dot{x}_1(t)\\
\dot{x}_2(t)\\
\vdots\\
\dot{x}_{n-1}(t)\\
\dot{x}_n(t)\\
\end{array}
\right ] &=&
\left [ 
\begin{array}{c c c c c}
0 & 0 & 0 & \ldots & -a_0\\
1 & 0 & 0 & \ldots & -a_1\\
0 & 1 & 0 & \ldots & -a_2\\
\vdots & \vdots &  \ddots & 0 & \vdots\\
0 & 0 & \ldots & 1 & -a_{n-1}\\

\end{array}
\right ]\cdot
\left [ 
\begin{array}{c}
x_1(t)\\
x_2(t)\\
\vdots \\
x_{n-1}(t)\\
x_{n}(t)\\
\end{array}
\right ]+
\left [ 
\begin{array}{c}
b_0-a_0b_n \\
b_1-a_1b_n\\
b_2-a_2b_n\\
\vdots\\
b_{n-1}-a_{n-1}b_n\\
\end{array}
\right ]\cdot
u(t),\\
y(t) &= &
\left [ 
\begin{array}{c c c c}
0 & 0 & \ldots & 1\\
\end{array}
\right ] \cdot
\left [ 
\begin{array}{c}
x_1(t)\\
x_2(t)\\
\vdots \\
x_{n-1}(t)\\
x_{n}(t)\\
\end{array}
\right ]+
\left [ 
\begin{array}{c}
b_n \\
\end{array}
\right ] \cdot
u(t).
\end{eqnarray*}

%\newpage
%\vspace*{-9mm}
\bsp{Das Blockdiagramm\index{Blockdiagramm} der Regelungsnormalform der UTF $H(s)=\frac{\frac{1}{LC}}{s^2+s\frac{R}{L}+\frac{1}{LC}}$ von Abb.~\ref{ZVD_ABB_003} ist:}
\begin{figure}[!htb]
  \vspace*{-4mm}\begin{center}
    \bild{/zvd/ZVD011.ps,width=0.63}\caption{Blockdiagramm} \vspace*{-3mm}
\end{center}
\vspace*{-6mm}
\end{figure}

%\newpage
\subsubsection{Jordan'sche Normalform}\index{Jordansche Normalform@Jordan'sche Normalform}
Die {\bf Jordan'sche Normalform (Diagonalform)}\index{Diagonalform} eignet sich besonders gut zur {\bf Stabilit"atsanalyse}\index{Stabilitatsanalyse@Stabilit\"atsanalyse} \cite{UNB:89}. Eine reine Diagonalform kann aber nur bei Systemen mit einfachen Polen in der UTF\index{UTF} aufgestellt werden. Dazu wird die UTF\index{UTF} mittels 
Partialbruchzerlegung\index{Partialbruchzerlegung} dargestellt:\\
\begin{equation*}
H(s)=\frac{Y(s)}{U(s)}=\frac{b_{n} s^{n} + b_{n-1} s^{n-1} +\cdots+b_{1} s + b_{0}}{s^{n} + a_{n-1} s^{n-1} + \cdots + a_{1} s + a_{0}}= \alpha_0 + \frac{\alpha_1}{s-p_1}+ \frac{\alpha_2}{s-p_2}+\ldots  + \frac{\alpha_n}{s-p_n} .
\end{equation*}


\nit Die Diagonalform ist ($m=n$): 
\begin{eqnarray*}
\left [ 
\begin{array}{c}
\dot{x}_1(t)\\
\dot{x}_2(t)\\
\vdots\\
\dot{x}_{n-1}(t)\\
\dot{x}_n(t)\\
\end{array}
\right ] &=&
\overbrace{\left [ 
\begin{array}{c c c c c}
p_1 & 0 & 0 & \ldots & 0\\
0 & p_2 & 0 & \ldots & 0\\
0 & 0 & p_3 & \ldots & 0\\
\vdots & \vdots &  \ddots & p_{n-1} & \vdots\\
0 & 0 & \ldots & 0 & p_n\\
\end{array}
\right ]}^{\boldsymbol{A}} \cdot
\left [ 
\begin{array}{c}
x_1(t)\\
x_2(t)\\
\vdots \\
x_{n-1}(t)\\
x_n(t)\\
\end{array}
\right ]+
\overbrace{\left [ 
\begin{array}{c}
1\\
1\\
1\\
\vdots\\
1\\
\end{array}
\right ] }^{\boldsymbol{B}} \cdot
u(t),\\
y(t) &= &
\underbrace{\left [ 
\begin{array}{c c c c}
\alpha_1 & \alpha_2 & \ldots & \alpha_n\\
\end{array}
\right ] }_{\boldsymbol{C}}  \cdot
\left [ 
\begin{array}{c}
x_1(t)\\
x_2(t)\\
\vdots \\
x_{n-1}(t)\\
x_n(t)\\
\end{array}
\right ]+
\underbrace{\left [ 
\begin{array}{c}
\alpha_0 \\
\end{array}
\right ]}_{\boldsymbol{D}} \cdot
u(t).
\end{eqnarray*}
\nit Eine weitere Darstellungsform der Diagonalform ergibt sich durch vertauschen von $\boldsymbol{B}$ und $\boldsymbol{C}$.
\bsp{UTF $H(s)=\frac{3s+7}{(s+2)(s+1)(s+5)}= \frac{1}{s+1}+\frac{\frac{-1}{3}}{s+2}+\frac{\frac{-2}{3}}{s+5}$}
Die Jordan'sche Normalform (mit $\boldsymbol{B}$ und $\boldsymbol{C}$ vertauscht) ist dann:\\
\begin{eqnarray*}
\left [ 
\begin{array}{c}
\dot{x}_1(t)\\
\dot{x}_2(t)\\
\dot{x}_3(t)\\
\end{array}
\right ] &=&
\left [ 
\begin{array}{c c c}
-1 & 0 & 0\\
0 & -2 & 0\\
0 & 0 & -5\\
\end{array}
\right ]\cdot
\left [ 
\begin{array}{c}
x_1(t)\\
x_2(t)\\
x_3(t)\\
\end{array}
\right ]+
\left [ 
\begin{array}{c}
1\\
\frac{-1}{3}\\
\frac{-2}{3}\\
\end{array}
\right ]\cdot
u(t),\\
y(t) &= &
\left [ 
\begin{array}{c c c }
1 &  1 & 1\\
\end{array}
\right ] \cdot
\left [ 
\begin{array}{c}
x_1(t)\\
x_2(t)\\
x_3(t)\\
\end{array}
\right ]+
\left [ 
\begin{array}{c}
$0$ \\
\end{array}
\right ] \cdot
u(t).
\end{eqnarray*}

\begin{figure}[!htb]
\begin{center}
\vspace*{-4mm}\bild{/zvd/ZVD_jordan1.eps,width=0.71}\caption{Jordan'sche Normalform von $H(s)=\frac{3s+7}{(s+2)(s+1)(s+5)}$}
\end{center}% MATLAB print -s -depsc2 xxx.eps % letter format
\end{figure}


%\newpage
\section{Stabilit"at}\index{Stabilitat@Stabili\"at}
Ein LTI-System\index{LTI-System} ist asymptotisch stabil\index{asymptotisch stabil}, wenn alle Eigenwerte\index{Eigenwert} der Systemmatrix\index{Systemmatrix\} $\boldsymbol{A}$ einen negativen Realteil\index{Realteil!negativ} besitzen\footnote{Wenn das
charakteristische Polynom nur Eigenwerte mit negativen Realteilen hat.}\index{charakteristisches Polynom}\index{Polynom!charakteristisches}:\\~\\ 
\myboxx{\begin{equation}
\left | \lambda\boldsymbol{I} - \boldsymbol{A} \right |   =0 \rightarrow \forall~\lambda \quad\Re \{\lambda\}<0
\end{equation}}\\~\\

\nit Umgekehrt gilt diese Aussage nicht, d.h., eine asymptotisch stabiles LTI-System bedeutet nicht, dass alle Eigenwerte der Systemmatrix $\boldsymbol{A}$ des Systems einen negativen Realteil\index{Realteil!negativ} besitzen.

\bsp{Ein kausales LTI-System ist gegeben durch $H(s)=\frac{1}{s^2+2s+1}$}
Z.B. ist die Regelungsnormalform\index{Regelungsnormalform} von $H(s)$ ist:\\
\begin{eqnarray*}
\left [ 
\begin{array}{c}
\dot{x}_1(t)\\
\dot{x}_2(t)\\
\end{array}
\right ] &=&
\left [ 
\begin{array}{c c}
$0$ & $1$\\
$-1$ & $-2$\\
\end{array}
\right ]\cdot
\left [ 
\begin{array}{c}
x_1(t)\\
x_2(t)\\
\end{array}
\right ]+
\left [ 
\begin{array}{c}
$0$\\
$1$\\
\end{array}
\right ]\cdot
u(t),\\
y(t) &= &
\left [ 
\begin{array}{c c}
$1$ & $0$\\
\end{array}
\right ] \cdot
\left [ 
\begin{array}{c}
x_1(t)\\
x_2(t)\\
\end{array}
\right ]+
\left [ 
\begin{array}{c}
$0$ \\
\end{array}
\right ] \cdot
u(t).
\end{eqnarray*}

$\left | \lambda\boldsymbol{I} - \boldsymbol{A} \right |   =0 \rightarrow \left [
\begin{array}{cc}
  \lambda -0 & -1\\
 1 & \lambda+2 \\
\end{array}
\right ]=\lambda(\lambda+2)+1=0$$\rightarrow\lambda_1=-1;~\lambda_2=-1$\\~\\
\nit Alle Eigenwerte der Systemmatrix\index{Systemmatrix\} $\boldsymbol{A}$ haben einen negativen Realteil, d.h. die inneren Systemzust"ande sind stabil und somit ist auch das LTI-System asymptotisch stabil.\\

%\newpage
%\vspace*{-9mm}
\bsp{LTI-System}
\begin{figure}[!htb]
\vspace*{-4mm}
\begin{center}\bild{/zvd/ZVD_stabilitaet.eps,width=0.991}\caption{Blockdiagramm eines LTI-System}\end{center}% MATLAB print -s -depsc2 xxx.eps % letter format
\end{figure}
\begin{eqnarray*}
\left [ 
\begin{array}{c}
\dot{x}_1(t)\\
\dot{x}_2(t)\\
\end{array}
\right ] &=&
\left [ 
\begin{array}{c c}
0 & 1\\
2 & 1\\
\end{array}
\right ]\cdot
\left [ 
\begin{array}{c}
x_1(t)\\
x_2(t)\\
\end{array}
\right ]+
\left [ 
\begin{array}{c}
1\\
-1\\
\end{array}
\right ]\cdot
u(t),\\
y(t) &= &
\left [ 
\begin{array}{c c}
1 & -1\\
\end{array}
\right ] \cdot
\left [ 
\begin{array}{c}
x_1(t)\\
x_2(t)\\
\end{array}
\right ]+
\left [ 
\begin{array}{c}
$0$ \\
\end{array}
\right ] \cdot
u(t).
\end{eqnarray*}
$\left | \lambda\boldsymbol{I} - \boldsymbol{A} \right |   =0 \rightarrow \left [
\begin{array}{cc}
  \lambda -0 & -1\\
 -2 & \lambda-1 \\
\end{array}
\right ]=\lambda^2-\lambda-2=0$$\rightarrow\lambda_1=-1;~\lambda_2=2$\\~\\
\nit Ein Eigenwert der Systemmatrix\index{Systemmatrix\} $\boldsymbol{A}$ hat einen positiven Realteil ($\lambda_2=2$), d.h. dieser innere
 Systemzustand ist nicht stabil. \\
$H(s)=\boldsymbol{C}\klam{s\boldsymbol{I}-\boldsymbol{A}}^{-1}\boldsymbol{B}+\boldsymbol{D}=\frac{2}{s+1}$.
Das bedeutet, dass das gesamte System asymptotisch stabil ist. Achtung! Das gesamte System ist asymptotisch stabil, weil der innere, instabile Pol (bei $+2$) durch die "aussere Beschaltung ($\boldsymbol{B}$, $\boldsymbol{C}$ und $\boldsymbol{D}$) eliminiert wird\index{Polelimination}. Das erreicht man aber nur bei {\bf absoluter Genauigkeit} der Elimination\index{Elimination} und ist in der Praxis nicht m"oglich.\index{Genauigkeit!absolut}

%\newpage

\section{Beobachtbarkeit und Steuerbarkeit}\index{Beobachtbarkeit}
Das dynamische Verhalten eines {\bf analogen LTI-Systems} wird durch
die {\bf Zustandsgr\"ossen} und die {\bf Eingangssignale}
vollst\"andig beschrieben \cite{UNB:89}. Die Begriffe {\bf
  Steuerbarkeit}\index{Steuerbarkeit} und {\bf
  Beobachtbarkeit}\index{Beobachtbarkeit} lassen sich n\"aherungsweise erkl\"aren  \cite{UNB:89}:
\begin{enumerate}
 \item[] {\bf Beobachtbarkeit}\\ Gibt es {\bf Zust\"ande}\index{Zustand} \vector{x}$(t)$ die keinen {\bf Einfluss} auf die {\bf Ausg"ange}\index{Ausgang} \vector{y}$(t)$ haben? Wenn ja, kann man aus dem Verhalten von  \vector{y}$(t)$ nicht auf die {\bf Zust\"ande} \vector{x}$(t)$ schliessen. $\rightarrow$ Das System ist {\bf nicht beobachtbar!}
 \item[] {\bf Steuerbarkeit}\\ Gibt es {\bf Zust\"ande}\index{Zustand} von \vector{x}$(t)$ die nicht von den {\bf Eing"angen}\index{Eingang} \vector{u}$(t)$ beeinflusst werden? Wenn ja, dann ist das System {\bf nicht steuerbar!}
\end{enumerate}





\subsection{Steuerbarkeit}\index{Steuerbarkeit}
Kann eine Zustandsraumdarstellung ({\boldmath $A$, $B$, $C$} und  {\boldmath $D$}) mit einer {\bf Transformationsmatrix}\index{Transformationsmatrix}
$\boldsymbol{T}$ diagonalisiert werden (\mat{T^{-1}T=I}), z.B. $\boldsymbol{A}_\text{diagonal}=\boldsymbol{\hat{A}}=\boldsymbol{T^{-1}AT}$, wobei $\boldsymbol{\hat{B}}=\boldsymbol{T^{-1}B}$, $\boldsymbol{\hat{C}}=\boldsymbol{CT}$ und  $\boldsymbol{\hat{D}}=\boldsymbol{D}$, \\
\begin{figure}[!htb]
\vspace*{-4mm}\begin{center}
  \bild{/zvd/ZVD002.ps,width=0.95}\vspace*{-3mm}\caption{Ist $\boldsymbol{A}_\text{diagonal}=\boldsymbol{\hat{A}}$, so nennt man die Zustandsraumdarstellung auch {\bf Parallelform} \cite{GIR:RAB:STE:05} und entspricht somit der Jordan'schen Normalform (Diagonalform).}\index{Parallelform}\index{Diagonalform}
\end{center}
\vspace*{-6mm}
\end{figure}\\
so kann die Steuerbarkeit und Beobachtbarkeit von {\bf Eingr\"ossensystemen}\index{Eingrossensystemen@{Eingr\"ossensystemen}} ($\boldsymbol{B}$ ist dann eine ($n\times 1$)-Matrix und \vector{u}$(t)$=$u(t)$) wie folgt definiert werden \cite{GIR:RAB:STE:05, UNB:89}:\\~\\
\myboxx{Ein {\bf
    Eingr\"ossensystem}\index{Eingrossensystem@{Eingr\"ossensystem}}
  mit {\bf einfachen Eigenwerten}\index{Eigenwert!einfach} ist genau dann
  {\bf vollst\"andig
    steuerbar}\index{steuerbar!vollstandig@{vollst\"andig}}, wenn nach
  der Transformation auf {\bf Parallelform}\index{Parallelform}
  ($\boldsymbol{A}_\text{diagonal}=\boldsymbol{\hat{A}}=\boldsymbol{T^{-1}AT}$), alle
  Elemente von $\boldsymbol{\hat{B}}=\boldsymbol{T^{-1}B}$ nicht Null sind.
}\\~\\
Einfache Eigenwerte sind vorhanden, wenn
    sich $\boldsymbol{A}$ diagonalisieren l\"asst. F\"ur  {\bf Mehrgr\"ossensystemen}\index{Mehrgrossensystem@{Mehrgr\"ossensystem}} lautet die Steuerbarkeit \cite{UNB:89}:\\~\\
\myboxx{Ein {\bf
    Mehrgr\"ossensystem}\index{Mehrgrossensystem@{Mehrgr\"ossensystem}}
  ($m>1$) mit {\bf einfachen Eigenwerten}\index{Eigenwert!einfach} ist
  genau dann {\bf vollst\"andig
    steuerbar}\index{steuerbar!vollstandig@{vollst\"andig}}, wenn nach
  der Transformation
  ($\boldsymbol{A}_\text{diagonal}=\boldsymbol{T^{-1}AT}$), in
  jeder Zeile von $\boldsymbol{\hat{B}}=\boldsymbol{T^{-1}B}$ mindestens
  ein Element nicht Null ist.
}\\~\\
Allgemein ist die Steuerbarkeit definiert als \cite{UNB:89}:\\~\\
\myboxx{Ein {\bf LTI-System}\index{LTI-System} ist {\bf vollst\"andig
    steuerbar}, wenn f\"ur jeden {\bf
    Anfangszustand}\index{Anfangszustand} \vector{x}$(t_0)$ eine {\bf
    Steuerfunktion}\index{Steuerfunktion} \vector{u}$(t)$ vorhanden
  ist, die das System innerhalb einer {\bf endlichen} Zeitspanne
  $t_0\leq t\leq t_1$ in den {\bf Endzustand}\index{Endzustand}
  \vector{x}$(t_1 )$ bringt.
}\\
\subsubsection{Steuerbarkeitsmatrix}\index{Steuerbarkeitsmatrix}
Die {\bf Steuerbarkeit}\index{Steuerbarkeit} l"asst sich auch von Systemen mit {\bf mehrfachen} Eigenwerten\index{Eigenwert!mehrfach} berechnen. Dazu wird der {\bf Rang}\index{Rang} der Steuerbarkeitsmatrix bestimmt \cite{UNB:89}. Ist der der Rang
kleiner als die Ordnung $n$ des Systems, so ist das System nicht {\bf vollst"andig
steuerbar}. Die Steuerbarkeitsmatrix (f"ur allgemeine Systeme) ($n\times n$-Matrix) ist \cite{LUT:WEN:05, UNB:89}:\\~\\
\myboxx{\begin{equation}
Q_{Steuerbarkeit}=\left [ \boldsymbol{B~~AB~~ A^2B~\ldots~ A^{n-1}B}  \right ]
\end{equation}}\\~\\
\nit wobei $ \boldsymbol{A} $ die Systemmatrix\index{Systemmatrix} ($n\times n$-Matrix) und $ \boldsymbol{B}$ die Steuermatrix\index{Steuermatrix} sind. Ist die
Determinante\index{Determinante} von $Q_{Steuerbarkeit}$ ungleich Null ($\left|  Q_{Steuerbarkeit}\right| \neq 0$ ), so ist das System {\bf vollst"andig steuerbar}\index{steuerbar!vollstandig@vollst\"andig}. Ein vollst"andig steuerbares System l"asst sich in der Steuerungsnormalform\index{Steuerungsnormalform} (Regelungsnormalform)\index{Regelungsnormalform} darstellen \cite{LUT:WEN:05}.


\subsection{Beobachtbarkeit}\index{Beobachtbarkeit} 
Allgemein ist die Beobachtbarkeit definiert als \cite{UNB:89}:\\~\\
\myboxx{Ein {\bf LTI-System}\index{LTI-System} ist {\bf vollst\"andig
    beobachtbar}, wenn bei bekannter \"ausserer Beeinflussung
  \mat{B}\vector{u}$(t)$ und den bekannten Matrizen {\boldmath $A$}
  und {\boldmath $C$} aus dem Ausgangsvektor \vector{y}$(t)$ \"uber
  ein endliches Zeitintervall $t_0\leq t\leq t_1$ den Anfangszustand
  \vector{x}$(t_0)$ eindeutig bestimmen kann.
}\\~\\
Wiederum kann nach der Diagonalisierung von \mat{A} mit \mat{T}, $\boldsymbol{A}_\text{diagonal}=\boldsymbol{\hat{A}}=\boldsymbol{T^{-1}AT}$,  die {\bf Beobachtbarkeit}\index{Beobachtbarkeit} definiert werden als \cite{GIR:RAB:STE:05}:\\~\\
\myboxx{Ein {\bf
    Eingr\"ossensystem}\index{Eingrossensystem@{Eingr\"ossensystem}}
  mit {\bf einfachen Eigenwerten}\index{Eigenwert!einfach} ist genau
  dann {\bf vollst\"andig
    beobachtbar}\index{beobachtbar!vollstandig@{vollst\"andig}}, wenn
  nach der Transformation auf {\bf Parallelform}\index{Parallelform}
  ($\boldsymbol{A}_\text{diagonal}=\boldsymbol{\hat{A}}=\boldsymbol{T^{-1}AT}$),
  alle Elemente von $\boldsymbol{\hat{C}}=\boldsymbol{CT}$ nicht Null
  sind.
}\\~\\
F\"ur  {\bf Mehrgr\"ossensystemen}\index{Mehrgrossensystem@{Mehrgr\"ossensystem}} lautet die Steuerbarkeit \cite{UNB:89}:\\~\\
\myboxx{Ein {\bf
    Mehrgr\"ossensystem}\index{Mehrgrossensystem@{Mehrgr\"ossensystem}}
  ($m>1$) mit {\bf einfachen Eigenwerten}\index{Eigenwert!einfach} ist
  genau dann {\bf vollst\"andig
     beobachtbar}\index{beobachtbar!vollstandig@{vollst\"andig}}, wenn nach
  der Transformation
  ($\boldsymbol{A}_\text{diagonal}=\boldsymbol{T^{-1}AT}$), in
  jeder Spalte von $\boldsymbol{\hat{C}}=\boldsymbol{CT}$ mindestens
  ein Element nicht Null ist.
}\\

\subsubsection{Beobachtbarkeitsmatrix}\index{Beobachtbarkeitsmatrix}
Analog zur Steuerbarkeitsmatrix l"asst sich von einem System die {\bf Beobachtbarkeitsmatrix} (f"ur allgemeine Systeme) ($n\times n$-Matrix) aufstellen \cite{GIR:RAB:STE:05, LUT:WEN:05, UNB:89}:\\~\\
\myboxx{\begin{equation}
Q_{Beobachtbarkeit}=\left [ \boldsymbol{
\begin{array}{c}
 C\\
 CA\\
CA^2\\
\vdots \\
CA^{n-1}\\
\end{array}}\right ]
\end{equation}}\\~\\
\nit wobei $ \boldsymbol{A} $ die Systemmatrix\index{Systemmatrix} ($n\times n$-Matrix) und $ \boldsymbol{C}$ die Beobachtungsmatrix\index{Beobachtungsmatrix} sind. Ist die
Determinante\index{Determinante} von $Q_{Beobachtbarkeit}$ nicht Null ($\left | Q_{Beobachtbarkeit}\right| \neq 0$ ), so ist das System {\bf vollst"andig beobachtbar}\index{beobachtbar!vollstandig@{vollst\"andig}}. Ein vollst"andig beobachtbares System l"asst sich in der Beobachtungssnormalform\index{Beobachtungsnormalform}  darstellen \cite{LUT:WEN:05}.

\subsection{Bemerkung zur Beobachtbarkeit und Steuerbarkeit} 
Aus den Definitionen der {\bf Steuerbarkeit}\index{Steuerbarkeit} und
der {\bf Beobachtbarkeit}\index{Beobachtbarkeit} erkennt man
\cite{UNB:89}, dass nur bei {\bf vollst\"andig steuerbaren} und {\bf
   vollst\"andig beobachtbaren} Systemen die {\bf Zustandsraumdarstellung} und die
{\bf
  \"Ubertragungsmatrix}\index{Ubertragungsmatrix@{\"Ubertragungsmatrix}}
\mat{H(s)} gleichwertig und ineinander \"uberf\"uhrbar sind
\cite{UNB:89}.
%\newpage
%\vspace*{-8mm}
\bsp{Beobachtbarkeit und Steuerbarkeit \cite{HSU:95}}
Das System, beschrieben durch
\begin{eqnarray*}% eigenvektoren unb89 s.42
 y(t) &=& x_1(t) -x_2(t),\\
  \dot{x}_1(t)&=&u(t)+x_1(t)+2x_2(t),\\
  \dot{x}_2(t)&=&u(t)+3x_2(t),
\end{eqnarray*}
\vspace*{-3mm}ergibt die {\bf Matrizen}\index{Matrizen}:
\begin{equation*}
\boldsymbol{A}=\left [
\begin{array}{cc}
 1 & 2\\
 0 & 3\\
\end{array}
\right ],~
\boldsymbol{B}=\left [
\begin{array}{c}
 1 \\
 1\\
\end{array}
\right ],~
\boldsymbol{C}=\left [
\begin{array}{cc}
 1 & -1\\
\end{array}
\right ]\quad
\text{und}\quad
\boldsymbol{D}=\left [
\begin{array}{c}
 0\\
\end{array}
\right ].\end{equation*}\\ 
Mit der {\bf Transformationsmatrix}\index{Transformationsmatrix} \mat{T} diagonalisieren wir \mat{A}, d.h., $\boldsymbol{A}_\text{diagonal}=\boldsymbol{T^{-1}AT}$. Z.B. ist mit \\
\begin{equation*}
\boldsymbol{T}=\left [
\begin{array}{cc}
 1 & 1\\
 0 & 1\\
\end{array}
\right ]\quad\text{und}\quad
\boldsymbol{T}^{-1}=\left [
\begin{array}{cc}
 1 & -1\\
 0 & 1\\
\end{array}
\right ]\quad\rightarrow\quad\boldsymbol{A}_\text{diagonal}=\left [
\begin{array}{cc}
 1 & 0\\
 0 & 3\\
\end{array}
\right ]
.\end{equation*}
\begin{eqnarray*}
\boldsymbol{\hat{B}}&=&\boldsymbol{T^{-1}B}=\left [
\begin{array}{cc}
 1 & -1\\
 0 & 1\\
\end{array}
\right ]\cdot\left [
\begin{array}{c}
 1 \\
 1\\
\end{array}
\right ]=\left [
\begin{array}{c}
 0 \\
 1\\
\end{array}
\right ]\quad\text{und}\\ 
\boldsymbol{\hat{C}}&=&\boldsymbol{CT}=
\left [
\begin{array}{cc}
 1 & -1\\
\end{array}
\right ]\cdot\left [\begin{array}{cc}
 1 & 1\\
 0 & 1\\
\end{array}
\right ]=\left [
\begin{array}{cc}
 1 & 0 \\
\end{array}
\right ].
\end{eqnarray*}\\
Das System ist weder vollst"andig beobachtbar noch vollst"andig steuerbar!\\
Die Beobachtbarkeitsmatrix\index{Beobachtbarkeitsmatrix} und die Steuerbarkeitsmatrix\index{Steuerbarkeitsmatrix} sind:\\
\begin{equation*}
Q_{Beobachtbarkeit}=\left [ \boldsymbol{
\begin{array}{c}
 C\\
 CA\\
\end{array}}\right ]=\left [ 
\begin{array}{cc}
1 & -1 \\
1 & -1 \\
\end{array}\right ]\quad \text{und} \quad
Q_{Steuerbarkeit}=\left [ \boldsymbol{
\begin{array}{cc}
 B & AB
\end{array}}\right ]=\left [ 
\begin{array}{cc}
1 & 3 \\
1 & 3 \\
\end{array}\right ]
\end{equation*}
Der Rang\index{Rang} der Beobachtbarkeitsmatrix\index{Beobachtbarkeitsmatrix} und der 
Steuerbarkeitsmatrix\index{Steuerbarkeitsmatrix} ist jeweils 1, d.h., dass jeweils nur ein Zustand\index{Zustand} beobachtbar und steuerbar (da der Rang jeweils 1 ist). Die Determinanten der  Beobachtbarkeitsmatrix\index{Beobachtbarkeitsmatrix} und der Steuerbarkeitsmatrix sind jeweils Null. Mit der folgenden \mb\!\!sfolge erhalten wir die gleichen Resultate:\\
\begin{verbatim}
>>A=[1 2; 0 3];
>>B=[1;1];
>>C=[1 -1];
>>Q_S=ctrb(A,B) % Steuerbarkeitsmatrix
Q_S =
     1     3
     1     3
>> Q_B=obsv(A,C) % Beobachtbarkeitsmatrix
Q_B =
     1    -1
     1    -1
>>rank(Q_S)
ans =
     1
>> rank(Q_B)
ans =
     1
\end{verbatim}\index{rank@{\tt rank}}\index{ctrb@{\tt ctrb}}\index{obsv@{\tt obsv}}
 Die {\bf UTF}\index{UTF} ist\\
\begin{eqnarray*}
H(s)&=&\boldsymbol{C}\klam{s\boldsymbol{I}-\boldsymbol{A}}^{-1}\boldsymbol{B}+\boldsymbol{D}=
\left [
\begin{array}{cc}
 1 & -1\\
\end{array}
\right ]
\left [
\begin{array}{cc}
 s-1 & -2\\
0 & s-3\\
\end{array}
\right ]^{-1}
\left [
\begin{array}{c}
 1 \\
1\\
\end{array}
\right ]
+[0]=\\
&=&
\left [
\begin{array}{cc}
 1 & -1\\
\end{array}
\right ]
\frac{
\left [
\begin{array}{cc}
 s-3 & 2\\
0 & s-1\\
\end{array}
\right ]}{(s-1)(s-3)}
\left [
\begin{array}{c}
 1 \\
1\\
\end{array}
\right ]
=
\frac{
\left [
\begin{array}{cc}
 (s-3) & -(s-3)\\
\end{array}
\right ]\left [
\begin{array}{c}
 1 \\
1\\
\end{array}
\right ]}{(s-1)(s-3)}
=0.
\end{eqnarray*}
%\newpage
\section{Abschliessende Worte}
Die Beschreibungen dieses Kapitels beziehen sich auf {\bf analoge
  LTI-Systeme}\index{LTI-System!analog}, die sich durch
{\bf Netzwerke}\index{Netzwerke} mit {\bf konzentrierten
  Elementen}\index{konzentrierte Elemente} realisieren lassen und
daher durch {\bf gew\"ohnliche
  Differentialgleichungen}\index{Differentialgleichung!gewohnliche@{gew\"ohnliche}}
darstellbar sind.  Die {\bf
  Zustandsraumdarstellung}\index{Zustandsraumdarstellung} hat viele
Vorteile; sie gibt uns {\bf Information}\index{Information} \"uber das
``{\bf Innenleben}''\index{Innenleben} eines Systems; sie l\"asst uns
Systeme mit mehreren Ein- und Ausg\"angen in einer {\bf
  standartisierten Form} beschreiben; und man kann die {\bf
  Zustandsraumdarstellung} auf {\bf diskrete}, {\bf nichtlineare}, und
{\bf zeitvariante Systeme} erweitern
\cite{HSU:95}.\index{System!diskret}\index{System!nichtlinear}\index{System!zeitvariant}
Die hier aufgezeigten Techniken lassen sich auch mehrheitlich f\"ur
zeitdiskrete LTI-Systeme, welche mit {\bf
  Differenzengleichungen}\index{Differenzengleichung} f\"ur die
Zustandsraumdarstellung beschrieben werden, verwenden.\\~\\

\bsp{Wir berechnen die {\bf UTF}\index{UTF} \mat{H}$(s)$ des Systems auf verschiedene Arten und bestimmen die Beobachtbarkeit und Steuerbarkeit in diesem abschliessenden Beispiel, wobei wir der Einfachheit halber annehmen, dass $i_L(0)=0$ und $u_C(0)=0$ ist.}\label{ZRD_BSP_gross} % unb:89 seite 60
\begin{figure}[!htb] 
  \vspace*{-2mm}\begin{center}
    \bild{/zvd/ZVD008.ps,width=0.71}\vspace*{-3mm}\caption{$RLC$-Schaltung\index{RLC-Schaltung@{$RLC$-Schaltung}}
      mit {\bf idealem
        Einheitsverst\"arker}\index{Einheitsverstarker@{Einheitsverst\"arker}!ideal} mit
      $RC=1$~s und $L/R=1$~s.}\label{ZVD_ABB_last_example}
\end{center}
\vspace*{-6mm}
\end{figure}
\begin{enumerate}
 \item[]{\bf Impedanzrechnung}\index{Impedanzrechnung}
\begin{equation*}
  H(s)=\frac{Y(s)}{U(s)}=\frac{R+sL}{2R+sL}\cdot\frac{\frac{1}{sC}}{R+\frac{1}{sC}}=\frac{s\frac{L}{R}+1}{s\frac{L}{R}+2}\cdot\frac{1}{sCR+1}=\frac{1}{s+2}
\end{equation*}
 \item[]{\bf Differentialgleichung}\index{Differentialgleichung}
\begin{eqnarray}
 u(t) &=& CR\cdot\dot{y}(t)+y(t)+R\cdot i_L(t),\label{ZRD_FORM_BSP1}\\
 R\cdot i_L(t) +L\cdot \frac{\partial i_L(t)}{\partial t}&=& CR\cdot\dot{y}(t)+y(t),\label{ZRD_FORM_BSP2}\\
 R\klam{i_L(t) +\frac{\partial i_L(t)}{\partial t}}&=& \dot{y}(t)+y(t), \nonumber \\
 y(t) &=& R\cdot i_L(t),\nonumber \\
 u(t) &=& CR\cdot\dot{y}(t)+2\cdot y(t),\nonumber \\
 u(t) &=& \dot{y}(t)+2\cdot y(t)\FT U(s)=sY(s)+2Y(s)\nonumber \\
 &\rightarrow& H(s)=\frac{1}{s+2}.\nonumber
\end{eqnarray}

\item[]{\bf Zustandsraumdarstellung im
    Zeitbereich}\index{Zustandsraumdarstellung!im Zeitbereich}\\ Mit den
  Zust\"anden $x_1(t)\equiv i_L(t)$ und $x_2(t)\equiv u_c(t)\equiv
  y(t)$ folgt mit Formel~\ref{ZRD_FORM_BSP1} und \ref{ZRD_FORM_BSP2}
\begin{eqnarray*}
 u(t) &=& CR\cdot\dot{x}_2(t)+x_2(t)+R\cdot x_1(t)\\
 & & \rightarrow \dot{x}_2(t)=\frac{u(t)}{CR}-\frac{x_2(t)}{CR}-\frac{x_1(t)}{C} \\
R\cdot x_1(t) +L\cdot\dot{x}_1(t)&=& CR\cdot\dot{x}_2(t)+x_2(t) = u(t)-R\cdot x_1(t)\\
 & & \rightarrow \dot{x}_1(t)=\frac{u(t)}{L}-\frac{2\cdot x_1(t)\cdot R}{L} \\
\end{eqnarray*}
\begin{eqnarray*}
\rightarrow \left [
\begin{array}{c}
 \dot{x}_1(t) \\
 \dot{x}_2(t) \\
\end{array}
\right ]&=&
\left [
\begin{array}{cc}
 \frac{-2R}{L} & 0\\
\frac{-1}{C} & \frac{-1}{CR}\\
\end{array}
\right ]
\left [
\begin{array}{c}
 x_1(t) \\
 x_2(t) \\
\end{array}
\right ]
+
\left [
\begin{array}{c}
 \frac{1}{L} \\
 \frac{1}{CR} \\
\end{array}
\right ]u(t)\\
y(t)&=&[0 \quad 1]
\left [
\begin{array}{c}
 x_1(t) \\
 x_2(t) \\
\end{array}
\right ]
+[0] u(t).
\end{eqnarray*}
Somit sind:
\begin{equation*}
\boldsymbol{A}=\left [
\begin{array}{cc}
 \frac{-2R}{L} & 0\\
\frac{-1}{C} & \frac{-1}{CR}\\
\end{array}
\right ],\quad
\boldsymbol{B}=\left [
 \begin{array}{c}
  \frac{1}{L} \\
 \frac{1}{CR} \\
\end{array}
\right ],\quad \boldsymbol{C}=[ 0 \quad 1], \quad \boldsymbol{D}=[0]. 
\end{equation*}


 \item[]{\bf Zustandsraumdarstellung im Frequenzbereich}\index{Zustandsraumdarstellung!im Frequenzbereich}\\
Mit den {\bf Matrizen}\index{Matrizen}  {\boldmath $A$, $B$, $C$} und  {\boldmath $D$} erhalten wir (da alle Anfangsbedingungen Null sind):
\begin{eqnarray*}
\left [
\begin{array}{c}
 sX_1(s) \\
 sX_2(s) \\
\end{array}
\right ]
&=&\left [
\begin{array}{cc}
 \frac{-2R}{L} & 0\\
\frac{-1}{C} & \frac{-1}{CR}\\
\end{array}
\right ]
\left [
\begin{array}{c}
 X_1(s) \\
 X_2(s) \\
\end{array}
\right ]+
\left [
 \begin{array}{c}
  \frac{1}{L} \\
 \frac{1}{CR} \\
\end{array}
\right ] \cdot U(s) \quad\text{und}\\ 
Y(s)&=&
 [ 0 \quad 1]\left [
\begin{array}{c}
 X_1(s) \\
 X_2(s) \\
\end{array}
\right ]+[0]\cdot U(s). 
\end{eqnarray*}


Demzufolge ist $H(s)$:
\begin{eqnarray*}
H(s) &=& \boldsymbol{C}\klam{s\boldsymbol{I}-\boldsymbol{A}}^{-1}\boldsymbol{B}+\boldsymbol{D}=
\left [ 
\begin{array}{c c}
0  & 1\\
\end{array}
\right ]\cdot
\left [
\begin{array}{cc}
 s+\frac{2R}{L} & 0\\
\frac{1}{C}& s+\frac{1}{CR}\\
\end{array}
\right ]^{-1}\cdot
\left [ 
\begin{array}{c}
 \frac{1}{L}\\
\frac{1}{CR}\\
\end{array}
\right ] + [0]=\\
&=&
\left [ 
\begin{array}{c c}
0  & 1\\
\end{array}
\right ]\cdot
\frac{\left [
\begin{array}{cc}
 s+\frac{1}{CR} & 0\\
\frac{-1}{C}& s+\frac{2R}{L}\\
\end{array}
\right ]}{\klam{s+\frac{1}{CR}}\klam{s+\frac{2R}{L}}}\cdot
\left [ 
\begin{array}{c}
 \frac{1}{L}\\
\frac{1}{CR}\\
\end{array}
\right ]
=\frac{\left [
\begin{array}{cc}
 \frac{1}{C} & \klam{s+\frac{2R}{L}} \\
\end{array}
\right ]}{\klam{s+\frac{1}{CR}}\klam{s+\frac{2R}{L}}}\cdot
\left [ 
\begin{array}{c}
 \frac{1}{L}\\
\frac{1}{CR}\\
\end{array}
\right ]\\
&=&\frac{\frac{s}{CR}+\frac{1}{LC}}{\klam{s+\frac{1}{CR}}\klam{s+\frac{2R}{L}}}=\frac {s+\frac{R}{L}} {CR\klam{s+\frac{2R}{L}} \klam{s+\frac{1}{CR}} }=\frac{(s+1)}{1\cdot (s+2)(s+1)}=\frac{1}{s+2}.
\end{eqnarray*}

 \item[]{\bf Steuerbarkeit}\index{Steuerbarkeit}\\ Mit den {\bf Eigenwerten}\index{Eigenwert} $\lambda_1=-1$ und  $\lambda_2=-2$ ergibt sich z.B. mit\\
\begin{equation*}
\boldsymbol{T}=\left [
\begin{array}{cc}
 1  & 0\\
\frac{1}{C} & 1\\
\end{array}
\right ],~
\boldsymbol{T}^{-1}=\left [
 \begin{array}{cc}
  1 & 0\\
 \frac{-1}{C} & 1 \\
\end{array}
\right ],~
\boldsymbol{A}_{\text{diagonal}}=\boldsymbol{T}^{-1}\boldsymbol{AT}
=\left [
 \begin{array}{cc}
  \frac{-2R}{L}& 0\\
 0 & \frac{-1}{CR} \\
\end{array}
\right ] 
=\left [
 \begin{array}{cc}
  -2 & 0\\
 0 & -1 \\
\end{array}
\right ].
\end{equation*}
Somit ist \mat{\hat{B}}:
\begin{equation*}
\boldsymbol{\hat{B}}=\boldsymbol{T^{-1}B}=\left [
\begin{array}{cc}
 1  & 0\\
\frac{-1}{C} & 1\\
\end{array}
\right ]
\left [
\begin{array}{c}
 \frac{1}{L}\\
\frac{1}{CR}\\
\end{array}
\right ]=
\left [
\begin{array}{c}
 \frac{1}{L}\\
\frac{1}{CR}-\frac{1}{LC}\\
\end{array}
\right ]=
\left [
\begin{array}{c}
 \frac{1}{L}\\
 0 \\
\end{array}
\right ]
\end{equation*}
Das System ist mit diesen Zust\"anden nicht vollst"andig steuerbar. \\
\begin{equation*}
Q_{Steuerbarkeit}=\left [ \boldsymbol{
\begin{array}{cc}
 B & AB
\end{array}}\right ]=\left [ 
\begin{array}{cc}
\frac{1}{L} & -\frac{2R}{L^2} \\
\frac{1}{CR} & \frac{-1}{L}-\frac{1}{C^2R^2} \\
\end{array}\right ] 
\end{equation*}
$\rightarrow \text{Rang}(Q_{Steuerbarkeit})=1 \quad\text{und} \quad \left |  Q_{Steuerbarkeit}\right | =0. $
\item[]{\bf Beobachtbarkeit}\index{Beobachtbarkeit}\\
\mat{\hat{C}} ist:
\begin{equation*}
\boldsymbol{\hat{C}}=\boldsymbol{CT}=\left [
\begin{array}{cc}
 0  & 1\\
\end{array}
\right ]
\left [
\begin{array}{cc}
 1 & 0\\
\frac{1}{C} & 1 \\
\end{array}
\right ]=\left [
\begin{array}{cc}
 \frac{1}{C} & 1 \\
\end{array}
\right ].
\end{equation*}
Das System ist mit diesen Zust\"anden vollst"andig beobachtbar. \\
\begin{equation*}
Q_{Beobachtbarkeit}=\left [ \boldsymbol{
\begin{array}{c}
 C  \\
CA\\
\end{array}}\right ]=\left [ 
\begin{array}{cc}
0 & 1 \\
\frac{-1}{C} & \frac{-1}{CR} \\
\end{array}\right ] 
\end{equation*}
$\rightarrow \text{Rang}(Q_{Beobachtbarkeit})=2 \quad\text{und} \quad \left |  Q_{Beobachtbarkeit}\right | =\frac{1}{C}. $
\end{enumerate}

\aufg Bestimmmen Sie f\"ur Beipiel~\ref{kapitel_ZVD}.\ref{ZRD_BSP_gross} die Steuerbarkeit und Beobachtbarkeit, wenn Sie die Zust\"ande $x_1(t)\equiv u_R(t)$ und $x_2(t)\equiv y(t)$ w\"ahlen.% jetzt steuerbar! aber immer noch nicht beobachtbar

%\newpage
\aufg
\begin{figure}[!htb]
\vspace*{-4mm}
\begin{center}\bild{/zvd/ZVD_probepruefung.eps,width=0.991}\caption{Blockdiagramm}\end{center}% MATLAB print -s -depsc2 xxx.eps
\end{figure}
\begin{enumerate}
\item[a)] Stellen Sie die Matrizen ${\boldmath A,~B,~C}$ und ${\boldmath D}$ der Zustandsraumdarstellung des obenstehenden Blockdiagrammes auf. Verwenden Sie dabei die Zust\"ande $x_1(t)$, $x_2(t)$ und $x_3(t)$ des Blockdiagramms.
\item[b)] Wir betrachten von hier an nur die reduzierte Zustandsraumdarstellung (ohne Eingang $In_2(s)$).  Stellen Sie aus der Zustandsraumdarstellung (${\boldmath A_R,~B_R,~C_R}$ und ${\boldmath D_R})$ die UTF $H(s)=\frac{Out_1(s)}{In_1(s)}$ auf.
\item[c)] Von nun an sind: $a=1,~b=1,~c=1,~d=-1,~e=-2,~f=-3$ sowie $g=0$. Bestimmen Sie die Eigenwerte der Systemmatrix ${\boldmath A}$.\index{Eigenwert}
\item[d)] Sind die inneren Systemzust\"ande stabil?\index{Stabilitat@Stabilit\"at!innere}
\item[e)] Bestimmen Sie die Stabilit"at des Gesamtsystem $H(s)=\frac{Out_1(s)}{In_1(s)}$.
\item[f)] Bestimmen Sie die Beobachtbarkeit der reduzierten ZRD.\index{Beobachtbarkeit}
\item[g)] Bestimmen Sie die Steuerbarkeit der reduzierten ZRD.\index{Steuerbarkeit}
\end{enumerate}

\newpage
\subsection{Zustandskurve (Trajektorie)}\index{Zustandskurve}\index{Trajektorie}
F\"ur {\bf nichtlineare} Systeme\index{System!nichtlinear} wird
h\"aufig die Darstellung mittles {\bf Zustandskurve}, {\bf Phasenbahn}
oder {\bf Trajektorie}\index{Phasenbahn} verwendet \cite{UNB:89}. Bei
einem System mit 2 {\bf Zust\"anden} $x_1$ und $x_2$ werden die {\bf
  Zust\"ande}\index{Zustand} in einem rechtwinkligen Koordinatensystem
f\"ur verschiedene Startpunkte $x_1(0)$ und $x_2(0)$ eingetragen.  \bsp{Mit $\boldsymbol{A}=\left [
\begin{array}{cc}
 -3/2 & -1/2\\
 1 & 0\\
\end{array}
\right ]$ von Beispiel~\ref{kapitel_ZVD}.\ref{ZRD_BSP_016} und den folgenden \mb\hspace*{-1.6mm}en erhalten wir die Abb.~\ref{ZVD_ABB_phasenbahn}.}
\begin{small}
\begin{verbatim}
function []= Phasenbahn(); 
  hold off; n=2;
  for i=-n:n,
    for j=-n:n,
      x0=[i;j];
      [t,x]=ode45('Trajektorie',[0 10],x0); 
      plot(x(:,1),x(:,2),'LineWidth',2);
      title('Zustandskurven (Phasenbahn, Trajektorie)',... 
            'Fontsize',12,'FontName','Times');grid on; hold on; 
      xlabel('\rightarrow x_1','Fontsize',12,'FontName','Times');
      ylabel('\rightarrow x_2','Fontsize',12,'FontName','Times');
    end
  end

function xDot = Trajektorie(t,x) 
  xDot=[-3/2*x(1)-x(2)/2; x(1)];
\end{verbatim}
\end{small}
\begin{figure}[!htb] % matlab phasenbahn.m % unb:89 s. 215
  \vspace*{-4mm}\begin{center}
    \bild{/zvd/ZVD007.eps,width=0.76}\vspace*{-3mm}\caption{Das {\bf LTI-System} 2.~Ordnung $(2s^2+3s+1)\cdot Y(s)=U(s)$ hat ein {\bf ``Phasenportr\"at''} mit einem {\bf Knotenpunkt}(bei (0,0)), was auf ein {\bf stabiles System} hinweist \cite{UNB:89}.}\label{ZVD_ABB_phasenbahn}
\end{center}
\vspace*{-6mm}
\end{figure}\index{LTI} \index{Ordnung}\index{System!stabil} \index{Knotenpunkt}\index{Phasenportrat@{Phasenportr\"at}}  

\newpage
\section{Weitere Aufgaben zur Zustandsraumdarstellung}
\begin{enumerate}
\item {\bf Zustandsraumdarstellung \cite{HSU:95}}\\
  Bestimmen Sie die Zustandsraumdarstellung der folgenden Schaltung.\\
\begin{figure}[!htb]
\vspace*{-4mm}\begin{center}
  \bild{/zvd/ZVD005.ps,width=0.8}
\end{center}
\vspace*{-6mm}
\end{figure}\\
Verwenden Sie als Zustandsgr\"osse $x_1(t)\equiv i_1(t)$ und
$x_2(t)\equiv u_c(t)$, wobei die Ausgangsgr\"ossen $y_1(t)\equiv
i_1(t)$ und $y_2(t)\equiv i_2(t)$ sind. Die Eingangsgr\"ossen sind $u_1(t)$ und $u_2(t)$.
\item {\bf Zustandsraumdarstellung \cite{GIR:RAB:STE:05}}
\\
\begin{figure}[!htb]
\vspace*{-4mm}\begin{center}
  \bild{/zvd/ZVD006.ps,width=0.8}\caption{$RLC$-Schaltung}\label{ZRD_FIG_06}
\end{center}
\vspace*{-14mm}
\end{figure}\\
\begin{enumerate}
\item Stellen Sie die Zustandsraumdarstellung von
  Abb.~\ref{ZRD_FIG_06} auf, wobei der Ausgang $y(t)$ ist und die
  Zust\"ande $i_1(t)$, $i_2(t)$ und $u_c(t)$ verwendet werden sollen.
 \item Bestimmen Sie aus der Zustandsraumdarstellung die UTF.
\end{enumerate}
\item {\bf Zustandsraumdarstellung aus Differentialgleichung \cite{HSU:95}}
\begin{enumerate}
\item Finden Sie die Zustandsraumdarstellung des analogen LTI-Systemes
  $\ddot{y}(t)+3\dot{y}(t)+2y(t)=u(t)$, wobei die Zust\"ande
  $x_1(t)\equiv y(t)$ und $x_2(t)\equiv \dot{y}(t)$ verwendet werden
  sollen.
 \item Ist das System beobachtbar?
 \item Ist das System steuerbar?
\end{enumerate}
\item {\bf Zustandsraumdarstellung aus Differentialgleichung \cite{GIR:RAB:STE:05}} % s.42
\begin{enumerate}
\item Finden Sie die Zustandsraumdarstellung
  (Regelungsnormalform)\index{Regelungsnormalform} des analogen
  LTI-Systemes $\ddot{y}(t)+4\dot{y}(t)+5y(t)=2\ddot{u}(t)+7u(t)$,
  indem Sie die {\bf UTF}\index{UTF} $H(s)=\frac{Y(s)}{U(s)}$
  aufstellen, das {\bf Blockschaltbild}\index{Blockschaltbild} der
  {\bf Regelungsnormalform} skizzieren und die Matrizen {\boldmath $A$, $B$,
    $C$} und {\boldmath $D$} bestimmen.
\item Finden Sie eine Transformationsmatrix \mat{T}, so dass \mat{A} diagonalisiert werden kann ($\boldsymbol{A}_\text{diagonal}=\boldsymbol{T^{-1}AT}$).
 \item Ist das System beobachtbar?
 \item Ist das System steuerbar?
\end{enumerate}

\clearpage
\item {\bf Zustandsraumdarstellung (ZRD)}\index{ZRD} 

 \begin{center}\bild{/zvd/ZRD_WS0607.eps,width=0.95}\end{center}
\begin{enumerate}
\item[a)] Stellen Sie die Matrizen {\boldmath $A,~B,~C$} und {\boldmath $D$} der Zustandsraumdarstellung des obenstehenden Blockdiagramms auf. Verwenden Sie dabei die Zust\"ande $x_1(t)$, und $x_2(t)$ des Blockdiagramms. (Beachten Sie, dass das Blockdiagramm {\bf nicht} in einer Normalform dargestellt ist.)
\item[b)] Welche Gr"osse hat die "Ubertragungs{\bf matrix} {\boldmath $H(s)$} der ZRD von a)?
\item[c)] Bestimmen Sie die UTF $H(s)=\frac{Out_1(s)}{In_1(s)}$ mit den Matrizen {\boldmath $A,~B,~C$} und {\boldmath $D$}. (Exakte, nummerische Werte sind f"ur die volle Punktzahl verlangt.)
\item[d)] Bestimmen Sie die Eigenwerte der Systemmatrix {\boldmath $A$} der ZRD.
\item[e)] Sind die inneren Systemzust\"ande der ZRD stabil?
\item[f)] Bestimmen Sie die Beobachtbarkeit der ZRD.
\item[g)] Bestimmen Sie die Steuerbarkeit der ZRD.
\end{enumerate}


\end{enumerate}







