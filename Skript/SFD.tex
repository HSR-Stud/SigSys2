% a) Beispiele schwierieg Vorleseung + 1x Pruefung einfuegen...
% 1x vereinfachung?? rundung??
% HI Schema aus Diss
% Sigma delta mit Siganl UTFund Noise UTF
% SFD mit Hoergerate und Latticefilter
% fundamentales SFD 2. Ordnung 
% fundamentales SFD fuer bsp 1.8
\renewcommand{\thesection}{\thechapter.\arabic{section}}
\setcounter{Aufgabe}{0}\setcounter{Beispiel}{0}
\chapter{Signalflussdiagramme \index{Signalflussdiagramm}}\label{KAP_SFD} 
\section{Einf"uhrung} 
Ein Signalflussdiagramm\index{Signalflussdiagramm}
(SFD)\index{SFD|see{Signalflussdiagramm}} ist eine graphische
Darstellung eines Systems, das durch ein Gleichungssystem beschrieben
wird. Genau so wie die Gleichungen gel"ost werden k"onnen, um
verschiedene Beziehungen zwischen Variablen des Systems zu finden,
kann das SFD ``gel"ost'' werden, um die gleichen Beziehungen zu
erhalten. Die graphische Darstellung des Systems durch das SFD\index{SFD} liefert
visuelle ``Einsicht'' in das System hinsichtlich der
Komplexit"at\index{Komplexitat@Komplexit\"at}, der
R"uckkopplungsschleifen \index{Ruckkopplungsschleife@{R"uckkopplungsschleife}} und der
Vorw"artspfade\index{Vorwartspfad@Vorw\"artspfad}. Das ist in einem
Gleichungssystem \index{Gleichungssystem}viel schwieriger zu erkennen.
SFD bilden eine Br\"ucke von den
\"Ubertragungsfunktionen\index{Ubertragungsfunktion@{\"Ubertragungsfunktion}} (UTF)\index{UTF}
zwischen Eingang und Ausgang (oder dazwischenliegenden Variablen des
Systems), zu m"oglichen Topologien\index{Topologie} des entsprechenden
Systems. Die Theorie und die Reduktionsregeln f"ur SFD wurden im Jahre
1953 von S.~J.~Mason eingef"uhrt \cite{MOS:97}. Die Theorie der SFD
ist ein hervorragendes Werkzeug f\"ur die {\it\textbf{Analyse}}\index{Analyse} und die {\it\textbf{Synthese}}\index{Synthese} von
Systemen.\\
\nit Bis jetzt haben wir unsere Systeme nicht n"aher spezifiziert,
weil die Theorie der SFD f"ur die meisten Systeme, ob sie nun
{\it\textbf{linear}} oder {\it\textbf{nichtlinear}},\index{System!nichtlinear}\index{System!linear}
{\it\textbf{zeitdiskret}} oder {\it\textbf{zeitkontinuierlich}} sind,\index{System!zeitdiskret}\index{System!zeitkontinuierlich}
angewandt werden kann. In "Ubereinstimmung mit der Behandlung der
{\it\textbf{linearen Systeme}} werden wir uns in den n"achsten
Abschnitten auf eine Besprechung der SFD, die sich auf
{\it\textbf{lineare, zeitinvariante Systeme}}\index{System!linear} (LTI-Systeme) beziehen,
beschr"anken. In vielen Beispielen sind zeitdiskrete
Systeme\index{System!zeitdiskret} oder deren
$z$-Transformation\index{z-Transformation@$z$-Transformation} verwendet worden.  F"ur
das Verst"andnis der SFD ist die $z$-Transformation aber
nicht notwendig.
\section{Definitionen und Konstruktionsregeln} 
Die urspr"ungliche und direkteste Anwendung der SFD ist die
Visualisierung der beschreibenden Gleichungen eines Systems in
graphischer Form. Bevor wir die Regeln, die wir f"ur die Konstruktion
von Signalflussdiagrammen ben"otigen, besprechen, m"ussen einige
grundlegende Bezeichnungen erkl"art werden. Unter Bezugnahme auf das
allgemeine SFD in Abb.~\ref{SFD1} werden wir die
folgenden Bezeichnungen verwenden: %\newpage
\begin{description}
\item[Knoten:] \index{Knoten}Stellt eine bestimmte Gr"osse (z.B. ein Signal) oder eine Variable dar.
\item[Zweig:] \index{Zweig}Stellt die funktionelle Abh"angigkeit
  zwischen den Variablen (zwischen den Knoten) dar. Der Wert einer
  Variablen, die durch einen Knoten dargestellt ist, berechnet sich
  durch Addition der Zweig, die in diesen Knoten {\it\textbf{einm"unden}}.
  Signale bewegen sich nur in der durch den Pfeil des Zweiges
  angegebenen Richtung (Zweigrichtung).
\item[Quelle:] \index{Quelle}Ein Knoten, in den keine Zweige
  einm"unden (zum Beispiel Knoten $X_1$ in Abb.~\ref{SFD1}). Eine Quelle stellt eine
  unabh"angige Variable dar.
\begin{figure}[htb!]
\begin{center} 
  \bild{/sfd/SFD1.fig.eps,width=0.75}\caption{Beispiel eines allgemeinen
Signalflussdiagramms}\label{SFD1}
\end{center}
\vspace*{-7mm}
\end{figure}

\item[Senke:] \index{Senke}Ein Knoten ohne weggehende Zweige (Knoten
  $X_5$). Eine Senke stellt eine Variable dar, von welcher keine andere
  Variable im System abh"angt.
\item[Gemischter Knoten:] \index{Knoten!gemischter}Ein Knoten mit
  hineinf"uhrenden und weggehenden Zweigen (Knoten $X_2, X_3, X_4$ und
  $X_6$).  
\item[Pfad:] \index{Pfad}Eine kontinuierliche Folge von Zweigen, welche alle
  in die gleiche Richtung zeigen.  
\item[Offener Pfad:] \index{Pfad!offener}Ein Pfad, bei dem
  jeder beteiligte Knoten nur einmal durchquert wird ($abcd$ oder $aeh$;
  $aef$ ist nicht offen).  
\item[Vorw"artspfad:] \index{Vorwartspfad@Vorw\"artspfad} Ein offener Pfad zwischen
  einer Quelle und einer Senke ($abcd$ oder $aehd$).  
\item[Schleife:] \index{Schleife}Ein geschlossener Pfad, welcher zum
  Ausgangsknoten zur"uckkehrt, wobei jeder beteiligte Knoten nur
  einmal durchlaufen wird, ausgenommen der Ausgangsknoten (Schleife
  $g$ und $ef$, aber nicht $egf$).  
\item[Eigenschleife:] \index{Eigenschleife}Eine
  (R"uckkopplungs)schleife, die aus einem Zweig und einem Knoten
  besteht (Schleife $g$).
\item[Zweigtransmittanz:] \index{Zweig!transmittanz}Die lineare
  Gr"osse, unabh"angig von ihrer Dimension, die einen Knoten eines
  Zweiges zum anderen Knoten in Beziehung setzt. Somit wird zum
  Beispiel ein Signal $X_k$, das einen Zweig zwischen $X_k$ und $X_j$
  durchquert, mit seiner Transmittanz (zum Beispiel seiner
  Verst"arkung) $t_{kj}$ multipliziert, so dass beim Knoten $X_j$ das
  Signal $t_{kj}X_k$ resultiert.  
\item[Schleifentransmittanz:] \index{Schleifen!transmittanz}Das Produkt der
  Zweigtransmittanzen in einer Schleife.
\end{description}
\newpage
\nit Mit diesen Definitionen
k"onnen wir nun die Konstruktionsregeln von Signalflussdiagrammen
zusammenfassen: 
\begin{enumerate}
\item Die Variablen (z.B. Signale) bei den Knoten eines
  SFD stellen die Variablen eines {\it\textbf{linearen}}
  Gleichungssystems dar, wobei die Zweigtransmittanzen die Konstanten
  oder Koeffizienten darstellen.
\item Signale durchqueren Zweige nur in
  Pfeilrichtung.
\item Ein Signal, das einen Zweig durchquert, wird mit
  dessen Transmittanz multipliziert. 
\item Der Wert der Variable, die durch einen Knoten dargestellt wird,
  ist die Summe aller Signale, die
  in diesen Knoten einm"unden. 
\item Der Wert der Variable, die durch einen Knoten dargestellt wird,
  wird auf alle weggehenden Zweige "ubertragen.
\end{enumerate}
\nit Das SFD eines physikalischen, linearen Systems ist nichts anderes
als eine graphische Darstellung des linearen Gleichungssystems, das
dieses System charakterisiert; es wird in der Tat direkt von diesem
Gleichungssystem abgeleitet. Da es viele verschiedene Formen gibt, wie
die Gleichungen eines Systems geschrieben werden k"onnen, gibt es auch
ebensoviele verschiedene Formen von SFD zur Beschreibung desselben
Systems.  Nat"urlich erzeugen gewisse Gleichungssysteme einfachere SFD
f"ur ein gegebenes System als andere.  Verallgemeinerungen sind jedoch
schwierig, weil die Form der Systemgleichungen, die das einfachste SFD
liefert, von der Topologie\index{Topologie}, der Komplexit"at und der
Zahl der unabh"angigen Variablen des Systems abh"angt. Der Erfolg bei
der Umwandlung eines komplexen Systems in ein einfach handhabbares SFD
h"angt somit zu einem gewissen Grad von der Erfahrung ab.
Andererseits, wie wir gleich sehen werden, haben die SFD folgenden
grossen Vorteil: Auch wenn Gleichungen gew"ahlt werden, die ein
unn"otig kompliziertes Diagramm ergeben, k"onnen die leicht
durchzuf"uhrenden Regeln, das Diagramm auf seine einfachstm"ogliche
Form reduzieren.\\  
\bsp{Betrachten wir das folgende Gleichungssystem:}
\begin{eqnarray}
 x_1[k] & = & x_0[k] + a\cdot x_2[k] + b\cdot x_2[k], \nonumber \\
 x_2[k] & = & c\cdot x_1[k], \label{mathSFD2}\\ 
 x_3[k] & = & d\cdot x_0[k] + e\cdot x_2[k]. \nonumber
\end{eqnarray}
\nit Das entsprechende SFD ist in Abb.~\ref{SFD2} gezeigt.
\newpage
\begin{figure}[htb!]
\vspace*{-3mm}\begin{center}
  \bild{/sfd/SFD2.fig.eps,width=0.57}\caption{Signalflussdiagramm
    zu den Gleichungen \ref{mathSFD2}}\label{SFD2}
\end{center}
\vspace*{-7mm}
\end{figure}
\nit Da wir uns hier mit zeitdiskreten
Systemen\index{System!zeitdiskret} befassen, sind die Knoten zeitdiskrete
Signale. Dies ist ausgedr"uckt durch die Notation
$x_i[k]$. Ein um $m$ Zeiteinheiten verz"ogertes Signal wird analog
mit $x_i[k-m]$ bezeichnet. Gem"ass der Terminologie ist $x_0[k]$ eine
Quelle und $x_3[k]$ eine Senke. Da die Transmittanz von $x_0[k]$
  zu $x_1[k]$ eins ist, haben wir die Bezeichnung $1$ weggelassen. {\it\textbf{Im Allgemeinen m"ussen Transmittanzen\index{Transmittanz} mit dem Wert eins nicht beschriftet werden}}. Wenn ein SFD
mindestens eine R"uckkopplungsschleife \index{Rue@{R\"u}ckkopplungs!schleife}
enth"alt, wie in Abb.~\ref{SFD2}, wird es manchmal
{\it\textbf{R"uckkopplungsdiagramm}} \index{Ruckkopplungsdiagramm@R{\"u}ckkopplungsdiagramm} (feedback
graph) genannt. Dagegen wird das SFD in Abb.~\ref{SFD3}, welches nur
Vorw"artspfade enth"alt, auch {\it\textbf{Kaskadendiagramm}}\index{Kaskadendiagramm}
genannt.
\begin{figure}[htb!]
\begin{center}
  \bild{/sfd/SFD3.fig.eps,width=0.58}\caption{Signalflussdiagramm, das nur Vorw"artspfade enth"alt
(``Kaskadendiagramm'')}\label{SFD3}
\end{center}
\vspace*{-7mm}
\end{figure}\\
%\newpage
\nit Das entsprechende Gleichungssystem f"ur Abb.~\ref{SFD3} lautet:
\begin{eqnarray}
 x_2[k] & = & t_{12}\cdot x_1[k], \nonumber\\
 x_3[k] & = & t_{13}\cdot x_1[k] + t_{23}\cdot x_2[k], \\
 x_4[k] & = & t_{24}\cdot x_2[k] + t_{34}\cdot x_3[k], \nonumber  
\end{eqnarray}\\
\nit wobei $x_1[k]$ eine Quelle und $x_4[k]$ eine Senke darstellt. Zu
beachten ist, dass eine Transmittanz $t_{jk}$ am Knoten $j$ beginnt
und am Knoten $k$ endet, was der Richtung des Pfeils dieses Zweiges
entspricht. Dies ist die  {\it\textbf{"ubliche Bezeichnung}} der Transmittanzen in
der Theorie der SFD, aber sie ist die Transponierte\index{Transponierte} der Bezeichnung,
wie sie im Allgemeinen in Matrixgleichungen gebraucht wird. Somit ist
zum Beispiel die $j$-te Zeile der Matrixgleichung\index{Matrixgleichung}
\begin{eqnarray*}
 \underline{x}[k] =  \boldmath{T} \underline{x}_k + \underline{t}_0 x_0
\end{eqnarray*}
 im Allgemeinen 
\begin{equation}
  x_j[k] = t_{j1} \cdot x_1[k] + t_{j2} \cdot x_2[k] + \ldots + t_{jj} \cdot x_j[k] + \ldots + t_{jn} \cdot x_n[k] + t_{j0} \cdot x_0 \label{zeile}
\end{equation}
und das entsprechende SFD (Abb.~\ref{SFD4}) von Formel~\ref{zeile} ist:\\
\begin{figure}[htb!]
\begin{center}
  \bild{/sfd/SFD4.fig.eps,width=0.5}\caption{Signalflussdiagramm der $j$-ten Zeile einer Matrixgleichung}\label{SFD4}
\end{center}
\vspace*{-7mm}
\end{figure}\\
\nit Dieser Unterschied der Bezeichnungen bei SFD und Matrixgleichungen mag trivial scheinen, f"uhrt aber oft zu Verwirrung.\\ ~\\ Da wir uns
mit zeitdiskreten Systemen\index{System!zeitdiskret} befassen, sind die entsprechenden
beschreibenden Gleichungssysteme  {\it\textbf{Differenzengleichungen}}\index{Differenzengleichung} (anstatt
 {\it\textbf{Differentialgleichungen}})\index{Differentialgleichung}. Die grundlegenden Komponenten, die f"ur die
Realisierung dieser Systeme ben"otigt werden, sind
Addierer\index{Addierer}, Multiplikatoren\index{Multiplikator} und
Verz"ogerungselemente\index{Verzoe@{Verz\"o}gerungselement}. Als Beispiel dient
das folgende Gleichungssystem und das entsprechende Blockdiagramm\index{Blockdiagramm}, das
in Abb.~\ref{SFD5} gezeigt ist. Zweigpunkte, Addierer, Multiplikatoren
und Verz"ogerungselemente \index{Verzoe@{Verz\"o}gerungselement} (Symbol $D$,
englisch: Delay) sollten in der Abbildung~\ref{SFD5} ersichtlich sein.\\
\begin{figure}[htb!]
\begin{center}
  \bild{/sfd/SFD5.fig.eps,width=0.75}\caption{Blockdiagramm\index{Blockdiagramm}, das den Gleichungen \ref{mathSFD5} entspricht}\label{SFD5}
\end{center}
\vspace*{-7mm}
\end{figure}
\begin{eqnarray}
 x_1[k] & = & a\cdot x_4[k] + x_0[k] \nonumber \\
 x_2[k] & = & x_1[k]\nonumber\\
 x_3[k] & = & b_0\cdot x_2[k] + b_1\cdot x_4[k] \label{mathSFD5}\\ 
 x_4[k] & = & x_2[k-1]\nonumber\\
 x_5[k] & = & x_3[k]\nonumber
\end{eqnarray}
\nit Wenn wir die Gleichungen \ref{mathSFD5} $z$-transformieren, erhalten
wir:
\begin{eqnarray}
 X_1(z) & = & a\cdot X_4(z) + X_0(z)\nonumber\\
 X_2(z) & = & X_1(z)\nonumber\\
 X_3(z) & = & b_0\cdot X_2(z) + b_1\cdot X_4(z) \label{mathSFD6}\\ 
 X_4(z) & = & z^{-1}\cdot X_2(z)\nonumber\\
 X_5(z) & = & X_3(z)\nonumber
\end{eqnarray}
\nit Das entsprechende Signalflussdiagramm ist in Abb.~\ref{SFD6} gezeigt.\\
\begin{figure}[htb!]
\begin{center}
  \bild{/sfd/SFD6.fig.eps,width=0.75}\caption{Signalflussdiagramm, das den Gleichungen \ref{mathSFD6} entspricht}\label{SFD6}
\end{center}
\vspace*{-7mm}
\end{figure}\\
\nit Es besteht eine direkte "Ubereinstimmung zwischen den Zweigen im
SFD (Abb.~\ref{SFD6}) und den Zweigen im Blockdiagramm
(Abb.~\ref{SFD5}). Ein wichtiger Unterschied zwischen den beiden ist,
dass Knoten im SFD sowohl Zweigpunkte als auch Addierer darstellen,
w"ahrend im Blockdiagramm ein spezielles Symbol f"ur Addierer
gebraucht wird. Ein Zweigpunkt \index{Zweig!punkt} im Blockdiagramm
ist im SFD durch einen Knoten dargestellt, der \textbf{\emph{nur
    einen}} hineinf"uhrenden Zweig und einen oder mehrere weggehende
Zweige hat (z.B. $X_2(z)$ in Abb.~\ref{SFD6}). Ein Addierer im
Blockdiagramm ist im SFD durch einen Knoten gekennzeichnet, der zwei
oder mehrere hineinf"uhrende Zweige hat (z.B. $X_1(z)$ in Abb.~\ref{SFD6}).  Trotzdem sind \textbf{\emph{SFD}}
und \textbf{\emph{Blockdiagramme}} gleichwertig als bildliche Darstellungen von
Differenzengleichungen, wobei SFD aber einfacher zu zeichnen sind. Wichtig
ist, dass bei Anwendung der SFD-Theorie beliebige Beziehungen direkt
herausgelesen werden k"onnen und dass Reduktionsregeln erlauben, die
Komplexit"at eines gegebenen Diagramms stark herabzusetzen. Die
wichtigsten Reduktionsregeln werden im Abschnitt~\ref{SFD_red_regeln} behandelt.\\
\aufg
Skizzieren Sie das SFD des folgenden, zeitdiskreten Gleichungssystems:\\
$x_2=x_1+x_3;\quad x_3=a_1\cdot x_5 + a_2\cdot x_6;\quad x_4=x_2;\quad
x_5=z^{-1}\cdot x_4;\quad x_6=z^{-1}\cdot x_5;\quad x_7=b_0\cdot x_4 +
x_8;\quad x_8=b_1\cdot x_5 + b_2\cdot x_6;\quad y=x_7$.


\newpage
\section{Reduktionsregeln \index{Reduktionsregeln}}\label{SFD_red_regeln} 
Wir haben im letzten Abschnitt gesehen, dass das SFD eines Systems
nicht mehr Information enth"alt, als die beschreibenden Gleichungen,
aus welchen es hergeleitet wurde.  Trotzdem gibt es viele gute
Gr"unde, die Systemgleichungen in eine "aquivalente graphische Form
umzuwandeln. Einerseits erlaubt sie die Visualisierung der
Signal"ubertragungswege im System, andererseits gibt sie die
Wechselwirkung zwischen den Systemvariablen \index{System!variablen}
an und zeigt die vorhandenen R"uckkopplungsschleifen. Der gr"osste
Vorteil sind aber die einfachen Regeln, die es erlauben, ein SFD
beliebiger Komplexit"at (abh"angig von den entsprechenden
Systemgleichungen, aus denen es hergeleitet wurde) durch sukzessive
Elimination \index{Elimination!sukzessive} von abh"angigen Variablen
(das heisst Knoten) zu vereinfachen. Somit ist die Reduktion des SFD
gleichwertig mit der L"osung der entsprechenden Systemgleichungen.
Dank systematischer Regeln ist die Reduktion des Diagramms aber oft
einfacher.  Wie im Folgenden gezeigt, entspricht fast jede der
Reduktionsregeln direkt einer gleichwertigen Umwandlung von linearen
Gleichungen, in welchen eine abh"angige Variable eliminiert wird. Zu
beachten ist, dass, der Einfachheit halber, Knotenvariable Skalare
sind. Der "Ubergang zu Zeitsequenzen oder zu $z$-transformierten
Variablen sollte offensichtlich sein.
\subsection{Regel 1:  Kettentransformation \index{Kettentransformation}} 
Die gesamte "Ubertragung einer Kaskade \index{Kaskade} von Zweigen
(d.h. einem Pfad) ist gleich dem Produkt der einzelnen Transmittanzen.
Ein Beispiel ist in Abb.~\ref{SFD7} zusammen mit den gleichwertigen
Gleichungen aufgezeigt. \\
\begin{figure}[htb!]
\vspace*{-3mm}\begin{center}
  \bild{/sfd/SFD7.fig.eps,width=0.68}\caption{Reduktionsregel 1:
Kettentransformation}\label{SFD7}
\end{center}
\vspace*{-7mm}
\end{figure}
\subsection{Regel 2:  Paralleltransformation \index{Paralleltransformation}}
Die gesamte Transmittanz paralleler Zweige ist gleich der Summe der
einzelnen Zweigtransmittanzen, wie in Abb.~\ref{SFD8} gezeigt.\\
\begin{figure}[htb!]
\vspace*{-3mm}\begin{center}
  \bild{/sfd/SFD8.fig.eps,width=0.61}\caption{Reduktionsregel 2: Paralleltransformation}\label{SFD8}
\end{center}
\vspace*{-7mm}
\end{figure}

\subsection{Regel 3: Entfernung eines Knotens \index{Knoten!entfernung}}
Der Anfangs- oder Endpunkt einer Transmittanz kann
entfernt oder verschoben werden, solange die Transmittanz zwischen den
interessierenden Knoten im System unver"andert bleibt (siehe auch
Regel 4). Ein Knoten kann, wie in Abb.~\ref{SFD9} gezeigt, entfernt werden.

\begin{figure}[htb!]
\begin{center}
  \bild{/sfd/SFD9.fig.eps,width=0.75}\caption{Reduktionsregel 3: Entfernung eines Knotens}\label{SFD9}
\end{center}
\vspace*{-7mm}
\end{figure}

\subsection{Regel 4: Transmittanzverschiebung \index{Transmittanz!-verschiebung}} 
\begin{figure}[htb!]
\vspace*{-3mm}\begin{center}
  \bild{/sfd/SFD10.fig.eps,width=0.55}\caption{Reduktionsregel 4: Verschiebung einer
Transmittanz. (a) Urspr"ungliches Signalflussdiagramm. (b)
Verschiebung des Anfangspunktes eines inneren Zweiges. (d)
Verschiebung des Endpunktes eines Zweiges, der eine Quelle enth"alt.}\label{SFD10}
\end{center}
\vspace*{-7mm}
\end{figure}
\nit Ausgehend vom SFD in Abb.~\ref{SFD10}a kann der
Anfangspunkt eines Zweiges verschoben werden (Abb.~\ref{SFD10}b), es
kann der Endpunkt eines inneren Zweiges verschoben werden
(Abb.~\ref{SFD10}c), und es kann der Endpunkt eines Zweiges, der eine
Quelle $x_4$ enth"alt, verschoben werden (Abb.~\ref{SFD10}d). Zu
beachten ist, dass eine neue Variable $x_3^{\mbox{,}}$ eingef"uhrt werden muss,
wenn der Endpunkt eines inneren Zweiges verschoben wird
(Abb.~\ref{SFD10}c).  Konsequenzen der
Transmittanz-Verschiebungsregeln sind die ``Y''-Transformation
\index{Y-Transformation}(Abb.~\ref{SFD11}a) und die
``Stern-zu-Masche''-Transformation
\index{Stern-zu-Masche-Transformation} (Abb.~\ref{SFD11}b).\\
\begin{figure}[htb!]
\begin{center}
  \bild{/sfd/SFD11.fig.eps,width=0.56}\caption{(a)
    ``Y''-Transformation. (b) ``Stern-zu-Masche''-Transformation.
    }\label{SFD11}
\end{center}
\vspace*{-7mm}
\end{figure}
\subsection{Regel 5: \label{abschnitt13} Pfadinversion \index{Pfad!inversion}} 
Um einen Pfad, der seinen Ursprung bei einer Quelle $x_i$ und den
Endpunkt bei einem Knoten $x_j$ hat, dessen Transmittanz $\mu$ ist, zu
invertieren, ersetzen wir ihn durch einen Pfad von $x_j$ nach $x_i$,
der eine Transmittanz $1/\mu$ hat.  Ferner werden alle anderen Zweige,
die urspr"unglich bei $x_j$ endeten, verschoben, so dass sie bei $x_i$
enden, und ihre Transmittanzen werden mit $-1/\mu$ multipliziert. Das
wird in Abb.~\ref{SFD12} gezeigt, wo der Pfad $a$ von $x_2$ nach $x_1$
(Abb.~\ref{SFD12}a) invertiert wird (Abb.~\ref{SFD12}b).\\
\begin{figure}[htb!]
\begin{center}
  \bild{/sfd/SFD12.fig.eps,width=0.75}\caption{Reduktionsregel 5: Pfadinversion. (a) Urspr"ungliches SFD (b) SFD mit invertiertem Pfad $a$}\label{SFD12}
\end{center}
\vspace*{-7mm}
\end{figure}\\
\nit Beachten Sie, dass die Inversion eines Pfades (dessen
Anfangspunkt nach Definition eine Quelle sein muss) den Effekt hat,
dass die Quelle vom einen Ende des Pfades zum anderen Ende verschoben
wird. Somit wird eine Folge von Inversionen in einem SFD sukzessiv die
entsprechende Quelle transferieren. Das ist in Abb.~\ref{SFD13}
gezeigt, wo der Pfad von $x_1$ nach $x_5$ in vier Schritten invertiert
wird. Somit wandert
die Quelle von $x_1$ nach $x_5$.\\
\begin{figure}[htb!]
\vspace*{-8mm}\begin{center}
  \bild{/sfd/SFD13a.ps,width=0.95}\caption{Schrittweise Pfadinversion. (a) Urspr"ungliches SFD. (b-e) Folge von Inversionen. Beachten Sie das Verschieben der Quelle durch jeden Knoten.}\label{SFD13}
\end{center}
\vspace*{-7mm}
\end{figure}

\subsection{Regel 6: Entfernung einer Eigenschleife}
Eine Eigenschleife, deren Transmittanz gleich $L$ ist, wird von einem
Knoten entfernt, indem die Transmittanzen von allen anderen Zweigen,
die in diesen Knoten m"unden, durch $(1-L)$ dividiert werden. Ein
Beispiel einer Entfernung einer Eigenschleife ist in Abb.\ref{SFD14} gezeigt.\\
\begin{figure}[htb!]
\vspace*{-2mm}\begin{center}
  \bild{/sfd/SFD14.fig.eps,width=0.75}\caption{Regel 6:
Entfernung einer Eigenschleife (a) Urspr"ungliches SFD mit
Eigenschleife $b$ (b) SFD mit entfernter Eigenschleife $b$}\label{SFD14}
\end{center}
\vspace*{-7mm}
\end{figure}

\subsection{Regel 7: Schleifenreduktion \index{Schleifen!reduktion}} 
\subsubsection{Einzelne Schleife}
Die Transmittanz einer
unabh"angigen Variablen $x_i$ (d.h. einer Quelle) zu einer abh"angigen
Variable (d.h. einem inneren Knoten oder einer Senke) in einem
SFD, das nur eine Schleife und einen offenen  Pfad enth"alt,
ist gleich 
\begin{equation}
 T_{ij} = \frac{P_{ij}}{1-L} \label{mathSFD15} 
\end{equation}
wobei $P_{ij}$ die Transmittanz des
Vorw"artspfades von $x_i$ nach $x_j$ und $L$ die Transmittanz der
Schleife ist. Die Formel~\ref{mathSFD15} l"asst sich mittels der L"osung des entsprechenden Gleichungssystems oder "aquivalent durch Transmittanzverschiebung und Entfernung der Eigenschleife beweisen. Ein Beispiel ist in Abb.~\ref{SFD15} gezeigt.
\begin{figure}[htb!]
\begin{center}
  \bild{/sfd/SFD15.fig.eps,width=0.85}\caption{Regel 7a: Entfernung einer Schleife. (a)
Urspr"ungliches SFD. (b) SFD mit entfernter Schleife}\label{SFD15}
\end{center}
\vspace*{-7mm}
\end{figure}
\subsubsection{Mehrere \textbf{\emph{sich nicht ber"uhrende}} Schleifen}
Bei einer Kaskade von sich nicht ber"uhrenden Schleifen (d.h., dass
sie keine gemeinsamen Knoten haben) ist die gesamte Transmittanz
gleich dem Produkt der einzelnen Transmittanzen, wie durch Formel~\ref{mathSFD15}
gegeben, n"amlich 
\begin{equation}
 T_{in} = \frac{P_{ij}}{1-L_j}\cdot \frac{P_{jk}}{1-L_k}\cdot\quad \ldots\quad\cdot\frac{P_{(n-1)n}}{1-L_n}. \label{mathSFD16} 
\end{equation}
\nit Wenn wir Formel~\ref{mathSFD16} auf das Beispiel in Abb.~\ref{SFD16}a anwenden, erhalten wir 
\begin{equation}
 T_{15} =  \frac{x_5}{x_1} = \frac{P_{13}}{1-L_1}\cdot \frac{P_{35}}{1-L_2}= \frac{P_{15}}{1-L_1-L_2+L_1\cdot L_2} = \frac{abcd}{1-be -df + bedf}.\label{mathSFD16z}
\end{equation}
Hier ist zu beachten, dass die zwei Kreise \textbf{\emph{keinen
    gemeinsamen}} Knoten haben. Wenn sie einen gemeinsamen Knoten
haben, erhalten wir ein anderes Resultat, wie es im n"achsten Fall
gezeigt wird.

\subsubsection{Mehrere \textbf{\emph{sich ber"uhrende}} Schleifen} 
F"ur den Fall, dass die zwei Schleifen mindestens einen \textbf{\emph{gemeinsamen
Knoten}} haben, wie es in Abb.\ref{SFD16}b gezeigt ist, ist die gesamte
Transmittanz gegeben durch: 
\begin{equation*}
 T_{15} =  \frac{x_5}{x_1} = \frac{P_{15}}{1-L_1-L_2} = \frac{abcd}{1-be -cf}.
\end{equation*}
Wenn wir dies mit Formel~\ref{mathSFD16z} vergleichen, sehen wir, dass
der Kreuzterm \index{Kreuzterm} $L_1L_2$ hier fehlt, weil die zwei
Kreise des SFD sich beim Knoten $x_3$ ber"uhren. {\it\textbf{Tats"achlich ist die
Frage, ob sich die Kreise ber"uhren oder nicht, sehr wichtig f"ur die
Auswertung des entsprechenden SFD.}} Diese und auch alle
 anderen F"alle k\"onnen in den zwei allgemeinen Regeln,
die als n"achstes besprochen werden, zusammengefasst werden.

\begin{figure}[htb!]
\begin{center}
  \bild{/sfd/SFD16.fig.eps,width=0.63}\caption{Mehrfache Schleifen: (a)
Sich \textbf{\emph{nicht ber"uhrende}} Schleifen, (b) Zwei sich \textbf{\emph{ber"uhrende}} Schleifen}\label{SFD16}
\end{center}
\vspace*{-7mm}
\end{figure}

\subsection{Regel 8: Die allgemeine Mehrfachschleifen-Reduktions\-regel f"ur
  einfache Pfade} Wir betrachten hier den Mehrfachschleifen-Fall, wo
\textbf{\emph{nur ein Pfad}} zwischen einer Quelle $x_i$ und einem abh"angigen Knoten
$x_j$ existiert, wobei dieser Pfad jede Schleife im SFD ber"uhrt,
(d.h., dass er wenigstens einen Knoten mit jeder Schleife gemeinsam
hat). Kurz ausgedr"uckt, lautet die Mehrfachschleifen-Reduktionsregel
f"ur einfache Pfade
\begin{equation}
 T_{ij} =   \frac{P_{ij}}{\Delta}.\label{mathSFD16b} 
\end{equation}
Die Gr"osse $\Delta$ ist die \textbf{\emph{Graph-}} oder \textbf{\emph{Netzwerkdeterminante}}.\index{Netzwerk!-determinante} \index{Grapdeterminante} $\Delta$ wird folgendermassen ermittelt:
\begin{DFrame}
$\Delta$ = 1- (Summe aller Schleifen\footnotemark) + (Summe aller Produkte zweier
Schleifen, die sich nicht ber"uhren) - (Summe
aller Produkte dreier Schleifen, die sich nicht ber"uhren) + $\ldots$ 
\end{DFrame}
\footnotetext{Der K"urze wegen sprechen wir hier von Schleifen.
  Nat"urlich sind Schleifentransmittanzen gemeint.}  F"ur einen Graph,
der $n$ Schleifen enth"alt, kann $\Delta$ ``mathematisch`` als
\begin{equation}
 \Delta = [(1-L_1)(1-L_2)(1-L_3)\ldots (1-L_n)]^{\mbox{'}}
\end{equation}
ausgedr"uckt werden, wobei der Apostroph bedeutet, dass nach Umformung
von $(1-L_1)(1-L_2)(1-L_3)\ldots (1-L_n)$ in eine Summe von Produkten
diejenigen Produkte, die sich ber"uhrende Schleifen enthalten, gleich
Null gesetzt werden.  Die Gr"osse $P_{ij}$ ist als Pfad von einer
unabh"angigen Variable\footnote{Wenn $x_i$ keine Quelle ist, kann sie
  z.B. durch Inversion in eine Quelle
  verwandelt werden, oder sonst kann die Transmittanz der realen Quelle
  (zum Beispiel $x_s$ nach $x_i$ $(T_{si})$ und nach $x_j$
$(T_{sj})$) separat erhalten werden.  Dann ist
$T_{ij}=T_{sj}/T_{si}$.} $x_i$ (Quelle) zu einer abh"angigen Variable
$x_j$ definiert.\\  Obwohl die Definition der Graph-Determinanten
ziemlich kompliziert scheint, werden einige Beispiele zeigen, wie
leicht sie berechnet werden kann.\\  

\bsp{} Als erstes Beispiel betrachten wir das SFD in Abb.~\ref{SFD18}.
Wenn wir Regel 8 anwenden, erhalten wir f"ur die UTF
\begin{eqnarray*}
  T_{17} & = & \frac{x_7}{x_1} = \frac{P_{17}}{\Delta}\\
  & = & \textstyle\frac{abcdef}{1-(L_1 + L_2 + L_3 + L_4 + L_5) + (L_1L_3 + L_1L_4 + L_1L_5 + L_2L_4 + L_2L_5 + L_3L_5)- L_1L_3L_5}\\
  & = & \textstyle\frac{abcdef}{1 -(bg+ch+di+ej+fk)+(bgdi+bgej+bgfk+chej+chfk+difk)-bgdifk}.
\end{eqnarray*}

\begin{figure}[htb!]
\begin{center}
  \bild{/sfd/SFD18.fig.eps,width=0.95}\caption{Mehrfachschleifen-Signalflussdiagramm mit einfachem
Vorw"artspfad und sich ber"uhrenden Schleifen}\label{SFD18}
\end{center}
\vspace*{-6mm}
\end{figure}
\bsp{} 
\nit Wenn wir drei sich
nicht ber"uhrende Schleifen betrachten, wie es in Abb.~\ref{SFD19} gezeigt
ist, l"asst sich die UTF durch Anwendung der
allgemeinen Reduktionsregel (Regel 8) direkt ablesen:

\begin{eqnarray*}
  T_{06} & = & \frac{x_6}{x_0} = \frac{P_{06}}{\Delta}\\
  & = & \frac{abcde}{1-(L_1 + L_2 + L_3) + (L_1L_2 + L_1L_3 + L_2L_3)- L_1L_2L_3}\\
  & = & \frac{abcde}{1-af-cg-eh+afcg+afeh+cgeh-afcgeh}.
\end{eqnarray*}
Sie muss also nicht wie in Regel 7 durch Kaskadierung der einzelnen
Schleifen errechnet werden. \\
\begin{figure}[htb!]
\vspace*{-2mm}\begin{center}
  \bild{/sfd/SFD19.fig.eps,width=0.85}\caption{Drei
    \textbf{\emph{sich nicht ber"uhrende}} Schleifen}\label{SFD19}
\end{center}
\vspace*{-7mm}
\end{figure}
\bsp{}
\nit Als letztes Beispiel betrachten wir das SFD in
Abb.~\ref{SFD20} mit mehreren Schleifen. Durch Anwendung von Regel 8
erhalten wir die "Ubertragungsfunktion:\\
\begin{equation*}
  T_{16} = \frac{x_6}{x_1} =  \frac{P_{16}}{\Delta} = \frac{abcde}{1-(L_1+L_2+L_3)+L_2L_3} = \frac{abcde}{1-bcdg -cf -eh +cfeh}.
\end{equation*}\\
\nit Der Tatsache, dass die Schleife $L_2$ innerhalb der Schleife $L_1$
liegt, kommt also keine spezielle Bedeutung zu. \\
\begin{figure}[htb!]
\vspace*{-3mm}\begin{center}
  \bild{/sfd/SFD20.fig.eps,width=0.65}\caption{Mehrfachschleifen-Signalflussdiagramm mit einfachem Pfad}\label{SFD20}
\end{center}
\vspace*{-7mm}
\end{figure}
 
\subsection{Regel 9: Die allgemeine Mehrfachschleifen-Reduktionsregel f"ur Mehrfachpfade (Mason's Regel)}\index{Mason's Regel}
Wenn ein SFD mehr als einen Pfad von der Quelle zur
Senke (oder zu einem dazwischenliegenden Knoten) enth"alt, wie es in
Abb.~\ref{SFD21} gezeigt ist, kann die UTF folgendermassen berechnet werden: 

\begin{eqnarray*}
  T_{16} & = & \frac{x_6}{x_1} = \frac{P_{16}}{\Delta} = T_1 + T_2 = \frac{P_1}{1-L_1}+ \frac{P_2}{1-L_2}=\frac{P_1}{\Delta_2}+ \frac{P_2}{\Delta_1}\\
  & = & \frac{P_1\Delta_1+P_2\Delta_2}{\Delta} = \frac{abc(1-eh) + def(1-bg)}{1-bg-eh+bgeh}
\end{eqnarray*}
\nit wobei $\Delta=\Delta_1\cdot\Delta_2, \Delta_1=1-L_2, \Delta_2=1-L_1,T_1 = P_1/(1-L_1), T_2 = P_2/(1-L_2)$ ist.  
\begin{figure}[htb!]
\begin{center}
  \bild{/sfd/SFD21.fig.eps,width=0.6}\caption{Mehrfachschleifen-Signalflussdiagramm mit Mehrfachpfad}\label{SFD21}\end{center}
\vspace*{-7mm}
\end{figure}
\nit Beachten Sie, dass jeder Pfad $P_k$ mit $\Delta_k$ multipliziert wird, das heisst, mit der Determinante desjenigen Teils des Graphs, der den
$k$-ten Pfad nicht ber"uhrt. Diese Modifikation von $P_k$ erlaubt uns
die Bedingung in Regel 8, dass der zu betrachtende Pfad ``jede
Schleife im SFD ber"uhrt'', fallen zu lassen. Wir
k"onnen daher die vorherigen Resultate (Regel 8) verallgemeinern, um
die folgende Formel f"ur die UTF eines beliebigen
SFD zu erhalten:
\begin{DFrame}
{\bf Mason's Regel}
\begin{equation}
  T_{ij} = \frac{\sum\limits_k P_k\cdot\Delta_k}{\Delta},\label{mason}
\end{equation}
wobei $P_k$ die Transmittanz des $k$-ten offenen Pfades
zwischen $x_i$ und $x_j$ ist; $x_i$ eine Quelle, aber $x_j$ nicht
notwendigerweise eine Senke sein muss, sondern auch ein gemischter
Knoten sein kann.  $\Delta$ ist die Graph- oder Netzwerkdeterminante.  $\Delta_k$
ist der Kofaktor des $k$-ten Pfades, also die Determinante desjenigen
Teils des Graphen, der den $k$-ten Pfad nicht ber"uhrt, das heisst, dass
er keinen gemeinsamen Knoten mit dem $k$-ten offenen Pfad hat.  
\end{DFrame}\\
\nit Die Gleichung~\ref{mason} stellt die allgemeinste Beziehung dar,
welche die UTF eines SFD definiert. Sie kann direkt, ohne vorg"angige
Reduktionsprozesse auf irgendein SFD angewandt werden.  Bei
Sensitivit"ats- oder Stabilit"atsbetrachtungen hingegen w"are es
n"utzlicher, das allgemeine SFD auf ein fundamentales SFD, zu
reduzieren (siehe Abschnitt~\ref{sfd_sec_fund}). Zuerst werden wir die Anwendung von Formel~\ref{mason}
mit einigen typischen Beispielen illustrieren.\\

\clearpage\vspace*{-9mm}
\bsp{} Betrachten Sie das SFD in Abb.~\ref{SFD22}. Wenn man
Formel~\ref{mason} anwendet, um die UTF zu berechnen, erh"alt man:
\begin{equation*}
  T_{18} = \frac{x_8}{x_1} = \frac{P_{18}\Delta_{18}}{\Delta}
\end{equation*}
wobei 
\begin{equation*}
  \Delta = 1 - \sum\limits_i L_i + \sum\limits_{i,j}L_i^{\mbox{,}}L_j^{\mbox{,}}-\sum\limits_{i,j,k}L_i^{\mbox{,,}}L_j^{\mbox{,,}}L_k^{\mbox{,,}}+\ldots
\end{equation*}
ist, und $\sum\limits_i L_i $ die Summe
der Transmittanzen aller Schleifen ist; $L_i^{\mbox{,}}L_j^{\mbox{,}}$ das Produkt der
Transmittanzen jedes Schleifenpaares ist, das keinen Knoten oder Zweig
gemeinsam hat; $\sum\limits_{i,j}L_i^{\mbox{,}}L_j^{\mbox{,}}$ die Summe all dieser Produkte ist; $L_i^{\mbox{,,}}L_j^{\mbox{,,}}L_k^{\mbox{,,}}$ das
Produkt der Transmittanzen jedes Schleifentripletts ist, das keinen
Knoten oder Zweig gemeinsam hat; und so weiter. \\
\begin{figure}[htb!]
\vspace*{-3mm}\begin{center}
  \bild{/sfd/SFD22.fig.eps,width=0.87}\caption{SFD mit mehreren Schleifen}\label{SFD22}
\end{center}\vspace*{-9mm}
\end{figure}\\
\nit F"ur einen komplexen Graphen wie denjenigen in Abb.~\ref{SFD22}, ist
es am sichersten, die Summen und Produkte von $\Delta$ systematisch zu
bestimmen. Folglich beginnen wir mit Zweig $a$ und f"uhren alle
Schleifen auf, die $a$ enthalten, und gehen weiter zu Zweig $b$, und
nehmen alle Schleifen, die $b$ enthalten, die noch nicht aufgef"uhrt
wurden und so weiter, um $\sum\limits_i L_i$ zu finden. Somit ist 
\begin{equation*}
 \sum\limits_i L_i = bch + cdek + cdefl + defgi + efj.
\end{equation*}
Als n"achstes finden wir
$\sum\limits_{i,j}L_i^{\mbox{,}}L_j^{\mbox{,}}$, indem wir die Summe
aller Produkte zweier Schleifen bilden und alle diejenigen weglassen,
die irgendeinen Knoten oder Zweig gemeinsam haben. Um
$\sum\limits_{i,j}L_i^{\mbox{,}}L_j^{\mbox{,}}$ zu berechnen,
untersuchen wir die Liste von $\sum\limits_i L_i$. Der erste Ausdruck
in $\sum\limits_i L_i$ ist $bch$. Die anderen Ausdr"ucke werden
durchsucht, um diejenigen zu finden, welche weder $b$ noch $c$ noch
$h$ enthalten. Der erste Term, welcher diese Bedingung erf"ullt, ist
$defgi$, aber wenn wir den Graph pr"ufen, merken wir, dass er einen
Knoten gemeinsam mit $bch$ hat.  Jedoch hat der letzte Term $efj$
keine Zweige oder Knoten gemeinsam mit $bch$, und deshalb erscheint
$bchefj$ in $\sum\limits_{i,j}L_i^{\mbox{,}}L_j^{\mbox{,}}$. Als
n"achstes betrachten wir den zweiten Term in $\sum\limits_i L_i$, das
heisst $cdek$, und pr"ufen wie oben.  Auf diese Art finden wir: 
\begin{equation*}
 \sum\limits_{i,j}L_i^{\mbox{,}}L_j^{\mbox{,}}=bch\cdot efj.
\end{equation*}
Da $\sum\limits_{i,j}L_i^{\mbox{,}}L_j^{\mbox{,}}$ nur einen Term
enth"alt, ist es unm"oglich, dass
$\sum\limits_{i,j,k}L_i^{\mbox{,,}}L_j^{\mbox{,,}}L_k^{\mbox{,,}}$
existiert. Somit ist 
\begin{equation*}
 \Delta = 1-\sum\limits_i L_i+\sum\limits_{i,j}L_i^{\mbox{,}}L_j^{\mbox{,}}=1 - (bch+cdek+cdefl+defgi+efj)+bchefj.
\end{equation*}
Wenn wir mit der Berechnung von $P_{18}$ und $\Delta_{18}$ fortfahren,
erhalten wir $P_{18}=abcdefg$. Da jede Schleife in $\sum\limits_i L_i$
mindestens einen Zweig gemeinsam mit $P_{18}$ hat, ist
$\Delta_{18}=1$, und $P_{18}\Delta_{18}=abcdefg$.  Die Tatsache, dass
$\Delta_{18}=1$ ist, ist aus dem SFD unmittelbar ersichtlich. Somit
erhalten wir $T_{18}=(P_{18}\Delta_{18})/\Delta=abcdefg/\Delta.$
Nat"urlich ist Formel~\ref{mason} allgemein genug, um die Analyse von
Mehrfachschleifen-Graphen mit einfachem Pfad ebenso einzuschliessen.
Wenn wir die Ausdr"ucke, die wir oben erhalten haben, einsetzen,
erhalten wir die UTF $T_{18}$.\\

\bsp{}
\nit Um die Anwendung von Formel~\ref{mason} auf echte
Mehrfachschleifen-Graphen mit Mehrfachpfad zu veranschaulichen,
betrachten wir nun das SFD in Abb.~\ref{SFD23}. Mit Formel~\ref{mason}
erhalten wir
\begin{equation*}
 T_{12} = \frac{x_2}{x_1} = \frac{\sum\limits_k P_k\Delta_k}{\Delta} = \frac{P_1\Delta_1+ P_2\Delta_2}{\Delta} = \frac{P_1(1-L_2)+P_2}{1-L_1-L_2} = \frac{a(1-ef)+cd}{1-db-ef}.
\end{equation*} $\Delta_2$ ist gleich eins, da $P_2= cd$ jede Schleife des Graphen
ber"uhrt.\\
\begin{figure}[htb!]
\vspace*{-3mm}\begin{center}
  \bild{/sfd/SFD23.fig.eps,width=0.61}\caption{Mehrfachschleifen-Graph mit Mehrfachpfad}\label{SFD23}
\end{center}
\vspace*{-9mm}
\end{figure}

\clearpage
\bsp{}
\nit Ein anderer Mehrfachschleifen-Graph mit Mehrfachpfad ist in
Abb.~\ref{SFD24} gezeigt. ~\\
\begin{figure}[htb!]
\vspace*{-5mm}\begin{center}
  \bild{/sfd/SFD24.fig.eps,width=0.61}\caption{Mehrfachschleifen-Graph mit Mehrfachpfad}\label{SFD24}
\end{center}\vspace*{-7mm}
\end{figure}~\\
\nit Mit Formel~\ref{mason} (Mason's Regel)\index{Mason's Regel} lassen sich die folgenden
UTF herauslesen:
\begin{eqnarray*}
  T_{12} & = & \frac{x_2}{x_1} = \frac{P_1\Delta_1 + P_2\Delta_2 + P_3\Delta_3}{\Delta} = \frac{a(1-cg)+de+dcf}{1-be-bcf-cg}\\
   T_{13} & = & \frac{x_3}{x_1} = \frac{P_4\Delta_4 + P_5\Delta_5}{\Delta} = \frac{ab+d}{1-be-bcf-cg}\\
  T_{14} & = & \frac{x_4}{x_1} = \frac{P_6\Delta_6 + P_7\Delta_7}{\Delta} = \frac{abc+dc}{1-be-bcf-cg}
\end{eqnarray*}
\nit {\it\textbf{Beachten Sie, dass die
    Netzwerkdeterminante\index{Netzwerk!-determinante} $\Delta$ f"ur
    einen gegebenen Graph unabh"angig von der UTF ist und deshalb nur
    einmal berechnet werden muss.}}\\
\nit {\it\textbf{Bemerkung:}} Z.B. kann die UTF von $x_2$ nach $x_3$ nicht ``direkt`` mit Mason's Formel bestimmt werden. Mit dem folgenden Zusammenhang
\begin{equation*}
 T_{23}=\frac{x_3}{x_2}=\frac{x_3}{x_1}\cdot\frac{x_1}{x_2}=\frac{T_{13}}{T_{12}}=\frac{\frac{P_4\Delta_4 + P_5\Delta_5}{\Delta} }{\frac{P_1\Delta_1 + P_2\Delta_2 + P_3\Delta_3}{\Delta} }=\frac{P_4\Delta_4 + P_5\Delta_5}{P_1\Delta_1 + P_2\Delta_2 + P_3\Delta_3}=\frac{ab+d}{a(1-cg)+de+dcf}
\end{equation*}
ist aber ersichtlich, dass f"ur die Berechnung von UTFs von einem Knoten aus zu einem anderen Knoten jeweils der Z"ahler der entsprechenden UTFs (hier $T_{12}$ und $T_{13}$) von der Quelle her bestimmt und durcheinander dividiert werden m"ussen.

\clearpage
\bsp{Allgemeines SFD}
\begin{figure}[htb!]\vspace*{-7mm}
\begin{center} 
  \bild{/sfd/SFD1.fig.eps,width=0.63}\caption{Allgemeinen SFD}\label{SFD_alles}
\end{center}
\vspace*{-9mm}
\end{figure}

\begin{enumerate}
\item[a)] Die UTF\index{UTF} zwischen $X_1$ und $X_4$ ist (mit Mason's Regel):\index{Mason's Regel}\\
\begin{equation*}
H_{14}=\frac{X_4}{X_1}=\frac{aeh+abc(1-g)}{1-ef-g}
\end{equation*}
\item[b)] Das folgende Gleichungssystem beschreibt das SFD von Abb.~\ref{SFD_alles}.
\begin{eqnarray*}
X_2 &=&a\cdot X_1+f\cdot X_3\\
X_3 &=&e\cdot X_2+g\cdot X_3\\
X_4 &=&h\cdot X_3+c\cdot X_6\\
X_5 &=&d\cdot X_4\\
X_6 &=&b\cdot X_2\\
\end{eqnarray*}
Nach Umformung der Gleichungen erhalten wir:
\begin{equation*}
X_4=h\cdot X_3+\frac{bc}{e}\cdot (1-g)\cdot X_3\quad\text{und}\quad X_3\cdot \frac{1-g}{e}=a\cdot X_1+f\cdot X3.
\end{equation*}
Somit ist $X_4=\frac{h+\frac{bc}{e}(1-g)}{\frac{1-g}{ae}-\frac{f}{a}}X_1=\frac{aeh+abc(1-g)}{1-g-ef}X_1$.
\item[c)] Mit der symbolischen Toolbox von \matlogo~ erhalten wir:\index{Toolbox!symbolisch}\index{MATLAB@{\matlogo}}
\begin{verbatim}
>> syms a b c d e f g h X1 X2 X3 X4 X5 X6
Loesungen=solve('X2=a*X1+f*X3','X3=e*X2+g*X3','X4=h*X3+c*X6',...
'X5=d*X4','X6=b*X2',X2,X3,X4,X5,X6);
>> Loesungen.X4
ans = 
-a*X1*(e*h+c*b-c*b*g)/(e*f-1+g)
\end{verbatim}
\end{enumerate}\index{syms@{\tt syms}}\index{solve@{\tt solve}}

\aufg
Bestimmen Sie die UTF $H_{65}=\frac{X_5}{X_6}$ von Abb.~\ref{SFD_alles}.

%\newpage
\section{Das fundamentale SFD \index{Signalflussdiagramm!fundamentales}}\label{sfd_sec_fund}Aus den vorherigen
Abschnitten ist ersichtlich, dass die Komplexit"at\index{Komplexitat@Komplexit{\"a}t}
eines SFD eher das Resultat der Komplexit"at eines bestimmten
Gleichungssystems, das ein gegebenes System beschreibt, als eine
Angabe f"ur die Komplexit"at des Systems selbst ist. Wir brauchen nur
die zwei SFD, die in Abb.~\ref{SFD25} gezeigt sind, zu betrachten, um
zu sehen, dass die Zahl der R"uckkopplungsschleifen keine Angabe der
Komplexit"at des Systems ist.  Die zwei Graphen stellen das gleiche
System dar, jedoch zeigt Abb.~\ref{SFD25}a zwei
R"uckkopplungsschleifen, w"ahrend
Abb.~\ref{SFD25}b nur eine zeigt.\\
\begin{figure}[htb!]
\vspace*{-6mm}\begin{center}
  \bild{/sfd/SFD25.fig.eps,width=0.95}\caption{Zwei Signalflussdiagramme von verschiedener Komplexit"at,
die die Gleichung $x_2/x_1=a/(1-a(b+c))$ darstellen. (a) Zwei Schleifen. (b) Eine
Schleife}\label{SFD25}
\end{center}\vspace*{-7mm}
\end{figure}\\
\nit Der Begriff der {\it\textbf{Ordnung}}\index{Ordnung} kann erkl"art werden,
indem man {\it\textbf{fundamentale Knoten}} \index{Knoten!fundamentale} eines SFD
definiert, das heisst diejenigen Knoten, welche entfernt werden
m"ussen, um alle R"uckkopplungsschleifen zu "offnen.  Die Entfernung
eines Knotens darf nicht mit der Eliminierung eines Knotens gem"ass
Regel 7 verwechselt werden. Sie muss hier als Gleichsetzen des Wertes
einer Knotenvariable mit Null oder gleichwertig mit der L"oschung
aller weggehenden Zweige interpretiert werden. Dann k"onnen wir
festlegen, dass die {\it\textbf{Ordnung eines SFD}} der {\it\textbf{Anzahl}} der {\it\textbf{fundamentalen
Knoten}} entspricht, das heisst der Zahl der Knoten, welche entfernt
werden m"ussen, um alle Schleifen aufzubrechen.  Beispiele von
Signalflussdiagrammen der
Ordnung 1, 2 und 3 sind in Abb.~\ref{SFD26} gezeigt.\\
\begin{figure}[htb!]
\vspace*{-3mm}\begin{center}
  \bild{/sfd/SFD26.fig.eps,width=0.81}\vspace*{-4cm}\hspace*{-8cm}
\begin{pspicture}(14,2)
\psset{arrowscale=2}\psset{arrowinset=0}
 \psline{->}(4.53,3.3)(4.53,3.6) 
\end{pspicture}\vspace*{4cm}
\caption{Beispiele fundamentaler Signalflussdiagramme verschiedener Ordnung. (a) 1. Ordnung (b) 2. Ordnung (c) 3. Ordnung}\label{SFD26}
\end{center}\vspace*{-7mm}
\end{figure}\\
\nit In einem komplexen SFD mit Ordnung gr"osser als 2 ist es oft
schwierig, die Ordnung zu bestimmen; man braucht eine
``trial-and-error'' Analyse, um die Anzahl fundamentaler Knoten zu
finden. Jedoch ist es im Allgemeinen nicht sehr wichtig, die genaue
Ordnung eines Systems ermitteln zu k"onnen. Wenn die Ordnung, die
bestimmt wurde, f"alschlicherweise zu hoch ist, ist die Analyse
grunds"atzlich davon nicht betroffen, obwohl sie schwieriger sein
kann, als es gewesen w"are, wenn wir erkannt h"atten, dass der Graph
eine niedrigere Ordnung besitzt. Gl"ucklicherweise ist in den meisten
F"allen von R"uckkopplungssystemen, die man in der Praxis antrifft,
die Ordnung kleiner oder gleich 2.  Eine grundlegende Charakteristik
eines fundamentalen Knotens ist, dass seine Eliminierung (im Gegensatz
zu seiner Entfernung) eine Division aller einm"undenden Zweige durch
den Term $(1-L)$ zur Folge hat, wobei $L$ die Schleifentransmittanz
darstellt. Alle anderen Knoten k"onnen durch Additionen und
Multiplikationen von Transmittanzen des urspr"unglichen Graphes
eliminiert werden\footnote{Durch Netzwerkgleichungen ausgedr"uckt,
  entspricht die Reduktion zu einem fundamentalen SFD einer
  Eliminierung von Variablen in den urspr"unglichen Gleichungen durch
  direkte Substitution\index{Substitution!direkte}. Es ist keine Division von Koeffizienten
  notwendig.}. Als eine Folge davon gilt: Wenn alle Transmittanzen im
urspr"unglichen Graph stabil sind (das heisst, dass es keine Pole auf
oder ausserhalb des Einheitskreises in der $z$-Ebene oder in der
rechten Halbebene des $s$-Bereichs gibt), sind auch alle fundamentalen
Transmittanzen stabil\footnote{Dies trifft nicht mehr zu, wenn wir ein
  SFD 2.~Ordnung zu einem SFD 1.~Ordnung
  reduzieren.}. F"ur die Stabilit"ats\index{Stabilitat@Stabilit\"at}- und
Sensitivit"ats\index{Sensitivitat@Sensitivit\"at}-Analyse sollte der einfachere
fundamentale Graph verwendet werden anstelle des komplexeren Graphen
mit Mehrfachschleifen, der direkt von einem gegebenen System erstellt
wurde.
\subsection{Das fundamentale Signalflussdiagramm 1. Ordnung}\index{Signalflussdiagramm!fundamentales} Im
Folgenden werden wir zuerst zeigen, dass die vier grundlegenden
Transmittanzen des fundamentalen SFD 1.~Ordnung in einem Schritt von
einem Graphen mit Mehrfachschleifen abgeleitet werden k"onnen und
nicht mehrere Schritte mit den Reduktionsregeln notwendig sind.\\
\begin{figure}[htb!]
\vspace*{-8mm}\begin{center}
  \bild{/sfd/SFD27.fig.eps,width=0.26}\caption{Fundamentales Signalflussdiagramm 1. Ordnung}\label{SFD27}
\end{center}
\vspace*{-7mm}
\end{figure}~\\
\nit Betrachten wir das fundamentale SFD 1.~Ordnung, das in Abb.~\ref{SFD27} gezeigt
ist. Neben den Eingangs- und Ausgangsknoten entspricht im
fundamentalen SFD die Anzahl fundamentaler Knoten der Ordnung des
Graphen. In diesem Fall hat es einen einzigen fundamentalen Knoten
$E_x$ (abgesehen von den Eingangs- und Ausgangsknoten $E_i$ und
$E_o$).  Des Weiteren gibt es in einem fundamentalen Graphen 1.
Ordnung nur vier m"ogliche Transmittanzen $t_{io}$, $t_{ix}$, $t_{xo}$
und $t_{xx}$. Die Bestimmung des fundamentalen SFD
reduziert sich auf die Berechnung dieser vier
Transmittanzen\index{Transmittanz}. Es wird nun in einem Beispiel
gezeigt, dass diese direkt durch Betrachten des urspr"unglichen
SFD erhalten werden k"onnen.
\bsp{}Im SFD in Abb.~\ref{SFD28} k"onnen wir den Knoten $x_7$
sofort als fundamentalen Knoten identifizieren. Unsere Aufgabe ist es
nun, die fundamentalen Transmittanzen von Abb.~\ref{SFD27} in Abh"angigkeit
der Zweigtransmittanzen von Abb.~\ref{SFD28} zu finden.  Die fundamentale
Transmittanz $t_{io}$ misst die totale "Ubertragung vom Eingang zum
Ausgang, ohne durch den Knoten $E_x$ zu gehen. Offensichtlich gibt es
in Abb.~\ref{SFD28} nur einen solchen Pfad, n"amlich den horizontalen vom
Eingang zum Ausgang; jedes Abschweifen nach unten f"uhrt zu $E_x$.
Somit ist $t_{io}=abcde$.\\  
\begin{figure}[htb!]
\begin{center}
  \bild{/sfd/SFD28.fig.eps,width=0.85}\caption{Signalflussdiagramm 1.
Ordnung}\label{SFD28}
\end{center}
\vspace*{-7mm}
\end{figure}\\
\nit Die totale Transmittanz $t_{ix}$ vom Eingang $x_1$ zu $E_x$, ohne
von $E_x$ auszugehen, umfasst zwei parallele Pfade. Im Vergleich mit
Abb.~\ref{SFD27} erhalten wir $t_{ix}=ab(h+cdk)$. Die Transmittanz
$t_{xo}$ vom fundamentalen Knoten $E_x$ zum Ausgangsknoten, ohne zum
Knoten $E_x$ zur"uckzukehren, ist die Summe der vier einzelnen
Transmittanzen, die die vier m"oglichen Pfade darstellen, die von
$E_x$ ausgehen: $t_{xo}=fbcde+gcde+ide+je$. Schliesslich ist $t_{xx}$
die Summe der Transmittanzen der vier Pfade, welche von $E_x$
wegf"uhren und wieder zu $E_x$ zur"uckkehren:
$t_{xx}=fb(h+cdk)+g(h+cdk)+idk+jk$. Beachten Sie, dass gewisse Terme
in den zwei letzten Gleichungen kombiniert werden k"onnen. Als
allgemeines Verfahren jedoch ist es sinnvoller, zuerst die Pfade, die
von jedem Knoten wegf"uhren, individuell zu betrachten.
\newpage
\bsp{}\label{SFD_bsp_fund}Zwei weitere Beispiele sollen helfen, die Reduktion eines
 SFD zu seiner fundamentalen Form zu illustrieren. Der
Graph in Abb.~\ref{SFD29}a hat trotz dreier R"uckkopplungsschleifen
die Ordnung eins, da die Entfernung des Knotens 3 alle Schleifen
"offnet. \\
\begin{figure}[htb!]
\begin{center}
  \bild{/sfd/SFD29.fig.eps,width=0.85}\caption{Signalflussdiagramm 1.  Ordnung.  (a) Urspr"unglicher Graph. (b)
"Aquivalenter fundamentaler Graph}\label{SFD29}
\end{center}
\vspace*{-7mm}
\end{figure}\\
\nit Der fundamentale Knoten des fundamentalen
Signalflussdiagramms muss deshalb Knoten~3 sein, und es wird die Form
des fundamentalen Graphen in Abb.~\ref{SFD29}b haben. Die Betrachtung von
Abb.~\ref{SFD29}a liefert die vier grundlegenden Transmittanzen:\\
\begin{description}
\item $t_{15}=$ (Leckpfad,\index{Leckpfad} das heisst alle Pfade vom Eingang zum Ausgang, welche nicht
durch Knoten 3 f"uhren) $= ah$ 
\item $t_{13}=$ (alle Pfade vom Eingang zum Knoten
3) $= ab$ 
\item $t_{35}=$ (alle Pfade vom Knoten 3 zum Ausgang) $= cd + eh + cgh$
\item $t_{33}=$ (alle Eigenschleifen des Knotens 3) $= eb + cf + cgb$
\end{description}
\nit Nach Ermittlung aller Transmittanzen irgendeines fundamentalen
SFD 1.~Ordnung folgt die UTF direkt
aus Abb.~\ref{SFD27}:
\begin{equation}
 \frac{E_o}{E_i}=t_{io}+\frac{t_{ix}t_{xo}}{1-t_{xx}}=\frac{t_{io}-t_{io}t_{xx}+t_{ix}t_{xo}}{1-t_{xx}}.\label{sfd_equ1}
\end{equation}
\nit Somit ergibt sich f"ur $T_{15}$ von Abbildung~\ref{SFD29} 
\begin{equation*}
 T_{15}=\frac{abcd+ah(1-cf)}{1-bcg-be-cf}.
\end{equation*}
\aufg
Kontrollieren Sie das Resultat $T_{15}=\frac{abcd+ah(1-cf)}{1-bcg-be-cf}$ von Bsp.~\ref{KAP_SFD}.\ref{SFD_bsp_fund} indem Sie direkt Mason's Regel (Formel~\ref{mason}) anwenden.

\newpage\section{Einbezug analoger Verst"arker}
Im Fall von ``sampled-data'' Systemen (z.B.  ``switched-capacitor''
(SC)\index{SC} oder ``switched-current''(SI)\index{SI} Filter), werden
zeitdiskrete, geschaltete Netzwerke, die aus Schaltern und
Kondensatoren bestehen, mit analogen Verst"arkern wie
Operationsverst"arkern (OP\index{OP}) oder Transkonduktanzverst"arkern
(OTA\index{OTA}) kombiniert. Wie bei den meisten anderen Filtertypen
(z.B. digitale Filter, aktive $RC$-Filter) sind SFD nicht nur f"ur die
Analyse n"utzlich, sondern auch f"ur die Synthese. Zuerst betrachten
wir die SFD von analogen aktiven
Bauteilen und ihre Kombination mit geschalteten Netzwerken.
\begin{figure}[htb!]
\vspace*{-3mm}\begin{center}
  \bild{/sfd/SFD30.fig.eps,width=0.7}\caption{Symbolische Darstellung einer typischen SC-Schaltung mit OP}\label{SFD30}
\end{center}
\vspace*{-7mm}
\end{figure}\\
\nit Betrachten wir die in Abb.~\ref{SFD30} dargestellte
SC-Netzwerktopologie.  Wir nehmen an, dass die Spannungsquelle am
Eingang ideal ist und dass, wegen der hohen Verst"arkung des OP, der
OP-Eingang $V_2$ virtuell auf Masse\index{Masse!virtuelle} ist. Wenn wir die Transmittanzen
und Signale $z$-transformieren, kann die Schaltung von
Abb.~\ref{SFD30} durch das Diagramm in Abb.~\ref{SFD31} dargestellt
werden, wobei die $Z_i(z)$ zeitabh"angige, zeitdiskrete
Impedanz-"Ubertragungsfunktionen sind.
\begin{figure}[htb!]
\begin{center}
  \bild{/sfd/SFD31.fig.eps,width=0.7}\caption{SC-Inverterschaltung mit $z$-transformierten
Impedanzen}\label{SFD31}
\end{center}
\vspace*{-7mm}
\end{figure}\\
\nit Um das SFD einer (zeitkontinuierlichen oder zeitdiskreten)
Schaltung, die einen OP enth"alt, zeichnen zu k"onnen, brauchen wir
das SFD des OP. Ein OP mit endlicher Verst"arkung $A$ kann, wie in
Abb.~\ref{SFD32}a oder b gezeigt, dargestellt werden. Eine einfache
Methode, unendliche Verst"arkung mit endlichen Transmittanzen
(Verst"arkung 1) zu modellieren, ist in Abb.~\ref{SFD32}c gezeigt.
Wie schon fr"uher erw"ahnt, ist die SFD-Darstellung eines Systems
nicht eindeutig. Sie h"angt vom Gleichungssystem, das das System oder
die Schaltung beschreibt, ab. Das kann sogar f"ur die einfache
``sampled-data'' OP-Inverterschaltung von Abb.~\ref{SFD31} gezeigt
werden. F"ur die Gleichungen:
\begin{eqnarray}
  1) & & I_1 = (V_1-V_2)/Z_1 \nonumber \\
  2) & & V_2 = -V_3/A \label{mathSFD33}\\
  3) & & V_3 = V_2-Z_2\cdot I_1 \nonumber
\end{eqnarray}
erhalten wir das SFD in Abb.~\ref{SFD33}.
\begin{figure}[htb!]
\vspace*{-8mm}\begin{center}
  \bild{/sfd/SFD32.ps,width=0.65}\caption{SFD-Darstellung eines OP mit endlicher und unendlicher
Verst"arkung}\label{SFD32}
\end{center}
\vspace*{-7mm}
\end{figure}

\begin{figure}[htb!]
\begin{center}
  \bild{/sfd/SFD33.fig.eps,width=0.42}\caption{SFD der Schaltung von Abb.~\ref{SFD31} gem"ass den Gleichungen \ref{mathSFD33}}\label{SFD33}
\end{center}
\vspace*{-7mm}
\end{figure}
F"ur eine andere Formulierung derselben Gleichungen, n"amlich\\
\begin{eqnarray}
  1) & & V_2 = V_1 -Z_1\cdot I_1 \nonumber \\
  2) & & V_3 = -A\cdot V_2 \label{mathSFD34}\\
  3) & & I_1 = (V_2-V_3)/Z_2  \nonumber
\end{eqnarray}
erhalten wir das SFD in Abb.~\ref{SFD34}.
\begin{figure}[htb!]
\begin{center}
  \bild{/sfd/SFD34.fig.eps,width=0.42}\caption{SFD der Schaltung von Abb.~\ref{SFD31} gem"ass den Gleichungen \ref{mathSFD34}}\label{SFD34}
\end{center}
\vspace*{-7mm}
\end{figure}
In diesem Beispiel wurde eine neue Variable $(V_2-V_3)$ eingef"uhrt.
Nat"urlich resultiert mit der Formel von Mason f"ur beide SFD aus
Abb.~\ref{SFD33} und Abb.~\ref{SFD34} die gleiche UTF, n"amlich:
\begin{equation}
 \frac{V_3}{V_1} = -\frac{Z_2}{Z_1}\cdot \left. \frac{A}{1+Z_2/Z_1+A}\right |_{A\rightarrow\infty} = -\frac{Z_2}{Z_1}. \label{mathSFD35}
\end{equation}
In zeitkontinuierlichen OP-Schaltungen ist dies die UTF eines
invertierenden Verst"arkers.  Eine andere L"osung in der SFD-Analyse
besteht darin, ein physikalisch vern"unftiges SFD aufzustellen, und
dann die individuellen Transmittanzen abzuleiten, indem die
physikalische Natur des Systems oder Netzwerks in Betracht gezogen
wird. Im Falle der Schaltung von Abb.~\ref{SFD31} ist das SFD, das in
Abb.~\ref{SFD35} gezeigt ist, eine vern"unftige Anordnung.\\
\begin{figure}[htb!]
\vspace*{-2mm}
\begin{center}
  \bild{/sfd/SFD35.fig.eps,width=0.39}\caption{Postuliertes SFD f"ur die Schaltung aus Abb.~\ref{SFD31}}\label{SFD35}
\end{center}
\vspace*{-7mm}
\end{figure}\\
\nit Von diesem SFD erhalten wir die Transmittanzen $t_{12}$ und $t_{32}$
folgendermassen: $V_2=t_{12}V_1+t_{32}V_3$. Folglich ist:
\begin{equation*}
 t_{12} =  \left.\frac{V_2}{V_1}\right |_{V_3=0}; \quad t_{32} =  \left.\frac{V_2}{V_3}\right |_{V_1=0}
\end{equation*}\\
\nit und aus Abb.~\ref{SFD31} folgt, dass
\begin{equation*}
 t_{12} =  \frac{Z_2}{Z_1+Z_2}\quad \mbox{und}\quad t_{32} = \frac{Z_1}{Z_1+Z_2}
\end{equation*}\\
\nit ist. Wir erkennen, dass dies nichts anderes ist als die Anwendung
des Superpositionsprinzipes\index{Superposition} auf die lineare
Schaltung von Abb.~\ref{SFD31}, in welcher wir $V_1$ und $V_3$ als
Ausg"ange von idealen Spannungsquellen\index{Spannungsquelle!ideal}
annehmen, was in der Praxis mehr oder weniger der Fall ist. Wenn wir
die Regel von Mason\index{Mason} anwenden, erhalten wir -- wie
erwartet -- die UTF nach Gleichung \ref{mathSFD35}.  Wenn die
Verst"arkung des OP unendlich und somit der Eingang
virtuell\index{Masse!virtuelle} auf Masse ist, und wenn wir das
SFD von Abb.~\ref{SFD32}c einsetzen, erhalten wir f"ur die Schaltung
von Abb.~\ref{SFD31} schliesslich das SFD von Abb.~\ref{SFD36}. Dies
gibt nat"urlich direkt die UTF f"ur unendliche
Verst"arkung\footnote{Da die Verst"arkung des OP unendlich ist, ist
  die Transmittanz des Pfades zwischen $I_1$ und $V_3$ beliebig,
  solange sie endlich und ungleich Null ist. Wir w"ahlen
  -1 $(\Omega)$ mehr aus intuitiven als irgendwelchen anderen Gr"unden.}, n"amlich $V_3/V_1=-Z_2/Z_1$.\\
\begin{figure}[htb!]
\vspace*{-3mm}\begin{center}
  \bild{/sfd/SFD36.fig.eps,width=0.45}\caption{SFD
f"ur die Schaltung von Abb.~\ref{SFD31} mit einem OP mit unendlicher
Verst"arkung}\label{SFD36}
\end{center}
\vspace*{-7mm}
\end{figure}
\newpage
\section{Inversion und Transposition} In Regel 5 in
Abschnitt~\ref{abschnitt13} f"uhrten wir den Begriff der Pfadinversion
einer Quelle ein und zeigten, wie die stufenweise Inversion eines
ganzen SFD mit einem ``invertierten'' SFD endet. Es ist wichtig,
zwischen Inversion und Transposition eines SFD zu unterscheiden, weil,
trotz "Ahnlichkeiten, das Resultat ziemlich verschieden ist. Das SFD
in Abb.~\ref{SFD37}a hat folgende UTF:
\begin{equation}
 T_{14} = \frac{x_4}{x_1} = \frac{abc}{1-bce-cd}.\label{mathSFD37}
\end{equation}
\begin{figure}[htb!]
\vspace*{-7mm}\begin{center}
  \bild{/sfd/SFD37.ps,width=0.55}\caption{Inversion und Transposition eines SFD}\label{SFD37}
\end{center}
\vspace*{-7mm}
\end{figure}\\
\nit Wenn wir die stufenweise Inversion gem"ass (Regel 5) durchf"uhren,
erhalten wir das SFD von Abb.~\ref{SFD37}b.  Typischerweise ergibt
sich eine nichtkausale \index{nichtkausal} Darstellung des
urspr"unglichen Systems oder Netzwerks, indem das Treibersignal $x_1$
zur Senke\index{Senke} und das Ausgangssignal zur Quelle\index{Quelle}
wird.  Des weiteren, wie allgemein nach einer Inversion, enth"alt das
resultierende SFD nur Vorw"artspfade und keine
R"uckkopplungsschleifen. Die resultierende nichtkausale
"Ubertragungsfunktion ist die Summe aller Pfade von der neuen Quelle
$x_4$ zur Senke $x_1$:\\
\begin{equation*}
 \frac{x_1}{x_4} = \frac{1}{abc} - \frac{e}{a} - \frac{d}{ba} = \frac{1-bce-cd}{abc} = \frac{1}{T_{14}}.
\end{equation*}
\nit Folglich ergibt sich aus der Inversion eines SFD die invertierte
UTF\index{Ubertragungsfunktion@{\"Ubertragungsfunktion}!invertierte}; sie ist einfach eine alternative (n"amlich nichtkausale)
Darstellung des urspr"unglichen Graphen\footnote{Diese
  ``nichtkausale'' Darstellung eines invertierten SFD wurde verwendet,
  um R"uckkopplungssysteme zu behandeln, ohne dass
  R"uckkopplungsschleifen ben"otigt wurden \cite{WAL:82}.}.  Sehen wir
uns nun die Definition eines
\textbf{\emph{transponierten}}\index{transponiert} SFD an.
Zusammengefasst besteht die Transposition aus drei Schritten,
n"amlich:
\begin{description}
\item i) Richtungsumdrehung aller Zweigtransmittanzen bei gleichbleibenden Transmittanzen
\item ii) Spiegelung des resultierenden SFD
\item iii) Bezeichnungswechsel von Eingangs- und Ausgangsknoten
\end{description}
\nit Dies wird f"ur das SFD von Abb.~\ref{SFD37}a gezeigt.
\begin{figure}[htb!]
\begin{center}
  \hspace*{1mm}\bild{/sfd/SFD38.ps,width=0.97}\caption{Transposition eines SFD}\label{SFD38}
\end{center}\vspace*{-7mm}
\end{figure}\\
\nit Die resultierende Topologie\index{Topologie}, die auch in
Abb.~\ref{SFD37}c gezeigt ist, ist die sogenannte transponierte
Version des urspr"unglichen SFD.  Ihre UTF ist
identisch mit derjenigen des urspr"unglichen SFD, die durch Gleichung
\ref{mathSFD37} gegeben ist, aber ihre Topologie ist verschieden. Wie
aus den zwei entsprechenden Blockdiagrammen in Abb.~\ref{SFD39}
ersichtlich ist, kann dies bedeutende praktische Folgen hinsichtlich
der Rundungsfehler\index{Rundungsfehler}, der ben"otigten
Bitl"angen\index{Bitlae@{Bitl\"a}ngen} etc.  haben. Somit ben"otigt zum Beispiel
die Konfiguration von Abb.~\ref{SFD39}a (welche derjenigen von
Abb.~\ref{SFD37}a entspricht) zwei Addierer, w"ahrend diejenige von
Abb.~\ref{SFD39}b (siehe Abb.~\ref{SFD37}c) nur einen ben"otigt.
{\it\textbf{Transposition ist deshalb eine wichtige Methode f"ur die Ableitung von
alternativen praktischen Topologien mit identischen
UTF.}}\\
\begin{figure}[htb!]
\vspace*{-6mm}\begin{center}
  \bild{/sfd/SFD39.ps,width=0.75}\caption{Blockdiagramm der Systeme,
die den SFD von a) Abb.~\ref{SFD37}a und b) Abb.~\ref{SFD37}c entsprechen}\label{SFD39}
\end{center}\vspace*{-7mm}
\end{figure}
\section{Skalierung \index{Skalierung}} 
Sehr oft muss der Signalpegel bei einem Knoten des Systems oder des
Netzwerks angeglichen werden, um zum Beispiel "Ubersteuerung zu
verhindern oder um den Dynamikbereich\index{Dynamikbereich} des
gesamten Systems zu verbessern.  Skalierung ist der Prozess, mit
welchem eine solche Angleichung ohne "Anderung der UTF durchgef"uhrt
werden kann. Am SFD k"onnen
schnell und einfach Skalierungen vorgenommen werden.\\
\nit Zuerst betrachten wir das
\textbf{\emph{Trennb"undel}}\index{Trennbue@{Trennb\"u}ndel} eines SFD.  Ein
Trennb"undel ist eine minimale Menge von Zweigen, welche, wenn sie
entfernt wird, den Graphen in genau zwei Teile trennt. Das
Trennb"undel wird die Knoten des Graphen in zwei Mengen $N_a$ und
$N_b$ trennen. Ohne Verlust der Allgemeinheit nehmen wir an, dass
$N_a$ die Menge ist, die den Eingangsknoten $x_0$ enth"alt. Die
Skalierung besteht aus der Multiplikation des Gewichts jedes Zweiges
des Trennb"undels mit dem Faktor $\lambda$ oder $1/\lambda$. Wenn ein
Zweig gegen $N_b$ gerichtet ist, wird sein Gewicht mit $\lambda$
multipliziert, wenn er von $N_b$ weggehend gerichtet ist, wird sein
Gewicht mit $1/\lambda$ multipliziert.  Es resultiert eine Skalierung
des Signalniveaus der Knoten in der Menge $N_b$ mit dem Faktor
$\lambda$ in Bezug zum Signalniveau am Eingang des Systems oder
Netzwerks. \\
\bsp{Wir betrachten das SFD, das in Abb.~\ref{SFD40}
  gezeigt ist, wobei das Trennb"undel $\{G_1,a_2,a_3,G_3\}$ durch den
  Rand des grauen Hintergrunds angegeben ist.}
\begin{figure}[htb!]
\vspace*{-3mm}\begin{center}
  \bild{/sfd/SFD40.ps,width=0.61}\caption{Graph,
    der durch das Trennb"undel $\{G_1,a_2,a_3,G_3\}$, in die Graphen
    $N_a=\{G_0,a_1\}$ und $N_b=\{G_2\}$ getrennt wird.}\label{SFD40}
\end{center}\vspace*{-7mm}
\end{figure}\\ 
\nit Da die Zweige $G_1$, $a_2$, und $a_3$ des Trennb"undels gegen die
Menge der Knoten von $N_b$ gerichtet sind, werden ihre Gewichte mit
$\lambda$ multipliziert, und da der Zweig $G_3$ von $N_b$ weggerichtet
ist, wird sein Gewicht mit $\lambda$ dividiert.  So werden die
Signalniveaus der Knoten $x_2$ und $x_3$ mit dem Faktor $\lambda$
skaliert. Es kann mit Hilfe der Regel von Mason\index{Mason} gezeigt
werden, dass die Skalierung keinen Einfluss auf die UTF des Systems
hat, wenn das Trennb"undel den Eingangsknoten (z.B. $x_0$) nicht vom
Ausgangsknoten (z.B. $x_4$) trennt, das heisst, wenn der
Ausgangsknoten nicht in der Menge $N_b$ ist. Wenn dies der Fall sein
sollte, wird die UTF mit dem
Faktor $\lambda$ multipliziert.\\
\nit Die Skalierung kann der Reihe nach auf mehrere verschiedene
Trennb"undel eines SFD angewandt werden, und die Menge der Knoten von
$N_b$ kann aus nur einem Knoten bestehen.  Des weiteren kann $\lambda$
eine Konstante sein, wie es f"ur Signalniveau-Skalierung der Fall sein
wird, $\lambda$ kann eine Funktion der Frequenz sein, oder $\lambda$
kann einfach -1 sein, wobei dann die invertierenden und
nichtinvertierenden Zweige in einer gew"unschten Art verteilt werden.
Der letzte Fall wird in Abb.~\ref{SFD41} gezeigt, wobei die drei
invertierenden Zweige (Abb.~\ref{SFD41}a) positive Multiplikatoren und
die Inverter Teil des Vorw"artspfades werden (Abb.~\ref{SFD41}b).
\begin{figure}[htb!]
\begin{center}
  \bild{/sfd/SFD41.fig.eps,width=0.63}\caption{Skalierung mit -1, um die invertierenden und
nichtinvertierenden Zweige in einer bestimmten Weise zu verteilen.}\label{SFD41}
\end{center}\vspace*{-7mm}
\end{figure}\\
\nit Wenn man den Dynamikbereich\index{Dynamikbereich} eines Systems
maximieren will, wird der Faktor $\lambda$ eine positive Konstante
sein.  Typischerweise wird die Skalierung mehrmals gebraucht, um die
unabh"angigen Signalniveaus bei jedem kritischen Knoten des Systems
oder Netzwerks einzustellen. Durch Skalierung der maximalen
Signalniveaus, die bei jedem Knoten gleich sein sollten, kann der
maximal m"ogliche Dynamikbereich erreicht werden. \textbf{\emph{Die
    sehr wirkungsvolle Operation der Skalierung kann also unter
    anderem dazu verwendet werden, den
    Dynamikbereich\index{Dynamikbereich} zu maximieren, Inverter zu
    entfernen und die Verst"arkung und die Signalniveaus innerhalb
    eines Systems zu "andern.}} Es ist interessant, dass die
Skalierung unabh"angig ist von der Topologie des Systems und keinen
Einfluss hat auf die differentielle Sensitivit"at\index{Sensitivitat@Sensitivit\"at} der
"Ubertragungsfunktion in Bezug auf "Anderungen der Zweigtransmittanzen
im SFD.

\clearpage
\section{Aufgaben zu den Signalflussdiagrammen}
\begin{enumerate}
\item Bestimmen Sie die Ordnung des folgenden Signalflussdiagramms:
\begin{center}
  \bild{/sfd/SFDauf1.fig.eps,width=0.3}
\end{center}
\item Gegeben ist folgendes Signalflussdiagramm:
  \begin{center}
  \bild{/sfd/SFDauf2.fig.eps,width=0.3}
\end{center}
Berechnen Sie mit der Formel von Mason die "Ubertragungsfunktion
$T_{13}=\frac{U_3}{U_1}$ aus obigem Signalflussdiagramm (SFD).

\item Gegeben ist folgendes Signalflussdiagramm:
  \begin{center}
  \bild{/sfd/SFDauf3.fig.eps,width=0.56}
\end{center}
  \begin{enumerate}
  \item Berechnen Sie die "Ubertragungsfunktion $T_{08} =
    \frac{x_8}{x_0}$.
  \item Berechnen Sie die "Ubertragungsfunktion $T_{07} =
    \frac{x_7}{x_0}$. (Hinweis: Das Resultat l"asst sich
    sofort aus $T_{08}$ bestimmen.)
  \end{enumerate}
\item Skalieren Sie das folgende SFD so, dass alle
  Inverter (Transmittanzen=-1) verschwinden.
  \begin{center}
  \bild{/sfd/SFDauf4.fig.eps,width=0.6}
\end{center}
\item Eine Bank zahlt f"ur ein Guthaben einen Jahreszins von $r$
  Prozent.  Das Jahr sei in $n$ Zins\-perioden der Dauer \(T_{\$}\)
  unterteilt (nach dem Motto: ``Time ist money$\ldots$''). Immer nach
  Ablauf einer Periode wird der Zins f"ur die vorhergehende Periode
  berechnet und dem Kapital zugeschlagen. Uns interessiert der Stand
  des Guthabens zu diesem Zeitpunkt \(k T_{\$}\). Der Kapitalwert
  \(y(k T_{\$})\) am Ende der $k$-ten Zinsperiode berechnet sich aus
  dem Kapitalwert am Ende der letzten Zinsperiode plus dem Zins,
  welcher dieses Kapital in der letzten Zinsperiode abgeworfen hat und der
  Summe der Ein- bzw.  Auszahlungen w"ahrend der letzten Zinsperiode
  \(f(k T_{\$})\).

  \begin{enumerate}
  \item Fassen Sie die beschriebenen Zusammenh"ange in einer
    Differenzengleichung\index{Differenzengleichung} zusammen.

  \item Stellen Sie die erhaltene Gleichung durch ein SFD dar.
  \end{enumerate}
  
\item Wir betrachten nun ein
  psycho-sozio-laboro-physio-human-universit"ares System: den
  ``gestressten Assistenten''. Er kann (leicht vereinfacht und im
  Arbeitspunkt linearisiert) durch folgende Gleichungen beschrieben
  werden:

  \begin{tabular}{ll}

    $SH = a \cdot AA - m \cdot EA$ & Stapelh"ohe auf seinem Pult = 
     $a$ mal anfallende \\ & Arbeiten minus $m$ mal erledigte Arbeiten \\

    $FZ = SFZ - b \cdot SH$ & Freizeit = Sollfreizeit minus $b$ mal
    Stapelh"ohe\\ & ($b$ = moralischer "Uberzeitkoeffizient) \\ 

    $SP = c \cdot FZ + e \cdot PHY$ & Sportt"atigkeit = $c$ mal
    Freizeit plus $e$ mal\\ & physische Gesundheit \\ 

    $ERH = g \cdot FZ + i \cdot PSY$ & Erholung = $g$ mal Freizeit
    plus $i$ mal\\ & psychische Gesundheit \\ 

    $PHY = d \cdot SP$ & Physische Gesundheit = $d$ mal
    Sportt"atigkeit \\ 

    $PSY = h \cdot ERH + k \cdot SP$ & Psychische Gesundheit = $h$ mal
    Erholung plus  \\ & $k$ mal Sportt"atigkeit \\ 

    $LEI = f \cdot PHY + j \cdot PSY$ & Leistungsf"ahigkeit = $f$ mal
    physische plus \\ & $j$ mal psychische Gesundheit\\ 

    $EA = l \cdot LEI + n \cdot AA$ & Erledigte Arbeit = $l$ mal
    Leistungsf"ahigkeit \\ &  plus $n$ mal anfallende Arbeiten \\ &($n$ =
    Termindruckfaktor)
  \end{tabular}

  Zeichnen Sie das SFD f"ur dieses System mit folgender
  Knotenanordnung:
 \begin{center}
  \bild{/sfd/SFDauf6.fig.eps,width=0.8}
\end{center}
 

\item Setzen Sie folgendes Gleichungssystem in ein SFD um:
  \begin{align*}
    x_2 &= -a \cdot x_1 + b \cdot x_3 - c \cdot x_4 \\
    d \cdot x_3 &= e \cdot x_1 + f \cdot x_2 - g \cdot x_3 \\
    x_4 &= x_3
  \end{align*}

\item Schreiben Sie f"ur die folgenden zwei Funktionsblockdiagramme
  jeweils ein Gleichungssystem auf und setzen Sie dieses in ein SFD um.

  \begin{enumerate}
  \item ~
    \begin{center}
  \bild{/sfd/SFDauf5.fig.eps,width=0.6}
\end{center}

  \item ~
    \begin{center}
  \bild{/sfd/SFDauf7.fig.eps,width=0.6}
\end{center}

  \end{enumerate}
\item Schreiben Sie zu folgendem, aus der analogen Signalverarbeitung
  stammenden, SFD das Gleichungssystem auf:
\begin{center}
  \bild{/sfd/SFDauf8.fig.eps,width=0.6}
\end{center}

  \emph{Fakultativ:} K"onnen Sie die dazugeh"orige passive
  $RLC$-Schaltung\index{RLC@$RLC$-Schaltung} zeichnen?
\item Stellen Sie f"ur die folgenden zwei SFD jeweils ein
  Gleichungssystem auf:

  \begin{enumerate}
  \item ~ 
  \begin{center}
  \bild{/sfd/SFDauf9.fig.eps,width=0.6}
  \end{center}
  \item ~
   \begin{center}
  \bild{/sfd/SFDauf10.fig.eps,width=0.8}
  \end{center}
  \end{enumerate}

\item Beweisen Sie, dass die beiden SFD dieselbe
  Graphdeterminante $\Delta$ besitzen:
  \begin{center}
  \bild{/sfd/SFDauf11.fig.eps,width=0.8}
   \end{center}

 \item Berechnen Sie im folgenden SFD die
   "Ubertragungsfunktion $H(z)=\frac{Y(z)}{X(z)}$.
   \begin{center}
       \bild{/sfd/SFDauf12.fig.eps,width=0.8}
 \end{center}
\newpage
\item Bestimmen Sie, welche der f"unf Signalflussdiagramme ii) bis vi)
  dieselbe "Ubertragungsfunktion $H(z)=\frac{Y(z)}{X(z)}$ wie das
  Signalflussdiagramm in i) besitzen. \vspace*{-4mm}
  \begin{center}
  \bild{/sfd/SFDauf13.fig.eps,width=0.95}
\end{center}
\newpage
\item Gegeben ist folgendes Signalflussdiagramm.\\
\begin{center}
    \bild{/sfd/SFDauf14.fig.eps,width=0.42}
  \end{center}
\vspace*{-4mm}
\begin{enumerate}  
  \item
    Berechnen Sie mit Hilfe von Mason's Regel die "Ubertragungsfunktion $T_{14}$.
  \item \label{Teil1} Setzen Sie $t_{14}=0$, $t_{34}=1$,
    $t_{12}=\frac{n_{12}}{\hat d}$, $t_{32}=\frac{n_{32}}{\hat d}$,
    wobei $n_{12}$, $n_{32}$ und $\hat d$ Polynome in $s$ sind.
    Berechne die "Ubertragungsfunktion $T_{14}$.\\ (Bemerkung:
    UTF von passiven (Teil-)Netzwerken besitzen nur Pole auf der
    negativen reellen Halbachse. Um dies hervorzuheben, wird der
    Nenner mit einem ~$\hat{}$~ gekennzeichnet.)
  \item
    Bestimmen Sie $T_{14}$, wenn $\beta \rightarrow \infty$.
\end{enumerate}
Betrachten Sie nun die folgende Schaltung:\\
\vspace*{-4mm}\begin{center}
    \bild{/sfd/SFDauf15.fig.eps,width=0.64}
  \end{center}
\begin{enumerate}
\item[(d)] \label{Teil2} (\emph{F"ur diese und die folgende Teilaufgabe
    ist die Schaltung open-loop zu betrachten, d.h. der Verst"arker
    kann weggedacht werden.})\\ 
    Berechne
    $t_{12}=\left.\frac{U_2}{U_1}\right|_{U_3=0}$ und
    $t_{32}=\left.\frac{U_2}{U_3}\right|_{U_1=0}$.

  \item[(e)]
    Berechnen Sie $t_{14}=\left.\frac{U_4}{U_1}\right|_{U_3=0}$ und $t_{34}=\left.\frac{U_4}{U_3}\right|_{U_1=0}$.

  \item[(f)]
    Bestimmen Sie nun die "Ubertragungsfunktion $T_{14} = \frac{U_4}{U_1}$ und betrachten Sie danach den Grenz"ubergang $\beta \rightarrow \infty$.

  \item[(g)]
    (\emph{Schlussfolgerungen})
    \begin{enumerate}
      \item
        Was bedeutet $t_{14} \neq 0$?
      \item
        Vergleichen Sie die Teilaufgaben \ref{Teil1}) und \ref{Teil2}d). Warum sind die Nenner $\hat d$ von $t_{12}$ und $t_{32}$ gleich?
    \end{enumerate}
\end{enumerate}

\item {\bf Mathematik (just for fun)}\\ Die Kettenbruchdarstellung\index{Kettenbruch} von $\alpha$ ist
$\alpha=1+\frac{1}{1+\frac{1}{1+\frac{1}{1+\ldots}}}$. In kurzer Notation wird $\alpha$ auch mit $\alpha=[1;1,1,1,\ldots]$ dargestellt. Wie gross ist $\alpha$?

\item {\bf R"atsel\index{Raetsel@{R\"atsel}} (just for fun)}\\ Verteilen Sie die Zahlen 1 bis 15
  so auf die Kreise (Pyramide mit f"unf Stufen), dass die Zahl in
  jedem Kreis die Differenz der Zahlen der zwei Kreise ergibt, auf
  welchen sie sitzt. Es gibt genau eine L"osung! Als Beispiel finden
  Sie die L"osung wenn man die Zahlen 1 bis 10 auf eine vierstufige
  Pyramide verteilt. 
\begin{center}
  \bild{/sfd/SFDraetsel.fig.eps,width=0.5}
\end{center}
  Bemerkung: F\"ur Pyramiden\index{Pyramiden}  mit 1-5 Stufen gibt es L\"osungen. F"ur Pyramiden mit 6-11
  Stufen gibt es keine L"osungen! F"ur gr"ossere Stufenpyramiden gibt
  es h"ochstwahrscheinlich keine L"osung.
\item {\bf R"atsel (just for fun)}\\ Wieviele Mal k"onnen Sie das Wort
  ``Sisyphus'' schreiben, wenn Sie vom Spitz bis zum Sockel der
  Pyramide f"ur die n"achste Stufe jeweils zwischen links und rechts
  w"ahlen k"onnen.
\begin{center}
  \bild{/sfd/sisyphus.fig.eps,width=0.3}
\end{center}
\item {\bf R"atsel der verborgenen Dreiecke\index{Raetsel@{R\"atsel}}\index{Dreieck} (just for fun)}\\ 
Wie viele Dreiecke sind in der untenstehenden Abbildung enthalten?\\
\begin{center}
{\psset{unit=1}
\begin{pspicture}(2,3)
\pspolygon[linewidth=0.1pt](0,2)(0.6666,2)(1,3) %1
\pspolygon[linewidth=0.1pt](0.66666,2)(1.33333,2)(1,3) %2
\pspolygon[linewidth=0.1pt](1.33333,2)(2,2)(1,3) %3
\pspolygon[linewidth=0.1pt](1.33333,2)(2,2)(1.5,1.5) %4
\pspolygon[linewidth=0.1pt](2,2)(1.5,1.5)(2,0) %5
\pspolygon[linewidth=0.1pt](1.5,1.5)(2,0)(1,1) %6
\pspolygon[linewidth=0.1pt](2,0)(1,1)(0,0) %7
\pspolygon[linewidth=0.1pt](1,1)(0,0)(0.5,1.5) %8
\pspolygon[linewidth=0.1pt](0,0)(0.5,1.5)(0,2) %9
\pspolygon[linewidth=0.1pt](0.5,1.5)(0,2)(0.66666,2) %10
\pspolygon[linewidth=0.1pt](0.5,1.5)(0.66666,2)(1.333333,2)(1.5,1.5)(1,1) %11
\end{pspicture}}
\end{center}

\newpage
\item{}~\\
 \vspace*{-2mm} \begin{center}    \bild{/sfd/SFD_P1.eps,width=0.95} \end{center} %print -depsc2 SFD1
\begin{enumerate}
\item[a)] Zeichnen Sie das SFD des obenstehenden Blockdiagrammes im Bildbereich ($X_i(s)\leftrightarrow x_i(t)$) auf, wobei Sie die folgende Anordnung verwenden sollen:\\
\vspace*{0.5cm}\\
\hspace*{9.8cm}$In_2(s)$\\
\vspace*{-0.3cm}\\
\hspace*{9.9cm}$\bullet$\\
\vspace*{1.8cm}\\
\hspace*{0cm}$\bullet$\hspace*{1.8cm}$\bullet$\hspace*{1.8cm}$\bullet$\hspace*{1.8cm}$\bullet$\hspace*{1.8cm}$\bullet$\hspace*{1.8cm}$\bullet$\hspace*{1.89cm}$\bullet$\hspace*{1.98cm}$\bullet$\\
\hspace*{-0.45cm}$In_1(s)$\hspace*{0.92cm}$sX_1(s)$\hspace*{0.92cm}$X_1(s)$\hspace*{0.93cm}$sX_2(s)$\hspace*{0.93cm}$X_2(s)$\hspace*{0.93cm}$sX_3(s)$\hspace*{0.93cm}$X_3(s)$\hspace*{0.92cm}$Out_1(s)$\\
\vspace*{2cm}\\
\item[b)] Bestimmen Sie alle Vorw\"artspfade, alle Kofaktoren sowie die Graphdeterminante $\Delta$ (Eingang $In_1(s)$ zu Ausgang $Out_1(s)$). \\
\item[c)] Wie gross ist die Anzahl fundamentaler Knoten?\\
\item[d)] Bestimmen Sie die UTF $H_{IO}(s)=\frac{Out_1(s)}{In_1(s)}$ sowie $H_{I2}(s)=\frac{X_2(s)}{In_1(s)}$.\\

\end{enumerate}

\end{enumerate}


