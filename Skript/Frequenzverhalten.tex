% SNR einbauen und Rauschzahl !!!!
% alte bode etc. von Moschytz uebererbeutren
% Bilder der Approximationen einzeichnen (je ein Beispiel) siehe auch UNterlagen von Werner Hinn
\renewcommand{\thesection}{\thechapter.\arabic{section}}
\setcounter{Aufgabe}{0}\setcounter{Beispiel}{0}
\chapter{Frequenzverhalten von analogen LTI-Systemen}\label{kapitel_Frequenzverhalten} 
\index{LTI-System!analog}
\index{Frequenzverhalten} 

\section{Einf\"uhrung}
In diesem Kapitel werden wir zu Beginn den {\bf D"ampfungsfaktor} und
{\bf Verst\"arkungs\-faktor}, sowie deren {\bf logarithmische
  Darstellung} betrachten. \\
\nit Der Zusammenhang von {\bf Frequenzgang}\index{Frequenzgang} (Amplituden- und Phasengang),\\ {\bf \"Uber\-tragungsfunktion (UTF)}\index{UTF}, Polen und Nullstellen wird kurz er\"ortert und es wird  anhand von verschiedenen Beispielen der Frequenzgang, die UTF, Pole und Nullstellen von {\bf Netzwerken}\index{Netzwerk} (z.B. Tiefpass, Bandpass, Allpass) gezeigt.\\
\nit Danach f\"uhren wir das {\bf Bode-Diagramm} und die {\bf
  Ortskurve} (Nyquist-Diagramm) ein und streifen kurz das
Nichols-Diagramm. Das {\bf Bode-Diagramm}\index{Bode-Diagramm} ist
eine graphische Methode, die es erlaubt, den {\bf
  Frequenzgang}\index{Frequenzgang} ({\bf
  Amplitudengang}\index{Amplitudengang} und den {\bf
  Phasengang}\index{Phasengang}) aus den {\bf Polen}\index{Pol} und
den {\bf Nullstellen}\index{Nullstellen} einer {\bf
  \"Uber\-tragungs\-funktion}\index{Ubertragungsfunktion@{\"Ubertragungsfunktion}}
aufzuzeichnen. Im {\bf Bode-Diagramm} wird die {\bf
  Amplitude}\index{Amplitude} und die {\bf Frequenz}\index{Frequenz}
{\boldmath $f$} {\bf logarithmisch}\index{logarithmisch}
aufgetragen.\\ \nit Die {\bf Ortskurve}\index{Ortskurve} verwendet die
{\bf komplexe Ebene}\index{komplexe Ebene}, wo alle komplexen Werte
des Frequenzganges in Abh\"angigkeit der {\bf Frequenz} {\boldmath
  $f$} eingetragen werden. Die {\bf Ortskurvendarstellung} wird z.T.
auch als {\bf Nyquist-Diagramm}\index{Nyquist-Diagramm} oder {\bf
  Nyquist-Ortskurve}\index{Nyquist-Ortskurve} bezeichnet
\cite{UNB:81}.  Beim weniger gebr\"auchlichen {\bf
  Nichols-Diagramm}\index{Diagramm!Nichols}\index{Nichols-Diagramm}
werden {\bf Amplitude}\index{Amplitude} und {\bf Phase}\index{Phase}
als {\bf kartesische Koordinaten} mit $\omega$ als Parameter
aufgezeichnet.  Das Nyquist-Diagramm (und das Nichols-Diagramm) werden
vor allem in der Regelungstechnik angewendet.\\
Die {\bf Stabilit\"atsbedingungen}\index{Stabilitat@{Stabilit\"at}} f\"ur LTI-Systeme werden kurz anhand des {\bf Nyquist-Kriteriums}\index{Nyquist-Kriterium} mit der Ortskurve und anhand des {\bf Bode-Diagramms} {\bf (Amplitudenrand und Phasenrand)} behandelt.\index{Amplitudenrand}\index{Phasenrand}\\
Die bei weitem gr"osste Verbreitung hat aber wegen seiner
Anschaulichkeit und einfachen Handhabung das {\bf Bode-Diagramm}
erfahren \cite{WYR:99:2}.

\newpage
\section{Logarithmische Darstellung}
\subsection {D"ampfung, D"ampfungsfaktor  und D"ampfungsmass}\index{Dampfungsmass@{D\"ampfungmass}}\index{Dampfungsfaktor@{D\"ampfungsfaktor}}
Die {\bf Energie} eines Nachrichtensignales betr"agt am Ende eines
"Ubertragungssytemes normalerweise nur noch ein {\bf Bruchteil} 
der eingespeisten Energie. Diese Abnahme wird als
{\bf D"ampfung}\index{Dampfung@{D\"ampfung}}
bezeichnet.
\bsp{}
\begin{figure}[htb]
\vspace*{-7mm}
\begin{center}
  \bild{/f_verhalten/F_VER_001.ps,width=0.7}\caption{Beispiel der Leistungsabnahme durch ein "Ubertragungssytem (z.B. Leitung, Filter) wobei $R_i=R_1=R_2=R_L=600~\Omega$ und $U_0=600$~mV. Somit ist $U_1=300$~mV, $I_1=500~\mu$A und $P_1=150~\mu$W. Mit $I_2=100~\mu$A ist $P_2=6~\mu$W. }
\end{center}
\vspace*{-6mm}
\end{figure} \\
\nit
Das Verh"altnis zwischen Eingangs- und
Ausgangsgr"ossen wird als {\bf D"ampfungsfaktor}\index{Dampfungsfaktor@{D\"ampfungsfaktor}}  \mat{D}
gekennzeichnet \cite{WYR:99:2}
\begin{equation}
 D_P  = \frac {P_1} {P_2}; \quad D_U   = \frac {U_1} {U_2};\quad D_I   = \frac {I_1} {I_2}. 
\end{equation} 
Die Indizes $P,U$ und $I$ stehen f"ur {\bf Leistung}\index{Leistung},
{\bf Spannung}\index{Spannung} und {\bf Strom}\index{Strom}.\\ In der Nachrichtentechnik
arbeitet man selten mit {\bf D"ampfungsfaktoren}.  Man dr"uckt das
Verh"altnis der Ausgangs- und Eingangsgr"ossen meistens {\bf
  logarithmisch} aus. \\ Durch das Logarithmieren des
{\bf D"ampfungsfaktors} \mat{D} erh"alt man das {\bf D"ampfungsmass}
\mat{a}:
\begin{eqnarray*} 
  a_P &=& \log \klam{\frac {P_1} {P_2}}.
\end{eqnarray*}\\ 
Das einheitenlose D"ampfungsmass  \mat{a} erh"alt die Pseudoeinheit
{\bf Bel}\footnote{A.G. Bell, amerikanischer Ingenieur, 1847-1922}.\index{Bel} F"ur
viele praktische F"alle ist das Bel \index{Bel} zu gross. Deshalb wurde das
{\bf Dezibel} \index{Dezibel} ({\bf dB}) eingef"uhrt:
\begin{equation}
 1\mbox{~B} = 10\mbox{~dB}.
\end{equation}\\ 
Somit folgt f"ur das D"ampfungsmass  \mat{a} in {\bf dB}:\\~\\
\myboxx{\begin{eqnarray}
 a_P = 10 \cdot \log \klam{\frac {P_1} {P_2}}, \\
 a_U = 20 \cdot \log \klam{\frac {U_1} {U_2}}, \\
 a_I = 20 \cdot \log \klam{\frac {I_1} {I_2}}. 
\end{eqnarray}}

\newpage
\vspace*{-9mm}\aufg 
Wieso stehen vor den D"ampfungsmassen $a_I$
und $a_U$ die Faktoren $20$ und nicht wie bei $a_P$ der Faktor $10$.
\vspace*{16mm}

{\bsp{~}
Ein Vierpol weist bei einer Eingangsspannung
von 24~V eine Ausgangsspannung von 9.5~V auf. Wie gross ist das
Strom-, Spannungs- und Leistungsd"ampfungsmass (in dB)? Der Last-,
sowie der Ein- und Ausgangswiderstand sind alle gleich gross.\\
\nit
\underline{L"osung:}\\ Weil alle Widerst"ande gleich gross sind,
folgt: $a_U = 20 \cdot \log \klam{\frac {24}{9.5}} = 8.05$~dB, $a_I = a_U = 8.05$~dB und $a_P = a_U = 8.05$~dB.\\~\\
\nit Neben dem {\bf dB}\index{dB} gibt es noch ein weiteres Mass, dass in der
{\bf Telefonie}\index{Telefonie} verbreitet war. Das {\bf Neper}\index{Neper}
({\bf Np})\index{Np} \index{Neper}ist definiert als:\\~\\
\myboxx{\begin{eqnarray*}
 a_P = \frac {1}{2} \cdot \ln \klam{\frac {P_1} {P_2}}, \\
 a_U = \ln \klam{\frac {U_1} {U_2}}, \\
 a_I = \ln \klam{\frac {I_1} {I_2}}. 
\end{eqnarray*}}\\ ~\\
Die Umrechnung zwischen {\bf dB} und {\bf Np} ist:\\~\\
\myboxx{\begin{eqnarray*}
 1\mbox{~dB} &=& \frac {\ln(10)} {20} \mbox{~Np} = 0.1151\mbox{~Np},\\
 1\mbox{~Np} &=& 20 \cdot \log(\mbox{e}) \mbox{~dB} = 8.686\mbox{~dB}.
\end{eqnarray*}}\\~\\

\newpage
\vspace*{-9mm}\aufg
Erg"anzen Sie die fehlenden Stellen der Tabelle~\ref{FREQ_TAB_01}. \\
\begin{table}[htb!]
\vspace*{-4mm}\begin{center}
{\footnotesize \begin{tabular}{|r|r|r|r|} \hline
\multicolumn{4}{|c|}{\bf Umrechnungstabelle dB $\Leftrightarrow$ Np} \\ \hline\hline
$L_{rel.}$ (dB) & $L_{rel.}$ (Np)  & $P_2/P_1$ & $U_2/U_1$ \\ \hline
 \hspace*{3cm} &   9.210  & $10^8$   & $10^4$       \\ \hline
 \hspace*{3cm} &          & $10^7$   & 3162 \\ \hline
  60 &          & $10^6$   & $10^3$\\ \hline
  50 &          & $10^5$   & 316.2\\ \hline
  40 &          & $10^4$   & $10^2$\\ \hline
  30 &          & $10^3$   & 31.62\\ \hline
  20 &          & $10^2$   & 10\\ \hline
      &          & $81$   & 9\\ \hline
      &          & $64$   & 8\\ \hline
      &          & $49$   & 7\\ \hline
      &          & $36$   & 6\\ \hline
  15 &          & 31.62    & 5.623\\ \hline
      &          & $25$   & 5\\ \hline
      &          & $16$   & 4\\ \hline
  10 &          & 10       & 3.162\\ \hline
      &          & $9$   & 3\\ \hline
   9 &          & 7.943    & 2.818\\ \hline
   8 &          & 6.310    & 2.512\\ \hline
   7 &          & 5.012    & 2.239\\ \hline
      &          & $4$   & 2\\ \hline
   6 &          & 3.981    & 1.995\\ \hline
   5 &          & 3.162    & 1.778\\ \hline
   4 &          & 2.512    & 1.585\\ \hline
   3 &          & 1.995    & 1.413 \\ \hline
   2 &          & 1.585    & 1.259\\ \hline
   1 &          & 1.259    & 1.122\\ \hline
   0 &  0       & 1        & 1\\ \hline
  -1 &          & 0.7943   & 0.8913  \\ \hline
  -2 &          &   & \\ \hline
  -3 &          & & \\ \hline
  -4 &          & & \\ \hline
  -5 &          & & \\ \hline
  -6 &          & & \\ \hline
  -7 &          & & \\ \hline
  -8 &          & & \\ \hline
  -9 & -1.036   & & \\ \hline
 -10 &          &          & 0.3162 \\ \hline
 -15 &          & 0.03162  & \\ \hline
 -20 &          & & \\ \hline
 -30 &          & & \\ \hline
 -40 &          & & \\ \hline
 -50 &          & & \\ \hline
 -60 &          & & \\ \hline
 -70 & \hspace*{3cm} & & \\ \hline
 -80 & -9.210   & \hspace*{3cm}& \hspace*{3cm}\\ \hline 
 \end{tabular}}
\end{center}\vspace*{-3mm}\caption{Umrechnungstabelle zwischen Neper, Dezibel, Spannungs- und Leistungsverh"altnissen}\index{Neper}\index{Dezibel}\label{FREQ_TAB_01}
\end{table}

\clearpage
\nit Neben den {\bf passiven} (d"ampfenden) {\bf Vierpolen} \index{Vierpol} spielen
{\bf aktive Bausteine} (Verst"arker) eine wesentliche Rolle. Sie verst"arken
das Signal. Wir betrachten hier nur den {\bf Betrag}\index{Betrag}
des Signales und lassen die {\bf Phase}\index{Phase}, wenn nicht anders angegeben, weg.
Allgemein kann also das ausgangsseitige Signal eines Vierpols kleiner
oder gr"osser sein als am Eingang. Man spricht von {\bf D"ampfung}, wenn $U_1 \geq U_2$. Es liegt eine {\bf Verst"arkung} vor, wenn $ U_1 \leq U_2$.\\ \nit
Bei {\bf korrekter} Anwendung der f"ur die {\bf D"ampfung} festgelegten
dB-Ausdrucksweise (Eingangsgr"osse stets "uber dem Bruchstrich)
f"uhrt eine Spannungs-, Strom- oder Leist\-ungs-{\bf Vergr"osserung} durch
einen Vierpol zu {\bf negativen} dB-Werten.\\ 
\bsp{~} 
Das {\bf D"ampfungsmass} 20~dB entspricht einem
D"ampfungsfaktor von 10 f"ur Spannungen oder einer Verst"arkung von
0.1.\\ Umgekehrt entspricht dem D"ampfungsmass von -20dB eine
Spannungsverst"arkung von 10. \\ ~\\
\nit
Die Verst"arkung negativ auszudr"ucken widerstrebt dem Ingenieur.
Deshalb wurde ein {\bf Verst"arkungsfaktor}\index{Verstarkungsfaktor@{Verst\"arkungsfaktor}} ("Ubertragungsfaktor)\index{Ubertragungsfaktor@{\"Ubertragungsfaktor}}  \mat{T}
definiert, bei welchem die Ausgangsgr"osse im Z"ahler steht:\\~\\ 
\myboxx{\begin{equation}
 \mbox{Spannungsverst"arkungsfaktor } T_U = \frac {U_2} {U_1}.
\end{equation}}\\~\\ 
Dadurch entstehen bei Verst"arkungen positive Werte. Bei der
Berechnung von "Ubertragungsketten muss jedoch beachtet werden, ob
die jeweilige Stufe {\bf d"ampft} oder {\bf verst"arkt}. (Das Minuszeichen muss
dann entsprechend verwendet werden.)
\bsp{~}
\begin{figure}[htb]
\vspace*{-8mm}
\begin{center}
  \hspace*{-1.9cm}
  \bild{/f_verhalten/daempf.fig.eps,width=0.7}\caption{D"ampfungsverlauf in einer "Ubertragungskette (f\"ur ausgew\"ahlte Frequenzen)}
\end{center}
\vspace*{-4mm}
\end{figure}  \\
\bem {\bf Serieschaltungen} \index{Serieschaltung} (multiplikative
Verkn"upfung) von elektrischen Gr"ossen (z.B.  Ver\-st"ark\-ungen,
Filterkaskaden, usw.) lassen sich mit dB-Werten durch
einfache {\bf Addition} der einzelnen dB-Werte darstellen.

\newpage
\subsection{Relativer \index{Pegel!relativ} und absoluter Pegel \index{Pegel!absolut}} 
H"aufig verwendet man den
{\bf relativen Pegel}. Es wird angegeben, um wieviele dB eine Spannung oder
Leistung an einem Ort gr"osser oder kleiner ist, als an einem
festgelegten Ort des "Ubertragungssystemes. Dabei verwendet man als
Referenz die dort vorherrschende Spannung oder Leistung.\\~\\
\myboxx{\begin{eqnarray}\vspace*{-3mm}
 (L_U)_{\mbox{rel.}} = 20 \cdot \log \klam{\frac {U_2} {U_1}}, \\
 (L_I)_{\mbox{rel.}} = 20 \cdot \log \klam{\frac {I_2} {I_1}}, \\
 (L_P)_{\mbox{rel.}} = 10 \cdot \log \klam{\frac {P_2} {P_1}}.
\end{eqnarray}}\\~\\
Bei {\bf absoluten Pegelangaben}\index{Pegel!absolut} in der Nachrichtentechnik legt man einen
{\bf Normgenerator}\index{Normgenerator} (Fig.~\ref{GRU09}) zugrunde, der bei einem
Innenwiderstand von 600~$\Omega$ eine Leistung von 1~mW an einen
Verbraucher $R$ von 600~$\Omega$ liefert.\\
\begin{figure}[htb]
   \vspace*{-1mm}
   \begin{center}
        \bild{/f_verhalten/GRU09.fig.eps,width=0.4}\caption{Normgenerator mit $P_2=U_2\cdot I_2=1$~mW. Somit ist $U_2=\sqrt{1\text{~mW}\cdot 600~\Omega}=0.7746$~V.}\label{GRU09}
  \end{center} \vspace*{-5mm}
\end{figure} \\
\nit  Somit folgt:\\~\\
\myboxx{\begin{eqnarray}\vspace*{-3mm}
 (L_U)_{\mbox{abs.}} = 20 \cdot \log \klam{\frac {U_2} {774.6 \mbox{~mV}}}, \\
 (L_I)_{\mbox{abs.}} = 20 \cdot \log \klam{\frac {I_2} {1.291 \mbox{~mA}}}, \\
 (L_P)_{\mbox{abs.}} = 10 \cdot \log \klam{\frac {P_2} {1 \mbox{~mW}}}.
\end{eqnarray}}
\newpage
\nit In der Praxis lassen sich Pegelangaben nicht immer entnehmen, ob die
Pegelart Leistung, Spannung, Strom, absolut oder relativ zugrunde
liegt. Zur Kennzeichnung sollte ein Buchstabe an das Kennzeichen dB
angef"ugt sein. Folgende Angaben sind in der {\bf Elektrotechnik}\index{Elektrotechnik} gebr"auchlich:\\ 
\begin{table}[htb]
\begin{center}\vspace*{-2mm}
\begin{tabular}{|l|l|l|}\hline
  & dBu & Spannungspegel bezogen auf 774.6~mV an 600~$\Omega$\\ \cline{2-3}
 \multicolumn{1}{|l|}{\raisebox{1.5ex}[-1.5ex]{$\mbox{dB}_{abs.}$}} & dBm & Leistungspegel bezogen auf 1~mW an 600~$\Omega$\\ \hline\hline
  & dBV & Spannungspegel bezogen auf 1~V\\ \cline{2-3}
  & dB$\mu$V & Spannungspegel bezogen auf 1~$\mu$V\\ \cline{2-3}
  & dBf & Leistungspegel bezogen auf $10^{-15}$~W\\ \cline{2-3} 
\multicolumn{1}{|l|}{$\mbox{dB}_{rel.}$}  & dBW & Leistungspegel bezogen auf 1~W\\ \cline{2-3}
  & dBk & Leistungspegel bezogen auf 1~kW\\ \cline{2-3}
  & dBr & relativer Pegel\\ \cline{2-3}
  & dB0 & Pegel auf 0~dB bezogen\\ \hline
\end{tabular}\caption{Tabelle  gebr"auchlicher  Pegelangaben}
\end{center}
\end{table}

\aufg
Mit einem {\bf Digitalvoltmeter}\index{Digitalvoltmeter} misst man die
Bezugsspannung und erh"alt als Resultat 2.218~dBu. Die zu
vergleichende Spannung ist 10~V. Wie gross ist der relative
Spannungspegel $(L_U)_{\mbox{rel.}}$ und wie gross ist die
Bezugsspannung in Volt?\vspace*{1cm}

%\aufg
%In der Akustik\index{Akustik} wird die Schallintensit"at\index{Schallintensitat@{Schallintensit\"at}} $p_0=20\mu$  als 0~dB %Schalldruck\index{Schalldruck} definiert. Wie gross ist der Schalldruck und die Schallintensit"at bei 100~dB \cite{SCH:03}.
\newpage
\section{Frequenzgang}
Wird bei der {\bf
  \"Ubertragungsfunktion}\index{Ubertragungsfunktion@{\"Ubertragungsfunktion}}
{\boldmath $H(s)$} eines {\bf LTI-Systems}\index{LTI-System}
{\boldmath $\sigma=0$} gesetzt, d.h., {\boldmath $s=j\omega$}, so
erh\"alt man den {\bf Frequenzgang}\index{Frequenzgang} {\boldmath
  $H(j\omega)$}. W\"ahrend die {\bf UTF}\index{UTF} $H(s)$ eine {\bf
  abstrakte, nicht messbare, mathematische} Beschreibungsform eine
LTI-Systemes ist, kann der Frequenzgang $H(j\omega)$ {\bf
  physikalisch} interpretiert und auch gemessen werden \cite{UNB:81}.
\subsection[Zusammenhang Frequenzgang $\leftrightarrow$ UTF]
{Der Zusammenhang zwischen Frequenzgang und "Ubertragungsfunktion
  (UTF)} Bei unseren Betrachtungen beschr"anken wir uns auf {\bf
  zeitinvariante}\index{zeitinvariant} LLF\footnote{linear
  lumped-parameter finite (LLF) networks}-Netzwerke\index{LLF}
(lineare Netzwerke mit konzentrierten
Elementen).\\
\begin{figure}[!htb]
\begin{center}
  \vspace*{-2mm}\bild{/filter/FIL2.fig.eps,width=0.5}\caption{LLF Netzwerk}
\end{center}
\vspace*{-6mm}
\end{figure}\\
Diese {\bf LTI-Systeme}\index{LTI} lassen sich mit einer {\bf Differentialgleichung}\index{Differentialgleichung} der Form
\begin{equation*}
a_{n}\frac{d^{n} y}{dt^{n}} + a_{n-1}\frac{d^{n-1} y}{dt^{n-1}} + \cdots +
a_{1}\frac{d y}{dt} + a_{0} y=
b_{m}\frac{d^{m} x}{dt^{m}} + b_{m-1}\frac{d^{m-1} x}{dt^{m-1}} + \cdots +
b_{1}\frac{d x}{dt} + b_{0} x
\end{equation*} 
und deren Laplace-Transformierten\index{Laplace} beschreiben:\\~\\
\myboxx{
\begin{equation*}
H(s)\equiv T(s)=\frac{Y(s)}{X(s)}=\frac{b_{m} s^{m} + b_{m-1} s^{m-1} +\cdots+b_{1} s 
+ b_{0}}{a_{n} s^{n} + a_{n-1} s^{n-1} + \cdots + a_{1} s + a_{0}}
=\frac{N(s)}{D(s)}.
\end{equation*}} \\~\\
 Die
"Ubertragungsfunktion des Netzwerkes $T(s)$, ist eine reelle,
rationale Funktion\index{Funktion!reelle, rationale} in $s$ der
Ordnung $n$.  Das Z"ahlerpolynom\index{Zae@{Z\"a}hlerpolynom} $N(s)$ und das
Nennerpolynom\index{Nennerpolynom} $D(s)$ sind Polynome mit reellen
und konstanten Koeffizienten, weil das Netzwerk aus linearen,
konzentrierten Elementen besteht \cite{MOS:89}. Die
Koeffizienten sind unabh"angig vom Eingangssignal.\\
{\bf\boldmath{Die Wurzeln\index{Wurzeln} der Gleichung $N(s)=0$ ergeben die $m$
 endlichen
  Nullstellen ($z_{i}, i=1,\ldots, m$); die Wurzeln von $D(s)$ ergeben
  die $n$ Pole des Systems ($p_{j}, j=1,\ldots, n$), die aus
  Stabilit"atsgr"unden\index{Stabilitat@{Stabilit\"at}} in der linken
  Halbebene\index{Halbebene!linke}
  (LHE)\index{LHE|see{Halbebene!linke}} liegen m"ussen.}}  Damit l"asst
sich $T(s)$ auch schreiben als:
\begin{equation}
T(s)=K \cdot\frac{{\displaystyle \prod_{i=1}^{m}} (s - z_{i})}
{{\displaystyle \prod_{j=1}^{n}} (s - p_{j})}
\end{equation}
wobei $K=b_{m}/a_{n}$ ist. {\bf\boldmath{ Die
  "Ubertragungsfunktion $T(s)$ ist also vollst"andig bestimmt durch ihre Pole
  und Nullstellen, sowie durch eine multiplikative Konstante $K$.}}  Da
die Wurzeln\index{Wurzeln} von Polynomen mit reellen Koeffizienten
entweder reell sind oder in konjugiert-komplexen Paaren auftreten \cite{MOS:89}, ist
es meist sinnvoll, die Systemfunktionen als Produkt von Faktoren 1.
und 2.~Ordnung mit reellen Koeffizienten darzustellen:
\begin{equation}
T(s)=K \cdot\frac{{\displaystyle\prod_{i=1}^{r}} (s^{2} + 2\sigma_{zi}\; s + 
\omega_{zi}^{2})
{\displaystyle\prod_{i=2r+1}^{m}}(s - z_{i})} 
{{\displaystyle\prod_{j=1}^{t}} (s^{2} + 2\sigma_{pj}\; s + \omega_{pj}^{2})
{\displaystyle\prod_{j=2t+1}^{n}}(s - p_{j})}
\end{equation}  
oder mit {\bf Polfrequenzen}\index{Polfrequenz} und {\bf Polg\"uten}\index{Polguten@{Polg\"uten}} geschrieben:\\~\\
\myboxx{\begin{equation} T(s)=K
    \cdot\frac{{\displaystyle\prod_{i=1}^{r}} (s^{2} +
      \displaystyle\frac{\omega_{zi}} {q_{zi}}\; s + \omega_{zi}^{2})
      {\displaystyle\prod_{i=2r+1}^{m}}(s - z_{i})}
    {{\displaystyle\prod_{j=1}^{t}} (s^{2} +
      \displaystyle\frac{\omega_{pj}}{q_{pj}}\; s + \omega_{pj}^{2})
      {\displaystyle\prod_{j=2t+1}^{n}}(s - p_{j})}.
\end{equation}}  \\~\\
Die Gr"ossen $\omega_{pj}$ bzw. $\omega_{zi}$ bezeichnet man als
{\bf Pol}- bzw. {\bf Nullstellenfrequenzen}\index{Nullstellen!frequenz}, $q_{pj}$
und $q_{zi}$ sind {\bf Pol}- und {\bf Nullstelleng"uten}\index{Nullstellen!gue@{g\"u}te}.
\subsection{Pol- und Nullstellendiagramme\index{Nullstellen!diagramm}}
In vielen F"allen ist es anschaulich und n"utzlich, die Pole $p_j$
bzw. die Nullstellen $z_i$ (Englisch:
zeros\index{zero|see{Nullstellen}}) in der komplexen $s$-Ebene
$(s=\sigma+j\omega)$ darzustellen: \\
\begin{figure}[!htb]% made by zero_pole_plot.m
\vspace*{-5mm}
\begin{center}
 \bild{/filter/FILn004.eps,width=0.72}\vspace*{-11mm}\caption{Pol- und Nullstellendarstellung in der $s$-Ebene. (Pol: {\tt x}, Nullstelle: {\tt o})}
\end{center}
\vspace*{-6mm}
\end{figure}\\
Mit dem \mb {\tt pzmap}\footnote{Achtung! In der Darstellung mit {\tt pzmap} sind mehrfache Pole oder Nullstellen nicht direkt ersichtlich!} lassen sich die Pole und Nullstellen einfach darstellen.\index{pzmap@{\tt pzmap}}\index{Pol!mehrfach}
\newpage\vspace*{-9mm}
\bsp{{\tt pzmap}}\index{pzmap@{\tt pzmap}}
\begin{figure}[!htb]% pzmap(poly([-1 1 -1+j -1-j]),poly([2 -2 -2+j -2-j]))
\vspace*{-6mm}
\begin{center}
 \bild{/f_verhalten/pzmap.eps,width=0.72}\vspace*{-3mm}\caption{Pol- und Nullstellendarstellung von {\tt pzmap(poly([-1 1 -1+j -1-j]),poly([2 -2 -2+j -2-j]))} in der $s$-Ebene. (Pol: x, Nullstelle o)}
\end{center}
\vspace*{-6mm}
\end{figure}\\
\nit Im Folgenden betrachten wir ein einzelnes Polynom 2.~Ordnung
(Abb.~\ref{Gdeut}):
\begin{figure}[!htb]
\vspace*{1mm}
\begin{center}
  %\bild{/filter/ALT/FIL4.ps,width=0.6}
{\psset{unit=1.05}
\begin{pspicture}(6,6)
\psset{linecolor=red,linewidth=2pt}
\psdots[dotsize=4pt 4,dotstyle=x]%
(1,1)(1,5)
\psset{linecolor=black,linewidth=1pt}

\psline{->}(0,3)(6,3) 
\psline{->}(3,0)(3,6) 
\uput[0](5.5,3.3){$\sigma$}\uput[0](1.9,5.7){$j\omega$}
\psdot[dotsize=12pt 12,dotstyle=square](5.5,5.5)\uput[0](5.2,5.5){$s$}

\psline{->}(3,3)(1,5)\uput[0](2,4){$\omega_p$}\rput[rb](0.9,2.6){$-\sigma_p$}
\uput[0](0.8,5.3){$p_1$}
\uput[0](0.4,0.6){$p_2=p_1^{\ast}$}
\uput[0](3.1,5){$j\tilde{\omega}_p$}
\uput[0](3.1,1){$-j\tilde{\omega}_p$}

\uput[0](2,3.3){$\alpha$}
\psarc{<->}(3,3){1}{135}{180}

\psline[linestyle=dotted,linecolor=red](1,5)(3.1,5)
\psline[linestyle=dotted,linecolor=red](1,5)(1,1) 
\psline[linestyle=dotted,linecolor=red](1,1)(3.1,1) 


\uput[0](5,4.7){$p_1=-\sigma_p+j\tilde{\omega}_p$}
\uput[0](5,4){$p_2=-\sigma_p-j\tilde{\omega}_p=p_1^{\ast}$}
\end{pspicture}}
\caption{Geometrische Deutung von Polfrequenz und Polg"ute eines konjugiert-komplexen Polpaares}\label{Gdeut} 
\end{center}
\vspace*{-9mm}
\end{figure}\index{konjugiert-komplex}\index{Pol!gute@-g\"ute}\index{Pol!frequenz@-frequenz}
\begin{equation}
(s-p_{1})\cdot(s-p_{2})=s^{2}+2\sigma_{p}s+(\sigma_{p}^{2}+\tilde{\omega}_{p}^{2}).
\end{equation}
F"ur die Polfrequenz erhalten wir:\\~\\
\myboxx{\begin{equation}
\omega_{p}=\sqrt{\sigma_{p}^{2}+\tilde{\omega}_{p}^{2}}.
\end{equation}}\\~\\
Die Polfrequenz ist also gleich dem Abstand des Pols vom Ursprung\index{Ursprung}.
Die Polg"ute ist definiert als:\\~\\
\myboxx{\begin{equation}
q_p=\frac{\omega_p}{2\cdot\sigma_p}\qquad\text{und somit gilt auch}\qquad q_p=\frac{1}{2\cdot\cos{\alpha}}.\label{F_VER_FORM_GUETE}\end{equation}}\\~\\
Betrachten wir die Grenzf"alle:\\
\hspace*{2cm}$\sigma_{p}=\omega_{p}$: (Doppelpol auf neg. reeller Achse)
\hfill $\Rightarrow \; q_{p}=\frac{1}{2}$\hspace*{4cm}~\\
\hspace*{2cm}$\sigma_{p}=0$:  (Polpaar auf imagin"arer Achse) \hfill
$\Rightarrow \; q_{p}=\infty$\hspace*{4cm}~\\
{\bf Man beachte, dass diese geometrischen Beziehungen nur f"ur ein
  konjugiert-komplexes Polpaar g"ultig sind.} Formell gelten sie
jedoch auch f"ur ein Polpaar auf der negativ reellen Achse, wobei dann
\begin{equation}
\omega_{p}=\sqrt{\sigma_{p1}\cdot\sigma_{p2}} \hspace{1cm} {\rm und}
\hspace{1cm}
q_{p}=\frac{\sqrt{\sigma_{p1}\cdot\sigma_{p2}}}{\sigma_{p1}+\sigma_{p2}} \leq
\frac{1}{2}
\end{equation}
wird. F"ur einen Einzelpol ist die G"ute nicht definiert; die
Polfrequenz entspricht dem Abstand zum Ursprung.\\ {\bf Pole}\index{Pol} gleicher
{\bf Polfrequenz}\index{Pol!frequenz@{-frequenz}} liegen auf konzentrischen Kreisen um
den Ursprung; Polpaare\index{Pol!paar} gleicher {\bf G"ute}\index{Gute@{G\"ute}} auf
Strahlenpaaren\index{Strahlenpaar}, die vom Ursprung ausgehen und
symmetrisch zur reellen Achse liegen. Diese Eigenschaften sind in
Abb.~\ref{bsp-lage} zusammengefasst.\\
\begin{figure}[!htb]   %alt \bild{/filter/FIL5.ps,width=0.6}
\begin{center}
{\psset{unit=1.2}
\begin{pspicture}(6,6)
\psline{->}(0,3)(6,3) 
\psline{->}(3,-0.5)(3,6) 
\rput[lb](6.1,2.9){$\sigma$}\rput[cb](3,6.2){$j\omega$}
\pscircle[linecolor=green,linewidth=0.4pt](3,3){2.5}
\rput[lb](5,4.5){\color{green} $\omega_p=~$konstant}

\psset{linewidth=0.4pt}
\psline[linecolor=blue](3,3)(0.5,5.5)\rput[rb](0.4,5.3){\color{blue} $q_p=~$konstant}\psline[linecolor=blue](3,3)(0.5,0.5)\rput[rb](0.4,0.3){\color{blue} $q_p=~$konstant}
\psset{linewidth=0.2pt}
\psarc{->}(3,3){2.55}{3}{15}\psarc{<-}(3,3){2.55}{-15}{-3}
\psarc{->}(3,3){2.55}{75}{90}\psarc{<-}(3,3){2.55}{90}{105}
\psarc{<-}(3,3){2.55}{165}{177}\psarc{->}(3,3){2.55}{183}{195}
\psarc{->}(3,3){2.55}{255}{270}\psarc{<-}(3,3){2.55}{270}{285}
\psset{linewidth=1pt}

\psset{linecolor=red} \psdots[dotsize=4pt 4,dotstyle=x]%
(5.5,3)(0.5,3)(5.55,3.04)(0.45,2.98)(2,4)(2,2)(1.232,4.768)(1.232,1.232)(1,5)(1,1)
\psset{linecolor=black}\textcolor{red}{
\rput[lb](5.6,2.47){Doppelter Pol}\rput[rb](0.4,2.47){Doppelter Pol}
\rput[rb](5.4,3.1){$q_p=\frac{-1}{2}$}\rput[lb](0.7,2.4){$q_p=\frac{1}{2}$}}

\rput[rb](0.4,2){$q_p>\frac{1}{2}$}\rput[lb](5.5,2){$q_p<\frac{-1}{2}$}
\rput[rb](0.4,3.2){$q_p>\frac{1}{2}$}\rput[lb](5.6,3.3){$q_p<\frac{-1}{2}$}
\rput[rb](2.9,5.55){$q_p\rightarrow\infty$}\rput[lb](3.1,5.6){$q_p\rightarrow-\infty$}
\rput[rt](2.9,0.35){$q_p\rightarrow\infty$}\rput[lt](3.1,0.4){$q_p\rightarrow-\infty$}

\psdot[dotsize=12pt 12,dotstyle=square](6,5.5)\uput[0](5.7,5.5){$s$}
\rput[rb](2.95,-0.6){$\longleftarrow $linke $s$-Halbebene (LHE) $\longrightarrow$}\rput[lb](3.05,-0.6){$\longleftarrow$ rechte $s$-Halbebene (RHE) $\longrightarrow$}
\end{pspicture}}
 \vspace*{6mm}
\caption{Lage der Pole gleicher Polfrequenz und gleicher G"ute}\label{bsp-lage}
\end{center}\index{RHE}\index{LHE}\index{linke s-Halbebene@linke $s$-Halbebene}\index{rechte s-Halbebene@rechte $s$-Halbebene}
\vspace*{-12mm}
\end{figure}~\\~\\

\nit F"ur die Nullstellen\index{Nullstellen} (Nullstelleng"uten,
Nullstellenfrequenzen) gelten die gleichen geometrischen Beziehungen.
Praktisch bedeutsam ist der Zusammenhang zwischen der
"Uber\-tragungs\-funktion\index{Ubertragungsfunktion@{\"Ubertragungsfunktion}} $T(s)$ und dem
einfach messbaren Frequenzgang\index{Frequenz!gang} $T(j\omega)$.
$T(j\omega)$ gibt das Verh"altnis des eingeschwungenen Ausgangssignals
zum anregenden harmonischen Eingangssignal der Frequenz $\omega$ an.
Im Folgenden werden wir uns mit den grundlegenden Beziehungen zwischen
$T(s)$ und $T(j\omega)$ befassen.

\clearpage
\subsection{Bestimmung des Frequenzganges aus der "Ubertra\-gungs\-funk\-tion}
Um das Verhalten eines Netzwerks im eingeschwungenen Zustand zu
ermitteln, ersetzen wir in der UTF $T(s)$ $s$ durch $j\omega$ und
variieren die anregende
Frequenz $\omega$ von $0$ bis $\infty$. {\bf\boldmath{Der praktisch bedeutsame Spezialfall\\~\\
  \myboxx{\begin{equation}
\left. T(s)\right|_{{\displaystyle s=j\omega}}=T(j\omega)
\end{equation}}~\\~\\ 
\nit wird als Frequenzgang bezeichnet.}}  $T(j\omega)$ kann in
Polarform\index{Polarform} folgendermassen dargestellt werden:
\begin{equation}
\left. T(s)\right|_{s=j\omega}=T(j\omega)=|T(j\omega)|\cdot e^{j\varphi(\omega)}
\end{equation}
wobei $|T(j\omega)|$ als {\bf Amplitudengang}\index{Amplitudengang} und
$\varphi(\omega)$ als {\bf Phasengang}\index{Phasen!gang} der
Netzwerkfunktion\index{Netzwerk!funktion} bei der Frequenz $\omega$
bezeichnet wird.  Wenn die Netzwerkfunktion eine
Spannungs\-"ubertragungs\-funktion\index{Spannungsubertragungsfunktion@{Spannungs\"ubertragungsfunktion}}\index{Ubertragungsfunktion@{\"Ubertragungsfunktion}} darstellt, ist es zweckm"assig mit dem
Logarithmus der Funktion zu rechnen. Die
Kaskadierung\index{Kaskadierung} mehrerer entkoppelter Teilnetzwerke
kann dadurch anstatt durch Multiplikation der
Teilfrequenzg"ange\index{Teilfrequenzgang} durch Addition der
logarithmischen Teilfrequenzg"ange berechnet werden.\\ Zur
{\bf logarithmischen Darstellung} fanden zwei Arten Verbreitung:
\begin{itemize}
\item  Darstellung in Neper und Radiant
       \begin{eqnarray}
       G_{\text{Neper}}(j\omega)=\ln{(T(j\omega))}&=&\ln{|T(j\omega)|}+ j\varphi(\omega)=\alpha_{\text{Neper}}(\omega) + j \varphi(\omega) \nonumber
       \end{eqnarray}
       wobei $G_{\text{Neper}}(j\omega)$ als log. Frequenzgang, $\alpha_{\text{Neper}}(\omega)$ als
       log. Amplitudengang in Neper und $\varphi(\omega)$ als
       Phasengang in Radiant bezeichnet wird.

\item  Darstellung in dB und Grad
       \begin{eqnarray}
       G_{\text{dB}}(j\omega) &=&  20 \log{|T(j\omega)|} + j \varphi(\omega)=\alpha_{\text{dB}}(\omega)  + j \varphi(\omega) \nonumber
       \end{eqnarray}       
       wobei $\alpha_{\text{dB}}(\omega)$ in dB angegeben wird.  Die Phase
       $\varphi(\omega)$ wird h"aufig in Grad, teils aber auch in
       Radianten angegeben.
\end{itemize}
Praktisch arbeitet man fast ausschliesslich mit der Darstellung in dB.
Aus den Beziehungen $|T(j\omega)|=e^{\ln{|T(j\omega)|}}=
10^{\log{|T(j\omega)|}}$ folgt:
\begin{eqnarray*} 
\alpha_{\text{dB}}=20 \log{(e)}\cdot\alpha_{\text{Neper}}&=&8.686\cdot \alpha_{\text{Neper}}. \mbox{ D.h., 1~Np}=8.686~\text{dB}.
\end{eqnarray*}
Weiter ist: $1 {\rm ~rad}=57.2958 \;{\rm Grad}=\frac{180}{\pi} \;{\rm Grad}$. Auch graphisch kann der Frequenzgang auf verschiedene Arten
aufgezeichnet werden:
\begin{itemize}
\item {\bf Nyquist-Diagramm}\index{Diagramm!Nyquist}\index{Nyquist-Diagramm}\\
      $T(j\omega)$ wird in Polarkoordinaten mit $\omega$ als Parameter 
      aufgezeichnet.
\item {\bf Bode-Diagramm}\index{Diagramm!Bode}\index{Bode-Diagramm}\\
      $\alpha_{\text{dB}}(\omega)$ und $\varphi(\omega)$ werden je in Funktion von 
      $\log{\omega}$ aufgezeichnet.
\end{itemize}
Beim veralteten
{\bf Nichols-Diagramm}\index{Diagramm!Nichols}\index{Nichols-Diagramm}
werden $\alpha_{\text{dB}}(\omega)$ und $\varphi(\omega)$ als
kartesische Koordinaten mit $\omega$ als Parameter aufgezeichnet. Das
Nyquist-Diagramm (und das Nichols-Diagramm) werden vor allem in der
Regelungstechnik angewendet.  Die gr"osste Verbreitung hat aber wegen
seiner Anschaulichkeit das Bode-Diagramm erfahren.  
\subsection{Bestimmung des Frequenzganges aus den Polen und Nullstellen der UTF}
Der Amplituden- und Phasengang der Funktion $T(s)$ kann auch {\bf graphisch}
aus den Polen und Nullstellen von $T(s)$ bestimmt werden. Gehen wir
von folgender Darstellung von $T(s)$ aus:
\begin{equation}
T(s)=K\cdot \frac{(s-z_{1})(s-z_{2})\cdots(s-z_{m})}
{(s-p_{1})(s-p_{2})\cdots(s-p_{n})}
\end{equation}
so erhalten wir mit
\begin{equation}
T(j\omega_{0})=K\cdot \frac{(j\omega_{0}-z_{1})(j\omega_{0}-z_{2})\cdots(j\omega_{0}-
           z_{m})}{(j\omega_{0}-p_{1})(j\omega_{0}-p_{2})\cdots(j\omega_{0}-
           p_{n})}
         =|T(j\omega_{0})| \cdot e^{j\varphi(\omega_{0})}
\end{equation}
den Frequenzgang bei einer beliebig w"ahlbaren Frequenz $\omega_{0}$.
Die einzelnen Faktoren in obiger Darstellung k"onnen als Vektoren in
der $s$-Ebene interpretiert werden:

\begin{figure}[!htb] % alt \bild{/filter/FIL7.ps,width=0.5}
\vspace*{1mm}
\begin{center}
{\psset{unit=0.8}
\begin{pspicture}(10,6)
\psline{->}(1,0)(9,0) \uput[0](9.1,-0.1){$\sigma$}
\psline{->}(5,-0.1)(5,5)  \uput[0](4.6,5.25){$j\omega$}

\psline(4.9,1)(5.1,1) \uput[0](5.1,0.95){$j\omega_0$}
\pscircle[linecolor=blue,linewidth=1.5pt](2,4){0.2} \uput[0](1,3.9){$z_i$}

\psline[linecolor=red,linewidth=1.5pt](2,4)(5,1) \uput[0](-2,2){\color{red}Betrag: $A_{z_i}=|j\omega_0-z_i|$}

\psline(2,4)(3.5,4)
\psarc[linecolor=MyGreen,linewidth=1.5pt]{->}(2,4){1.1}{0}{315} \uput[0](-2,5.6){\color{MyGreen}Phase: $\theta_{z_i}=\angle (j\omega_0-z_i)$}

\end{pspicture}}
\caption{Zusammenhang zwischen Frequenzgang bei $\omega_0$ und Pol- und Nullstellen (am Beispiel einer Nullstelle $z_i$)}
\end{center}
\vspace*{-12mm}\index{Nullstelle}\index{Frequenzgang}
\end{figure}~\\

\nit Somit kann die gesamte UTF wie folgt geschrieben werden:
\begin{equation}
T(j\omega_{0})=K\cdot\frac{A_{z_{1}}\cdot A_{z_{2}}\cdots A_{z_{m}}\cdot
e^{j(\theta_{z_{1}}+\cdots + \theta_{z_{m}})}}{A_{p_{1}}\cdot A_{p_{2}}\cdots
A_{p_{n}}\cdot e^{j(\theta_{p_{1}}+\cdots + \theta_{p_{n}})}}
\end{equation}
F"ur den Betrag ergibt sich:\\~\\
\myboxx{
\begin{equation}
|T(j\omega_{0})|=K\cdot\frac{{\displaystyle \prod_{i=1}^{m}}A_{z_{i}}}
{{\displaystyle \prod_{j=1}^{n}}A_{p_{j}}}
\end{equation}}\\~\\
und f"ur die Phase:\\~\\
\myboxx{
\begin{equation}
\varphi(\omega_{0})=\sum_{i=1}^{m}\theta_{z_{i}} - \sum_{j=1}^{n}\theta_{p_{j}}\label{F_VER_FORM_PHASE}
\end{equation}}\\~\\
\nit {\bf\boldmath{Der Amplitudengang\index{Amplitudengang} l"asst sich also bis
auf eine Konstante $K$ aus den Ab\-st"anden der Pol- und Nullstellen zum
Frequenzpunkt $j\omega_{0}$ bestimmen. Ebenso l"asst sich die Phase
direkt aus der Lage der Pol- und Nullstellen ermitteln.}}  

% ALT (Moschytz und etwas nicht ganz korreket)
%Bez"uglich der Phase k"onnen im weiteren noch folgende Aussagen gemacht werden, wobei wir kausale, stabile LTI-Systeme voraussetzen:\index{LTI-System}%\index{kausal}\index{stabil}
%\begin{itemize}
%\item $\theta_{z_{i}}$ nimmt mit wachsendem $\omega$ f"ur Nullstellen in der 
%      LHE monoton zu, f"ur solche in der rechte Halbebene\index{Halbebene!rechte} (RHE)\index{RHE|see{Halbebene!rechte}} monoton ab.
% \item Wird auf der Imagin"arachse bei wachsendem $\omega$ eine
%      Nullstelle durchlaufen, ergibt sich ein
 %     Phasensprung\index{Phasen!sprung} von $-\pi$ \cite{MEY:02, MOS:89}. % MEY:02 Seite 109
%\item $\theta_{p_{j}}$ nimmt mit wachsendem $\omega$ monoton zu und damit 
%      $\varphi(\omega)$ monoton ab.
%\end{itemize}


Der Betrag von $T(s)$, d.h. der Amplitudengang,  l"asst sich auch f"ur einen beliebigen Punkt der\index{Amplitudengang}
$s$-Ebene aus den Abst"anden von $s_{0}$ zu den Pol- und Nullstellen\index{Nullstellen}
bestimmen.  Mit der 3-D~Darstellung (Abb.~\ref{figure_Filter_3D}) dieser Betragsfunktion\index{Betragsfunktion}
erh"alt man ein recht anschauliches Bild.
\begin{figure}[!htb]
\begin{center}
  \bild{/filter/FIL8.ps,width=0.6}\caption{3-D~Darstellung der Betragsfunktion $\alpha_{\text{dB}}=20\log{\left|T(s)\right|}$ einer UTF mit zwei konjugiert-komplexen Polpaaren.}\label{figure_Filter_3D}\index{konjugiert-komplex}\label{Z_VER_ABB_3d}
\end{center}
\vspace*{-6mm}\index{Tschebyscheff}
\end{figure}\\
\nit Dabei entspricht der  {\bf Schnitt} l"angs der $j\omega$-Achse in Abb.~\ref{Z_VER_ABB_3d} dem
   {\bf Amplitudengang} der "Uber\-tragungs\-funktion.  Es ist
daraus leicht ersichtlich, wie sich eine Verschiebung der Pole bzw.
Nullstellen auf den Amplitudengang auswirken.
\aufg Was f\"ur ein Filterverhalten ist in Abb.~\ref{Z_VER_ABB_3d} dargestellt?\\~\\
\newpage
\bsp{3-D~Darstellung mit \matlogo~der Betragsfunktion $20\log{\left| UTF \right|}$ eines \\{\bf Butterworth-Tiefpasses 4.~Ordnung}}\index{Ordnung}\index{Butterworth}
\begin{figure}[!htb]% matlab UTF_gebirge.m
\vspace*{-6mm}\begin{center}
  \bild{/f_verhalten/butter_gebirge.eps,width=0.6}\caption{Frontalansicht der Betragsfunktion $=20\log{\left|T(s)\right|}$ einer UTF mit zwei konjugiert-komplexen Polpaaren eines Butterworth-Tiefpasses 4.~Ordnung.}\index{konjugiert-komplex}\label{F_VER_ABB_GEBIRGE1}
\end{center}
\vspace*{-6mm}
\end{figure}
\aufg Versuchen Sie mit Hilfe vom \mb {\tt surf} die Abbildungen~\ref{F_VER_ABB_GEBIRGE1} und \ref{F_VER_ABB_GEBIRGE2} nachzubilden, wobei die exakte Lage der Pole nicht wichtig ist, sie sich aber mit dem \mb {\tt buttap(4)} bestimmen l\"asst.\index{buttap@{\tt buttap}} 
\bsp{3-D~Darstellung mit \matlogo~der Betragsfunktion $20\log{\left| UTF \right|}$ eines \\{\bf Butterworth-Tiefpasses 4.~Ordnung}}\index{Ordnung}\index{Butterworth}
\begin{figure}[!htb]
\vspace*{-6mm}\begin{center}
  \bild{/f_verhalten/butter_gebirge_oben.eps,width=0.6}\caption{Ansicht von Oben der Betragsfunktion $=20\log{\left|T(s)\right|}$ einer UTF mit zwei konjugiert-komplexen Polpaaren eines Butterworth-Tiefpasses 4.~Ordnung.}\index{konjugiert-komplex}\label{F_VER_ABB_GEBIRGE2}
\end{center}
\vspace*{-6mm}
\end{figure}

\newpage
\section{Minimal- und nicht-minimalphasige Systeme}
Bei "Ubertragungsfunktionen von Zweitoren\index{Zweitor} (auch
Vierpole\index{Vierpol} gennant) unterscheidet man zwischen Minimal-
und
Nicht-Minimalphasennetzwerken.\index{Nicht-Minimalphasennetzwerken}
\index{Minimalphasennetzwerk!Nicht-} Wir beschr"anken unsere Betrachtungen auf kausale, stabile LTI-Systeme.

\subsection{Allpass}
Um den Unterschied zwischen Minimal- und Nicht-Minimalphasennetzwerken
zu er\-l\"autern, wollen wir den
Begriff der Allpassnetzwerke einf"uhren:\\~\\
\myboxx{{\bf\boldmath{Ein Allpass ist ein Netzwerk, bei dem der
      Amplitudengang f"ur alle $\omega$ konstant ist ($|T(j\omega)|=$konst.$\not 0$).}} }\\~\\

\nit Dies l"asst sich mit einer zur $j\omega$-Achse symmetrischen
Pol-Nullstellen\-konfiguration erreichen.  Die
"Uber\-tragungs\-funktion eines Allpasses lautet somit:\\~\\
\framebox[\textwidth]{ \parbox{0.9\textwidth}{
\begin{equation}
T_{A}(s)=K \frac{Q(-s)}{Q(s)}
\end{equation}}}\\~\\
wobei $Q(s)$ ein striktes Hurwitz-Polynom\index{Hurwitz!-Polynom} ist \cite{MOS:89}. 
\begin{figure}[!htb]\vspace*{-3mm}
\begin{center} % made with Allpass_plot.m
\hspace*{-1cm}\bild{/filter/FILn005a.eps,width=0.45}\hspace*{-1.5cm}\bild{/filter/FILn005b.eps,width=0.45}\hspace*{-1.5cm}\bild{/filter/FILn005c.eps,width=0.45}\vspace*{-9mm}\caption{Pol- und Nullstellenkonfiguration von Allp"assen 1., 2., und 3. Ordnung}
\end{center}
\vspace*{-6mm}
\end{figure}\\
\nit Aus der skizzierten Pol-Nullstellenverteilung\index{Nullstellen} ist
ersichtlich, dass ein Allpass einen streng monoton\index{monoton!streng} abfallenden Phasengang
besitzt. Jede beliebige (realisierbare) "Ubertragungsfunktion kann immer in ein
allpassfreies Netzwerk und in einen Allpass zerlegt werden, n"amlich \cite{FRE:BOS:04, UNB:81}:\\~\\
\myboxx{\begin{equation}
T(s)=T_{M}(s) \cdot T_{A}(s). 
\end{equation}}\\~\\
Das allpassfreie Netzwerk $T_{M}(s)$ besitzt die kleinste
Phasendrehung, die bei einem vorgeschriebenen Amplitudengang m"oglich
ist. Es wird deshalb als {\bf
  Minimalphasennetzwerk}\index{Minimalphasennetzwerk} bezeichnet
\cite{FRE:BOS:04}. Die Pole (in den Abbildungen mit 'x' markiert)
aller hier betrachteten {\bf Netzwerke}\index{Netzwerke} m\"ussen in
der linken \mat{s}{\bf -Halbebene}\index{Halbebene} sein, da ansonsten
die Netzwerke {\bf instabil}\index{instabil} sind.

\subsection{Minimalphasennetzwerke\index{Minimalphasennetzwerk}}  
Als
wesentliches Merkmal besitzt das Minimalphasennetzwerk {\bf keine
Nullstellen in der RHE} \cite{KIE:JAE:05}. Bode\index{Bode} hat gezeigt, dass bei
  Minimalphasennetzwerken ein eindeutiger Zusammenhang zwischen
  Amplituden- und Phasengang besteht \cite{MOS:89}. Somit kann mit einem
  {\bf Minimalphasennetzwerk} nur der Wunsch nach {\bf entweder} einem frei w"ahlbaren
  Amplituden- {\bf oder} Phasengang\index{Phasen!gang} erf"ullt werden.\\  Bei einem
{\bf Nicht-Minimalphasennetzwerk} k"onnen Amplituden- und Phasengang
unabh"angig voneinander gew"ahlt werden. Praktisch wird dies durch
Kaskadierung eines Minimalphasennetzwerkes und eines Allpasses
realisiert.  Allp"asse werden vor allem als
Laufzeitkorrekturglieder\index{Laufzeitkorrektur|see{Allpass}} und als
Verz"ogerungselemente\index{Verzogerung@{Verz\"ogerung}|see{korrektur}}
verwendet \cite{MOS:89}.
\bsp{Zerlegung eines Nicht-Minimalphasennetzwerk in ein Minimalphasennetzwerk und einen Allpass}\index{Allpass}
\begin{figure}[!htb]\vspace*{-3mm}
\begin{center} % made with zerlegung_phasen.m
\hspace*{-1.5cm}\bild{/f_verhalten/F_VER_004.eps,width=0.42}\hspace*{-0.5cm}\bild{/f_verhalten/F_VER_003.eps,width=0.42}\hspace*{-0.5cm}\bild{/f_verhalten/F_VER_002.eps,width=0.42}\vspace*{-9mm}\caption{Nicht-Minimalphasennetzwerk (links) = Minimalphasennetzwerk (Mitte) $\cdot$ Allpass (rechts)}
\end{center}
\vspace*{-6mm}
\end{figure}\\
\vspace*{-8mm}\bsp{Bode-Diagramm und Pole und Nullstellen von $H(s)=\frac{(s-\alpha+j\beta)(s-\alpha-j\beta)}{(s+\alpha+j\beta)(s+\alpha-j\beta)}$ mit $\alpha=1$ und $\beta=1$}
\begin{figure}[!htb]\vspace*{-6mm}
\begin{center} % made with zerlegung_phasen
\hspace*{-1.5cm}\bild{/f_verhalten/F_VER_005.eps,width=0.55}\hspace*{-1cm}\bild{/f_verhalten/allpass.eps,width=0.7}\vspace*{-1mm}\caption{Pole, Nullstellen und Bode-Diagramm von  $H(s)=\frac{(s-1+j)(s-1-j)}{(s+1+j)(s+1-j)}$}
\end{center}
\vspace*{-6mm}
\end{figure}\\
\newpage
\subsection{Bode-Diagramm und Pol- Nullstellenverteilung von verschiedenen UTF 2. Ordnung}\label{Z_VER_KAP_BSP}
F\"ur alle Bespiele a) bis f) gilt: $2\sigma_p=\frac{\omega_p}{q_p}$ (Formel~\ref{F_VER_FORM_GUETE}), wobei $|q_p|>\frac{1}{2}$ sein muss, damit die Pole konjugiert-komplex sind.\index{konjugiert-komplex}\index{Ordnung}\\
a) Tiefpass {\large $H(s)=\frac{K\cdot\omega_p^2}{s^2+2\sigma_p s+\omega_p^2}$}~\\
\begin{figure}[!htb]
\vspace*{-0.5cm}
\begin{center}
\bild{/f_verhalten/BSP_moschytz_1.eps,width=0.79}
\end{center}
\end{figure}~\index{Tiefpass}~\\
b) Bandpass~{\large $H(s)=\frac{K\cdot \omega_p\cdot s}{s^2+2\sigma_p s+\omega_p^2}$}\\
\begin{figure}[!htb]
\vspace*{-0.5cm}
\begin{center}
  \bild{/f_verhalten/BSP_moschytz_2.eps,width=0.79}
\end{center}
\end{figure}~\index{Bandpass}
  
 


\newpage 
\begin{figure}[!htb]
\vspace*{0cm}
\begin{center}
  \bild{/filter/FIL11.ps,width=0.8}
\end{center}
\vspace*{-6mm}
\end{figure}
% Hochpass mit endlichen Nullstellen machen!!!!
\newpage
\begin{figure}[!htb]
\vspace*{0cm}
\begin{center}
  \bild{/filter/FIL12.ps,width=0.8}
\end{center}
\vspace*{-6mm}
\end{figure}
\clearpage
\newpage
\section{Bode-Diagramm}\index{Bode-Diagramm}
Wie schon erw\"ahnt, wird im Bode-Diagramm der {\bf Betrag
  (Amplitude)}\index{Betrag}\index{Amplitude} und die {\bf
  Phase}\index{Phase} in Abh\"angigkeit der Frequenz ($f$ oder auch $\omega=2\pi f$ logarithmisch
aufgetragen) dargestellt. In diesem Abschnitt erarbeiten wir eine
Methode, um das Bode-Diagramm n\"aherungsweise schnell von Hand
aufzeichnen zu k\"onnen \cite{FRE:BOS:04}.\\ \nit Mit den heutigen Werkzeugen wie
\matlogo\index{Matlab@{\matlogo}} l\"asst sich das wesentlich
bequemer bewerkstelligen und exakter durchf\"uhren, doch ist eine
schnelle Absch\"atzung von Hand vielfach sehr hilfreich!\\
Zum Beispiel zeichnet der \mb {\tt bode} das Bode-Diagramm eines LTI-Systems auf und \mb {\tt bodemag} den Amplitudengang eines LTI-Systems auf.\index{bode@{\tt bode}}\index{bodemag@{\tt bodemag}}


\subsection{Konstanter Faktor (z.B. Impedanzverh\"altnis)}\label{F_VER_BODE_faktor}
\bsp{$H(s)=\alpha e^{j\beta}$ mit $\alpha=5$ und $\beta=\frac{\pi}{2}$.}
\begin{figure}[!htb]%matlab Bode04.m
\vspace*{-8mm}
\begin{center}
  \bild{/f_verhalten/Bode_04.eps,width=0.75}\vspace*{-3mm}\caption{Bode-Diagramm mit $H(s)=5\cdot e^{j\frac{\pi}{2}}$}
\end{center}
\vspace*{-6mm}
\end{figure}~\\

\nit Mit dem \mb {\tt bodemag} wir nur der Amplitudengang dargestellt.\index{bodemag@{\tt bodemag}}
\newpage

\subsection{Pol im Ursprung (z.B. Kapazit\"at)}\index{Kapazitat@{Kapazit\"at}}
\vspace*{-5mm}\bsp{$H(s)=\frac{\alpha}{s}$ mit $\alpha=10$.}
\begin{figure}[!htb]%matlab Bode04.m
\vspace*{-4mm}
\begin{center}
  \bild{/f_verhalten/Bode_05.eps,width=0.75}\vspace*{-3mm}\caption{Bode-Diagramm mit $H(s)=\frac{10}{s}$.}
\end{center}
\vspace*{-6mm}
\end{figure}


\subsection{Nullstelle im Ursprung (z.B. Induktivit\"at)}\index{Induktivitat@{Induktivit\"at}}
\vspace*{-5mm}\bsp{$H(s)=\alpha s$ mit $\alpha=2$.}
\begin{figure}[!htb]%matlab Bode04.m
\vspace*{-8mm}
\begin{center}
  \bild{/f_verhalten/Bode_06.eps,width=0.75}\vspace*{-3mm}\caption{Bode-Diagramm mit $H(s)=2 s$.}
\end{center}
\vspace*{-6mm}
\end{figure}

\newpage
\vspace*{-5mm}\subsection{Reeller Pol (z.B. Tiefpass 1.~Ordnung)}\index{Tiefpass}
\vspace*{-5mm}\bsp{$H(s)=\frac{1}{s+\alpha}$ mit $\alpha=\frac{1}{3}$.}
\begin{figure}[!htb]%matlab Bode04.m
\vspace*{-5mm}
\begin{center}
  \bild{/f_verhalten/Bode_07.eps,width=0.75}\vspace*{-3mm}\caption{Bode-Diagramm mit $H(s)=\frac{1}{s+\frac{1}{3}}$.}
\end{center}
\vspace*{-6mm}
\end{figure}

\bsp{$H(s)=\frac{\alpha}{s+ \alpha}$ mit $\alpha=\frac{1}{3}$.}
\begin{figure}[!htb]%matlab Bode04.m
\vspace*{-5mm}
\begin{center}
  \bild{/f_verhalten/Bode_07a.eps,width=0.75}\vspace*{-3mm}\caption{Bode-Diagramm mit $H(s)=\frac{\frac{1}{3}}{s+ \frac{1}{3} }$.}
\end{center}
\vspace*{-6mm}
\end{figure}

\newpage
\vspace*{-5mm}\subsection{Reelle Nullstelle}\index{Nullstellen!reell} 
\vspace*{-5mm}\bsp{$H(s)=s+\alpha$ mit $\alpha=\frac{1}{4}$.}
\begin{figure}[!htb]%matlab Bode04.m
\vspace*{-5mm}
\begin{center}
  \bild{/f_verhalten/Bode_08.eps,width=0.75}\vspace*{-3mm}\caption{Bode-Diagramm mit $H(s)=s+\frac{1}{4}$.}
\end{center}
\vspace*{-6mm}
\end{figure}

\vspace*{-7mm}\subsection{Konjugiert-komplexe Pole und Nullstellen}
\vspace*{-7mm}\bsp{Mit $H(s)=\frac{1}{(s^2+\frac{s}{q}+1^2)}$ und $q=100$ erh\"alt man folgendes Bode-Diagramm:}
\begin{figure}[!htb]%matlab Bode02.m
\vspace*{-4mm}
\begin{center}
  \bild{/f_verhalten/Bode_09a.eps,width=0.74}\vspace*{-3mm}\caption{Bode-Diagramm mit $H(s)=\frac{1}{(s^2+\frac{s}{100}+1^2)}$ (konjugiert-komplexes Polpaar)}\index{Polpaar!konjugiert-komplex}
\end{center}
\vspace*{-6mm}
\end{figure}\\

\newpage
\vspace*{-11mm}\bsp{Mit $H(s)=\frac{1}{(s^2+\frac{s}{q}+1^2)}$ und $q=10$ erh\"alt man folgendes Bode-Diagramm:}
\begin{figure}[!htb]%matlab Bode02.m
\vspace*{-6mm}
\begin{center}
  \bild{/f_verhalten/Bode_09b.eps,width=0.75}\vspace*{-3mm}\caption{Bode-Diagramm mit $H(s)=\frac{1}{(s^2+\frac{s}{10}+1^2)}$ (konjugiert-komplexes Polpaar)}\index{Polpaar!konjugiert-komplex}
\end{center}
\vspace*{-18mm}
\end{figure}\\


\bsp{Mit $H(s)=\frac{1}{(s^2+\frac{s}{q}+1^2)}$ und $q=2$ erh\"alt man folgendes Bode-Diagramm:}
\begin{figure}[!htb]%matlab Bode02.m
\vspace*{-6mm}
\begin{center}
  \bild{/f_verhalten/Bode_09c.eps,width=0.75}\vspace*{-3mm}\caption{Bode-Diagramm mit $H(s)=\frac{1}{(s^2+\frac{s}{2}+1^2)}$ (konjugiert-komplexes Polpaar)}\index{Polpaar!konjugiert-komplex}
\end{center}
\vspace*{-10mm}
\end{figure}\\~\\
\nit Bei einer UTF der Form $H(s)=\frac{K\cdot\omega_p^2}{s^2+s\frac{\omega_p}{q_p}+\omega_p^2}$ (konjugiert-komplexe Pole $\leftrightarrow |q_p|>\frac{1}{2}$, sowie $q_p,~\omega_p, K\in\mathbb{R}$), kann gezeigt werden, dass das Maximum des Amplitudenganges\index{Amplitudengang!Maximum} bei $\omega=\omega_p\sqrt{1-\frac{1}{2q_p^2}}$ liegt und den Wert $\frac{K\cdot q_p}{\sqrt{1-\frac{1}{4q_p^2}}}$ hat ($|q_p|$ muss daf"ur gr"osser als $1/\sqrt{2}$ sein).


\newpage
\vspace*{-6mm}\bsp{Mit $H(s)=(s^2+\frac{s}{q}+1^2)$ und $q=2$ erh\"alt man folgendes Bode-Diagramm:}
\begin{figure}[!htb]%matlab Bode02.m
\vspace*{-6mm}
\begin{center}

  \bild{/f_verhalten/Bode_09d.eps,width=0.75}\vspace*{-3mm}\caption{Bode-Diagramm
    mit $H(s)=(s^2+\frac{s}{2}+1^2) $ (konjugiert-komplexes
    Nullstellenpaar)}\index{Nullstellenpaar!konjugiert-komplex}
\end{center}
\vspace*{-6mm}
\end{figure}\\

\subsection{Weitere Beispiele von Bode-Diagrammen}\label{F_VER_W_BODE_BSP}
\vspace*{-6mm}\bsp{Mit $H(s)=\frac{s-1}{s+10}$ erh\"alt man folgendes Bode-Diagramm, wobei die \mb\!\!sfolge wie folgt lautet: {\tt s=tf('s'); H=(s-1)/(s+10); bode(H); grid on;} }
\begin{figure}[!htb]%matlab Bode01.m
\vspace*{-8mm}
\begin{center}
  \bild{/f_verhalten/Bode_01.eps,width=0.75}\vspace*{-3mm}\caption{Bode-Diagramm mit $H(s)=\frac{s-1}{s+10}$}
\end{center}
\vspace*{-6mm}
\end{figure}

\newpage
\bsp{Mit $H(s)=\frac{1}{(s+1)(s+1)}$ erh\"alt man folgendes Bode-Diagramm:}
\begin{figure}[!htb]%matlab Bode02.m
\vspace*{-8mm}
\begin{center}
  \bild{/f_verhalten/Bode_02.eps,width=0.75}\vspace*{-3mm}\caption{Bode-Diagramm mit $H(s)=\frac{1}{(s+1)(s+1)}$}
\end{center}
\vspace*{-6mm}
\end{figure}

\bsp{Mit $H(s)=\frac{1}{(s^2+\frac{s}{4}+1^2)}$ erh\"alt man folgendes Bode-Diagramm:}
\begin{figure}[!htb]%matlab Bode02.m
\vspace*{-6mm}
\begin{center}
  \bild{/f_verhalten/Bode_03.eps,width=0.75}\vspace*{-3mm}\caption{Bode-Diagramm mit $H(s)=\frac{1}{(s^2+\frac{s}{4}+1^2)}$ (konjugiert-komplexes Polpaar)}\index{Polpaar!konjugiert-komplex}
\end{center}
\vspace*{-6mm}
\end{figure}\\

\newpage
% etwas falsch
%\subsection{Bemerkung zum \mb {\tt bode}}
%Der \mb {\tt bode} zeichnet die Phase nicht in allen F\"allen gem\"ass
%unserer Definition (Formel~\ref{F_VER_FORM_PHASE} und Zusatzaussagen) auf\footnote{Es ist zu beachten, dass Phaseng\"ange die
%eine Verschiebung von $\pm k\cdot 2\pi$ ($k\in \mathbb{N}$) aufweisen,
%identische Verhalten haben.}.
%\bsp{$H(s)=\frac{s^2+4}{s^2+0.15s+2.25}=\frac{(s+2j)(s-2j)}{s^2+\frac{1.5}{10}s+1.5^2}$}
%\begin{figure}[!htb]\vspace*{-3mm}
%\begin{center} % made with bode ud bode2

%  \hspace*{-0.8cm}\bild{/f_verhalten/bode_falsch.eps,width=0.6}\hspace*{-1cm}\bild{/f_verhalten/bode2.eps,width=0.6}\vspace*{-1mm}\caption{Links
 %   ist der \mb {\tt bode}; rechts die neue Funktion {\tt bode2} (siehe Aufgabe~\ref{kapitel_Frequenzverhalten}.\ref{F_VER_AUF}). Das bedeutet, dass beim %Durchlaufen einer Nullstelle (auf der
%    positiven $j\omega$-Achse) der Befehl {\tt bode} den Wert $\pi$
%    nicht vom Phasengang subtrahiert.}\label{ABB_bode_bode2}
%\end{center}
%\vspace*{-6mm}
%\end{figure}\index{bode@{\tt bode}}~\\
%\nit Der Unterschied der Phaseng"ange (Abb.~\ref{ABB_bode_bode2}) ergibt sich, da der \mb {\tt bode} folgende Definition f"ur die Phase verwenden:\index%{Phase}\\
%\begin{equation}
%\varphi(\omega)=\arctan\klam{\frac{\Im \klam{H(j\omega)}}{\Re \klam{H(j\omega)}}}\label{Phase_def_2}
%\end{equation}\\
%Diese Definition ergibt meistens die gleichen Werte wie Formel~\ref{F_VER_FORM_PHASE}. Der Phasengang von Formel~\ref{Phase_def_2} bewegt sich zwischen %$\pm \pi$ und bei Nullstellen  auf der imagin"aren Achse weist der Phasengang von Formel~\ref{Phase_def_2} einen Sprung von $\pi$ auf.
%\aufg\label{F_VER_AUF}Schreiben Sie eine \matlogo-Funktion {\tt bode2}, die das Bode-Diagramm von einer UTF $H(s)$ richtig zeichnet. Der \mb {\tt bode2} soll %als Eingabe das Z\"ahler- und Nennerpolynom haben.\\ D.h. z.B, dass f\"ur $H(s)=\frac{s+1}{s^2+s\frac{2}{10}+2^2}$ der Befehl {\tt bode2([1 1],[1 2/10 4])}  %lautet.
%\ref{KAP_SYS_PHASE}
\newpage
\subsection{Regeln f\"ur die Approximation des Bode-Diagrammes\\ (Amplituden- und Phasengang)}
Mit den betrachteten Beispielen der Abschnitte~\ref{F_VER_BODE_faktor}
bis \ref{F_VER_W_BODE_BSP} k\"onnen wir einige Regeln aufstellen, um
den Verlauf des Bode-Diagramms durch {\bf Geraden}\index{Gerade} zu
approximieren \cite{FRE:BOS:04, GIR:RAB:STE:05, UNB:81}. Wir wollen uns dabei auf kausale, stabile und minimalphasige\index{minimalphasig} LTI-Systeme\index{LTI-System} beschr"anken.

\subsubsection{Grundregeln zur Approximation}
\begin{enumerate}
\item{} {\bf Konstanter Faktor:} \mat{H(s)=\alpha e^{j\beta}}\\ 
        Betrag = $20\log(\alpha)$ = konstant $\forall\omega$\\
        Phase = $\beta$ = konstant $\forall\omega$\\
        $\rightarrow$ Diese Approximation entspricht genau dem exakten Wert des Bode-Diagramms.
\item{} {\bf Pol im Ursprung:} \mat{H(s)=\frac{\alpha}{s}} \\
        Betrag = {\bf Gerade}\index{Gerade} mit {\bf Steigung}\index{Steigung} -20~dB/Dekade, Schnittpunkt der 0~dB-Linie bei $\omega=\alpha$\\
        Phase = $-\pi/2$ = konstant $\forall\omega$\\
        $\rightarrow$ Diese Approximation entspricht genau dem exakten Wert des Bode-Diagramms.
\item{} {\bf Nullstelle im Ursprung:} \mat{H(s)=\alpha s} \\
        Betrag = {\bf Gerade}\index{Gerade} mit {\bf Steigung}\index{Steigung} +20~dB/Dekade, Schnittpunkt der 0~dB-Linie bei $\omega=\frac{1}{\alpha}$\\
        Phase = $+\pi/2$ = konstant $\forall\omega$\\
        $\rightarrow$ Diese Approximation entspricht genau dem exakten Wert des Bode-Diagramms.
      \item{} {\bf Reeller Pol} 
        \begin{enumerate}\item{} \mat{H(s)=\frac{1}{s+\alpha}}\\
        Betrag = Konstante mit Wert -$20\log(\alpha)$ von $\omega=0$ bis $\omega=\alpha$, f\"ur $\omega>\alpha$ ergibt sich eine Gerade mit Steigung  -20~dB/Dekade durch den Punkt mit Amplitude -$20\log(\alpha)$ und  $\omega=\alpha$.\\
        Phase = Konstante mit Wert 0 bis $\omega<\frac{\alpha}{10}$,
        f\"ur $\omega>10\alpha$ Konstante mit Wert -$\pi/2$,
        dazwischen eine Gerade von Punkt $0,\omega=\frac{\alpha}{10}$
        und Punkt -$\pi/2,\omega=10\alpha$.
        \item{} \mat{H(s)=\frac{\alpha}{s+\alpha}}\\
        Betrag = Konstante mit Wert 0~dB von $\omega=0$ bis $\omega=\alpha$, f\"ur $\omega>\alpha$ ergibt sich eine Gerade mit Steigung  -20~dB/Dekade durch den Punkt mit Amplitude 0~dB und  $\omega=\alpha$.\\
        Phase = Konstante mit Wert 0 bis $\omega<\frac{\alpha}{10}$,
        f\"ur $\omega>10\alpha$ Konstante mit Wert -$\pi/2$,
        dazwischen eine Gerade von Punkt $0,\omega=\frac{\alpha}{10}$
        und Punkt -$\pi/2,\omega=10\alpha$.
        \end{enumerate}
\item{} {\bf Reelle Nullstelle} 
 \begin{enumerate}\item{} \mat{H(s)=s+\alpha}\\ 
        Betrag = Konstante mit Wert $20\log(\alpha)$ von $\omega=0$ bis $\omega=\alpha$, f\"ur $\omega>\alpha$ ergibt sich eine Gerade mit Steigung  +20~dB/Dekade durch den Punkt mit Amplitude $+20\log(\alpha)$ und  $\omega=\alpha$.\\
        Phase = Konstante mit Wert 0 bis $\omega<\frac{\alpha}{10}$, f\"ur $\omega>10\alpha$ Konstante mit Wert $+\pi/2$, dazwischen eine Gerade von Punkt $0,\omega=\frac{\alpha}{10}$ und Punkt $+\pi/2,\omega=10\alpha$.
\item{} \mat{H(s)=\frac{s+\alpha}{\alpha}}\\ 
        Betrag = Konstante mit Wert 0~dB von $\omega=0$ bis $\omega=\alpha$, f\"ur $\omega>\alpha$ ergibt sich eine Gerade mit Steigung  +20~dB/Dekade durch den Punkt mit Amplitude 0~dB und  $\omega=\alpha$.\\
        Phase = Konstante mit Wert 0 bis $\omega<\frac{\alpha}{10}$, f\"ur $\omega>10\alpha$ Konstante mit Wert $+\pi/2$, dazwischen eine Gerade von Punkt $0,\omega=\frac{\alpha}{10}$ und Punkt $+\pi/2,\omega=10\alpha$.
        \end{enumerate}
      \item{} {\bf Konjugiert-komplexe Pole:} 
\begin{enumerate}\item{} \mat{H(s)=\frac{1}{s^2+s\frac{\omega_p}{q_p}+\omega_p^2}}\\
        Betrag = Konstante mit Wert -$40\log(\omega_p)$ bis
        $\omega<\omega_p$, f\"ur $\omega>\omega_p$ eine Gerade mit
        Steigung -40~dB/Dekade (Startpunkt:
        ($\omega=\omega_p,-40\log(\omega_p)$)). Zus\"atzlich wird bei
        $\omega=\omega_p$ eine \"Uberh\"ohung eingezeichnet, die den Maximalwert bei $~20\log\klam{\frac{q_p}{\omega_p^2}}$ hat. Die \"Uberh\"ohung und die zwei Geraden werden zwischen $\frac{\omega_p}{2}$, $\omega_p$ und $2\omega_p$ mit zwei zus\"atzlichen Geraden verbunden. F\"ur grosse G\"uten\index{Gute@{G\"ute}} $q_p\gg 1$ stimmt diese Approximation gut. F\"ur kleine G\"uten ($q_p<1$) verschiebt sich das Maximum ein wenig von $\omega_p$ weg \cite{FRE:BOS:04, UNB:81}.\\
        Phase = Konstante mit Wert 0 f\"ur $\omega=0$ bis
        $\omega=\frac{\omega_p}{10^{\frac{1}{2q_p}}}$, ab
        $\omega>\omega_p\cdot 10^{\frac{1}{2q_p}}$ ist es eine
        Konstante mit $-\pi$. Dazwischen ist eine Gerade die beide
        Endpunkte verbindet. Bei $\omega=\omega_p$ ist die Phase genau
        -90 Grad ($-\pi/2$).
\item{} \mat{H(s)=\frac{\omega_p^2}{s^2+s\frac{\omega_p}{q_p}+\omega_p^2}}\\
   Betrag = Konstante mit Wert 0~dB bis
        $\omega<\omega_p$, f\"ur $\omega>\omega_p$ eine Gerade mit
        Steigung -40~dB/Dekade (Startpunkt:
        ($\omega=\omega_p$, 0~dB)). Zus\"atzlich wird bei
        $\omega=\omega_p$ eine \"Uberh\"ohung eingezeichnet, die den Maximalwert bei $~20\log\klam{q_p}$ hat. Die \"Uberh\"ohung und die zwei Geraden werden zwischen $\frac{\omega_p}{2}$, $\omega_p$ und $2\omega_p$ mit zwei zus\"atzlichen Geraden verbunden. F\"ur grosse G\"uten\index{Gute@{G\"ute}} $q_p\gg 1$ stimmt diese Approximation gut. F\"ur kleine G\"uten ($q_p<1$) verschiebt sich das Maximum etwas von $\omega_p$ weg.\\
        Phase = Konstante mit Wert 0 f\"ur $\omega=0$ bis
        $\omega=\frac{\omega_p}{10^{\frac{1}{2q_p}}}$, ab
        $\omega>\omega_p\cdot 10^{\frac{1}{2q_p}}$ ist es eine
        Konstante mit $-\pi$. Dazwischen ist eine Gerade die beide
        Endpunkte verbindet. Bei $\omega=\omega_p$ ist die Phase genau
        -90 Grad ($-\pi/2$).
        \end{enumerate}
      \item{} {\bf Konjugiert-komplexe Nullstellen:}
\mat{H(s)=s^2+s\frac{\omega_z}{q_z}+\omega_z^2} oder\\ \mat{H(s)=\frac{s^2+s\frac{\omega_z}{q_z}+\omega_z^2}{\omega_z^2}} \\
        Die Approximation f\"ur konjugiert-komplexe Nullstellen ist
        analog zur Approximation f\"ur konjugiert-komplexe Pole,
        wobei (im Vergleich zu den konjugiert-komplexen Polen) der Betrag an der 0~dB-Linie gespiegelt wird und die
        Phase an der 0~Grad-Linie gespiegelt wird.
\item{} Die {\bf Serieschaltung}\index{Serieschaltung} von Systemen
  erfolgt im Bode-Diagramm durch die {\bf
    Superposition}\index{Superposition} der Bode-Diagramme der
  einzelnen Teil-Systeme (Multiplikation entspricht Addition im dB-Bereich)
  \cite{FRE:BOS:04, MOS:89}. Die UTF von {\bf LTI-Systemen}\index{LTI-System} mit konzentrierten Elementen lassen sich immer in Teil-Systeme mit den oben (Punkte~1-7) beschriebenen Teil-UTFs zerlegen!
\end{enumerate}
%\newpage
\vspace*{-3mm}
\bsp{$H(s)=20\cdot \frac{s+1}{(s^2+\frac{2s}{10}+2^2)}$}
\begin{figure}[!htb]%matlab bode_approx
\vspace*{-7mm}
\begin{center}
  \bild{/f_verhalten/Bode_approx_alt_ok.eps,width=0.71}\vspace*{-8mm}\caption{Bode-Diagramm von $H(s)=20\cdot \frac{s+1}{(s^2+\frac{2s}{10}+2^2)}$ und die Approximationen der Phasen- und Amplitudeng\"ange der einzelnen Teil-Systeme}
\end{center}
\vspace*{-6mm}
\end{figure}
\newpage
\begin{figure}[!htb]%matlab bode_approx
\vspace*{-5mm}
\begin{center}
  \bild{/f_verhalten/Bode_approx_alt_ok_total.eps,width=0.73}\vspace*{-8mm}\caption{Bode-Diagramm von $H(s)=20\cdot \frac{s+1}{(s^2+\frac{2s}{10}+2^2)}$ und die Approximation des Phasen- und Amplitudenganges des Gesamtsystems}
\end{center}
\vspace*{-6mm}
\end{figure}
\aufg Bestimmen Sie mit \mb {\tt bode} f\"ur das System
$H(s)=\frac{\omega_p^2}{s^2+s\frac{\omega_p}{q_p}+\omega_p^2}$ mit
$\omega_p=2\cdot\pi\cdot 500$ und f\"ur verschiedene $q_p=0.1, 0.2, 0.5,
1, 2, 5, 10 \ldots$ die entsprechenden Bode-Diagramme. Beachten Sie
dabei das Verhalten der \"Uberh\"ohung in Abh\"angigkeit von $q_p$.

\aufg 
Zerlegen Sie die UTF $H(s)=\frac{24(s^2+\frac{s}{4}+1)}{s^2+11s+24}$ des analogen, kausalen, stabilen LTI-Systems in maximal vier sinnvolle und einfach zu approximierende Teilsysteme und stellen Sie die Approximationen der Bode-Diagramme der Teilsysteme, sowie des gesamten Systems, dar. 

\bsp{$H(s)=\frac{9(s^2+11s+10)}{10(s^2+0.3s+9)}$}
$H(s)=\frac{9(s^2+11s+10)}{10(s^2+0.3s+9)}=\underbrace{ \frac{3^2}{s^2+\frac{3}{10}s+3^2}}_{T_1(s)} \cdot \underbrace{ (s+1)}_{T_2(s)}\cdot \underbrace{ \klam{\frac{s}{10}+1}}_{T_3(s)}= \\ \underbrace{ \frac{1}{s^2+\frac{3}{10}s+3^2}\cdot (s+1)\cdot (s+10)\cdot \frac{9}{10}}_{\text{Alternative Darstellung}}$\\
\begin{figure}[!htb]
\begin{center}
\bild{/f_verhalten/bode_approx2.eps,width=0.88} \vspace*{-0.5cm}\bild{/f_verhalten/bode_approx1.eps,width=0.88}\caption{Approximation der UTF $H(s)=\frac{9(s^2+11s+10)}{10(s^2+0.3s+9)}$}
\end{center}
\vspace*{-6mm}
\end{figure}

\clearpage
\newpage
\section{Ortskurve (Nyquist-Diagramm)}\index{Ortskurve}\index{Nyquist-Diagramm}
Der \mb {\tt nyquist} zeichnet die Ortskurve-Diagramm eines
LTI-Systems auf. Zu beachten ist, dass normalerweise beim \mb {\tt nyquist} alle {\bf
  Kreisfrequenzen}\index{Kreisfrequenzen} von $-\infty$ \"uber 0 bis
$+\infty$ dargestellt werden.

\subsection{Konstanter Faktor (z.B. komplexes Widerstandsverh\"altnis)}
\bsp{$H(s)=\alpha e^{j\beta}$ mit $\alpha=\sqrt{10}$ und $\beta=\frac{-\pi}{4}$.}
\begin{figure}[!htb] %matlab nyquist04.m
\vspace*{-2mm}
\begin{center}
  \bild{/f_verhalten/Nyquist_04.eps,width=0.75}\vspace*{-3mm}\caption{Ortskurve  mit $H(s)=\sqrt{10}\cdot e^{j\frac{-\pi}{4}}$. Die Ortskurve ist ein fixer Punkt (Stern ``*'' in Abbildung) mit Imagin\"arteil -2.2361 und Realteil 2.2361.}\index{Realteil}
\end{center}
\vspace*{-6mm}
\end{figure}~\\
Die Hilfslinien in der Abbildung der Ortskurve werden mit {\tt grid on} eingeblendet.\index{grid on@{\tt grid on}} Die Hilfslinien zeigen jeweils einen konstanten Wert f"ur das geschlossene System auf (siehe Kapitel~\ref{NYQ-KRIT}) und helfen bei der Stabilit"atsbetrachtungen.

\newpage

\vspace*{-5mm}\subsection{Pol im Ursprung (z.B. Kapazit\"at)}\index{Kapazitat@{Kapazit\"at}}
\vspace*{-5mm}\bsp{$H(s)=\frac{\alpha}{s}$ mit $\alpha=10$.}
\begin{figure}[!htb]%matlab 
\vspace*{-4mm}
\begin{center}
  \bild{/f_verhalten/Nyquist_05.eps,width=0.75}\vspace*{-3mm}\caption{Ortskurve mit $H(s)=\frac{10}{s}$.}
\end{center}
\vspace*{-6mm}
\end{figure}


\vspace*{-5mm}\subsection{Nullstelle im Ursprung (z.B. Induktivit\"at)}\index{Induktivitat@{Induktivit\"at}}
\vspace*{-5mm}\bsp{$H(s)=\alpha s$ mit $\alpha=2$.}
\begin{figure}[!htb]%matlab
\vspace*{-8mm}
\begin{center}
  \bild{/f_verhalten/Nyquist_06.eps,width=0.75}\vspace*{-3mm}\caption{Ortskurve mit $H(s)=2 s$.}
\end{center}
\vspace*{-6mm}
\end{figure}

\newpage

\vspace*{-5mm}\subsection{Reeller Pol (z.B. Tiefpass 1.~Ordnung)}
\vspace*{-5mm}\bsp{$H(s)=\frac{1}{s+\alpha}$ mit $\alpha=\frac{1}{3}$.}
\begin{figure}[!htb]
\vspace*{-5mm}
\begin{center}
  \bild{/f_verhalten/Nyquist_07.eps,width=0.75}\vspace*{-3mm}\caption{Ortskurve mit $H(s)=\frac{1}{s+\frac{1}{3}}$.}
\end{center}
\vspace*{-6mm}
\end{figure}

\vspace*{-6mm}
\subsection{Reelle Nullstelle}
\vspace*{-5mm}\bsp{$H(s)=s+\alpha$ mit $\alpha=\frac{1}{4}$.}
\begin{figure}[!htb]
\vspace*{-5mm}
\begin{center}
  \bild{/f_verhalten/Nyquist_08.eps,width=0.75}\vspace*{-3mm}\caption{Ortskurve mit $H(s)=s+\frac{1}{4}$.}
\end{center}
\vspace*{-6mm}
\end{figure}

\newpage
\subsection{Weitere Ortskurven (Nyquist-Diagramm)}
\vspace*{-6mm}\bsp{Mit $H(s)=\frac{s-1}{s+10}$ erh\"alt man folgendes Nyquist-Diagramm, wobei die \mb\!\!sfolge wie folgt lautet: {\tt s=tf('s'); H=(s-1)/(s+10); nyquist(H); grid on;}}
\begin{figure}[!htb]%matlab Nyquist_01
\vspace*{-8mm}
\begin{center}
  \bild{/f_verhalten/Nyquist_01.eps,width=0.75}\vspace*{-3mm}\caption{Nyquist-Diagramm mit $H(s)=\frac{s-1}{s+10}$}
\end{center}
\vspace*{-6mm}
\end{figure}



\vspace*{-12mm}\bsp{Mit $H(s)=\frac{1}{(s+1)(s+1)}$ erh\"alt man folgendes Nyquist-Diagramm}
\begin{figure}[!htb]%matlab Nyquist_02
\vspace*{-8mm}
\begin{center}
  \bild{/f_verhalten/Nyquist_02.eps,width=0.75}\vspace*{-3mm}\caption{Nyquist-Diagramm mit $H(s)=\frac{1}{(s+1)(s+1)}$}
\end{center}
\vspace*{-9mm}
\end{figure}
\newpage
\vspace*{-12mm}\bsp{Mit $H(s)=\frac{1}{(s^2+\frac{s}{4}+1^2)}$ erh\"alt man folgendes Nyquist-Diagramm:}
\begin{figure}[!htb]%matlab nyquist03.m
\vspace*{-4mm}
\begin{center}
  \bild{/f_verhalten/Nyquist_03.eps,width=0.75}\vspace*{-3mm}\caption{Nyquist-Diagramm mit $H(s)=\frac{1}{(s^2+\frac{s}{4}+1^2)}$ (konjugiert-komplexes Polpaar)}\index{Polpaar!konjugiert-komplex}
\end{center}
\vspace*{-6mm}
\end{figure}
\vspace*{-6mm}
\newpage
\section{Nichols-Diagramm}\index{Nichols-Diagramm}
Der \mb {\tt nichols} zeichnet das Nichols-Diagramm eines LTI-Systems auf. Zu beachten ist, dass normalerweise beim \mb {\tt nichols} alle {\bf
  Kreisfrequenzen}\index{Kreisfrequenzen} von 0 bis
$+\infty$ dargestellt werden.
\vspace*{-3mm}
\subsection{Konstanter Faktor (z.B. komplexes Widerstandsverh\"altnis)}
\vspace*{-3mm}\bsp{$H(s)=\alpha e^{j\beta}$ mit $\alpha=\sqrt{10}$ und $\beta=\frac{-\pi}{4}$.}
\begin{figure}[!htb]%matlab nichols04.m
\vspace*{-8mm}
\begin{center}
  \bild{/f_verhalten/Nichols_04.eps,width=0.75}\vspace*{-3mm}\caption{Nichols-Diagramm  mit $H(s)=\sqrt{10}\cdot e^{j\frac{-\pi}{4}}$. Die Ortskurve ist ein fixer Punkt (Stern ``*'' in Abbildung) mit Koordinaten 10~dB und -45 Grad.}\index{Realteil}
\end{center}
\vspace*{-6mm}
\end{figure}


\newpage

\vspace*{-5mm}\subsection{Pol im Ursprung (z.B. Kapazit\"at)}\index{Kapazitat@{Kapazit\"at}}
\vspace*{-5mm}\bsp{$H(s)=\frac{\alpha}{s}$ mit $\alpha=10$.}
\begin{figure}[!htb]%matlab 
\vspace*{-3mm}
\begin{center}
  \bild{/f_verhalten/Nichols_05.eps,width=0.75}\vspace*{-3mm}\caption{Nichols-Diagramm  mit $H(s)=\frac{10}{s}$.}
\end{center}
\vspace*{-6mm}
\end{figure}


\vspace*{-5mm}\subsection{Nullstelle im Ursprung (z.B. Induktivit\"at)}\index{Induktivitat@{Induktivit\"at}}
\vspace*{-5mm}\bsp{$H(s)=\alpha s$ mit $\alpha=2$.}
\begin{figure}[!htb]%matlab
\vspace*{-8mm}
\begin{center}
  \bild{/f_verhalten/Nichols_06.eps,width=0.75}\vspace*{-3mm}\caption{Nichols-Diagramm mit $H(s)=2 s$.}
\end{center}
\vspace*{-6mm}
\end{figure}

\newpage
\vspace*{-6mm}\subsection{Reeller Pol (z.B. Tiefpass 1.~Ordnung)}
\vspace*{-5mm}\bsp{$H(s)=\frac{1}{s+\alpha}$ mit $\alpha=\frac{1}{3}$.}
\begin{figure}[!htb]%matlab 
\vspace*{-5mm}
\begin{center}
  \bild{/f_verhalten/Nichols_07.eps,width=0.75}\vspace*{-3mm}\caption{Nichols-Diagramm mit $H(s)=\frac{1}{s+\frac{1}{3}}$.}
\end{center}
\vspace*{-6mm}
\end{figure}
\vspace*{-6mm}

\subsection{Reelle Nullstelle}
\vspace*{-5mm}\bsp{$H(s)=s+\alpha$ mit $\alpha=\frac{1}{4}$.}
\begin{figure}[!htb]%matlab
\vspace*{-5mm}
\begin{center}
  \bild{/f_verhalten/Nichols_08.eps,width=0.75}\vspace*{-3mm}\caption{Nichols-Diagramm mit $H(s)=s+\frac{1}{4}$.}
\end{center}
\vspace*{-6mm}
\end{figure}

\subsection{Weitere Beispiele von Nichols-Diagrammen}
\bsp{Mit $H(s)=\frac{s-1}{s+10}$ erh\"alt man folgendes Nichols-Diagramm, wobei die \mb\!\!sfolge wie folgt lautet: {\tt s=tf('s'); H=(s-1)/(s+10); nichols(H); grid on;}}
\begin{figure}[!htb]%matlab Nichols_01
\vspace*{-8mm}
\begin{center}
  \bild{/f_verhalten/Nichols_01.eps,width=0.75}\vspace*{-3mm}\caption{Nichols-Diagramm mit $H(s)=\frac{s-1}{s+10}$}
\end{center}
\vspace*{-6mm}
\end{figure}

\bsp{Mit $H(s)=\frac{1}{(s+1)(s+1)}$ erh\"alt man folgendes Nichols-Diagramm}
\begin{figure}[!htb]%matlab Nt_02
\vspace*{-8mm}
\begin{center}
  \bild{/f_verhalten/Nichols_02.eps,width=0.75}\vspace*{-3mm}\caption{Nichols-Diagramm mit $H(s)=\frac{1}{(s+1)(s+1)}$}
\end{center}
\vspace*{-6mm}
\end{figure}


\newpage
\vspace*{-6mm}\bsp{Mit $H(s)=\frac{1}{(s^2+\frac{s}{4}+1^2)}$ erh\"alt man folgendes Nichols-Diagramm:}
\begin{figure}[!htb]%matlab nichols03.m
\vspace*{-4mm}
\begin{center}
  \bild{/f_verhalten/Nichols_03.eps,width=0.75}\vspace*{-3mm}\caption{Nichols-Diagramm mit $H(s)=\frac{1}{(s^2+\frac{s}{4}+1^2)}$ (konjugiert-komplexes Polpaar)}\index{Polpaar!konjugiert-komplex}
\end{center}
\vspace*{-6mm}
\end{figure}
\newpage
\section{Nyquist-Kriterium (Stabilit\"atsbestimmung mittels Ortskurve)}\index{Nyquist-Kriterium}\index{Ortskurve}\label{NYQ-KRIT}
Das {\bf Nyquist-Kriterium} wird vor Allem f\"ur die Stabilit\"atsbestimmung\index{Stabilitat@{Stabilit\"at}} in regelungstechnischen Problemstellungen verwendet \cite{UNB:81}. Die Idee des {\bf Nyquist-Kriteriums}\index{Nyquist-Kriterium} ist es, anhand der {\bf Ortskurve} von $H(s)$ ({\bf offener Regelkreis})\index{Regelkreis!offen} eine Aussage \"uber die Stabilit\"at\index{Stabilitat@{Stabilit\"at}} vom {\bf geschlossenen Regelkreis}\index{Regelkreis!geschlossen} zu machen. Wir betrachten jeweils ein {\bf LTI-System}\index{LTI} 
\begin{equation*}
  H(s)=\frac{D(s)}{N(s)},
\end{equation*}
wobei $D(s)$ und $N(s)$ {\bf teilerfremd}\index{teilerfremd} sind und die
{\bf Ordnung}\index{Ordnung} von $N(s)$ gr\"osser oder gleich der
Ordnung von $D(s)$ ist.\\ 
\begin{figure}[!htb]
\vspace*{-2mm}
\begin{center}
  \bild{/f_verhalten/regelkreis.ps,width=0.85}\vspace*{-3mm}\caption{Offener und geschlossener Regelkreis mit Regelstrecke $H(s)$ und Regler ``-1'' \cite{WOS:88}.}\index{Regelstrecke}\index{Regler}
\end{center}
\vspace*{-6mm}
\end{figure}\\
F\"ur den geschlossenen Regelkreis ergibt sich eine {\bf UTF}\index{UTF} von 
\begin{equation*}
  H_{\text{geschlossen}}(s)=\frac{H(s)}{1+H(s)}=\frac{D(s)}{D(s)+N(s)}.
\end{equation*}\\
Die {\bf charakteristische Gleichung}\index{charakteristische Gleichung}\index{Gleichung!charakteristisch} des 
 geschlossenen Regelkreises (mit Regler  ``-1'') ist somit
\begin{equation*}
  1+H(s)=D(s)+N(s)=0.
\end{equation*}
Von den Abschnitten~\ref{SYS_SEC_HURWITZ} und \ref{SYS_SEC_STABIL} wissen wir, das {\bf LTI-Systeme} mit Polen in der linken $s$-Halbebene\index{Halbebene} {\bf asymptotisch stabil} sind.\index{stabil}\index{asymptotisch stabil} Somit k\"onnen wir sagen: Sind alle Nullstellen von  $1+H(s)=D(s)+N(s)$ in der linken $s$-Halbebene\index{Halbebene}, so ist der geschlossene Regelkreis {\bf asymptotisch stabil}.\\
\nit In der Ortskurve von  $1+H(s)$ muss zur Stabilit\"atskontrolle gepr\"uft werden, ob der {\bf kritische Nullpunkt}\index{kritische Punkt} ``0'' beim Durchlaufen von $\omega=0$ bis  $\omega=+\infty$, links oder rechts liegengelassen wird \cite{WOS:88}. Das {\bf Nyquist-Kriterium}\index{Nyquist-Kriterium}  stattdessen betrachtet die Ortskurve des {\bf offenen Systems} $H(s)$ bez\"uglich des kritischen Punktes ``-1'', somit erhalten wir die vereinfachte Form des Nyquist-Kriteriums \cite{UNB:81}:\\~\\
\myboxx{Ist der {\bf offene} Regelkreis $H(s)$ {\bf asymptotisch
    stabil}\index{asymptotisch stabil}, so ist der {\bf geschlossene}
  Regelkreis $1+H(s)=D(s)+N(s)$ asymptotisch stabil, wenn die {\bf
    Ortskurve} des {\bf offenen} Regelkreises den kritischen Punkt
  (-1,$j0$) weder umkreist noch durchl\"auft.
}\\~\\
Die Betrachtungen lassen sich einfach auf {\bf Regler}\index{Regler}
mit $-H_R(s)$ (anstelle ``-1'') erweitern \cite{FRE:BOS:04, UNB:81}.
\bsp{$H(s)=\frac{3}{(s+1)(s+3)(s+0.1)}$}
\begin{figure}[!htb]
\vspace*{-2mm}% Nyqyist_04.m
\begin{center}
  \bild{/f_verhalten/Nyquist_Krit.eps,width=0.71}\vspace*{-3mm}\caption{Ortskurve von $H(s)=\frac{3}{(s+1)(s+3)(s+0.1)}$ hergestellt mit dem \mb {\tt nyquist}.}
\end{center}
\vspace*{-6mm}
\end{figure}\\
\nit Der offene Regelkreis $H(s)$ ist asymptotisch stabil, da alle
Pole in der linken $s$-Halbebene sind. Die {\bf
  Ortskurve}\index{Ortskurve} des offenen Regelkreises $H(s)$ geht
nicht durch den Punkt\footnote{Der \mb {\tt nyquist} zeichnet jeweils
  ein rotes Plus ``+'' im Punkt (-1,$j0$) ein.} (-1,$j0$) und umkreist
ihn auch nicht.  Somit ist der geschlossene Regelkreis, mit der {\bf
  charakteristische Gleichung}\index{charakteristische Gleichung}
$H(s)+1=3+(s+1)(s+3)(s+0.1)=0$, {\bf asymptotisch
  stabil}\index{asymptotisch stabil}.\\
Man h\"atte aber auch die Nullstellen von
$H(s)+1=3+(s+1)(s+3)(s+0.1)=s^3+4.1s^2+3.4s+3.3=0$ berechnen k\"onnen.
Sie sind bei $p_1=-3.3834$, $p_{2,3}=-0.3583\pm 0.9203j$\footnote{Der
  \mb {\tt roots} bestimmt die {\bf Nullstellen}\index{Nullstellen}
  eines {\bf Polynoms}\index{Polynom}. Z.B.  {\tt roots([1 41. 3.4 3.3
    ])} ergibt -3.3834, -0.3583 + 0.9203i und -0.3583 - 0.9203i}. Das
System ist damit {\bf asymptotisch stabil}, da alle Pole in der linken
{\bf Halbebene}\index{Halbebene} sind.\\ Das {\bf
  Nyquist-Kriterium}\index{Nyquist-Kriterium} erlaubt es, ohne die
Nullstellen von $H(s)+1=0$ explizit berechnen zu m\"ussen, die
Stabilit\"at anhand der Ortskurve von $H(s)$ zu bestimmen.  \newpage


\section{Stabilit\"atsbestimmung mittels Bode-Diagramm}
\myboxx{Gem\"ass \cite{UNB:81} gilt mit der Bedingung, dass wenn der
  {\bf offene} Regelkreis $H(s)$ nur Pole in der linken $s$-Halbebene
  hat (und ein oder zwei Pole im Ursprung $s=0$), der {\bf
    geschlossene} Regelkreis genau dann {\bf asymptotisch
    stabil} ist, wenn $H(j\omega)$ f\"ur die {\bf Durchgangsfrequenz}\index{Durchgangsfrequenz} $\omega_D$ bei der  Amplitude $20\log(|H(j\omega_D)|)=0$~dB ist, und eine Phase $>-\pi$ hat. }\\~\\
Aus dieser Aussage ergeben sich folgende zwei Begriffe; der {\bf
  Amplitudenrand}\index{Amplitudenrand} und der {\bf
  Phasenrand}\index{Phasenrand}.
\subsection{Amplitudenrand}\index{Amplitudenrand}
Der {\bf Amplitudenrand}\index{Amplitudenrand} ist der Abstand des Amplitudenganges zur 0~dB-Linie bei der Kreisfrequenz $\omega$, wo die Phase gleich $-\pi$ (-180 Grad) ist.
\subsection{Phasenrand}\index{Phasenrand}
Der   {\bf Phasenrand}\index{Phasenrand} ist der Abstand das Phasenganges zur -$\pi$-Linie bei der Kreisfrequenz $\omega$, wo die Amplitude gleich 0~dB ist.
\vspace*{-3mm}\bsp{$H(s)=\frac{3}{s(s+1)(s+3)}$}
\begin{figure}[!htb]
\vspace*{-5mm}% Phasen_rand.m
\begin{center}
  \bild{/f_verhalten/Rand.eps,width=0.88}\vspace*{-3mm}\caption{Die 0~dB-Linie wird bei $\omega_D=0.7682$ geschnitten und die Phase schneidet die -$\pi$-Linie bei $\omega=1.7321$. Der Amplitudenrand betr\"agt somit 12~dB und der Phasenrand ist 0.6651 Radiant (38.11 Grad).}
\end{center}
\vspace*{-8mm}
\end{figure}\\
\nit Damit eine System stabil ist, m\"ussen Phasen- und Amplitudenrand
$>0$ sein. Je gr\"osser der Phasen- und Amplitudenrand ist, desto
``stabiler'' ist das System \cite{UNB:81}.  
\newpage
\vspace*{-9mm}
\bsp{Mit dem \mb {\tt margin} wird der Phasen- und Amplitudengang dargestellt. {\tt margin([3],[1 4 3 0])} ergibt die folgende Abbildung:}
\begin{figure}[!htb]
\vspace*{-5mm}% margin([3],[1 4 3 0])}}

\begin{center}
  \bild{/f_verhalten/Phasen_amp_rand_bsp.eps,width=0.71}\vspace*{-3mm}\caption{{\tt margin([3],[1 4 3 0])}}
\end{center}
\vspace*{-8mm}
\end{figure}\\
\index{margin@{\tt margin}}

\bsp{Mit dem \mb {\tt sisotool(tf([3],[1 4 3 0]))} erhalten wir:}
\begin{figure}[!htb]
\vspace*{-5mm}% margin([3],[1 4 3 0])}}
\begin{center}
  \bild{/f_verhalten/sisotool_bsp.eps,width=0.71}\vspace*{-3mm}\caption{{\tt sisotool(tf([3],[1 4 3 0]))}}
\end{center}
\vspace*{-8mm}
\end{figure}\\
\index{sisotool@{\tt sisotool}}
\newpage
\vspace*{-9mm}
\bsp{Mit dem \mb {\tt allmargin} wird ebenfalls der Phasen- und Amplitudengang berechnet. Mit {\tt allmargin(tf([3],[1 4 3 0]))} erhalten wir:}\vspace*{-5mm}
\begin{verbatim}
GainMargin: 4.0000

GMFrequency: 1.7321
PhaseMargin: 38.1045
PMFrequency: 0.7682
DelayMargin: 0.8657
DMFrequency: 0.7682
Stable: 1
\end{verbatim}\index{allmargin@{\tt allmargin}}\index{tf@{\tt tf}}
\section{Abschliessende Worte}
Neben dem {\bf Bode-Diagramm}, der {\bf Ortskurve (Nyquist-Diagramm)} und dem {\bf Nichols-Diagramm} gibt es weitere Diagramm, wie z.B. das {\bf Hall-Diagramm}\index{Hall-Diagramm} \cite{UNB:81}.

\begin{table}[htb]
\begin{center}\vspace*{-3mm}
{
\begin{tabular}{|p{7cm}|p{7cm}|}\hline
{\bf Bode-Diagramm} & {\bf Ortskurve}  \\ \hline\hline
 Doppelt logarithmische Darstellung & Frequenzgang in der komplexen Ebene \\ \hline
 N\"utzlich f\"ur die Synthese von Systemen & N\"utzlich f\"ur die Analyse von Systemen (Nyquist-Kriterium)  \\ \hline
Zuordnung von Frequenzen $f$ zu Amplitude und Phase eindeutig& Frequenz kein Parameter der Ortskurve\\ \hline
Serieschaltung von Systemen $\rightarrow$ Addition der Amplituden- und Phaseng\"ange der verschiedenen Teil-Systeme & --  \\ \hline
\end{tabular}\caption{Vergleich zwischen Bode-Diagramm und Ortskurve (Nyquist-Diagramm)}

}
\end{center}
\index{Nyquist-Kriterium}\index{Serieschaltung}\index{Ortskurve}\index{Bode-Diagramm}
\end{table}
\vspace*{-2mm}
\subsection{Umrechnungen mittels {\sc\bf Matlab}}
\vspace*{-5mm}
\begin{table}[htb]
\begin{tabular}{|p{3cm}||p{2.5cm}|p{2.5cm}|p{2.5cm}|p{2.5cm}|}\hline

%\begin{tabular}{llllllll}\hline
 & UTF & Zustands\-raum\-darstellung & Pole, Nullstellen, Verst\"arkung & Partial\-bruch\-zerlegung  \\ \hline\hline
UTF & --  & {\tt tf2ss} & {\tt tf2zp, roots} & {\tt residue}  \\ \hline
Zustands\-raum\-darstellung & {\tt ss2tf}  & -- & {\tt ss2zp} & -- \\ \hline
Pole, Nullstellen, Verst\"arkung & {\tt zp2tf, poly} & {\tt zp2ss}&-- &--  \\ \hline
 Partial\-bruch\-zerlegung & {\tt residue}  & -- & -- & -- \\ \hline
\end{tabular}\vspace*{-3mm}\caption{Umrechnung zwischen verschiedenen {\bf LTI-System-Darstellungen} mittels \mb\!\!en}

\end{table}\index{LTI-System}\index{tf2ss@{\tt tf2zp}}\index{tf2ss@{\tt tf2zp}}
\index{roots@{\tt roots}}\index{residue@{\tt residue}}
\index{ss2tf@{\tt ss2tf}}\index{ss2zp@{\tt ss2zp}}
\index{zp2tf@{\tt zp2tf}}\index{poly@{\tt poly}}

\newpage
\section{Weitere Aufgaben zum Frequenzverhalten von analogen LTI-Systemen}
\begin{enumerate}
\item{\bf dB-Werte}\\
Von einem LTI-System mit dem Eingangssignal $x(t)$ und dem Ausgangssignal $y(t)$ ist die Verst\"arkung $A$ in dB gesucht, wenn gilt ($t_0\in\mathbb{R}$):\\~\\
\begin{tabular}{l l}
 (a) $y(t)= x(t)$ & (e) $y(t)=100e^{-j\phi} x(t)$, ~~$\phi\in\mathbb{R}$\\
 (b) $y(t)=10\cdot x(t)$ & (f) $y(t)=\sqrt{2}\cdot x(t-t_0)$\\
 (c) $y(t)=0.1\cdot x(t+t_0)$ & (g) $y(t)=x(t)/\sqrt{2}$\\
 (d) $y(t)=-0.01\cdot x(t-t_0)$  & (h) $y(t)=\sqrt{10}\cdot x(t+t_0)$  \\
\end{tabular}

\item {\bf Zusammenhang zwischen \mat{L_U}, \mat{L_I} und \mat{L_P}}\\ Am Eingang
  und Ausgang eines Vierpoles werden Strom und Spannung gemessen (DC).\index{DC}
  $L_U$ wird mit 6.02~dB und $L_I$ mit 0~dB angegeben. Bestimmen Sie
  $L_P$ aus $L_I$ und $L_U$ und stellen Sie einen allgemein g"ultigen
  formellen Zusammenhang zwischen $L_P, L_I$ und $L_U$ her
  $(L_P=f(L_U,L_I))$.

\item{\bf Bode-Diagramm}\\ Skizzieren Sie das
  Bode-Diagramm\index{Bode-Diagramm} (Amplitudengang und
  Phasengang)\index{Amplitudengang}\index{Phasengang} f\"ur folgende UTF\index{UTF} und kontrollieren Sie Ihre Resultate mit dem \mb {\tt bode}\index{bode@{\tt bode}}:\\% HSU:95 s.265 GIR:RAB:STE:05
\begin{tabular}{l l}
 (a) $H(s)=\frac{1+s}{10} $ \hspace*{3.7cm}& (c) $H(s)=\frac{10^3\cdot (1+s)}{(100+s)(10+s)} $\\
 (b) $H(s)=\frac{20}{20+s} $ & (d) $H(s)=\frac{(s+1000)}{(10+s)^2} $\\
 
\end{tabular}
   
\item {\bf Mathematik (just for fun)}\\ Einer Ihrer Freunde bietet
  Ihnen eine Wette an. Als Sie zum ersten Mal den
  Englischkurs zusammen besuchen, z"ahlen sie 27 Teilnehmer. Nun
  wettet Ihr Freund, dass es in der Klasse mindestens zwei Personen
  gibt, die am selben Tag Geburtstag haben. W"urden Sie auf die
  Wette eingehen? \\ Berechnen Sie ab wievielen Personen in einer
  Gruppe die Wahrscheinlichkeit, dass zwei Personen am selben Tag
  Geburtstag haben, gr"osser als 50\% ist. Wir nehmen an, dass das
  Jahr 365 Tage hat und dass die Wahrscheinlichkeit, dass jemand an
  einem bestimmten Tag Geburtstag hat, f"ur alle Tage $1/365$
  betr"agt.  

\item {\bf Mathematik (just for fun)}\\ Bestimmen Sie die Anzahl
  Stellen von $19^{1999}$ sowie die letzten zwei Stellen dieser Zahl. Wie lautet
  die Formel f"ur die letzten zwei Stellen von $19^n$ mit $n \in
  \mathbb{N}$?

\item {\bf Bode-Diagramm und Gruppenlaufzeit \cite{FRE:BOS:04}}\index{Gruppenlaufzeit}\index{Bode}
\begin{enumerate}
\item{} Skizzieren Sie das Bode-Diagramm von $H(s)=\frac{9.9\cdot
    s}{s^2+9.9\cdot s +1}$ und berechnen Sie die Gruppenlaufzeit
  $\tau_G(\omega)$ von $H(s)$.
  \item{} Was f\"ur ein Filter ist $H(s)$?
  \item{} Kontrollieren Sie Ihre Resultate mit Hilfe von \matlogo.
\end{enumerate}
\newpage

\item {\bf Mathematik (just for fun)}\\ Wenn man sich die Frage
  stellt, wie lange eine K"ustenlinie oder Grenze ist, kommt man auf
  {\bf Helge von Kochs}\index{Kochs!Helge von} ``{\bf
    Schneeflocke}''\index{Schneeflocke} von 1904 zu sprechen.\\ Kochs
  ``Schneeflocke'' kann durch einen iterativen Prozess aufgebaut
  werden, indem jede gerade Strecke im n"achsten Schritt durch eine
  neue Strecke mit vier gleichlangen Teilstrecken ersetzt wird (siehe "Ubergang von
  $k=0$ zu $k=1$).  Wie lange ist die Strecke bei $k=5$ und bei
  $k=n\in\mathbb{N}$ im Verh"altnis zu $k=0$.
\begin{figure}[htb]
  \begin{center}\vspace*{+2mm}
  \bild{/f_verhalten/koch.fig.eps,width=0.5}\vspace*{-1mm}\caption{Aufbau von Helge von Kochs ``Schneeflocke''}
\end{center}
 \end{figure}

\item {\bf Bode-Diagramm}\\ Versuchen Sie mit \matlogo\index{Matlab@{\matlogo}} alle Bode-Diagramme von Abschnitt~\ref{Z_VER_KAP_BSP} nachzuvollziehen und ver\"andern Sie die jeweiligen Parameter\index{Parameter} (z.B. variieren Sie $\sigma_p$ beim Tiefpass)\index{Tiefpass}.
  

\item {\bf Mathematik (just for fun)}\\ Schreiben Sie alle nat\"urlichen Zahlen ($\mathbb{N}$) von 2 startend in einer Reihe auf. Unter jede Zahl schreiben Sie die jeweilige Primfaktorzerlegung. Unter jede Zerlegung, die aus einer geraden Anzahl Primfaktoren besteht, schreiben Sie {\bf G}, unter die, welche eine ungerade Anzahl Primfaktoren haben, schreiben Sie {\bf U}. Es ist jedoch zu beachten, dass unter die Zahlen, die einen mehrfachen Primfaktoren aufweisen, z.B. $4=2^2$ oder $24=2^3\cdot 3$, kein Eintrag {\bf G} oder {\bf U} erfolgt. Ihre Aufstellung sieht nun wie folgt aus:\\
\begin{table}[!htb]
\begin{center}
{\footnotesize
\begin{tabular}{|c||c|c|c|c|c|c|c|c|c|c|c|c|}\hline
 $n$             & 2 & 3 & 4     & 5 & 6          & 7 & 8     & 9     & 10         & 11 & 12         & $\ldots$ \\ \hline\hline
 Primfaktoren    & 2 & 3 & $2^2$ & 5 & $2\cdot 3$ & 7 & $2^3$ & $3^2$ & $2\cdot 5$ & 11 & $2^2\cdot 3$ & $\ldots$ \\ \hline
     & U & U & -     & U & G          & U & -     & -     & G          & U  & -          & $\ldots$ \\ \hline
\end{tabular}
}
\end{center}\vspace*{-2mm}
\end{table}\\
\nit {\bf Beweisen} Sie, wenn $n\rightarrow\infty$, dass die Anzahl
{\bf G}'s und {\bf U}'s gleich gross ist. Sollte Ihnen das gelingen,
haben Sie die {\bf Riemannsche Vermutung}\index{Riemannsche Vermutung}
bewiesen, ein Problem, das die mathematische Welt seit mehr als
150~Jahren besch\"aftigt (und noch nicht bewiesen ist!).

\newpage
\item {\bf Bode-Diagramm}\\ Bestimmen Sie den  {\bf
  Amplitudenrand}\index{Amplitudenrand} und den {\bf
  Phasenrand}\index{Phasenrand} von \\ $H(s)=\frac{3}{(s+1)(s+3)(s+0.1)}$. % ca 1 und 10dB
 

\item {\bf Mathematik (just for fun)}\\ L\"osen Sie das folgende {\bf
    Sudoku}\index{Sudoku}. (In jeder Zeile und jeder Spalte m\"ussen
  alle Zahlen 1 bis 9 genau je ein Mal vorkommen. In jedem der neun
  kleineren 3x3-Quadrate muss 1 bis 9 auch genau ein Mal vorkommen.)

\begin{table}[!htb]
\begin{center}
{\large
\begin{tabular}{||c|c|c||c|c|c||c|c|c||}\hline\hline
  & 6 &   & 5 &   &   &   & 1 & 8 \\ \hline
  & 3 &   &   &   &   &   &   &  \\ \hline
  &   & 5 & 4 & 3 &   &   &   &  \\ \hline\hline
  &   &   & 8 &   &   &   & 9 &  \\ \hline
2 &   &   &   & 5 &   &   &   & 3 \\ \hline
  & 7 &   &   &   & 1 &   &   &  \\ \hline\hline
  &   &   &   & 1 & 2 & 4 &   &  \\ \hline
  &   &   &   &   &   &   & 6 &  \\ \hline
9 & 1 &   &   &   & 6 &   & 7 &  \\ \hline\hline
\end{tabular}
}
\end{center}\vspace*{-2mm}
\end{table}


\end{enumerate}
% sudoku