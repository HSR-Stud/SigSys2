\renewcommand{\thesection}{\thechapter.\arabic{section}}
\setcounter{Aufgabe}{0}\setcounter{Beispiel}{0}
\chapter{Filtertheorie}\label{Kapitel_Filter}
\section{Einf"uhrung in die analoge Filtertheorie}
Ein elektrisches Filter ist ein Netzwerk, das ein Eingangssignal in
gew"unschter Art und Weise in ein Ausgangssignal verwandelt. Die
Signale k"onnen im Zeit- oder im Frequenzbereich betrachtet werden,
dementsprechend k"onnen die Anforderungen im Zeit- oder
Frequenzbereich definiert sein. Filter sind mehrheitlich
{\bf frequenzselektive}, {\bf lineare Netzwerke}, welche gewisse Frequenzbereiche
"ubertragen und andere d"ampfen. Kaum ein elektronisches Ger"at kommt
ohne Filter aus.  Telefon, TV\index{TV}, Radio,\index{Radio}
Radar und Daten"ubertragung sind einige Beispiele aus dem Gebiet der
Nachrichtentechnik\index{Nachrichtentechnik}, in denen Filter eine wesentliche Rolle spielen
\cite{MOS:89}.\\ Die frequenzselektiven Filter, mit denen wir uns im
weiteren besch"aftigen wollen, lassen sich in die f"unf wichtigsten
Grundtypen unterteilen \cite{ZVE:67}:\vspace*{3mm} \\ 
\hspace*{2cm}$\bullet$ Tiefpass\index{Tiefpass} (TP),\\
\hspace*{2cm}$\bullet$ Hochpass\index{Hochpass} (HP),\\
\hspace*{2cm}$\bullet$ Bandpass\index{Band!pass} (BP),\\
\hspace*{2cm}$\bullet$ Bandsperre\index{Band!sperre} (BS),\\
\hspace*{2cm}$\bullet$ Allpass\index{Allpass} (AP).\vspace*{3mm} \\ 
Es gibt weitere Filtertypen, die sich auf die f"unf % Kammfilter???
Grundtypen zur"uck\-f"uhren lassen \cite{SCH:VAL:01}.  Seit dem ersten
Konzept von elektrischen Filtern im Jahre 1915, wo unabh"angig
voneinander G.A.~Campbell in Amerika und K.W.~Wagner in Europa das
elektrische Wellenfilter\index{Wellenfilter} erfanden, wurden etliche
Filterbauarten\index{Filter!bauarten} entwickelt. In der
Tabelle~\ref{ubersichtstabelle} sind einige Filterbauarten in Form
einer "Ubersicht dargestellt.\\ 
\nit Je nach Anforderungen bez"uglich Frequenzbereich, Selektivit"at,
Sensitivit"at, Stabilit"at\index{Stabilitat@Stabilit\"at}, Leistungsbedarf,
dynamischen Bereich, Abmessungen (Fl"ache), Kosten,
Erh"altlichkeit, St"uckzahl, usw.,
stellt die eine oder andere Bauart die bessere L"osung dar \cite{LIN:BRA:LEH:85, MOS:89, SCH:VAL:01}.\\ 
\nit In diesem Kapitel betrachten wir ausschliesslich analoge
Filter\index{Filter!analog}. Der \"Ubergang zu den digitalen
Filtern\index{Filter!digital} ist relativ einfach, sobald die
$z$-Transformation\index{z-Transformation@{$z$-Transformation}}
eingef\"uhrt ist und das Abtasttheorem\index{Abtasttheorem} betrachtet
wurde. Mit der bilinearen Transformation\index{bilineare
  Transformation}\index{Transformation!bilineare} kann aus einem
analogen Filter das entsprechende digitale Filter entwickelt werden
\cite{OPP:SCH:89}. Die Methode mit der {\bf bilinearen Transformation} wird
auf IIR-Filter ({\bf I}nfinite {\bf I}mpulse {\bf R}esponse)\index{IIR-Filter}\index{Filter!IIR-} angewendet. F\"ur das Design von FIR-Filter\index{FIR-Filter}\index{Filter!FIR-} ({\bf F}inite {\bf I}mpulse {\bf R}esponse) verwendet
man z.B. die ``Windowing''-Methode\index{Windowing} \cite{ING:PRO:97}.
\clearpage
%\normalsize
\begin{table}[htb]
\begin{center}
\hspace*{0cm}
\begin{tabular}{|c|c|c|l|l|}\hline
  & & & & Filter mit Signalprozessor \hfill\cite{MOS:89}\\ \cline{5-5}
  & & & & Filter mit Gate Logic \hfill\cite{MOS:89}\\ \cline{5-5}
  & & & \rule{6.2mm}{0pt}
      \begin{rotate}{90}%
        \hspace*{-2.5mm} {\bf {\footnotesize \shortstack{diskrete\\ Werte}} }%
      \end{rotate} 
        & etc.  \\ \cline{4-5}
  & & & & SC-Filter (Switched Capacitor) \hfill\cite{MOS:89}\\ \cline{5-5}
  & & & & SI-Filter (Switched Current) \hfill\cite{MOS:89}\\ \cline{5-5}
  & & & & N-Pfad Filter \hfill\cite{MOS:89}\\ \cline{5-5}
  & \rule{2.2mm}{0pt}
    \begin{rotate}{90}%
      \hspace*{0.2cm} {\bf {\large zeitdiskret}}%
    \end{rotate} 
    & \rule{2.2mm}{0pt}
      \begin{rotate}{90}%
        \hspace*{5mm} {\bf {\small AKTIV} }%
      \end{rotate} 
      &  \rule{6.2mm}{0pt}
      \begin{rotate}{90}%
        \hspace*{0cm} {\bf {\footnotesize \shortstack{analoge\\ Werte}} }%
      \end{rotate} 
      & etc. \\ \cline{2-5} \cline{2-5}

  & & &  & $LC$-Filter ``Simulation'' \hfill\cite{LIN:BRA:LEH:85, MOS:89}\\ \cline{5-5} 
  & & &  & Gekoppelte Filterstrukturen \hfill\cite{MOS:89}\\ \cline{5-5}
  & & &  & Filter in Kaskadenbauweise \hfill\cite{MOS:89}\\ \cline{5-5}
  & &  \rule{2.2mm}{0pt}
      \begin{rotate}{90}%
        \hspace*{-2mm} {\bf {\small Aktiv $RC$} }%
      \end{rotate} & \rule{6.2mm}{0pt}
                      \begin{rotate}{90}%
                       \hspace*{1mm} {\bf {\footnotesize \shortstack{konz.\\ Bauteile}} }%
                       \end{rotate} &  etc. \\ \cline{3-5}

  & & & & \shortstack{SAW-Filter (Surface Acoustic Wave) \hspace*{1cm}\cite{MOS:89}\\ 
                       (Oberfl"achenwellenfilter)\hfill~}\\ \cline{5-5}
  & & & & Hohlraumresonatoren (Topfkreisfilter) \hfill \cite{MOS:89}\\ \cline{5-5}
  & & & \rule{6.2mm}{0pt}
        \begin{rotate}{90}%
        \hspace*{-1mm} {\bf {\footnotesize  \shortstack{verteilte\\ Bauteile}} }%
        \end{rotate} & etc.   \hfill\cite{ZVE:67}    \\ \cline{4-5}
  & & &  & Keramische Filter  \hfill\cite{LIN:BRA:LEH:85, MOS:89}\\ \cline{5-5}
  & & &  & Quarz-Filter  \hfill\cite{MOS:89}\\ \cline{5-5}
  & & &  & $RLC$-Filter \hfill\cite{MOS:89}\\ \cline{5-5}
  & & &  & $RC$-Filter  \hfill\cite{LIN:BRA:LEH:85}\\ \cline{5-5} 
  & & &  & $LC$-Filter  \hfill\cite{LIN:BRA:LEH:85}\\ \cline{5-5} 
  & & &  & $LC$-Filter mit Quarz  \hfill\cite{LIN:BRA:LEH:85, ZVE:67}\\ \cline{5-5} 
 
\rule{3.5mm}{0pt}
\begin{rotate}{90}%
  \hspace*{2.5cm} {\bf {\Large Filterbauarten}}%
\end{rotate}  

  &  \rule{2.2mm}{0pt}
    \begin{rotate}{90}%
      \hspace*{1.2cm} {\bf {\large kontinuierlich}}%
    \end{rotate} 
    &  \rule{2.2mm}{0pt}
      \begin{rotate}{90}%
        \hspace*{1.6cm} {\bf {\small Passiv} }%
      \end{rotate} 
      & \rule{6.2mm}{0pt}
      \begin{rotate}{90}%
        \hspace*{1mm} {\bf {\footnotesize  \shortstack{konzentrierte\\ Bauteile}} }%
      \end{rotate} & etc.    \hfill\cite{ZVE:67}        \\ \hline

\end{tabular}\caption{"Ubersicht "uber verschiedene Filterbauarten f"ur die Frequenzselektion\label{ubersichtstabelle}}
\end{center}
\end{table}\vspace*{-6mm}
\nit
Das Hauptgewicht unserer Betrachtungen wird nicht bei den
einzelnen Bauarten liegen, sondern bei der Approximation der gew"unschten
Filterfunktion durch eine
{\bf realisierbare "Ubertragungsfunktion}.\\
Dazu werden wir uns zuerst mit den Zusammenh"angen zwischen
dem Frequenzgang und der "Ubertragungsfunktion besch"aftigen.
Anschliessend werden wir einige Tief\-pass\-approximationen kennen lernen
und erl"autern, wie daraus mit Hilfe von {\bf Frequenztransformationen} Hochpass-, Band\-pass- und Bandsperren-Filter
generiert werden k"onnen.\index{Frequenztransformationen}

\clearpage
\section[Zusammenhang Frequenzgang $\leftrightarrow$ UTF]
{Der Zusammenhang zwischen Frequenzgang und "Ubertragungsfunktion
  (UTF)} Bei unseren Betrachtungen wollen wir uns auf
LLF\footnote{linear lumped-parameter finite (LLF) networks}-Netzwerke\index{LLF}
(lineare Netzwerke mit konzentrierten
Elementen) beschr"anken.\\
\begin{figure}[!htb]
\begin{center}
  \vspace*{-2mm}\bild{/filter/FIL2.fig.eps,width=0.5}\caption{LLF Netzwerk}
\end{center}
\vspace*{-6mm}
\end{figure}\\
Diese lassen sich bekanntlich mit einer Differentialgleichung\index{Differentialgleichung} der Form
\begin{equation*}
a_{n}\frac{d^{n} y}{dt^{n}} + a_{n-1}\frac{d^{n-1} y}{dt^{n-1}} + \cdots +
a_{1}\frac{d y}{dt} + a_{0} y=
b_{m}\frac{d^{m} x}{dt^{m}} + b_{m-1}\frac{d^{m-1} x}{dt^{m-1}} + \cdots +
b_{1}\frac{d x}{dt} + b_{0} x
\end{equation*} 
und deren Laplace-Transformierten\\~~\\
\myboxx{
\begin{equation*}
H(s)\equiv T(s)=\frac{Y(s)}{X(s)}=\frac{b_{m} s^{m} + b_{m-1} s^{m-1} +\cdots+b_{1} s 
+ b_{0}}{a_{n} s^{n} + a_{n-1} s^{n-1} + \cdots + a_{1} s + a_{0}}
=\frac{N(s)}{D(s)}
\end{equation*}} \\~~\\
beschreiben.\\ Die
"Ubertragungsfunktion des Netzwerkes $T(s)$, ist eine reelle,
rationale Funktion\index{Funktion!reelle, rationale} in $s$ der
Ordnung $n$.  Das Z"ahlerpolynom\index{Zae@{Z\"a}hlerpolynom} $N(s)$ und das
Nennerpolynom\index{Nennerpolynom} $D(s)$ sind Polynome mit reellen
und konstanten Koeffizienten, weil das Netzwerk aus linearen,
konzentrierten Elementen besteht \cite{MOS:89}. Die
Koeffizienten sind unabh"angig vom Eingangssignal.\\
{\bf\boldmath{Die Wurzeln\index{Wurzeln} der Gleichung $N(s)=0$ ergeben die $m$
 endlichen
  Nullstellen ($z_{i}, i=1,\ldots, m$); die Wurzeln von $D(s)$ ergeben
  die $n$ Pole des Systems ($p_{j}, j=1,\ldots, n$), die aus
  Stabilit"atsgr"unden\index{Stabilitat@Stabilit\"at} in der linken
  $s$-Halbebene\index{Halbebene!linke}
  (LHE)\index{LHE|see{Halbebene!linke}} liegen m"ussen.}}  Damit l"asst
sich $T(s)$ auch schreiben als:
\begin{equation*}
T(s)=K \cdot\frac{{\displaystyle \prod_{i=1}^{m}} (s - z_{i})}
{{\displaystyle \prod_{j=1}^{n}} (s - p_{j})}
\end{equation*}
wobei $K=b_{m}/a_{n}$ ist. {\bf\boldmath{ Die
  "Ubertragungsfunktion $T(s)$ ist also vollst"andig bestimmt durch ihre Pole
  und Nullstellen, sowie durch eine multiplikative Konstante $K$.}}  Da
die Wurzeln\index{Wurzeln} von Polynomen mit reellen Koeffizienten
entweder reell sind oder in konjugiert-komplexen Paaren auftreten \cite{MOS:89}, ist
es meist sinnvoll, die Systemfunktionen als Produkt von Faktoren 1.
und 2.~Ordnung mit reellen Koeffizienten darzustellen:
\begin{equation*}
T(s)=K \cdot\frac{{\displaystyle\prod_{i=1}^{r}} (s^{2} + 2\sigma_{zi}\; s + 
\omega_{zi}^{2})
{\displaystyle\prod_{i=2r+1}^{m}}(s - z_{i})} 
{{\displaystyle\prod_{j=1}^{t}} (s^{2} + 2\sigma_{pj}\; s + \omega_{pj}^{2})
{\displaystyle\prod_{j=2t+1}^{n}}(s - p_{j})}
\end{equation*}  
oder anders geschrieben:
\begin{equation*}
T(s)=K \cdot\frac{{\displaystyle\prod_{i=1}^{r}} (s^{2} + 
\displaystyle\frac{\omega_{zi}}
{q_{zi}}\; s + \omega_{zi}^{2}) {\displaystyle\prod_{i=2r+1}^{m}}(s - z_{i})} 
{{\displaystyle\prod_{j=1}^{t}} (s^{2} + 
\displaystyle\frac{\omega_{pj}}{q_{pj}}\; s + 
\omega_{pj}^{2}) {\displaystyle\prod_{j=2t+1}^{n}}(s - p_{j})}.
\end{equation*}


\section{Bestimmung des Frequenzganges aus der "Ubertra\-gungs\-funk\-tion}
Um das Verhalten eines Netzwerks im eingeschwungenen Zustand zu
ermitteln, ersetzen wir in der UTF $T(s)$ $s$ durch $j\omega$ und
variieren die anregende
Frequenz $\omega$ von $0$ bis $\infty$. {\bf\boldmath{Der praktisch bedeutsame Spezialfall\\~~\\
  \framebox[\textwidth]{ \parbox{0.9\textwidth}{
\begin{equation}
\left. T(s)\right|_{{\displaystyle s=j\omega}}=T(j\omega)
\end{equation} 
}}\\~~\\ \nit wird als Frequenzgang bezeichnet.}}  $T(j\omega)$ kann in
Polarform\index{Polarform} folgendermassen dargestellt werden:
\begin{equation}
\left. T(s)\right|_{s=j\omega}=T(j\omega)=|T(j\omega)|\cdot e^{j\varphi(\omega)}
\end{equation}
wobei $|T(j\omega)|$ als Amplitudengang\index{Amplitudengang} und
$\varphi(\omega)$ als Phasengang\index{Phasen!gang} der
Netzwerkfunktion\index{Netzwerk!funktion} bei der Frequenz $\omega$
bezeichnet wird.  Wenn die Netzwerkfunktion eine
Spannungs\-"ubertragungs\-funktion\index{Spannungsubertragungsfunktion@{Spannungs\"ubertragungsfunktion}}\index{Ubertragungsfunktion@{\"Ubertragungsfunktion}!Spannungs-} darstellt, ist es zweckm"assig mit dem
Logarithmus der Funktion zu rechnen. Die
Kaskadierung\index{Kaskadierung} mehrerer entkoppelter Teilnetzwerke
kann dadurch anstatt durch Multiplikation der
Teilfrequenzg"ange\index{Teilfrequenzgang} durch Addition der
logarithmischen Teilfrequenzg"ange berechnet werden. {\bf\boldmath{Eine weitere
  wichtige Gr"osse ist die Gruppenlaufzeit\index{Gruppenlaufzeit}
  $\tau_{g}(\omega)$,
  die wie folgt definiert ist \cite{MOS:89}:}}\\~~\\
\myboxx{
\begin{equation*}
\tau_{g}(\omega)=\frac{ \bold{-} \partial \varphi(\omega)}{\partial \omega}.
\end{equation*}
}\\~~\\ \nit Zur Veranschaulichung wollen wir ein Beispiel dazu betrachten. Die
Phase in Abb.~\ref{Zphase} hat eine konstante, negative Steigung.
Somit ergibt sich f"ur $\tau_{g}(\omega)$ ein von $\omega$
unabh"angiger, konstanter Wert.

\clearpage
\begin{figure}[!htb]% alt FIL6.ps
\begin{center}

{\psset{unit=0.63}
\begin{pspicture}(20,15)
% system
\psset{linewidth=1.5pt}
\psline(3,12)(7,12)\psline(3,14)(7,14)
\psline(3,12)(3,14)\psline(7,12)(7,14)
\psset{linewidth=1pt}
\psdots[dotsize=3pt 3,dotstyle=*]%
(1,13)(9,13)
\psline(1,13)(3,13)\psline(7,13)(9,13)
\uput[0](3.15,12.9){LTI-System}
\uput[0](0.4,13.6){$x(t)$}
\uput[0](8.4,13.6){$y(t)$}
% Amplitude
\psline{->}(12,11)(18,11)
\psline{->}(12,11)(12,14) 
\psline[linewidth=1.5pt,linecolor=red](12,13)(17.5,13)\uput[0](17.6,13){0~dB}
\uput[0](13,10.3){$\longrightarrow \omega~ [\text{rad}]$}
\rput[lb]{L}(11.8,11.2){$\rightarrow |H(j\omega)|~ [\text{dB}]$}
% Phase
\psline{->}(12,6)(18,6) 
\psline{->}(12,6)(12,9) 
\psline[linewidth=1.5pt,linecolor=red](12,8)(17.5,6)\psline(11.9,8)(12.1,8)\rput[lb](12.2,8.1){0~rad}
\uput[0](13,5.3){$\longrightarrow \omega~ [\text{rad}]$}
\rput[lb]{L}(11.8,6.2){$\rightarrow \varphi(\omega)~ [\text{rad}]$}
% gruppenlaufzeit
\psline{->}(12,1)(18,1) 
\psline{->}(12,1)(12,4) 
\psline[linewidth=1.5pt,linecolor=red](12,3)(17.5,3)
\uput[0](13,0.3){$\longrightarrow \omega~ [\text{rad}]$}
\rput[lb]{L}(11.8,1.2){$\rightarrow \tau_{g}(\omega)~ [\text{s}]$}
% x(t)
\psline{->}(2,6)(8,6) 
\psline{->}(2,6)(2,9) 
\psline[linewidth=1.5pt,linecolor=red](2,6)(3,6)(3,8)(4,9)(5,7)(7,7)
\uput[0](3,5.3){$\longrightarrow t ~[\text{s}]$}
\rput[lb]{L}(1.8,6.2){$\rightarrow x(t)$}
% y(t)
\psline{->}(2,1)(8,1) 
\psline{->}(2,1)(2,4) 
\psline[linewidth=1.5pt,linecolor=red](2,1)(4,1)(4,3)(5,4)(6,2)(8,2)
\uput[0](3,0.3){$\longrightarrow t ~[\text{s}]$}
\rput[lb]{L}(1.8,1.2){$\rightarrow y(t)$}

\psline{->}(3,2)(4,2) 
\psline(3,1)(3,6)\rput[lb](3.2,2.2){$\tau_{g}$}
\end{pspicture}}
\caption{Zusammenhang zwischen Phasengang $\varphi(\omega)$ und Gruppenlaufzeit $\tau_g(\omega)$ und die Auswirkung auf das Ausgangssignal $y(t)$\label{Zphase}}
\end{center}
\vspace*{-9mm}
\end{figure}~\\ \nit Im Zeitbereich hat dies zur Folge, dass das Eingangssignal unverformt
aber um $\tau_{g}$ verz"ogert am Ausgang erscheint. Der Grund liegt
darin, dass alle Frequenzkomponenten des Eingangssignals gleich
verz"ogert werden.  Ist $\tau_{g}$ nicht f"ur alle Frequenzen gleich,
so erscheint ein aus mehreren Frequenzen bestehendes Eingangssignal am
Ausgang verzerrt.

\newpage
\subsection{Bestimmung der UTF aus dem Frequenzgang}
In vielen Anwendungsf"allen sind die Filterspezifikationen durch den
Amplitudengang gegeben. Um ein Netzwerk mit vorgeschriebenem
Amplitudengang zu realisieren, m"ussen wir zuerst die
"Ubertragungsfunktion $T(s)$ finden.  Gegeben ist $|T(j\omega)|$,
gesucht wird die entsprechende Funktion $T(s)$.  Da $T(s)$ das Verh"altnis
zweier Polynome in $s$ mit reellen Koeffizienten ist, gilt:
\begin{equation}
T^{*}(j\omega)=T(-j\omega)
\end{equation}
wobei $T^{*}(j\omega)$ die konjugiert-komplexe Funktion von
$T(j\omega)$ bezeichnet.  Ferner gilt:
\begin{equation*}
|T(j\omega)|^{2}=T(j\omega) \cdot T^{*}(j\omega)=T(j\omega) \cdot T(-j\omega)=T(s) \cdot T(-s)|_{s=j\omega}
\end{equation*}
\begin{equation}
T(s) \cdot T(-s)=\left. |T(j\omega)|^{2} 
\right|_{{\displaystyle\omega^{2}=-s^{2}}}\label{eq: tbetr}
\end{equation}
{\bf\boldmath{ $|T(j\omega)|^{2}$ wird dabei immer eine Funktion in $\omega^{2}$
  sein, da der Amplitudengang eine gerade Funktion ist \cite{MOS:89}}.}  Unser
Problem besteht nun darin, eine Funktion $T(s)$ aus einem gegebenen
Produkt $T(s)\cdot T(-s)$ zu separieren. Dieses Problem ist nicht
eindeutig l"osbar \cite{MOS:89}.  Um die Sache etwas zu vereinfachen
schreiben wir:
\begin{equation}
T(s)=\frac{N(s)}{D(s)}
\end{equation}
eingesetzt in Formel~\ref{eq: tbetr} ergibt:
\begin{equation}
\frac{N(s)}{D(s)} \cdot \frac{N(-s)}{D(-s)}=\left. |T(j\omega)|^{2}
\right|_{{\displaystyle\omega^{2}=-s^{2}}}
\end{equation}
Aus Stabilit"atsgr"unden\index{Stabilitat@Stabilit\"at} muss $D(s)$ ein
Hurwitz-Polynom\index{Hurwitz!-Polynom} sein \cite{MOS:89}. Beschr"anken wir uns auf Minimalphasennetzwerke, so
muss auch $N(s)$ ein Hurwitz-Polynom sein \cite{UNB:81}.
\bsp{Veranschaulichung des Vorgehens}
\begin{equation*}
|T(j\omega)|^{2}=\frac{4+\omega^{2}}{1+\omega^{6}}
\end{equation*}
Ersetzen wir nun $\omega^{2}$ durch $-s^{2}$ , so erhalten wir:
\begin{eqnarray*}
|T(j\omega)|^{2}&=&\frac{4-s^{2}}{1-s^{6}}=\frac{(2-s)(2+s)}
                    {(1-s^{3})(1+s^{3})}=\frac{(s+2)(s-2)}{(s-1)(s^{2}+s+1)(s+1)(s^{2}-s+1)}\\
&=&\frac{N(s)}{D(s)}\cdot\frac{N(-s)}{D(-s)}
\end{eqnarray*}
Daraus resultiert folgendes Minimalphasennetzwerk:
\begin{equation*}
T(s)=\frac{s+2}{(s+1)(s^{2}+s+1)}=\frac{s+2}{s^{3}+2s^{2}+2s+1}=\frac{N(s)}
     {D(s)}
\end{equation*}
oder als Nicht-Minimalphasennetzwerk:
\begin{equation*}
T(s)=\frac{s-2}{s^{3}+2s^{2}+2s+1}=\frac{N(-s)}{D(s)}.
\end{equation*}
Sind sowohl Amplituden- als auch Phasengang vorgeschrieben, so
bestimmt man zuerst eine minimalphasige "Ubertragungsfunktion, die den
vorgeschriebenen Amplitudengang aufweist. Der vorgeschriebene
Phasengang wird dann durch zus"atzliche Allp"asse erreicht.
\clearpage
%%%%% Kapitel 3
\section{Approximation im Frequenzbereich}
Eine der wichtigsten Aufgabe der Filtertheorie\index{Filter!theorie}
ist die Bestimmung einer "Ubertra\-gungs\-funk\-tion, die einen
vorgegebenen Frequenzgang gew"ahrleistet. Wir wollen uns im Folgenden
mit der Approximation eines vorgegebenen Amplitudengangs befassen. In
der Filtertheorie wird das Problem in zwei Stufen gel"ost: 
{\bf \begin{itemize}
\item Approximation des gegebenen Amplitudengangs durch eine rational gebrochene
      Funktion.
\item Realisierung (Synthese) dieser "Ubertragungsfunktion durch ein Netzwerk.
\end{itemize}
}
\paragraph{Das Toleranzschema}~\\ 
Die technischen Anforderungen an die "Ubertragungseigenschaften eines
Filters werden h"aufig im Frequenzbereich mit Hilfe eines
Toleranzschemas (Abb.~\ref{Fspec}) beschrieben.
\begin{figure}[!htb]%alt \bild{/filter/FIL13.ps,width=0.5}\
\vspace*{-3mm}
\begin{center}
{\psset{unit=0.8}
\begin{pspicture}(15,10)
\psline{->}(-0.5,0)(13,0)
\psline{->}(0,-0.5)(0,8) 
\uput[0](1.5,-0.5){DB}\uput[0](8.5,-0.5){SB}\uput[0](-0.7,-0.5){0}
\uput[0](-1.5,8.5){D\"ampfung}
\uput[0](-1.5,7){$A(\Omega)$}\uput[0](13.1,-0.1){$\Omega$}
\uput[0](3.6,-0.5){$\Omega_D$}\uput[0](6.6,-0.5){$\Omega_S$}

\psline[linewidth=1.5pt](7,0)(7,4)(10,4)(10,6)(13,6) \uput[0](9.5,2.7){Matrize}
\psline[linewidth=1.5pt](0,1)(4,1)(4,8) \uput[0](0.5,3){Stempel}

\psplot[linecolor=red,linewidth=2pt]{0}{3.1}{x 4.2 mul RadtoDeg cos -0.5 mul 0.5 add}
\psline[linecolor=red,linewidth=2pt](3.05,0.05)(4,1)(7,4)(9.8,6.8)
\rput(9,7.2){\color{red}m\"oglicher D\"ampfungsverlauf}

\psline{<->}(8,0)(8,4)\uput[0](8.1,2){$A_{\text{min}}$}
\psline(-0.1,1)(0,1)\rput[rB](-0.2,0.8){$A_{\text{max}}$}



\end{pspicture}}
\vspace*{4mm}\caption{Filterspezifikation mittels Toleranzschema \label{Fspec}}
\end{center}
\vspace*{-6mm}
\end{figure}~\\

\nit {\bf\boldmath{Im Durchlassbereich\index{Durchlassbereich} (DB) bestimmt der Stempel\index{Stempel} die maximal zul"assige
  D"ampfung\index{Dampfung@{D\"ampfung}} $A_{\max}$; im
  Sperrbereich\index{Sperrbereich} (SB) bestimmt die
  Matrize\index{Matrize} die minimal n"otige D"ampfung $A_{\min}$.}} Nun gilt es eine D"ampfungsfunktion\index{Dampfungsfunktion@D\"ampfungsfunktion} zu finden, die einerseits das
Toleranzschema nicht verletzt und andererseits auf eine realisierbare
Funktion f"uhrt.\\ Die Filterfunktionen HP, BP und BS lassen sich
durch entsprechende Frequenztransformation (Abb.~\ref{bsb tp-hp}) aus
einer TP-Funktion gewinnen.
\begin{figure}[!htb]
\begin{center}
  \bild{/filter/FIL14.ps,width=0.5}\vspace*{-3mm}\caption{Beispiel: Toleranzschema-Transformation eines TP in einen HP \label{bsb tp-hp}}
\end{center}
\vspace*{-6mm}
\end{figure}
Es gen"ugt somit, wenn wir im weiteren einige Standardapproximationen des
Tiefpasses betrachten. Auf die HP, BP und BS Transformationen kommen wir im
Abschnitt~\ref{Frequtrans} zu sprechen.
\paragraph{Die Frequenznormierung}~\\
Um das Rechnen mit unhandlichen Zahlen zu umgehen und um die
Approximation zu vereinheitlichen, f"uhren wir {\bf normierte Frequenzen}\index{Frequenz!normiert}
ein.  Betrachten wir dazu das TP-Toleranzschema\index{Toleranzschema} in Abb.~\ref{norm}.
\begin{figure}[!htb]
\vspace*{-1mm}
\begin{center}
  \bild{/filter/FIL15.fig.eps,width=0.5}\vspace*{-3mm}\caption{Beispiel einer Normierung der Frequenzachse, wobei $\Omega_{D}<\Omega_{S}<2\Omega_{D}$\label{norm}}
\end{center}
\vspace*{-6mm}
\end{figure}
Die normierten Gr"ossen erhalten wir somit zu:\\~\\
\myboxx{
\begin{equation}
S=\frac{s}{\omega_{r}} \hspace{2cm} \Omega=\frac{\omega}{\omega_{r}} 
\hspace{2cm} \sigma'=\frac{\sigma}{\omega_{r}}
\end{equation}}~\\~\\
\nit Prinzipiell l"asst sich die Normierung mit einer beliebigen
Referenzfrequenz durch\-f"uh\-ren. Bei TP und HP ist jedoch die
Normierung bez"uglich der Grenzfrequenz des Durchlassbereiches
$\omega_{D}$, also $\omega_{r}=\omega_{D}$ zweckm"assig. Bei BP und
BS w"ahlen wir zur Normierung die Mittenfrequenz $\omega_{m}$. {\bf\boldmath{
  Zur Entnormierung wird in der normierten Funktion $S$ durch
  $s/\omega_{r}$ ersetzt.}}
\subsection{Tiefpassapproximation}
Um im Folgenden einige Standardapproximationen des Tiefpasses behandeln zu
k"onnen, betrachten wir zuerst die Amplitudencharakteristik des idealen
Tiefpasses mit der normierten Grenzfrequenz 1 (Abb.~\ref{amp-id}).
\begin{figure}[!htb]
\begin{center}
  \bild{/filter/FIL16.fig.eps,width=0.4}\caption{Amplitudengang eines idealen Tiefpasses \label{amp-id}}
\end{center}
\vspace*{-6mm}
\end{figure}~\\
Alle Signale mit Frequenzen im Durchlassbereich (DB)\index{Durchlassbereich} werden ohne
D"ampfung "ubertragen, w"ahrend Signale mit Frequenzen im Sperrbereich\index{Sperrbereich}
(SB) kein Ausgangssignal ergeben.  {\bf Eine solche Charakteristik
  kann mit einem physikalisch realisierbaren Netzwerk nicht ideal
  nachgebildet werden.}  Diese Tatsache kann anhand einfacher
"Uberlegungen gezeigt werden. Bekanntlich erscheint am Ausgang eines
mit einem Dirac-Impuls\index{Dirac!Impuls} angeregten Systems die Laplace-R"ucktransformierte
der "Ubertragungsfunktion $T(s)$, n"amlich\\~\\
\mybox{\begin{equation}
f_{\delta}(t)\equiv h(t)={\cal L}^{-1} \{ T(s) \}.
\end{equation}}\\~~\\
\nit Die Impulsantwort $h(t)$ des idealen Tiefpasses\index{Tiefpass!ideal} ist
somit von der Form $\frac{\sin{x}}{x}$ (Abb.~\ref{stoss-id}) und erstreckt
sich von $-\infty \leq t \leq \infty$. Aus Gr"unden der
Kausalit"at\index{Kausalitat@{Kausalit\"at}} (Wirkung nicht vor Ursache) ist das
Ausgangssignal eines Systems vor dem Impuls Null. Das Ausgangssignal
des reellen Tiefpasses kann somit nicht ideal sein. Der ideale
Tiefpass\index{Tiefpass!ideal} ist also physikalisch nicht
realisierbar.

\begin{figure}[!htb] %ehemals FIL17.ps
\vspace*{-4mm}
\begin{center}
\bild{/filter/FIL_impuls_TP.eps,width=0.6}\vspace*{-3mm}
\caption{Impulsantwort des idealen Tiefpasses $h(t)=\frac{\sin(t)}{\pi t}$\label{stoss-id}}   
\end{center}
\vspace*{-12mm}
\end{figure}~\\

\nit Die Amplitudenfunktion $|T(j\Omega)|$ ist eine gerade Funktion\index{Funktion!gerade}, also
eine Funktion von $\Omega^{2}$ \cite{MOS:89}, denn es gilt:
$|T(j\Omega)|=\sqrt{\Re [T(j\Omega)]^{2}+\Im [T(j\Omega)]^{2}}$. Weil
$|T(j\Omega)| \geq 0$ ist, arbeiten wir im Folgenden mit
$|T(j\Omega)|^{2}$ um Wurzelausdr"ucke zu vermeiden.
Wir machen f"ur $|T(j\Omega)|^{2}$ den Ansatz:\\~\\
\myboxx{\begin{equation}
|T(j\Omega)|^{2}=\frac{1}{1+K(\Omega^{2})}
\end{equation}}\\~~\\
\nit Im Falle des Tiefpasses gilt f"ur die charakteristische Funktion\index{Funktion!charakteristische} $K(\Omega^{2})$:\\
$
\begin{array}{lrl}
\hspace*{1cm}\mbox{Sperrbereich (SB):}\hspace{2cm}& K(\Omega^{2})\gg 1 &\hspace{1cm}\mbox{f"ur } \Omega>1 \\
\hspace*{1cm}\mbox{Durchlassbereich (DB):} & 0\leq K(\Omega^{2})\ll 1 &\hspace{1cm}\mbox{f"ur } 0\leq\Omega<1
\end{array}
$
\subsection{Approximation nach Butterworth}
Die Approximation nach Butterworth\index{Butterworth} geht von
folgendem Potenzansatz\index{Potenzansatz} f"ur die
charakteristische Funktion\index{Funktion!charakteristische} aus:\\~~\\
\myboxx{\begin{equation}
K(\Omega^{2})=(\Omega^{2})^{n}
\end{equation}}\\~~\\ 
\nit wobei $n$ die Ordnung\index{Ordnung} des Filters und $\Omega$ die normierte Frequenz ist.
Damit wird der Amplitudengang des Butterworth-Filters $n$.~Ordnung:\\~~\\
\myboxx{\begin{equation}
|T(j\Omega)|=\frac{1}{\sqrt{1+\Omega^{2n}}}.
\end{equation}}\\~~\\
\nit Wir sehen, dass der Amplitudengang mit zunehmendem $\Omega$ {\bf streng monoton} abnimmt.

\subsubsection{Eigenschaften der Approximation }
{\bf\boldmath{ Die Bedingungen f"ur einen idealen Tiefpass sind im
  Durchlassbereich mit $K(\Omega^{2})=0$ und im Sperrbereich mit
  $K(\Omega^{2})=\infty$ erf"ullt.}}  Beim Butterworth-Ansatz sind
s"amtliche Nullstellen von $K(\Omega^{2})$ im Ursprung\index{Ursprung}, die Pole liegen alle im Unendlichen. Damit ergeben sich folgende Eigenschaften:\\
\begin{itemize}
\item Durchlassbereich
\begin{itemize}
\item F"ur $\Omega=0$ wird f"ur s"amtliche $n$: $\quad |T(0)|=\maxs{\Omega}|T(j\Omega)|=T_{\max}=1$.
    
\item F"ur $\Omega=1$ wird f"ur s"amtliche $n$: $\quad |T(j)|=\frac{T_{\max}}{\sqrt{2}}=\frac{1}{\sqrt{2}}$ entsprechend einem Abfall von 3.01~dB.
\item $|T(j\Omega)|$ kann f"ur $|\Omega|<1$ in eine Binominalreihe entwickelt werden \cite{BRO:SEM:91}:
      \begin{eqnarray*}
      |T(j\Omega)|&=&\frac{1}{\sqrt{1+\Omega^{2n}}}=f(\Omega^{2})=1-\frac{1}{2}\Omega^{2n}+\frac{3}{8}\Omega^{4n}-\frac{5}{16}
      \Omega^{6n}+\cdots\\       
      \end{eqnarray*}
      Folglich sind alle Ableitungen bei $\Omega=0$ gleich Null, d.h.
      \[
      \frac{df}{d(\Omega^{2})}=\frac{d^{2}f}{d(\Omega^{2})^{2}}=\cdots
     =\frac{d^{n-1}f} {d(\Omega^{2})^{n-1}}=0 \hspace{1cm} \mbox{f"ur}
      \hspace{.5cm} \Omega^{2} \rightarrow 0.
      \]
      
      {\bf Der Amplitudengang hat also keine
        Welligkeit\index{Welligkeit}\index{Welligkeit|see{Rippel}}, er ist ``maximal flach''.}
\end{itemize}

\item Sperrbereich
\begin{itemize}
\item F"ur $\Omega\gg1$ wird $|T(j\Omega)|\approx \frac{1}{\Omega^{n}}$. Der Amplitudenabfall mit steigendem $\Omega$ wird also umso
      steiler, je gr"osser die Filterordnung\index{Ordnung!Filter-}\index{Filter!ordnung} ist.  Der Amplitudengang
      f"ur $\Omega \rightarrow \infty$ in dB wird:
      $\alpha_{\text{dB}}(\Omega) \approx - 20 \; n \log{\Omega}$. Es
      ergibt sich eine asymptotische Steilheit von  \boldmath{ $-n \cdot 20\mbox{~\bf{dB/Dekade}}$, bzw. $-n\cdot 6.02\mbox{\bf{~dB/Oktave}}$}.

\end{itemize}
\item Allgemein
\begin{itemize}
\item $|T(j\Omega)|$ ist f"ur $0<\Omega<\infty$ eine streng-monoton fallende Funktion\index{monoton!streng}. 
\end{itemize}
\end{itemize}

\begin{figure}[!htb]% made mit matlabfile butterworth_plot.m
\vspace*{-3mm}
\begin{center}
  \bild{/filter/FILn001.eps,width=0.71}\vspace*{-2mm}\caption{$|T(j\Omega)|^2$ der normierten Butterworth-TP-Filter der Ordnung: 1,2,3,10 und 20.}\label{butterworth_quadrat}
\end{center}
\vspace*{-6mm}
\end{figure}
\aufg
a) Erstellen Sie mit Hilfe von \matlogo~ eine identische Grafik wie Abb.~\ref{butterworth_quadrat}.\\ b) Wie gross ist $|T(j\Omega)|$ f"ur ein Butterworth-TP-Filter mit $n=123$ bei $\Omega=2$?\\
% Loesung: a) siehe butterworth_plot
% b)T(jO)=1/sqrt(1+2^(2*123))=9.403954806578300e-38 entspricht -740.5dB=10log10(1+2^(2*123))




\subsubsection{\boldmath{Bestimmung von $T(S)$ aus der Funktion $|T(j\Omega)|$}}
Ausgehend von\\~~\\
\myboxx{\begin{equation}
\left. |T(j\Omega)|^{2}\right|_{{\displaystyle\Omega^{2}=-S^{2}}}=
\frac{1}{1+(-S^{2})^{n}}=T(S) \cdot T(-S)=\frac{1}{D(S) \cdot D(-S)}
\end{equation}}\\~~\\
wird\\~~\\
\myboxx{\begin{equation}
D(S) \cdot D(-S)=1+ (-S^{2})^{n} 
\end{equation}}\\~~\\
\nit wobei D(S) ein Hurwitz-Polynom sein muss.  Mit dem Ansatz\\~~\\
\myboxx{\begin{equation}
D(S)=\prod_{j=1}^{t}(S^{2}+a_{j}\cdot S+b_{j})\prod_{j=2t+1}^{n}(S-c_{j})
\end{equation}}\\~~\\
\nit bestimmen wir das Produkt $D(S) \cdot D(-S)$ und erhalten durch
Koeffizientenvergleich die gesuchten Koeffizienten. Betrachten wir
zur Erl"auterung das Butterworth-Filter\index{Butterworth!Filter} 2.~Ordnung:
\begin{eqnarray*}
 |T(j\Omega)|^{2}=\frac{1}{1+S^{4}}&\mbox{somit ist} & D(S) \cdot D(-S)=S^{4}+1.\\
\mbox{Ansatz: } D(S)=S^{2}+a_{1}S+b_{1}  & \longrightarrow & D(S) \cdot D(-S)=S^{4}+(2b_{1}-a_{1}^{2})S^{2}+b_{1}^{2}
\end{eqnarray*}
Durch Koeffizientenvergleich erhalten wir $a_{1}=\sqrt{2}$, $b_{1}=1$
und
\[
T(S)=\frac{1}{S^{2}+\sqrt{2}S+1}.
\]
$T(S)$ ist eine {\bf Allpol-Funktion}, d.h. $T(S)$ hat keine endlichen
Nullstellen\index{Nullstellen}. Man spricht daher auch von einem {\bf Allpolnetzwerk}\index{Allpol!netzwerk@-netzwerk}
oder {\bf Allpolfilter}.\index{Allpol!filter@-filter}


\begin{figure}[!htb]% made mit matlabfile butterworth_plot.m
\vspace*{-3mm} % FILn002.eps alt
\begin{center}
\bild{/filter/FIL_butt_amp.eps,width=0.71}\vspace*{-2mm}\caption{Amplitudeng"ange\index{Butterworth!Amplitudengang} der normierten Butterworth-Tiefpassfilter der Ordnung: 1,2,3,10 und 20.}\label{fig_fil_butterworth}
\end{center}
\vspace*{-6mm}
\end{figure}

\clearpage

\begin{figure}[!htb]% made mit matlabfile buutterbrot.m
\vspace*{-3mm}
\begin{center}
\bild{/filter/FIL_butt_phase.eps,width=0.71}\vspace*{-2mm}\caption{Phaseng"ange der normierten Butterworth-Tiefpassfilter der Ordnung: 1,2,3,10 und 20.}
\end{center}
\vspace*{-6mm}
\end{figure}

\begin{figure}[!htb]% made mit matlabfile buutterbrot.m
\vspace*{-3mm}
\begin{center}
\bild{/filter/FIL_butt_tauG.eps,width=0.71}\vspace*{-2mm}\caption{Gruppenlaufzeiten der normierten Butterworth-Tiefpassfilter der Ordnung: 1,2,3,10 und 20.}
\end{center}
\vspace*{-6mm}
\end{figure}


\clearpage
\subsubsection{Bestimmung der Pollage\index{Pol!lage@-lage}}
Aus der Beziehung
\[
|T(S)|^{2}=\frac{1}{1+(-S^{2})^{n}}=\frac{1}{D(S) \cdot D(-S)}
\]
kann die Lage der Pole bestimmt werden.  Bei den Polen $P_{k}$ wird
\[
D(S) \cdot D(-S)|_{S=P_{k}}=1+(-S^{2})^{n}|_{S=P_{k}}=0
\]
d.h. 
\[
(-1)^{n} P_{k}^{2n}=-1=e^{j(2k-1)\pi} \hspace{1cm} k=1,2,\ldots, 2n
\]
somit liegen die Pole bei ($k=1,2,\ldots, 2n$):\\~~\\
\myboxx{
\begin{equation}
P_{k}=e^{j\frac{(2k+n-1)}{2n}\pi}= \sigma'_{k}+j\tilde{\Omega}_{k} \label{eq: pollage_butterworth}
\end{equation}
\begin{eqnarray}
\mbox{bzw.}\hspace{1cm} \sigma'_{k}&=&\cos{\left(\frac{2k+n-1}{2n}\pi\right)}\\
\mbox{bzw.}\hspace{1cm}\tilde{\Omega}_{k}&=&\sin{\left(\frac{2k+n-1}{2n}\pi\right)}
\end{eqnarray}}\\~~\\
\nit {\bf{\boldmath d.h. alle Pole von $|T(j\Omega)|^{2}$ liegen auf dem
Einheitskreis im Abstand $\pi/n$ (Abb.~\ref{pollage-bw}).}}  Ist $n$
gerade, ergeben sich keine reellen Pole ; ist $n$ ungerade\index{ungerade}, gibt es 2
reelle Pole bei $\pm 1$.

\begin{figure}[!htb]
\vspace*{-2mm}% alt /filter/FIL19.ps
\begin{center}
{\psset{unit=0.7}

\begin{pspicture}(5,6)

%%%%%%%%% n=2

\psline{->}(-7,3)(-1,3) 
\psline{->}(-4,0)(-4,6) 
\uput[0](-1.7,3.6){$\sigma'$}\uput[0](-5.3,5.5){$j\Omega$}\uput[0](-2.2,2.6){1}
\pscircle[linewidth=1pt,linecolor=red](-4,3){2}
\psdots[dotsize=4pt 4,dotstyle=x]%
(-5.41,4.41)(-5.41,1.59)(-2.59,4.41)(-2.59,1.59)
\psdot[dotsize=12pt 12,dotstyle=square](-6.5,5.5)\uput[0](-7,5.5){$S$}
\uput[0](-3,5.5){$n=2$}

\uput[0](-5.73,3.2){$\pi/2$}
\psarc{<->}(-4,3){1.6}{135}{225}
\psline(-4,3)(-5.71,4.71)
\psline(-4,3)(-5.71,1.29)

\uput[0](-6.6,0){$D(S)$}\uput[0](-3.6,0){$D(-S)$}
{\Huge
\rput{90}(-5.5,0.2){\uput[0](0,0){\{}}
\rput{90}(-2.5,0.2){\uput[0](0,0){\{}}}

%%%%%%%%% n=3

\psline{->}(0,3)(6,3) 
\psline{->}(3,0)(3,6) 
\uput[0](5.3,3.6){$\sigma'$}\uput[0](1.7,5.5){$j\Omega$}\uput[0](4.8,2.5){1}
\pscircle[linewidth=1pt,linecolor=red](3,3){2}
\psdots[dotsize=4pt 4,dotstyle=x]%
(5,3)(1,3)(4,4.7321)(2,4.7321)(4,1.2679)(2,1.2679)
\psdot[dotsize=12pt 12,dotstyle=square](0.5,5.5)\uput[0](0,5.5){$S$}
\uput[0](4,5.5){$n=3$}

\uput[0](1.37,3.4){$\pi/3$}
\psarc{<->}(3,3){1.6}{120}{180}
\psline(3,3)(1.8,5.07) 

\uput[0](0.4,0){$D(S)$}\uput[0](3.4,0){$D(-S)$}
{\Huge
\rput{90}(1.5,0.2){\uput[0](0,0){\{}}
\rput{90}(4.5,0.2){\uput[0](0,0){\{}}}

%%%%%%%%% n=4

\psline{->}(7,3)(13,3) 
\psline{->}(10,0)(10,6) 
\uput[0](12.3,3.6){$\sigma'$}\uput[0](8.7,5.5){$j\Omega$}\uput[0](11.8,2.6){1}
\pscircle[linewidth=1pt,linecolor=red](10,3){2}
\psdots[dotsize=4pt 4,dotstyle=x]%
(11.85,3.77)(11.85,2.23)(8.15,3.77)(8.15,2.23)(10.77,4.85)(10.77,1.15)(9.23,4.85)(9.23,1.15)
\psdot[dotsize=12pt 12,dotstyle=square](7.5,5.5)\uput[0](7,5.5){$S$}
\uput[0](11,5.5){$n=4$}

\uput[0](8.5,3.9){$\pi/4$}\psline(9.5,3.8)(8.5,2.6)
\psarc{<->}(10,3){1.6}{157.5}{202.5}
\psline(10,3)(8.15,3.772)
\psline(10,3)(8.15,2.23)

\uput[0](7.6,0){$D(S)$}\uput[0](10.2,0){$D(-S)$}
{\Huge
\rput{90}(8.7,0.2){\uput[0](0,0){\{}}
\rput{90}(11.3,0.2){\uput[0](0,0){\{}}}
\end{pspicture}


}
\caption{Pollage von normierten Butterworth-Tiefpassfiltern der 2.,~3. und 4.~Ordnung}\label{pollage-bw}
\end{center}
\vspace*{-6mm}
\end{figure}\index{Pol!lage@-lage}


\nit F"ur das Nennerpolynom $D(S)$ unserer "Ubertragungsfunktion brauchen wir nur
die Pole in der LHE zu ber"ucksichtigen. $D(S)$ wird somit
\begin{equation}
D(S)=\prod_{i=1}^{n} (S-P_{i}),
\end{equation}
wobei $P_{i}$ die Pole in der LHE bezeichnen.  Zur Veranschaulichung
betrachten wir wieder das Butterworth-Filter 2.~Ordnung:
\[ 
\left. T(S)\cdot T(-S)=\frac{1}{1+S^{4}}\right|_{{\displaystyle S=j\Omega}}=|T(j\Omega)|^{2}.
\]
Die Pole von $|T(j\Omega)|^{2}$ liegen nach Formel~\ref{eq: pollage_butterworth} somit bei:
\begin{equation*}
P_{1,2}=e^{j \frac{4\mp 1}{4}\pi}=-\frac{1}{\sqrt{2}}\pm j\frac{1}{\sqrt{2}},\quad\text{und}\quad
P_{3,4}=e^{j \frac{8\mp 1}{4}\pi}=+\frac{1}{\sqrt{2}}\mp j\frac{1}{\sqrt{2}}.
\end{equation*}
In der LHE liegen nur die Pole $P_{1}$ und $P_{2}$. $D(S)$ wird also:
\[
D(S)=(S-P_{1})\cdot(S-P_{2})=S^{2}+\sqrt{2}S+1\quad\mbox{und damit}\quad T(S)=\frac{1}{S^{2}+\sqrt{2}S+1}.
\]
Auf diese Art lassen sich auch Nennerpolynome von Filtern h"oherer Ordnung bestimmen.
\subsubsection{Bestimmung der Filterordnung}
Zur Bestimmung der notwendigen Filterordnung\index{Filter!ordnung} gehen wir vom
Toleranzschema in Abb.~\ref{tptol} aus.\\~~\\
\begin{figure}[!htb]
\vspace*{-3mm}
\begin{center}
  \bild{/filter/FIL20.fig.eps,width=0.5}\caption{Beispiel eines Tiefpass Toleranzschemas \label{tptol}}
\end{center}
\vspace*{-6mm}
\end{figure}\\~~\\
\nit Man beachte, dass der Butterworth-Ansatz\index{Butterworth!Ansatz} automatisch
zur Normierung auf die 3~dB-Grenz\-frequenz\footnote{Exakt m"usste es die $1/\sqrt{2}$-Grenz\-frequenz, oder die $20\log{(1/\sqrt{2})}=3.01~\text{dB}$-Grenz\-frequenz heissen. Der Einfachheit halber, verwenden wir aber den Ausdruck ``3~dB-Grenz\-frequenz''.} f"uhrt (siehe Figuren~\ref{fig_fil_butterworth} und \ref{butterworth_quadrat}). Der
D"ampfungsverlauf eines Butterworth-Filters $n$.~Ordnung ergibt sich
zu:\\~~\\
\myboxx{\begin{equation}
A(\Omega)=20 \cdot \log{\frac{1}{|T(j\Omega)|}}=10\cdot\log{(1+\Omega^{2n})}.
\end{equation}}\\~~\\
\nit Bei den Frequenzen $\Omega_{D}$ und $\Omega_{S}$ m"ussen folgende Bedingungen 
erf"ullt sein:
\begin{eqnarray*}
A(\Omega_{D})=10\log{(1+\Omega_{D}^{2n})}=A_{\max} &\text{ und }& A(\Omega_{S})=10\log{(1+\Omega_{S}^{2n})}=A_{\min}.
\end{eqnarray*}
Die Bedingungen lassen sich in folgende Form umformen:
\begin{eqnarray*}
\Omega_{S}^{2n}=10^{A_{\min}/10}-1 &\text{ und }&\Omega_{D}^{2n}=10^{A_{\max}/10}-1
\end{eqnarray*}
Aus dem Quotient dieser beiden Bedingungen 
\[
\frac{\Omega_{S}^{2n}}{\Omega_{D}^{2n}}=\frac{10^{A_{\min}/10}-1}
{10^{A_{\max}/10}-1}=\left(\frac{\Omega_{S}}{\Omega_{D}}\right)^{2n}
\]
aufgel"ost nach $n$ erhalten wir die notwendige Filterordnung zu:\\~~\\
\myboxx{
\begin{equation}
n \geq\frac{\log{\left[\displaystyle\frac{10^{A_{\min}/10}-1}
{10^{A_{\max}/10}-1}\right]}}
{2 \cdot \log{\left(\frac{\Omega_{S}}{\Omega_{D}}\right)}}.\label{eq: butterwort_ordnung}
\end{equation}}
\bsp{Bestimmung der Filterordnung $n$ eines Butterworth-Filters}
$A_{\max}=0.1$~dB, $A_{\min}=30$~dB, $f_{D}=2$~kHz, $f_{S}=3$~kHz $\longrightarrow$ $\Omega_{D}=1,$ $\Omega_{S}=1.5$.
\[\mbox{Damit ist}\quad
n\geq\frac{\log{\displaystyle\left[\frac{10^{30/10}-1} {10^{0.1/10}-1}\right]}}{2 \cdot \log{\left(\frac{1.5}{1}\right)}}=13.15.
\]
Somit erf"ullt ein Butterworth-Filter 14.~Ordnung die
gestellten Bedingungen. Die notwendige Filterordnung kann auch aus
dem Nomogramm im Anhang (Abb.~\ref{nomo-BW}) ermittelt werden.
\aufg
a) Bestimmen Sie mit Hilfe des Nomogramms (Abb.~\ref{nomo-BW}) die Filterordnung des Butter\-worth-Filters f"ur $A_{\max}\text{=0.04~dB}$, $A_{\min}=\text{40~dB}$, $f_{D}=\text{4~kHz}$ und $f_{S}=\text{10~kHz}$.\\ 
% Loesung: Ordnung 8\\
b) "Uberpr"ufen Sie Ihr Resultat von a) mit Hilfe vom \mb {\tt buttord}.\\
% buttord(2*pi*4000,2*pi*10000,0.04,40,'s')=8
c) Verifizieren Sie  die Resultate von \matlogo~und vom Nomogramm mit Formel~\ref{eq: butterwort_ordnung}. Wie gross muss die Filterordnung gem"ass Formel~\ref{eq: butterwort_ordnung} sein?
% n=log10((10^(4)-1)/(10^0.004-1))/2/log10(2.5)=7.5811
\clearpage
\subsection{Approximation durch kritisch-ged"ampfte Filter (Gauss-Filter)}
Tiefpassfilter $n$. Ordnung mit  {\bf kritischer D"ampfung}\index{Filter!kritisch-gedampft@{kritisch-ged\"ampft}} (Gauss-Filter\index{Gauss!-Filter}) haben jeweils einen $n$-fachen Pol auf der negativen $\sigma$-Achse.\\ \nit Somit kann die Impuls- und die Sprungantwort nicht oszillieren und die kritisch ged"ampften Filter weisen im \"Ubergangsbereich eine geringe Flankensteilheit auf.\\ \nit Gauss-Filter\index{Gauss-Filter} h"oherer Ordnung entstehen z.~B. durch Serieschaltung\index{Serieschaltung} identischer Filter 1.~Ordnung\index{Ordnung}. Das kritisch ged"ampfte Filter\index{Filter!kritisch-gedampft@{kritisch-ged\"ampft}} $n$.~Ordnung ist:~\\~\\
\myboxx{\begin{equation}
H(s)=\frac{1}{\klam{1+\frac{s}{\omega_c}}^n},
\end{equation}}~\\~\\
wobei $\omega_c$ den 3-dB Punkt\index{3-dB!Punkt} jedes der $n$ Teilfilter bezeichnet. 
\bsp{Will man bei der Kreisfrequenz $\omega_D$ des kritisch ged"ampften Tiefpassfilters $n.$~Ordnung eine D"ampfung von $\alpha $~dB haben, so muss $\omega_c$ (der $n$ identischen Teilfilter) wie folgt gew"ahlt werden:}~\\
\myboxx{\begin{equation}
\omega_c=\frac{\omega_D}{\sqrt{10^{\frac{\alpha}{10\cdot n}}-1}} 
\end{equation}}~\\ \index{Kreisfrequenz}\index{Dampfung@D\"ampfung}

\bsp{Normierte UTF $H(S)$ eines kritisch-ged"ampften Filters $5.$~Ordnung (normiert auf: $H(j)=\frac{1}{\sqrt{2}}\rightarrow \omega_c=2.5933$)}
\begin{equation*}
H(S)=\frac{117.2829}{S^5+12.9663\cdot+ S^4   67.2502\cdot S^3 +  174.3977\cdot S^2 +  226.1297\cdot S +  117.2829}
\end{equation*}
Weitere normierte UTF k"onnen im Kapitel~\ref{anhang_kritisch} gefunden werden.

\bsp{Kritisch-ged"ampftes Filter (Gauss-Filter) 3.~Ordnung}
\begin{figure}[!htb]
\vspace*{-6mm}
\begin{center}
  
{\psset{unit=0.8}
\begin{pspicture}(14,5)
\psline(0,1)(14,1) 

\psline(0,4)(2.5,4)\psline(3.5,4)(6.2,4) 
\psline(7.2,4)(9.9,4)\psline(10.9,4)(14,4) 


\psdots[dotsize=3pt 3,dotstyle=*]%
(0,1)(14,1)(0,4)(14,4)(4,1)(7.7,1)(11.4,1)(4,4)(7.7,4)(11.4,4)

\psline(11.4,2.6)(11.4,4) 
\psline(11.4,2.4)(11.4,1) 

\psline(10.9,2.6)(11.9,2.6) 
\psline(10.9,2.4)(11.9,2.4) 

\psline(7.7,2.6)(7.7,4) 
\psline(7.7,2.4)(7.7,1) 

\psline(7.2,2.6)(8.2,2.6) 
\psline(7.2,2.4)(8.2,2.4) 

\psline(4,2.6)(4,4) 
\psline(4,2.4)(4,1) 

\psline(3.5,2.6)(4.5,2.6) 
\psline(3.5,2.4)(4.5,2.4) 

\multido{\dx=1+3.7}{4}{%
\psdot[dotsize=15pt 15,dotstyle=triangle,dotangle=270](\dx,4)%
}
\psframe(2.5,3.7)(3.5,4.3)
\psframe(6.2,3.7)(7.2,4.3)
\psframe(9.9,3.7)(10.9,4.3)

\uput[0](2.7,4.7){$R$}
\uput[0](6.3,4.7){$R$}
\uput[0](10,4.7){$R$}

\uput[0](0.7,4){$1$}\uput[0](4.4,4){$1$}\uput[0](8.1,4){$1$}\uput[0](11.8,4){$1$}

\uput[0](4.5,2.5){$C$}\uput[0](8.2,2.5){$C$}\uput[0](11.9,2.5){$C$}

\rput[rb](-0.2,2.2){$x(t)$}\psline{->}(0,3.8)(0,1.2) 
\rput[lb](14.2,2.2){$y(t)$}\psline{->}(14,3.8)(14,1.2) 

\end{pspicture}}

\vspace*{-8mm}
\caption{Kritisch-ged"ampftes Filter (Gauss-Filter) bestehend aus drei identischen, mit idealen Einheitsverst"arkern entkoppelten (wirkungsfreien) $RC$-Tiefpassfiltern mit $\omega_c=\frac{1}{RC}$}\index{wirkungsfrei}
\end{center}
\vspace*{-6mm} 
\end{figure}\index{RC-Tiefpass@{$RC$-Tiefpass}}



\begin{figure}[!htb]
\vspace*{-7mm}
\begin{center}
  \bild{/filter/FIL_kritisch_amp.eps,width=0.71}\vspace*{-3mm}\caption{Amplitudeng"ange von normierten, kritisch-ged"ampften Filtern (Gauss-Filter) der Ordnung: 1,~2,~3,~10,~100 und 1000. Aus der Abbildung ist ersichtlich, dass kritisch-ged"ampfte Filter nicht beliebig schnell vom Durchlassbereich in den Sperrbereich wechseln k"onnen (selbst nicht mit extrem hoher Ordnung).}
\end{center}
\vspace*{-6mm} 
\end{figure}\index{Durchlassbereich}\index{Sperrbereich}

\begin{figure}[!htb]
\vspace*{-7mm}
\begin{center}
  \bild{/filter/FIL_krit_phase.eps,width=0.71}\vspace*{-3mm}\caption{Phaseng"ange $\Theta(\omega)$ von normierten, kritisch-ged"ampften Filtern der Ordnung: 1,~2,~3,~10 und 20.}
\end{center}
\vspace*{-6mm} 
\end{figure}


\begin{figure}[!htb]
\vspace*{-7mm}
\begin{center}
  \bild{/filter/FIL_krit_tauG.eps,width=0.71}\vspace*{-3mm}\caption{Gruppenlaufzeiten $\tau_G(\omega)$ von normierten, kritisch-ged"ampften Filtern der Ordnung: 1,~2,~3,~10 und 20.}
\end{center}
\vspace*{-6mm} 

\end{figure}
\begin{figure}[!htb]
\vspace*{-7mm}
\begin{center}
  \bild{/filter/FIL_krit_tauP.eps,width=0.71}\vspace*{-3mm}\caption{Phasenlaufzeiten $\tau_P(\omega)$ von normierten, kritisch-ged"ampften Filtern der Ordnung: 1,~2,~3,~10 und 20.}
\end{center}
\vspace*{-6mm} 
\end{figure}




\clearpage
\subsection{Approximation\index{Approximation} nach Tschebyscheff\index{Tschebyscheff} (engl.~Che\-by\-shev)}
Beim Butterworth-Ansatz waren s"amtliche Nullstellen der Funktion
$K(\Omega^2)$ im Ursprung. Durch eine Verteilung der Nullstellen im
Durchlassbereich\index{Durchlassbereich} erreicht man eine bessere
Approximation der idealen Tiefpasscharakteristik.
Dies geschieht mit dem Ansatz von Tschebyscheff\index{Tschebyscheff}\footnote{Wir verwenden auch die Bezeichnung Tschebyscheff-I. Eine weitere Approximationsart sind die inversen Tschebyscheff-Filter, auch Tschebyscheff-II Filter genannt.}\index{Tschebyscheff!invers}:\\~\\
\myboxx{\begin{equation}
K(\Omega^{2})=e^{2} \cdot C_{n}^{2}(\Omega), 
\end{equation}}\\~\\
wobei $e$ eine Konstante (Rippelfaktor\index{Rippel!faktor}) und
$C_{n}$ ein Tschebyscheff-Polynom\index{Tschebyscheff!polynom} erster
Art der Ordnung $n$ ist\index{Ordnung} \cite{GRA:RYZ:00, VAL:82}. F"ur $|\Omega|\leq 1$ ist $C_{n}(\Omega)$ definiert als:\\~\\
\myboxx{\begin{equation}
C_{n}(\Omega)=\cos{(n\arccos{\Omega})}. 
\end{equation}}\\~\\
Daraus folgt, dass $C_{0}(\Omega)=1$ und $C_{1}(\Omega)=\Omega$.  Zur
Bestimmung der Polynome h"oherer Ordnung l"asst sich eine
Rekursionformel herleiten. Dazu machen wir die Substitution $\Omega=
\cos{\theta}$ (bzw. $\theta=\arccos{\Omega})$. Mit Hilfe des trigonometrischen Additions\-theorems folgt:\\
$
\begin{array}{llclcl} 
\qquad & C_{n+1}(\Omega) &=& \cos{[(n+1)\theta]} &=&
\cos{(n\theta)}\cos{\theta}- \sin{(n\theta)}\sin{\theta},\\ 
\qquad &C_{n-1}(\Omega) &=&
\cos{[(n-1)\theta]} &=& \cos{(n\theta)}\cos{\theta}+ 
\sin{(n\theta)}\sin{\theta}.
\end{array} 
$\\
Durch Addition von $C_{n-1}$ und $C_{n+1}$ erhalten wir:\\
\[
C_{n+1}(\Omega)+C_{n-1}(\Omega)=2\cos{(n\theta)}\cos{\theta}=
2\cdot C_{n}(\Omega)\cdot\Omega,
\]
und $C_{n+1}(\Omega)=2\;\Omega\; C_{n}(\Omega)-C_{n-1}(\Omega)$. Damit ist die Rekursionsformel:\\~\\
\myboxx{\begin{equation} 
C_{n}(\Omega)=2\;\Omega\;C_{n-1}(\Omega)-C_{n-2}(\Omega). 
\end{equation}}\\~\\
Daraus lassen sich nun die Tschebyscheff-Polynome $C_{n}$ f"ur $n\geq 2$
berechnen:
\begin{eqnarray*} 
C_{2}(\Omega)=2\Omega^{2}-1, & \hspace*{0.5cm}& C_{3}(\Omega)=4\Omega^{3}-3\Omega, \hspace{0.5cm}\mbox{usw. (Siehe auch Tabelle~\ref{c-poly}.)}  
\end{eqnarray*}
F"ur $|\Omega|\geq1$ gehen die hergeleiteten Polynome in die Funktion\\~\\
\myboxx{\begin{equation}
C_{n}(\Omega)=\cosh{(n\; {\rm Arcosh}\Omega)} 
\end{equation}}\\~\\ \nit "uber.  Mit dem Polynom $C_{n}$ und dem Rippelfaktor $e$ wird der
Amplitudengang des Tschebyscheff-Filters:\\~\\
\myboxx{\begin{equation}
|T(j\Omega)|=\frac{1}{\sqrt{1+e^{2}C_{n}^{2}}}. 
\end{equation}}\\~\\
\nit Das Tschebyscheff-Filter ist ebenfalls ein {\bf Allpolfilter}\index{Allpolfilter}.
Zwischen der Konstanten $e$ und der maximalen D"ampfung $A_{\max}$ gilt
die Beziehung $A_{\max}=10\log{(1+e^{2})}$ [dB], also $e=\sqrt{10^{A_{\max}/10}-1}$.
F"ur $A_{\max}=0.1\mbox{dB}$ eines Tschebyscheff-Filter wird $e=0.1526$.
\subsubsection{Eigenschaften der Approximation}
{\bf{\boldmath Im Durchlassbereich\index{Durchlassbereich} schwankt
    das Tschebyscheff-Polynom in den Grenzen $\pm 1$
    (Abb.~\ref{g-c-poly}). Im Sperrbereich nimmt $C_n$ monoton mit
    $\Omega$ zu.}}
\begin{figure}[!htb] % alt\bild{/filter/FIL21.ps,width=0.8
\vspace*{-3mm}
\begin{center}% made mit Cn_polynome.m
  \bild{/filter/Cn.eps,width=0.95}\caption{Graphen der
    Tschebyscheff-Polynome $C_{n}(\Omega)$ im
    Durchlassbereich ($|\Omega |\leq 1$)  f"ur $n=1, ~2,~\ldots~6$.\label{g-c-poly}}
\end{center}
\vspace*{-6mm} 
\end{figure}~\\
\nit Somit ergeben sich f"ur die Approximation folgende Eigenschaften:
\begin{itemize} 
\item Durchlassbereich 
\begin{itemize} 
\item F"ur  $\Omega=0$ wird f"ur ungerade $n$: $|T(0)|=\maxs{\Omega}|T(j\Omega)|=1=T_{\max}$,\\
\hspace*{3.4cm}f"ur gerade $n$: $|T(0)|=\frac{T_{\max}}{\sqrt{1+e^{2}}}=\frac{1}{\sqrt{1+e^{2}}}$.

\item F"ur $\Omega=1$ wird f"ur s"amtliche $n$: $|T(j)|=\frac{T_{\max}} {\sqrt{1+e^{2}}}=\frac{1}{\sqrt{1+e^{2}}}$.
\item F"ur $|\Omega|<1$ wird $|T(j\Omega)|$ bei den Nullstellen von
$C_{n}^{2}$ maximal, minimal bei den Maxima von $C_{n}^{2}$. Weil alle Rippel\index{Rippel} im Durchlassbereich gleiche Amplitude aufweisen, spricht man
auch von einem "Aquirippel-Filter. 
\end{itemize}
\item Sperrbereich 
\begin{itemize} 
\item F"ur $\Omega\gg1$ wird $|T(j\Omega)|\approx \frac{1}{e\;C_{n}(\Omega)}$. 
Weil $C_{n}(\Omega)$ ein Polynom der Ordnung $n$ ist, wird der Abfall umso
steiler, je gr"osser die Ordnung des Filters ist.
F"ur $\Omega\gg1$ ist $C_{n}(\Omega)\approx 2^{n-1}\cdot\Omega^{n}$
und der Amplitudengang wird:
\begin{eqnarray*}
\alpha_{\text{dB}}(\Omega)&\approx&-20\cdot\log{e}-20\cdot\log{C_{n}(\Omega)},\\
\alpha_{\text{dB}}(\Omega)&\approx&-20\cdot\log{e}-(n-1)\cdot 20\cdot\log{2}-20\cdot n \cdot\log{\Omega}.
\end{eqnarray*}
\nit Der Amplitudengang hat somit gleich wie beim Butterworth-Filter eine 
asymptotische Steilheit von \boldmath{$-n\cdot 20~\mbox{\bf{dB/Dekade}},$ bzw. $-n\cdot 6.02~\mbox{\bf{dB/Oktave}}$}.
\begin{figure}[!htb]% made with tschebyscheff_plot.m
\begin{center}\vspace*{-4mm} 
  \bild{/filter/FILn003.eps,width=0.71}\vspace*{-3mm}\caption{Amplitudeng"ange f"ur normierte Tschebyscheff-Tiefpassfilter mit $A_{\text{max}}=2$~dB der Ordnung: 1,2,3 und 10.}\index{Amplitudengang!Tschebyscheff}\index{Tschebyscheff!Amplitudengang}
\end{center}
\vspace*{-6mm} 
\end{figure}
\end{itemize}
\end{itemize}


\begin{figure}[!htb]% made with tschb1.m
\begin{center}\vspace*{-6mm} 
  \bild{/filter/FIL_ts1_phase.eps,width=0.71}\vspace*{-3mm}\caption{Phaseng"ange f"ur normierte Tschebyscheff-Tiefpassfilter mit $A_{\text{max}}=2$~dB der Ordnung: 1,2,3 und 10.}\index{Phasengang}
\end{center}
\vspace*{-6mm} 
\end{figure}

\clearpage


\begin{figure}[!htb]% made with tschb1.m
\begin{center}\vspace*{-0mm} 
  \bild{/filter/FIL_ts1_tauG.eps,width=0.71}\vspace*{-3mm}\caption{Gruppenlaufzeiten f"ur normierte Tschebyscheff-Tiefpassfilter mit $A_{\text{max}}=2$~dB der Ordnung: 1,2,3 und 10.}\index{Gruppenlaufzeit}
\end{center}
\vspace*{-6mm} 
\end{figure}


\subsubsection{{\boldmath Bestimmung von $T(S)$ aus der Funktion $|T(j\Omega)|$}}
Ausgehend von:
\begin{equation}
|T(j\Omega)|^{2}|_{j\Omega=S}=\frac{1}{1+e^{2}C_{n}^{2}(-jS)}
=\frac{1}{D(S)\cdot D(-S)}
\end{equation}
wird
\begin{equation}
D(S) \cdot D(-S)=1 + e^{2}C_{n}^{2}(-jS)    
\end{equation}
wobei $D(S)$ ein Hurwitz-Polynom\index{Hurwitz!-Polynom} sein muss.

\bsp{Tschebyscheff-Filter 2.~Ordnung}
\begin{eqnarray*}
|T(S)|^{2}=\frac{1}{1+e^{2}(-2S^{2}-1)^{2}}=
\frac{K}{S^{4}+S^{2}+\frac{(1+e^{2})}{4e^{2}}}&=&\frac{K}{D(S)\cdot D(-S)}\\
D(S)\cdot D(-S)=S^{4}+S^{2}+\frac{(1 + e^{2})}{4e^{2}}&=&(S^{2}+a_{1}S+b_{1})(S^{2}-a_{1}S+b_{1})
\end{eqnarray*}
und somit
\begin{equation*}
a_{1}=\sqrt{\frac{\sqrt{1+e^{2}}}{e}-1} \quad \mbox{und} \quad
b_{1}=\frac{\sqrt{1+e^{2}}}{2e}.
\end{equation*}
Der Rippelfaktor\index{Rippel!faktor} $e$ kann aus der geforderten maximalen
Durchlassd"ampfung $A_{\max}$ bestimmt werden.  F"ur
$A_{\max}=0.1\;\text{dB}\;(e=0.1526)$ ergibt sich f"ur $D(S)$ somit:
\begin{equation*}
D(S)=S^{2}+a_{1}S+b_{1}=S^{2}+2.372 S+3.313.
\end{equation*}
Eine Zusammenstellung der Nennerpolynome der Tschebyscheff-Filter befindet
sich im Anhang~\ref{anhang_Tschebyscheff}.
\subsubsection{Lage der Pole}
{\bf{\boldmath Die Pole eines Tschebyscheff-Filters $P_{kT}=\sigma'_{kT}+j\tilde{\Omega}_{kT}$ liegen auf einer Ellipse} \cite{MOS:89}}.\index{Ellipse} Sie lassen sich wie folgt aus 
den auf dem Einheitskreis liegenden Butterworth-Polen\index{Butterworth!Pole} 
$P_{kBW}=\sigma'_{kBW}+j\tilde{\Omega}_{kBW}$ bestimmen ($k=1, 2, \ldots, n$):\\
\begin{eqnarray}
\sigma'_{kT}=\sinh{\left[\frac{1}{n}{\rm Arsinh}\left(\frac{1}{e}\right)\right]}
\cdot \sigma'_{kBW}&=&\sinh{\left[\frac{1}{n}{\rm Arsinh}\left(\frac{1}{e}\right)\right]}
\cdot \cos{\left(\frac{2k+n-1}{2n}\pi\right)}, \nonumber \\
\tilde{\Omega}_{kT}=\cosh{\left[\frac{1}{n}{\rm Arsinh}\left(\frac{1}{e}\right)
\right]}\cdot \tilde{\Omega}_{kBW}&=&\cosh{\left[\frac{1}{n}{\rm Arsinh}\left(\frac{1}{e}\right)
\right]}\cdot \sin{\left(\frac{2k+n-1}{2n}\pi\right)}. \nonumber 
\end{eqnarray}
{\bf{\boldmath F"ur das Nennerpolynom $D(S)$ unserer "Ubertragungsfunktion
  $T(S)$ m"ussen wir wiederum nur die Pole in der LHE
  ber"ucksichtigen.}}  Betrachten wir zur Erl"auterung ein
Tschebyscheff-Filter 3.~Ordnung mit 3~dB Rippel im
Durchlassbereich. Wir gehen f"ur die Berechnung vom Butter\-worth-Filter
3.~Ordnung aus, dessen 3 Pole in der LHE auf dem Einheitskreis\index{Einheitskreis}
liegen (Abb.~\ref{Pole-BW}).
\begin{figure}[!htb]%matlab butterwort_pole_plot.m
\vspace*{-3mm}
\begin{center}
  \bild{/filter/FILn006a.eps,width=0.7}\vspace*{-15mm}\caption{Pole des normierten Butterworth-Filters 3.~Ordnung \label{Pole-BW}}
\end{center}
\vspace*{-6mm} 
\end{figure}\\
\nit Die Pole f"ur das Tschebyscheff-Filter ergeben sich somit zu:
\begin{equation}
P_{T}=\sinh{(\lambda)}\cdot\sigma'_{BW}+j\cosh{(\lambda)}\cdot
\tilde{\Omega}_{BW}
\end{equation}
dabei ist $\lambda=\left(\frac{1}{n}\right){\rm
  Arsinh}\left(\frac{1}{e}\right)$ mit $e=\sqrt{10^{A_{\max}/10}-1}$
und $n=3$. Mit den Beziehungen $\sinh{(x)}=(e^{x}-e^{-x})/2$,
$\cosh{(x)}=(e^{x}+e^{-x})/2$ und ${\rm
  Arsinh}(x)=\ln{(x+\sqrt{x^{2}+1})}$ \cite{BRO:SEM:91}, wird $\lambda={\rm
  Arsinh}(1)/3=0.2938$, $\sinh{(\lambda)}=0.2980$ und
$\cosh{(\lambda)}=1.043$. Die Tschebyscheff-Pole liegen somit bei:
\begin{equation*}
P_{1,3T}=-0.1493\pm j\;0.9038,\quad \text{und} \quad P_{2T}=-0.2986.
\end{equation*}
Die Tschebyscheff-Pole sind dabei wie in (Abb.~\ref{lage bw-t}) dargestellt
aus den Butterworth-Polen hervorgegangen.
\begin{figure}[!htb]% matlab pole_plot_butt_tsch.m
\begin{center}
  \bild{/filter/FIL24.ps,width=0.43}\caption{Lage der Butterworth- und Tschebyscheff-Pole f"ur $n=3$ und $A_{\max}=3$~dB.\label{lage bw-t}}
\end{center}
\vspace*{-6mm} 
\end{figure}\\
Das Nennerpolynom $D(S)$ wird somit:
\begin{equation*}
D(S)=(S-P_{1})(S-P_{3})\cdot (S-P_{2})=S^{3}+0.59724 S^{2}+0.92835S +0.25059.
\end{equation*}
Auf diese Art lassen sich auch Nennerpolynome von Filtern
h"oherer Ordnung\index{Ordnung!hohere@h\"ohere} bestimmen.  Merke: Die Pole des Tschebyscheff-Filters
lassen sich aus den Polen des Butterworth-\-Filters {\bf approximieren},
in dem man annimmt, dass die Realteile\index{Realteil} der Pole des Butterworth-\-Filters mit einen Faktor
$\varepsilon$ multipliziert werden
(Abb.~\ref{approx T}: Stauchung zur imagin"aren Achse hin). Dabei ist $\varepsilon$ die Exzentrizit"at\index{Exzentrizitae@{Exzentrizit\"a}t} der
entstehenden Ellipse\index{Ellipse}, gegeben durch:
\begin{equation*}
\varepsilon=\tanh{\left[\left(\frac{1}{n}\right){\rm Arsinh}\left(\frac{1}{e}\right)\right]}.
\end{equation*}
\aufg
Bestimmen Sie die Ordnung sowie die normierte UTF des Tschebyscheff-Tiefpass\-filters f"ur: $\Omega_D=1$, $\Omega_S=3/2$, $A_{\max}=1$~dB und $A_{\min}=16$~dB.\\ \index{cheb1ord@{\tt cheb1ord}}
% LsgOrdnung [n,wn]=cheb1ord(2,3,1,16,'s') n=4, wn=2
% cheb1ap(n,1)=-0.1395 +/-0.9834j ; -0.3369+/-0.4073j
 

\bsp{Approximation der  Tschebyscheff-Pole aus den Butterworth-Polen}
F"ur das Butterworth-Filter 3.~Ordnung mit $A_{\text{max}}=3~\text{dB}$ ist $\varepsilon=0.2856$ (Abb.~\ref{approx T}).
\begin{figure}[!htb]
\vspace*{-3mm}
\begin{center}
  \bild{/filter/FIL25.ps,width=0.39}\caption{Approximation des Tschebyscheff-Filters durch Stauchung der 
$\sigma'$-Achse \label{approx T}}
\end{center}
\vspace*{-6mm} 
\end{figure}

\subsubsection{Bestimmung der Filterordnung}
Zur Bestimmung der notwendigen Filterordnung $n$ gehen wir wiederum vom
Tief\-pass\-toleranzschema (Abb.~\ref{tptol}) aus. Der
D"ampfungsverlauf eines Tschebyscheff-Filters $n.$~Ordnung wird:
\begin{equation}
A_{\text{dB}}(\Omega)=20\;\log{\frac{1}{|T(j\Omega)|}}=
10\;\log{(1+e^{2}C_{n}^{2}(\Omega))}.
\end{equation}
\nit An der Stelle $\Omega_{D}$ wird $A(\Omega_{D})=A(1)=A_{\max}=10\;\log{(1+e^{2}C_{n}^{2}(1))}=
10\;\log{(1+e^{2})}$.\\
\nit F"ur $\Omega\geq1$ gilt $C_{n}(\Omega)=\cosh{(n\;{\rm Arcosh}\Omega)}$.
Somit wird $A(\Omega)$ an der Stelle $\Omega_{S}/\Omega_{D}$:
\[
A(\Omega_{S}/\Omega_{D})=A_{\min}=10\;\log{[1+e^{2}(\cosh{(n\;{\rm Arcosh}\klam{\Omega_{S}/\Omega_{D}}})^{2}]}
\]
Mit $e^{2}=10^{A_{\max}/10}-1$ wird $A_{\min}=10\;\log{[1+(10^{A_{\max}/10}-1)
(\cosh{(n\;{\rm Arcosh}\klam{\Omega_{S}/\Omega_{D}})})^{2}]}$.
Bzw. $10^{A_{\min}/10}-1=(10^{A_{\max}/10}-1) (\cosh{(n\;{\rm Arcosh}\klam{\Omega_{S}/\Omega_{D}})})^{2}$
aufgel"ost nach $n$:\\ ~~\\
\myboxx{\begin{equation}
n\geq\frac{{\rm Arcosh}\sqrt{\displaystyle\frac{10^{A_{\min}/10}-1}
{10^{A_{\max}/10}-1}}}{{\rm Arcosh}\klam{\Omega_{S}/\Omega_{D}}}.\label{ordnung_tschebyscheff}
\end{equation}}\\~\\
\nit Die Identit"at ${\rm Arcosh}(x)=\ln{[x+\sqrt{x^{2}-1}]}$ kann bei der Berechnung der Ordnung hilfreich sein \cite{BRO:SEM:91}.
\bsp{Bestimmung der Filterordnung\index{Filter!ordnung}}
$A_{\max}=0.1$~dB, $A_{\min}=30$~dB, $f_{D}=2$~kHz und $f_{S}=3$~kHz. $\longrightarrow \; n\geq 6.26$. 
D.h. ein Tschebyscheff-Filter 7.~Ordnung erf"ullt die gestellten
Bedingungen.  Die notwendige Filterordnung\index{Filter!ordnung} l"asst sich auch aus dem
Nomogramm im Anhang (Abb.~\ref{nomo-Tsche}) oder mit dem \matlogo-Befehl {\tt cheb1ord} bestimmen.\index{cheb1ord@{\tt cheb1ord}}
% Matlabbefehl [n,wn]=cheb1ord(1,1.5,0.1,30,'s')
\aufg
Bestimmen Sie mit dem Nomogramm (Abb.~\ref{nomo-Tsche}), dem \matlogo-Befehl {\tt cheb1ord} und Formel~\ref{ordnung_tschebyscheff} die Ordnung des Tschebyscheff-TP-Filter mit $A_{\max}=1$~dB, $A_{\min}=50$~dB, und $\Omega_{S}/\Omega_D=3$.
\clearpage
\subsubsection{Der Einfluss der Filterordnung und des Rippelfaktors auf den  
  Amplitudengang der Tschebyscheff-Filter}
Wenn wir die Amplitudeng"ange f"ur ein fixes $n$ bei wachsendem
Rippelfaktor $e$ betrachten, stellen wir eine Verbesserung der
Charakteristik im Sperrbereich fest (Abb.~\ref{amp fix n}). 
\vspace*{-3mm} \begin{figure}[!htb] % matlab tscheby_plot_e_n_vergleich
 \begin{center}
  \bild{/filter/FILn013.eps,width=0.66}\vspace*{-2mm}\caption{Amplitudeng"ange f"ur fixes $n=4$ bei wachsendem Rippelfaktor $e=0.25, 0.5, 0.75$ und 1}\label{amp fix n}
\end{center}
\end{figure}\\
Ebenfalls bringt eine Erh"ohung der Filterordnung $n$ bei fixem
Rippelfaktor $e$ eine Verbesserung des Sperrbereichs (Abb.~\ref{amp
  fix e}). Somit haben wir zwei M"oglichkeiten, das Verhalten im
Sperrbereich zu verbessern.\\
\vspace*{-3mm} % matlab tscheby_plot_e_n_vergleich
\begin{figure}[!htb]
 \begin{center}
  \bild{/filter/FILn012.eps,width=0.66}\vspace*{-2mm}\caption{Amplitudeng"ange f"ur fixes $e=1$ bei wachsender Ordnung $n=1, 2, 3$ und 7}
 \label{amp fix e}
\end{center}
\vspace*{-6mm} 
\end{figure}
\newpage
\subsection{Approximation nach Cauer\index{Cauer}} 
Mit dem Cauer-Filter wird der D"ampfungsverlauf sowohl im Durchlassbereich als
auch im Sperrbereich im tschebyscheff'schen Sinn approximiert. Man spricht
deshalb auch von Complete-Chebyshev - oder Chebyshev-Cauer-Filtern (CC-Filter).
\begin{figure}[!htb]
\vspace*{-3mm}
\begin{center}
  \bild{/filter/FIL28.ps,width=0.5}\caption{Amplitudengang eines Cauer-Tiefpassfilters 5.~Ordnung}
\end{center}
\vspace*{-6mm} 
\end{figure}\\
\nit Ausgegangen wird vom Ansatz:\\~~\\
\myboxx{\begin{eqnarray}
K(\Omega^{2})=e^{2}  R_{n}^{2} (\Omega)
\end{eqnarray}}\\~\\
\nit wobei $e$ wieder eine Konstante und $R_{n} (\Omega)$ eine
tschebyscheff-rationale Funktion der Ordnung $n$ ist \cite{VAL:82}.\index{Ordnung}.
Das
charakteristische Polynom\index{Polynom!charakteristische} $K(\Omega)=f(\Omega)/p(\Omega)$ ist eine rationale Funktion. Es gilt:\\
\begin{tabular}{lll}
\hspace*{0.5cm} DB:& $f(\Omega)=0$ \hspace {1cm} & D"ampfungs- oder 
Reflexionsnullstellen $\Omega_{1}$, $\Omega_{3}$, $\ldots$\\
\hspace*{0.5cm} SB:& $p(\Omega)=0$ & D"ampfungspole $\Omega_{2}$,$\Omega_{4}$, $\ldots$
\end{tabular}\\
Die Funktion $R_{n}(\Omega)$ kann eine gerade oder ungerade Funktion sein.
Als gerade Funktion hat sie die Form:\\~\\
\myboxx{\begin{equation*}
R_{n}(\Omega)=k\frac{(\Omega^{2}-\Omega^{2}_{1})(\Omega^{2}-\Omega^{2}_{3})
\ldots (\Omega^{2}-\Omega^{2}_{n-1})}{(\Omega^{2}-\Omega^{2}_{2})
(\Omega^{2}-\Omega^{2}_{4})\ldots (\Omega^{2}-\Omega^{2}_{n})}=k\prod_{i=1}^{\frac{n}{2}}\frac{(\Omega^{2}-\Omega^{2}_{2i-1})}{(\Omega^{2}-\Omega^{2}
_{2i})}
\end{equation*}}\\~~\\
Als ungerade Funktion hat sie die Form:\\~~\\
\myboxx{\begin{equation*}
R_{n}(\Omega)=k\Omega\frac{(\Omega^{2}-\Omega^{2}_{1})(\Omega^{2}-\Omega^{2}_{3})
\ldots (\Omega^{2}-\Omega^{2}_{n-2})}{(\Omega^{2}-\Omega^{2}_{2})
(\Omega^{2}-\Omega^{2}_{4})\ldots (\Omega^{2}-\Omega^{2}_{n-1})}=k\Omega\prod_{i=1}^{\frac{n-1}{2}}\frac{(\Omega^{2}-\Omega^{2}_{2i-1})}{(\Omega^{2}-\Omega^{2}
_{2i})}
\end{equation*}
}\\~\\
Ferner gilt: $R_{n}(1)=1$. F"ur die Berechnung der Frequenzen $\Omega_{k}$ ist von Cauer eine
geschlossene L"osung angegeben worden \cite{DAN:74}. Sie f"uhrt zu
Jakobi-elliptischen Funktionen\index{Funktion!Jakobi-elliptisch}. Die Cauer-Filter werden deshalb auch
als elliptische Filter\index{Filter!elliptische}\index{Filter!Cauer-|see{Cauer}} bezeichnet. Die Werte
f"ur die Pol- und Nullstellen\index{Nullstellen} k"onnen aus Filtertabellen (z.B. \cite{ZVE:67}) entnommen oder
einfach mit dem \mb {\tt ellip} berechnet werden. Mit
der Funktion $R_{n}$ und dem Rippelfaktor $e$ wird der Amplitudengang
des Cauer-Filters:\\~\\
\myboxx{\begin{equation}
|T(j\Omega)|=\frac{1}{\sqrt{1+e^{2}R_{n}^{2}(\Omega)}}
\end{equation}}~\\~\\
Zwischen der Konstanten $e$ und der max. zul"assigen D"ampfung $A_{\max}$ im
Durchlassbereich gilt $e=\sqrt{10^{A_{\max}/10}-1}$.\\
\subsubsection{Eigenschaften der Approximation}
{\bf{\boldmath Im Durchlassbereich schwankt die tschebyscheff-rationale Funktion\index{Funktion!tschebyscheff-rationale}
  $R_{n} (\Omega)$ zwischen den Grenzen $\pm 1$. Im Sperrbereich
  oberhalb $\Omega_{s}$ schwankt der Betrag von $R_{n} (\Omega)$
  zwischen den Grenzen $L$}} (gegeben durch die min. zul"assige
D"ampfung im SB) {\bf{\boldmath und $\infty$}} \cite{MOS:89}.
\begin{figure}[!htb]
\vspace*{-3mm}%\bild{/filter/FIL29.ps,width=0.7} alt 
\begin{center}% Rn_polynom
  \bild{/filter/Rn.eps,width=0.95}\caption{Die tschebyscheff-rationale Funktionen $R_{n} (\Omega,L)$  f"ur $n=1,~2,~3$ und $4$.}
\end{center}
\vspace*{-6mm} 
\end{figure}~\\
\nit Daraus resultieren nun folgende Eigenschaften:
\begin{itemize}
\item Durchlassbereich
\begin{itemize}
\item F"ur $\Omega=0$ wird f"ur ungerade $n$: $|T(0)|=\maxs{\Omega}|T(j\Omega)|=1=T_{\max},$\\
      \hspace*{3.4cm}f"ur gerade $n$: $|T(0)|=\frac{T_{\max}}{\sqrt{1+e^{2}}}=\frac{1}{\sqrt{1+e^{2}}}$.
\item F"ur $\Omega=1$ wird f"ur s"amtliche $n$: $|T(j)|=\frac{T_{\max}}{\sqrt{1+e^{2}}}=\frac{1}{\sqrt{1+e^{2}}}$.
\item F"ur $|\Omega| < 1$ wird $|T(j\Omega)|$ maximal bei den Nullstellen von
      $R_{n}^{2}$, minimal bei den Maxima von $R_{n}^{2}$. Alle Rippel im
      Durchlassbereich weisen gleiche Amplitude auf. Es gilt wiederum: $e^2=10^{A_{\max}/10}-1$.
\end{itemize}
\item Sperrbereich
\begin{itemize}
\item F"ur $\Omega > 1$ schwankt der Amplitudengang nach dem "Ubergangsbereich
      zwischen $0$ und $\frac{1}{\sqrt{1+e^{2}L^{2}}}$. Die Rippel im
      Sperrbereich weisen auch alle gleiche Amplitude auf, wobei $(eL)^2=10^{A_{\min}/10}-1$.
\end{itemize}
\end{itemize}


\begin{figure}[!htb]% made with cauer2.m
\begin{center}\vspace*{-0mm} 
  \bild{/filter/FIL_cauer_amp.eps,width=0.71}\vspace*{-3mm}\caption{Amplitudeng"ange f"ur normierte Cauer-Tiefpassfilter mit $A_{\text{max}}=10$~dB, $A_{\text{min}}=90$~dB  der Ordnung: 1,2,3,4 und 6.}\index{Cauer!-Amplitudengang}
\end{center}
\vspace*{-6mm} 
\end{figure}

\begin{figure}[!htb]% made with cauer2.m
\begin{center}\vspace*{-0mm} 
  \bild{/filter/FIL_cauer_phase.eps,width=0.71}\vspace*{-3mm}\caption{Phaseng"ange f"ur normierte Cauer-Tiefpassfilter mit $A_{\text{max}}=10$~dB, $A_{\text{min}}=90$~dB  der Ordnung: 1,2,3,4 und 6. Die Phasenspr"unge von $+\pi$ ergeben sich jeweils beim Durchgang durch eine Nullstelle auf der $j\omega$-Achse}\index{Cauer!-Phasengang}\index{Phasensprung}
\end{center}
\vspace*{-6mm} 
\end{figure}

\clearpage
\begin{figure}[!htb]% made with cauer2.m
\begin{center}\vspace*{-0mm} 
  \bild{/filter/FIL_cauer_tauG.eps,width=0.71}\vspace*{-3mm}\caption{Gruppenlaufzeiten f"ur normierte Cauer-Tiefpassfilter mit $A_{\text{max}}=10$~dB, $A_{\text{min}}=90$~dB  der Ordnung: 1,2,3,4 und 6.}\index{Cauer!-Gruppenlaufzeit}
\end{center}
\vspace*{-6mm} 
\end{figure}


\subsubsection{{\boldmath{Bestimmung der Filterordnung, der Lage der Pol- und Nullstellen   und der "Ubertragungsfunktion $T(S)$}}}
Die notwendige Filterordnung des Cauer-Filters l"asst sich anhand von
$A_{\max}$, $A_{\min}$ und $\Omega_S/\Omega_D$ mit \matlogo-Befehl {\tt
  ellipord} oder aus dem Nomogramm\index{Nomogramm} im Anhang
(Abb.~\ref{nomo-CC}) bestimmen. Die Formel f"ur die Ordnung lautet \cite{ING:PRO:97}:\\
\myboxx{
\begin{equation}
\hspace*{-3mm}n\geq\frac{K\left(\left( \frac{\Omega_D}{\Omega_S}\right)^2\right)  K\left(1-\frac{10^{A_{\max}/10}-1}{10^{A_{\min}/10}-1}\right) } {K\left(1-\left(\frac{\Omega_D}{\Omega_S}\right)^2\right )K\left(\frac{10^{A_{\max}/10}-1}{10^{A_{\min}/10}-1} \right)},\text{mit}\quad K(k)=\int_0^{\frac{\pi}{2}}\frac{d\theta}{\sqrt{1-k\sin^2\theta}}\label{ordnung_cauer}
\end{equation}}\\~\\
wobei $K(k)$ das vollst"andige elliptische Integral\index{elliptisches Integral!vollstae@{vollst\"a}ndiges} erster Gattung in der Legendreschen Normalform\index{Legendresche Normalform} ist \cite{BRO:SEM:91}.\\
\nit Ebenso lassen sich die Pol- und
Nullstellen mit dem \matlogo-Befehl {\tt ellipord} bestimmen oder aus Filtertabellen entnehmen.
Da das Cauer-Filter endliche Nullstellen besitzt, ist es {\bf kein
  Allpolfilter}.

\clearpage
\bsp{Bestimmung der Filterordnung $n$ eines Cauer-Filters}
$A_{\max}=0.1$~dB, $A_{\min}=30$~dB, $f_{D}=2$~kHz und $f_{S}=3$~kHz.\\~\\
\nit Mit den genannten Spezifikationen ergibt sich mit Formel~\ref{ordnung_cauer} $n=4.0653$. Somit ist die ben"otigte
Filterordnung\index{Ordnung}\index{Filter!ordnung|see{Ordnung}}
$n=5$, was wir auch mit dem \mb {\tt ellipord} oder dem Nomogramm
(Abb.~\ref{nomo-CC}) finden. F"ur die gleichen Spezifikationen
ben"otigen wir ein Butterworth-Filter 14.~Ordnung oder ein
Tschebyscheff-Filter 7.~Ordnung.  Die normierten Pole, Nullstellen und Konstante
$K$ erhalten wir zum Beispiel mit dem \mb {\tt [Z,P,K]=ellip(5,0.1,30,1,'s')} zu\\
\begin{equation*}
\begin{aligned}
\text{Pole: }\\
P_{1}  &=&- 0.76123,\hspace*{2.1cm}\\    
P_{2,3}&=&- 0.38597 \pm j 0.86132, \\
P_{4,5}&=&- 0.080464 \pm j 1.0538. 
\end{aligned}\hspace*{-5mm}
\begin{aligned}
\text{Nullstellen:}\\
Z_{1,2}&=& \pm j 1.2616,\\
Z_{3,4}&=& \pm j 1.7798,\\
{}
\end{aligned}\hspace*{-5mm}
\begin{aligned}
\text{Konstante:}\\
K&=& 0.1502,\\
{}\\
{}
\end{aligned}
\end{equation*}\\
Sie liegen wie folgt in der $S$-Ebene:
\begin{figure}[!htb]
\vspace*{-4mm}
\begin{center}
  \bild{/filter/FILn010.eps,width=0.71}\vspace*{-5mm}\caption{Lage der Pol- und Nullstellen beim Cauer-Filter 5.~Ordnung f"ur $A_{\max}=0.1$~dB, $A_{\min}=30$~dB, $f_{D}=2$~kHz und $f_{S}=3$~kHz}
\end{center}
\vspace*{-6mm}
\end{figure}\\
Es resultiert damit folgende normierte "Ubertragungsfunktion $T(S)$:
\begin{equation*}
T(S)=K \cdot\frac{{\displaystyle \prod_{i=1}^{4}} (S-Z_{i})}
{{\displaystyle \prod_{j=1}^{5}} (S-P_{j})}=\frac{0.152\cdot (S^{2}+1.592)(S^{2}+3.168)}{(S+0.761)(S^{2}+0.772 S+0.891)(S^{2}+0.161S+1.117)}.
\end{equation*}

\clearpage
F"ur $|T(j\Omega)|$ ergibt sich damit folgende Form:
\begin{figure}[!htb] % made with Cauer_plot.m
\begin{center}
  \bild{/filter/FILn009.eps,width=0.71}\caption{Amplitudengang des Cauer-Tiefpassfilters 5.~Ordnung f"ur $A_{\max}=0.1$~dB, $A_{\min}=30$~dB, $f_{D}=2$~kHz und $f_{S}=3$~kHz}
\end{center}
\vspace*{-6mm} 
\end{figure}\\~\\
\nit Mit der Verwendung der Filtertabellen werden wir uns im 
Kapitel~\ref{ent-mit-tab} etwas eingehender befassen. Die Verwendung von \matlogo-Befehlen
ist jedoch wesentlich einfacher als die Verwendung der Tabellen.






\clearpage
\subsection{Approximation nach Bessel}
Bei den bisher betrachteten Approximationen wurde versucht, den
idealen Amplitudengang m"oglichst gut nachzubilden. Wir stellten keine
Anforderungen an den Phasengang.  Im  wollen wir Tiefp"asse
mit m"oglichst linearer Phase, d.h. konstanter Gruppenlauf
realisieren.  Nach dem Verschiebungssatz der Laplace-Transformation\index{Laplace}
ben"otigt man zur Bildung eines Laufzeitgliedes mit der Laufzeit
$T_{0}$ eine "Ubertragungsfunktion der Form:\\~\\
\myboxx{\begin{equation}
T(S)=K e^{-ST_{0}}
\end{equation}}\\~~\\
\nit bzw. $\ln{T(S)}=\ln{K}-j\Omega T_{0}=\alpha(\Omega)+j \varphi(\Omega)$. F"ur die Gruppenlaufzeit gilt somit:\\~~\\
\myboxx{\begin{equation}
\tau_{g}(\Omega)=\frac{-d\varphi(\Omega)}{d\Omega}=T_{0}=\text{konstant.}
\end{equation}}\\~\\
Ohne Einschr"ankung  k"onnen wir $T_{0}=1$
w"ahlen. Setzen wir weiter $K=1$, so erhalten wir:
\begin{equation}
T(S)=e^{-S}=\frac{1}{e^{S}} \approx \frac{1}{D(S)}.
\end{equation}
Unsere Aufgabe besteht nun darin, die transzendente
Funktion\index{Funktion!transzendente} $e^{S}$ durch ein
Hurwitz-Polynom\index{Hurwitz!-Polynom} m"oglichst gut zu approximieren.
Wir gehen nun von folgendem Ansatz aus $D(S) \approx  e^{S}=\cosh{(S)} + \sinh{(S)}=g(S) + u(S)$. Mit
\begin{eqnarray*}
\text{ }\cosh{(S)}=1 + \frac{S^{2}}{2!} + \frac{S^{4}}{4!} +\ldots \text{ und } \sinh{(S)}=\frac{S}{1!} + \frac{S^{3}}{3!} + \frac{S^{5}}{5!} +\ldots\quad. \nonumber
\end{eqnarray*}
Damit $D(S)$ ein Hurwitz-Polynom wird, m"ussen bei der Kettenbruchzerlegung\index{Kettenbruchzerlegung} von
$\frac{g(S)}{u(S)}$ (bzw. $\frac{u(S)}{g(S)}$ ) alle Koeffizienten existieren
und positiv sein.
\begin{eqnarray}
\frac{\cosh{(S)}}{\sinh{(S)}} &=& \frac{1 + \frac{S^{2}}{2!} + \frac{S^{4}}{4!}
+\ldots}{\frac{S}{1!} + \frac{S^{3}}{3!} + \frac{S^{5}}{5!} +\ldots}
\nonumber\\
&=& \underbrace{
\displaystyle\frac{1}{S}+
\displaystyle\frac{1}{
\displaystyle\frac{3}{S}+
\displaystyle\frac{1}{
\displaystyle\frac{5}{S}+\ldots+
\displaystyle\frac{1}{
\displaystyle\frac{2n-1}{S}}}}
}_{\mbox{$n$~Glieder}}
\end{eqnarray}
Im Sinne einer Approximation brechen wir die Kettenbruchzerlegung nach
$n$ Gliedern ab. Das resultierende Polynom $g_{n}(S) + u_{n}(S)$ wird
als Bessel-Polynom $B_{n}(S)$ bezeichnet. Es ist ein striktes
Hurwitz-Polynom der Ordnung $n$ und stellt eine Approximation von
$D(S)$ dar.  Zur Erl"auterung betrachten wir ein Filter 3.~Ordnung. Es
m"ussen zur Bildung von $D(S)$ also die ersten 3 Glieder der
Kettenbruchzerlegung\index{Kettenbruchzerlegung} ber"ucksichtigt
werden:
\[
\frac{g_{3}(S)}{u_{3}(S)} 
=\frac{1}{S}+
\displaystyle\frac{1}{
\displaystyle\frac{3}{S}+
\displaystyle\frac{1}{
\displaystyle\frac{5}{S}}}
=\frac{6S^{2}+15}{S^{3}+15S}
\]
Damit wird $D(S)=g_{3} (S) + u_{3} (S)=S^{3}+6S^{2}+ 15 S + 15$
und f"ur $T(S)$ erhalten wir:
\[
T(S)=\frac{K}{S^{3}+6S^{2}+15S+15}
\]
mit der gew"ahlten Normierung $T(0)=1$ wird $K=15$. F"ur die Gruppenlaufzeit
folgt:
\begin{equation*}
\tau_{g}(\Omega)=-\frac{d\varphi(\Omega)}{d\Omega}=-\frac{d \left[ -\arctan{\left(\frac{15\Omega-\Omega^{3}}{15-6\Omega^{2}}\right)}\right]}{d\Omega}=\frac{6\Omega^{4}+45\Omega^{2}+225}{\Omega^{6}+6\Omega^{4}
        +45\Omega^{2}+225}.
\end{equation*}
An den speziellen Punkten $\Omega=0$ und $\Omega=
1$ ist $\tau_{g}(0)=1$ und $\tau_{g}(1)=0.99625$.
\subsubsection{Bestimmung der Bessel-Polynome mittels Rekursionsformel}
Die Bessel-Polynome $B_{n}(S)$ k"onnen auch mit folgender
Rekursionsformel\index{Rekursionsformel}
bestimmt werden:\\~~\\
\myboxx{
\begin{equation}
B_{n}(S)=(2n-1)B_{n-1}+S^{2}B_{n-2}\label{eq: bessel_rekursion}
\end{equation}}\\~~\\
wobei $B_0 (S)=1$ und $B_1(S)=S+1$.\\ Zu Beginn der Betrachtungen
haben wir $T_{0}=1$ gesetzt, d.h. die hergeleiteten
"Uber\-tragungs\-funktionen sind alle auf eine
Gruppenlaufzeit\index{Gruppenlaufzeit} von 1 normiert.  Um die
D"ampfungsverl"aufe der Bessel-Filter mit denen der Butterworth-Filter
vergleichen zu k"onnen, m"ussen wir die D"ampfung auch auf
$A(1)=3~\mbox{dB}$ normieren. Dazu muss die Gruppenlaufzeit die Werte
gem"ass (Tab.~\ref{werte-glaufz}) annehmen, d.h., $S$ wird durch $ST_0$ umnormiert.

\bsp{Berechnung von $T_0$ f"ur das Bessel-Filter 3.~Ordnung}
\begin{eqnarray}
|T(ST_0)|_{S=j\cdot 1}&=&\left|\frac{15}{(ST_0)^3+6\cdot(ST_0)^2+15\cdot ST_0+15} \right|_{S=j\cdot 1}\nonumber \\ 
&=&\frac{15}{\left|-jT_0^3-6T_0^2+15jT_0+15 \right|}=\frac{1}{\sqrt{2}}
\end{eqnarray}
Somit wird $(15-6T_0^2)^2+(15T_0-T_0^3)^2=2\cdot 15^2$ und $\rightarrow T_0=1.7557$. Werte von $T_0$ f\"ur Bessel-Filter anderer Ordnung sind in Tabelle~\ref{werte-glaufz} zu finden.
\bsp{Umnormierung einer UTF eines Bessel-Filters 2.~Ordnung}
Normiert auf $T_{0}=1$ ergibt sich mit dem Bessel-Polynom\index{Bessel!Polynom} $B_2(S)$ aus der Tabelle~\ref{b-poly}:\\
\parbox[t]{0.7\textwidth}{
\[
T(S)=\frac{3}{S^{2}+3S+3}.
\]
}\\
Aus der Tabelle~\ref{werte-glaufz} entnehmen wir $T_{0}=1.3617$, damit geht $S$ in $ST_{0}$ 
"uber. Normiert auf $A(1)=3\mbox{~dB}$ erhalten wir:\\
\parbox{0.9\textwidth}{
\[
T(S)=\frac{3}{(1.3617S)^{2}+3(1.3617S)+3}=\frac{1.6180}{S^{2}+2.2032S+1.6180}.
\]
}\\
\\
Abb.~\ref{fg-Bessel} zeigt den Amplitudengang der auf $A(1)=3$~dB normierten
Bessel-Tiefpassfilter verschiedener Ordnung.
\begin{figure}[!htb]
\vspace*{-3mm}
\begin{center}
\bild{/filter/FILn007.eps,width=0.71}\caption{Amplitudeng"ange der normierten Bessel-Tiefpassfilter der Ordnung: 1,2,3 und 10.\label{fg-Bessel}}
\end{center}
\vspace*{-6mm}
\end{figure}

\begin{figure}[!htb]
\vspace*{-3mm}
\begin{center}
\bild{/filter/FIL_bess_phase.eps,width=0.71}\caption{Phaseng"ange der normierten Bessel-Tiefpassfilter der Ordnung: 1,2,3,10 und 20.}
\end{center}
\vspace*{-6mm}
\end{figure}


\clearpage

\begin{figure}[!htb]
\vspace*{1mm}
\begin{center}
\bild{/filter/FIL_bess_tauG.eps,width=0.71}\caption{Gruppenlaufzeiten der normierten Bessel-Tiefpassfilter der Ordnung: 1,2,3,10 und 20.}\index{Gruppenlaufzeit}
\end{center}
\vspace*{-6mm}
\end{figure}

\begin{figure}[!htb]
\vspace*{0mm}
\begin{center}
\bild{/filter/FIL_bess_tauP.eps,width=0.71}\caption{Phasenlaufzeiten der normierten Bessel-Tiefpassfilter der Ordnung: 1,2,3,10 und 20.}\index{Phasenlaufzeit}
\end{center}
\vspace*{-6mm}
\end{figure}


\clearpage
\subsection{Vergleich der Tiefpassapproximationen}
In diesem Abschnitt betrachten wir die Vor- und Nachteile der vorgestellten
Tief\-pass\-approximationen. In der Praxis werden h"aufig
nicht nur Anforderungen an den Amplitudengang, sondern auch an die
Linearit"at des Phasengangs gestellt. Wir werden unser Augenmerk daher
speziell auf den Amplituden- und Phasengang und das daraus
resultierende Zeitverhalten der
Filter richten.\\
Da bei Cauer-Filtern\index{Cauer} der Frequenzgang nicht nur von der
Filterordnung, sondern auch noch von der geforderten D"ampfung im
Sperrbereich abh"angt, lassen sie sich nicht "ubersichtlich in
allgemein g"ultiger Form darstellen. Wir werden uns daher bei der
Diskussion auf die Filtertypen Butterworth\index{Butterworth}, Tschebyscheff\index{Tschebyscheff} und Bessel\index{Bessel}
beschr"anken.

\subsubsection{Vergleich der Pollagen}
Es ist interessant, die Lage der Pole der drei gezeigten Approximationen
miteinander zu vergleichen (Abb.~\ref{lage-pole}).
\begin{figure}[!htb]
\begin{center}
  \bild{/filter/FIL32.ps,width=0.3}\caption{Lage der Pole in der $S$-Ebene f"ur Filter 3. Ordnung \label{lage-pole}}
\end{center}
\vspace*{-6mm}
\end{figure}~~\\
\nit Die Pole f"ur maximal flachen Amplitudengang (Butterworth) liegen auf einem
Halbkreis.  Liegen die Pole n"aher der $j\Omega$-Achse (Ellipse), so
erhalten wir einen konstanten Rippel\index{Rippel} im DB (Tschebyscheff). Liegen jedoch die Pole
weit entfernt von der $j\Omega$-Achse, so wird die Phase linearer (Bessel-Filter).

\subsubsection{Vergleich der Frequenzg"ange}
Um das Verhalten im Frequenzbereich vergleichend beurteilen zu k"onnen,
stellen wir den Amplitudengang, den Phasengang und die daraus abgeleitete
Gr"osse, die Gruppenlaufzeit, der verschiedenen Filter einander gegen"uber.
\begin{figure}[!htb] % plot_delay_phase_utf.m ohne cauer und cheb2ap macht nicht sinn
\vspace*{-3mm}\begin{center}
  \bild{/filter/FILn020.eps,width=1.1}\vspace*{-7mm}\caption{Vergleich der Filtercharakteristiken mit a) Bessel-, b) Butterworth-, c) Tschebyscheff- (0.1~dB) und d) Tschebyscheff- (2~dB) Approximation, wobei die Ordnungen $n=1,2,3,5,8$ und 10 dargestellt sind.}
\end{center}
\vspace*{-6mm}
\end{figure}

\begin{itemize}
\item Das Bessel-Filter weist den flachsten "Ubergang zwischen Durchlass und  
Sperrbereich auf. Es stellt somit die schlechteste N"aherung an den idealen  
Tiefpassamplitudengang dar. Daf"ur besitzt es eine sehr lineare Phase und  
daraus abgeleitet eine nahezu konstante Gruppenlaufzeit.

\item Das Butterworth-Filter zeigt mit steigender Filterordnung einen merklich  
steileren "Ubergang zwischen DB und SB. Die Gruppenlaufzeit zeigt bei der  
Grenzfrequenz schon eine leicht "Uberh"ohung\index{Uberhoehung@{\"Uberh\"ohung}}.

\item Das Tschebyscheff-Filter erf"ullt unsere W"unsche bez"uglich Amplitudengang
noch besser. Erkaufen m"ussen wir uns dies jedoch mit einem noch
schlechteren Gruppenlaufzeitverhalten. Lassen wir im Durchlassbereich noch
gr"osseren Rippel zu, verl"auft der "Ubergang zwischen DB und SB noch
steiler. Das Verhalten der Gruppenlaufzeit\index{Gruppenlaufzeit} verschlechtert sich jedoch abermals
erheblich.
\end{itemize}
{\bf Es l"asst sich also feststellen, dass bei einem Minimalphasennetzwerk bei
immer besserer Ann"aherung an den idealen Amplitudengang sich das
Zeitverhalten zunehmend verschlechtert}. Dieser Sachverhalt ist auch f"ur das hier
nicht aufgef"uhrte Cauer-Filter zutreffend.

\subsubsection{Vergleich des Zeitverhaltens}
Bei den verschiedenen Approximationen (mit Ausnahme des
Bessel-Filters) haben wir uns auf eine m"oglichst gute Nachbildung des
idealen Amplitudenganges\index{Amplitudengang!idealer} konzentriert.
Es gibt aber auch viele Anwendungen, bei denen das Zeitverhalten des
Filters von grossem Interesse ist.  Bevor wir die Sprung- und
Impulsantworten\footnote{F\"ur die Ausdr"ucke Sprung-(antwort) wird
  auch Schritt-(antwort) verwendet. Ebenfalls ist Stoss-(antwort)
  anstelle von Impuls-(antwort) in
  Gebrauch.\index{Sprungantwort}\index{Impuls!antwort}\index{Stossantwort|see{Impulsantwort}}\index{Schrittantwort|see{Sprungantwort}}}
der verschiedenen Tiefpassapproximationen vergleichen, wollen wir kurz
die notwendigen Definitionen betrachten.\\
\paragraph{Definitionen der Sprung- und Impulsantwort}~\\
\nit {\bf a) Die Sprungantwort\index{Sprungantwort}}\\
\nit Als Sprungantwort $f_{u}(t)=h(t)\ast u(t)$ bezeichnet man die
Antwort eines Systems, auf einen Einheitsschritt $u(t)$.  Praktisch
auftretenden Sprungantworten lassen sich durch fol\-gen\-de 4
Gr"os\-sen n"a\-he\-rungs\-wei\-se beschreiben:\\
\begin{tabular}{lll}
\hspace*{9mm}$t_{r}:$ \hspace*{1cm}& Anstiegszeit\index{Anstiegszeit}& (rise time)\index{rise time}\\
\hspace*{9mm}$t_{d}:$             & Verz"ogerungszeit\index{Verzogerungszeit@{Verz\"ogerungszeit}} & (delay time)\index{delay time}\\
\hspace*{9mm}$t_{s}:$     & Einschwingzeit\index{Einschwingzeit} & (settling time)\\
\hspace*{9mm}$o:$                 & "Uberschwingen\index{Uberschwingen@{\"Uberschwingen}} & (overshoot)\index{overshoot}\\
\hspace*{9mm}$\Delta:$            & "Restfehler & (residual error)\index{residual error} \\
\end{tabular}\\
\begin{figure}[!htb]
\vspace*{-3mm}% matlab plot_impuls_sprung_character.m FIL35
\begin{center}
  \bild{/filter/FILn025.eps,width=0.67}\caption{Die charakteristischen Gr"ossen der Sprungantwort (anhand der TP-Sprungantwort des normierten Butterworth-Filters der Ordnung 9)}
\end{center}
\vspace*{-6mm}
\end{figure}~\\
\nit Die Einschwingzeit $t_s$ wird meist mit einem Restfehler von $\Delta=1\%$ angegeben. Im
Falle eines Tiefpasses mit nicht zu grossem "Uberschwingen $(o<5\%)$
gilt zwischen der Anstiegszeit $t_r$ und der
3~dB-Bandbreite\index{3-dB!Bandbreite} ungef"ahr die Beziehung:
\[
t_{r} \approx \frac{2.2}{\omega_{c}}.
\]
Die Verz"ogerungszeit $t_d$ entspricht in etwa der mittleren
Gruppenlaufzeit des
Filters.\\
\nit {\bf b) Die Impulsantwort $h(t)$}\\
\nit Als Impulsantwort\index{Impuls!antwort} $f_{\delta}(t)\equiv h(t)$
bezeichnet man das Ausgangssignal eines Systems, das am Eingang mit
einem Dirac-Impuls\index{Dirac!Impuls} $\delta(t)$ angeregt wurde. Meist wird
f"ur die Bezeichnung der Impulsantwort $h(t)$ verwendet.
Praktisch auftretende Impulsantworten\index{Impuls!antwort} weisen etwa die in
Abb.~\ref{grossen-stoss} dargestellte Form auf.\\
\begin{figure}[!htb]
\begin{center}% matlab plot_impuls_sprung_character.m 
  \bild{/filter/FILn024.eps,width=0.7}\vspace*{-3mm}\caption{Die charakteristischen Gr"ossen einer Impulsantwort (anhand der  TP-Impulsantwort des normierten Butterworth-Filters der Ordnung 9)}
\label{grossen-stoss}
\end{center}
\vspace*{-6mm}
\end{figure}\\
\begin{tabular}{ll}
\hspace*{9mm}$t_{m}:$ \hspace*{1cm} & Mittlere Impulsl"ange \\
\hspace*{9mm}$t_{d}:$              & Verz"ogerungszeit \\
\end{tabular}\\
\nit F"ur einen Tiefpass mit nicht zu grossem "Uberschwingen gilt f"ur die
mittlere Impulsl"ange $t_m$:
\[
t_{m} \approx \frac{2.2}{\omega_{c}}.
\]
\nit Das Maximum der Impulsantwort erscheint etwa um die mittlere
Gruppenlaufzeit $t_d$ des Systems verz"ogert.\\
\nit {\bf c) Der Zusammenhang zwischen Sprung- und Impulsantwort}\\
\nit Zwischen der Sprung- und der Impulsantwort gilt (Kapitel~\ref{Kapitel_Signale_zeitbereich}):
\begin{equation}
f_{u}(t)=\int_{-\infty}^{t} h(\tau) d\tau\quad\text{bzw.}\quad h(t)=\frac{\partial f_{u}(t)}{\partial t}.
\end{equation}\\
Dies ist leicht aus den entsprechenden Gleichungen im Frequenzbereich zu
sehen:
\begin{eqnarray}
F_{u}(S)=\frac{1}{S}H(S)\qquad&\text{und}&\qquad F_{\delta}(S)=H(S).\nonumber
\end{eqnarray}                 
\paragraph{Einfluss des Frequenzganges auf das Zeitverhalten}~\\
Sowohl die aus der Phase abgeleitete Gruppenlaufzeit wie auch der
Amplitudengang nehmen (erw"unscht oder unerw"unscht) Einfluss auf das
Zeitverhalten eines Systems. Dies, weil z.B. ein Impuls im Zeitbereich im
Frequenzbereich eine konstante Amplitudendichte "uber alle Frequenzen
aufweist. Der Amplitudengang "andert nun die spektrale Zusammensetzung, die
Gruppenlaufzeit die zeitliche Lage der einzelnen Frequenzkomponenten des
Eingangssignals.

\paragraph{Vergleich der verschiedenen Tiefpassapproximationen}~\\
Beim Vergleich der Frequenzg"ange\index{Frequenz!gang} haben wir
festgestellt, dass mit einer besseren Ann"aherung an den idealen
Amplitudengang sich das Gruppenlaufzeitverhalten zunehmend
verschlechtert. Diese Tatsache findet auch im Zeitverhalten ihren
Niederschlag.\\
\newpage
\paragraph{Sprungantworten}\index{Sprungantwort}~\\
\begin{figure}[!htb]% alle bilder mit plot_impuls_sprung.m
\vspace*{-3mm}
\begin{center}
  \bild{/filter/FILn014.eps,width=0.75}\caption{Sprungantwort von Bessel-Filtern der Ordnung $n=1,2,3,5,8$ und 10}
\end{center}
\vspace*{-6mm}
\end{figure}

\begin{figure}[!htb]
\vspace*{-3mm}
\begin{center}
  \bild{/filter/FILn016.eps,width=0.75}\caption{Sprungantwort von Butterworth-Filtern der Ordnung $n=1,2,3,5,8$ und 10}
\end{center}
\vspace*{-6mm}
\end{figure}
\newpage
\begin{figure}[!htb] % plot_impuls_spung.m
\begin{center}
  \bild{/filter/FILn018.eps,width=0.75}\caption{Sprungantwort von Tschebyscheff-Filtern mit 0.1~dB Rippel der Ordnung $n=1,2,3,5,8$ und 10}\label{FIL_ABB_SPRUNG_tscheby}
\end{center}
\vspace*{-9mm}
\end{figure}
\aufg
Bestimmen Sie die Sprungantworten von Abb.~\ref{FIL_ABB_SPRUNG_tscheby} mit Hilfe des \mb {\tt step}.\index{step@{\tt step}}
\paragraph{Impulsantworten}\index{Impuls!antwort}~\\
\begin{figure}[!htb]
\vspace*{-3mm}
\begin{center}
  \bild{/filter/FILn015.eps,width=0.75}\caption{Impulsantwort von Bessel-Filtern der Ordnung $n=1,2,3,5,8$ und 10}
\end{center}
\vspace*{-6mm}
\end{figure}
\newpage
\begin{figure}[!!htbp]
\vspace*{-3mm}
\begin{center}
  \bild{/filter/FILn017.eps,width=0.75}\caption{Impulsantwort von Butterworth-Filtern der Ordnung $n=1,2,3,5,8$ und 10}
\end{center}
\vspace*{-6mm}
\end{figure}

\begin{figure}[!!htbp]
\vspace*{-3mm}
\begin{center}
  \bild{/filter/FILn019.eps,width=0.75}\caption{Impulsantwort von Tschebyscheff-Filtern mit 0.1~dB Rippel}\label{impuls_tscheby}
\end{center}
\vspace*{-6mm}
\end{figure}
\aufg
Bestimmen Sie die Impulsantworten von Abb.~\ref{impuls_tscheby} mit Hilfe des \mb {\tt impulse}.\\
\newpage
\nit Die Frage, welches Filter das Beste ist, l"asst sich nicht generell
beantworten. Es h"angt davon ab, ob gutes Frequenz- und / oder gutes
Zeitverhalten erw"unscht ist.\\ ~\\ \nit Bei der Daten"ubertragung sind nur
kleine Verzerrungen zugelassen, da die "Uberlagerung zeitlich
aufeinanderfolgender Impulse zu Fehlern f"uhren kann. Es ist neben
der Frequenzselektivit"at\index{Frequenz!selektivitae@{selektivit\"a}t} auch ein gutes
Gruppenlaufzeitverhalten\index{Gruppenlaufzeit} wichtig. Solche Filter
werden durch Kaskadierung zweier Sektionen realisiert. Die erste Sektion formt
den erforderlichen Amplitudengang mit einem bekannten Filtertyp, die
zweite korrigiert das unzureichende Gruppenlaufzeitverhalten mit einem Allpassnetzwerk.
\paragraph{Bemerkung zu weiteren Filterapproximationsarten}~\\
Es existiert unter anderem eine ``inverse''
Tschebyscheff-Approximation\index{Tschebyscheff!invers}, welche, im
Gegensatz zur ``normalen''
Tschebyscheff-Approximation\index{Tschebyscheff!Approximation}, den
Rippel im Sperrbereich besitzt und im Durchlassbereich flach ist
(siehe \mb {\tt cheb2ap}). Die Ordnung dieser inversen
Tschebyscheff-Filter ist gleich gross wie die vom Tschebyscheff-Filter
\cite{VAL:82}. Das ``inverse'' Tschebyscheff-TP-Filter ist aber kein
Allpol-Filter. \\ Es gibt weitere Approximationsm"oglichkeiten die wir
hier nicht besprechen. Den interessierten Leser verweise ich auf
\cite{VAL:82, ZVE:67}.  \clearpage